\chapter{Nichtfunktionale Anforderungen}
\label{ch:nfa}

\setcounter{nfanr}{10}
\newcommand{\nfano}{\ifnum\value{nfanr}<10 00\else\ifnum\value{nfanr}<100 0\fi\fi\arabic{nfanr}}
\newcommand\nfa[2]{\namedlabel{nfa:#1}{/NFA\nfano/}\addtocounter{nfanr}{10}: & #2 \\ [1ex] }

\section{Graph von Ansicht}

Falls eine der in \ref{nfa:maxknoten}  und \ref{nfa:maxkanten} beschriebenen Anforderungen überschritten wird, kann es zu (polynomiell) längeren Berechnungszeiten kommen als in \ref{nfa:berechzeit} beschrieben. Falls es zusätzlich zu ein Mangel von Hardware-Resourcen kommen sollte, wird die derzeit ausgeführte Funktion abgebrochen und dem Nutzer eine Fehlermeldung ausgegeben.

Alle in \ref{nfa:maxknoten}, \ref{nfa:maxknotentotal}, \ref{nfa:maxkanten} und \ref{nfa:berechzeit} beschriebenen Anforderungen gelten für die in \autoref{sec:hardware} aufgelisteten Hardware-Anforderungen. Bei Verwendung leistungstärkerer Hardware, kann erwartet werden, dass größere Graphen/Graphdateien verwendet werden können und die Berechnung des Layouts für Graphen gleicher Größe kürzer dauert.\\


\begin{tabular}{lp{0.9\linewidth}}
  \nfa{maxknoten}{Beim Layouten werden mindestens Graphen mit bis zu 1000 Knoten unterstützt. Dies entspricht nicht der maximalen Knotenzahl in einer unterstützten Graphdatei (siehe dafür \ref{nfa:maxknotentotal}).}
  \nfa{maxknotentotal}{Es werden mindestens Graphdateien mit bis zu 100.000 Knoten insgesamt unterstützt.}
  \nfa{maxkanten}{Die maximal unterstützte Kantenanzahl pro Graph entspricht mindestens der 3-4 fachen Knotenzahl aus \ref{nfa:maxknoten} und \ref{nfa:maxknotentotal}.}

  \nfa{algowechsel}{Der Algorithmus, welcher das Graphlayout berechnet, soll einfach auswechselbar sein.}
  \nfa{heuristic}{Durch heuristische Methoden soll ein möglichst optimales Ergebnis bei der Einhaltung gegebener Constraints erreicht werden.}
\end{tabular}

\section{User Interface}\label{sec:nfaui}
\setcounter{nfanr}{100}
\begin{tabular}{lp{0.9\linewidth}}
  \nfa{sprachwechsel}{Das Programm soll so entworfen werden, dass ein einfacher Sprachwechsel der \gls{gui} möglich ist.}
  \nfa{cmdfehler}{Falls das Programm über die Kommandozeile gestartet wird und falsche Parameter übergeben werden, stürzt das Programm nicht ab. Es wird eine informative Fehlermeldung zurückgegeben.}
\end{tabular}

\section{Graphlayout für \gls{joana}}\label{sec:nfajoana}
\setcounter{nfanr}{200}

\begin{tabular}{lp{0.9\linewidth}}
  \nfa{constraints}{Für Constraints die nicht durch den Nutzer festgelegt wurden (siehe \ref{fa:constraints} und \ref{fa:constraintsettings}) soll } %TODO: Punkte unten anpassen
  \nfa{berechzeit}{Die Berechnung eines Graphlayouts ist abhängig von der Knoten- und Kantenzahl, sowie der Hardware des benutzten Systems (vgl. Mindestanforderungen in \autoref{sec:hardware}), und überschreitet bei einer Anzahl von 1000 Knoten und 4000 Kanten 2 Minuten nicht. Bei einer Knotenzahl von 500 und einer Kantenzahl von 2000 wird bei der Berechnung eine Minute nicht überschritten.}
  \nfa{kantenabstand}{Kanten sollen nicht direkt übereinander liegen, sondern immer mit einem gewissen Mindestabstand nebeneinander.}
  \nfa{cutknotenknoten}{Die graphische Darstellung eines Knoten darf sich nicht mit der eines anderen Knoten überschneiden, so dass Teile eines Knoten verdeckt wären.}
  \nfa{cuttext}{Beschriftungen von Knoten dürfen sich nicht mit anderen Objekten überschneiden, sodass Teile der Beschriftung verdeckt/unkenntlich werden.}
  \nfa{kantekreuz}{Die Anzahl der sich kreuzenden Kanten wird mittels heuristischer Verfahren möglichst gering gehalten.}
\end{tabular}
