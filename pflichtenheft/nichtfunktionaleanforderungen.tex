\chapter{Nichtfunktionale Anforderungen}
\label{ch:nfa}

\newcounter{nfanr}[chapter]
\setcounter{nfanr}{10}
\newcommand{\nfano}{\ifnum\value{nfanr}<10 00\else\ifnum\value{nfanr}<100 0\fi\fi\arabic{nfanr}\addtocounter{nfanr}{10}}
\renewcommand\thesubsubsection{/NFA\ifnum\value{nfanr}<10 000\else\ifnum\value{nfanr}<100 00\else\ifnum\value{nfanr}<1000 0\fi\fi\fi\arabic{nfanr}/}
\newcommand\nfa[2]{\namedlabel{nfa:#1}{\textbf{/NFA\nfano/}}: & #2 \\ [1ex] }

%TODO: Durchlesen und alle unklaren Begriffe/Fachwörter ins Glossar hinzufügen
\section{Graph von Ansicht}

\begin{tabular}{lp{0.9\linewidth}}
  \nfa{maxknoten}{Es werden Graphen mit bis zu 10.000 Knoten unterstützt. Dies entspricht nicht der maximalen Knotenzahl in einer unterstützten Graphdateien.}
  \nfa{maxknotentotal}{Es werden Graphdateien mit nicht mehr als 1.000.000 Knoten insgesamt unterstützt.}
  \nfa{sprachwechsel}{Das Programm soll so entworfen werden dass ein einfacher Sprachwechsel der GUI möglich ist.}
  \nfa{algowechsel}{Der Algorithmus welcher das Graphlayout entwirft soll einfach auswechselbar sein.}
  %TODO: Ein testbares Maß für maximale Kantenkreuzungen finden.
\end{tabular}

\section{\gls{joana}-Plugin}

\begin{tabular}{lp{0.9\linewidth}}
  \nfa{berechzeit}{Die Berechnung eines Graphlayouts dauert höchstens 2 min.}
  \nfa{graphsicht}{Es existieren zwei Graphansichten. Einmal ein \gls{callgraph} welcher die Abhängigkeiten der Methoden darstellt und ein \gls{methgraph} welcher den Steuer- und Datenfluss innerhalb einer Methode visualisiert.}
  \nfa{layerspecs}{Bei einem \gls{methgraph} soll der Entry-Knoten auf dem obersten Layer dargestellt werden, und ein Layer darunter die Parameter für die Methode in fester Reihenfolge.}
  \nfa{methaufrufe}{Bei einem Knoten welcher einen Methodenaufruf darstellt, sollen die dazugehörigen Knoten (Parameter und Rückgabewert) ein Layer unter dem Methodenaufruf stehen.}
  \nfa{feldzugr}{Feldzugriffe werden bei \gls{joana} immer gleich dargestellt. Diese Zugriffe sollen erkannt werden und die dazugehörigen Knoten sollen automatisch immer im gleichen Muster angezeigt werden.}
  \nfa{kantenabstand}{Kanten sollen nicht direkt übereinander liegen, sondern immer mit einem gewissen Mindestabstand nebeneinander.}
  \nfa{cutknotenknoten}{Die graphische Darstellung eines Knoten, darf sich nicht mit der eines anderen Knoten überschneiden, so dass Teile eines Knoten verdeckt sind.}
  \nfa{cutknotenkante}{Kanten dürfen nicht durch Knoten verlaufen.}
  \nfa{cuttext}{Beschriftungen von Knoten oder Kanten dürfen sich nicht mit anderen Objekten überschneiden, sodass Teile der Beschriftung verdeckt/unkenntlich werden.}
\end{tabular}
