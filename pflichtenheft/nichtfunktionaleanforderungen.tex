\chapter{Nichtfunktionale Anforderungen}
\label{ch:nfa}


\setcounter{nfanr}{10}
\newcommand{\nfano}{\ifnum\value{nfanr}<10 00\else\ifnum\value{nfanr}<100 0\fi\fi\arabic{nfanr}}
\newcommand\nfa[2]{\namedlabel{nfa:#1}{/NFA\nfano/}\addtocounter{nfanr}{10}: & #2 \\ [1ex] }

%TODO: Durchlesen und alle unklaren Begriffe/Fachwörter ins Glossar hinzufügen
\section{Graph von Ansicht}

\begin{tabular}{lp{0.9\linewidth}}
  \nfa{maxknoten}{Es werden Graphen mit bis zu 1000 Knoten unterstützt. Dies entspricht nicht der maximalen Knotenzahl in einer unterstützten Graphdateien.}
  \nfa{maxknotentotal}{Es werden Graphdateien mit nicht mehr als 10.000 Knoten insgesamt unterstützt.}
  \nfa{maxkanten}{Die maximal unterstützte Kantenanzahl pro Graph entspricht der 3-4 fachen Knotenzahl.}
  \nfa{algowechsel}{Der Algorithmus, welcher das Graphlayout bestimmt, soll einfach auswechselbar sein.}
\end{tabular}

\section{User Interface}\label{sec:nfaui}
\setcounter{nfanr}{100}
\begin{tabular}{lp{0.9\linewidth}}
  \nfa{sprachwechsel}{Das Programm soll so entworfen werden, dass ein einfacher Sprachwechsel der \gls{gui} möglich ist.}
  \nfa{cmdfehler}{Falls das Programm über die Kommandozeile gestartet wird und falsche Parameter übergeben werden, stürzt das Programm nicht ab. Es wird eine informative Fehlermeldung zurückgegeben.} %TODO: Ähnliche Definition für GUI?
\end{tabular}

\section{Graphlayout für \gls{joana}}\label{sec:nfajoana}
\setcounter{nfanr}{200}

\begin{tabular}{lp{0.9\linewidth}}
  \nfa{berechzeit}{Die Berechnung eines Graphlayouts dauert höchstens 2 Minuten.}
  \nfa{graphsicht}{Es existieren zwei Graphansichten. Einmal ein \gls{callgraph} welcher die Abhängigkeiten der Methoden darstellt und ein \gls{methgraph} welcher den Steuer- und Datenfluss innerhalb einer Methode visualisiert.}
  \nfa{layerspecs}{Bei einem \gls{methgraph} soll der Entry-Knoten auf dem obersten Layer dargestellt werden, und ein Layer darunter die Parameter für die Methode in fester Reihenfolge.}
  \nfa{methaufrufe}{Bei einem Knoten, welcher einen Methodenaufruf darstellt, sollen die dazugehörigen Knoten (Parameter und Rückgabewert) ein Layer unter dem Methodenaufruf stehen.}
  \nfa{feldzugr}{Feldzugriffe werden bei \gls{joana} immer gleich dargestellt. Diese Zugriffe sollen erkannt werden und die dazugehörigen Knoten sollen automatisch immer im gleichen Muster angezeigt werden.}
  \nfa{kantenabstand}{Kanten sollen nicht direkt übereinander liegen, sondern immer mit einem gewissen Mindestabstand nebeneinander.}
  \nfa{cutknotenknoten}{Die graphische Darstellung eines Knoten darf sich nicht mit der eines anderen Knoten überschneiden, so dass Teile eines Knoten verdeckt wären.}
  \nfa{cuttext}{Beschriftungen von Knoten dürfen sich nicht mit anderen Objekten überschneiden, sodass Teile der Beschriftung verdeckt/unkenntlich werden.}
  %TODO: Ein testbares Maß für maximale Kantenkreuzungen finden.
\end{tabular}
