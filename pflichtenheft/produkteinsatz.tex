\chapter{Produkteinsatz}

\section{Anwendungsbereiche}
Das Produkt soll zur Visualisierung von Programmgraphen eingesetzt werden. Der Nutzer soll dadurch die Abhängigkeiten und den Steuerfluss eines Programms besser verstehen.

\section{Zielgruppen}
Institute und Forschungsgruppen, die sich mit der Analyse von Programmen beschäftigen. Das Produkt kann auch zu Lehrzwecken an Schulen oder Hochschulen angewendet werden, um Schülern und Studenten den Aufbau von Programmen näher zu bringen. %TODO Satz noch spezifizieren

\section{Betriebsbedingungen}
Das Produkt besteht aus einem Graphviewer (Graph von Ansicht) und wird mit einem JOANA-Plugin ausgeliefert. Somit benötigt es keine weiteren Installationen (abgesehen von der \gls{jre}), damit es funktioniert. Es benötigt auch keine aktive Internetverbindung oder ein Netzwerk. Es wird lediglich eine Graphdatei als Input benötigt. Das Programm ist für Linux und Windows Betriebssysteme ausgelegt und funktioniert nur auf diesen garantiert (siehe \nameref{ch:umgebung}).
%TODO Plugin geneauer beschreiben, Verweis auf die genau beschreibung der Plugin Struktur.