\chapter{Produkteinsatz}\label{produkteinsatz}

\section{Anwendungsbereiche}
Das Produkt soll zur Visualisierung von Programmgraphen eingesetzt werden.
Der Nutzer soll dadurch die Abhängigkeiten und den Steuerfluss eines Programms besser verstehen.
Eine mögliche Anwendung ist das Finden von Sicherheitslücken, welche durch Abhängigkeiten von sicherheitsrelevanten Daten mit unvertraulichen Quellen/Senken in Programmen sichtbar werden.
Bei Präsentationen kann das Produkt benutzt werden um dem Publikum einen (Ausschnitt eines) Graphen vorzustellen.

\section{Zielgruppen}
Institute und Forschungsgruppen, die sich mit der Analyse von Programmen beschäftigen.
Das Produkt kann auch zu Lehrzwecken an Schulen oder Hochschulen angewendet werden, um Schülern und Studenten Programmabhänghigkeitsgraphen näher zubringen.

\section{Betriebsbedingungen}
Das Produkt besteht aus einem Graphviewer (Graph von Ansicht) und wird mit einem JOANA-Plugin ausgeliefert.
Somit benötigt es keine weiteren Installationen (abgesehen von der \gls{jre}).
Es benötigt auch keine aktive Internetverbindung oder ein Netzwerk.
Es wird lediglich eine Graphdatei als Input benötigt.
Das Programm ist für Linux und Windows Betriebssysteme ausgelegt und funktioniert nur auf diesen garantiert (siehe \nameref{ch:umgebung}).
