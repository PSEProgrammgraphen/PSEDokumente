\chapter{Funktionale Anforderungen}
\label{ch:funktionen}

\newcounter{fanr}[chapter]
\setcounter{fanr}{10}
\newcommand{\fano}[1]{\subsection{#1}\addtocounter{fanr}{10}}
\newcommand{\subfano}[1]{\subsubsection{#1}\addtocounter{fanr}{1}}
\renewcommand\thesubsection{/FA\ifnum\value{fanr}<10 000\else\ifnum\value{fanr}<100 00\else\ifnum\value{fanr}<1000 0\fi\fi\fi\arabic{fanr}/}
\renewcommand\thesubsubsection{/FA\ifnum\value{fanr}<10 000\else\ifnum\value{fanr}<100 00\else\ifnum\value{fanr}<1000 0\fi\fi\fi\arabic{fanr}/}

\section{Muss-Funktionen} %TODO: Evtl. Umbennen

\fano{Graph aus Graph-Datei laden}\label{fa:laden}
\textbf{Ziel:} Der in der Graph-Datei beschriebene Graph soll visuell dargestellt werden. \\
\textbf{Kategorie:} \gls{io} \\
\textbf{Vorbedingung:} Der Nutzer hat eine korrekte Graph-Datei in seinem Dateissystem, auf welche zugegriffen werden kann. \\ % TODO: 'korrekt' vlt. in einem anderen Abschnitt definieren (GraphML)
\textbf{Nachbedingung (Erfolg):} Der Graph wird dargestellt. \\
\textbf{Nachbedingung (Fehlschlag):}
Eine Fehlermeldung wird ausgegeben, dass die Datei nicht geöffnet werden konnte, bzw. dass die Datei keinen korrekten Graph beschreibt. \\
\textbf{Auslösende Ereignisse:}
\begin{enumerate}[nolistsep, label=(\alph*)]
  \item Der Nutzer ruft das Programm auf und übergibt den Pfad zur Graph-Datei als Argument.
  \item Der Nutzer wählt die Graph-Datei über das Menü aus, nachdem das Programm geöffnet wurde.
\end{enumerate}
\textbf{Beschreibung:}
\begin{enumerate}[nolistsep]
  \item (a) Nutzer öffnet das Programm mit dem Pfad zur Graph-Datei als Argument.
  \item (a) Das Programm öffnet sich.
  \item Ein Fenster mit Präferenzmöglichkeiten bzgl. der Darstellung und Interpretation des Graphen öffnet sich.
  \item Der Nutzer wählt die gewünschten Einstellungen und bestätigt diese.
  \item Die Darstellung des Graphen wird berechnet und schließlich angezeigt.
\end{enumerate}
\textbf{Alternativen:}
\begin{enumerate}[nolistsep]
  \item (b) Der Nutzer wählt die Funktion Datei-->Öffnen.. aus dem Menübalken aus. %TODO: Pfeil schöner
  \item (b) Der Nutzer wählt die Graph-Datei über ein Dateiverzeichnis-Menü aus.%TODO: Dateiverzeichnis-Menü auch in GUI-Entwurf oder klar?
\end{enumerate}


\fano{Sichtfeld verschieben}\label{fa:ziehen} %Ref GUI Sichtfeld
\textbf{Ziel:} Das Sichtfeld soll verschoben werden. \\
\textbf{Kategorie:} Navigation \\
\textbf{Vorbedingung:} Ein Graph wurde geladen und das Sichtfeld deckt nicht alle Elemente des Graphens ab. \\
\textbf{Nachbedingung (Erfolg):}  Das Sichtfeld deckt nun einen anderen Teil des Graphen ab.\\
\textbf{Nachbedingung (Fehlschlag):} Das Sichtfeld deckt einen Randabschnitt ab und es wurde versucht das Sichtfeld über den Rand hinaus zu bewegen. \\
Das Sichtfeld wird nicht über den Rand bewegt. Keine Fehlermeldung. \\
\textbf{Auslösende Ereignisse:}
\begin{enumerate}[nolistsep, label=(\alph*)]
  \item Der Nutzer klickt und zieht mit der mittleren Maustaste in einem leeren Bereich des Sichtfeldes.
  \item Der Nutzer betätigt eine der zum Verschieben des Sichtfeldes designierten Tasten (siehe FAXXXX). %TODO: Referenz auf Tasten Auswahl funktion
  \item Der Nutzer bewegt die Scroll-Balken am Rand des Sichtfeldes. % TODO: Referenzt auf GUI Entwurf, wo scrollbalken sichtbar sind
\end{enumerate}
\textbf{Beschreibung:}
\begin{enumerate}[nolistsep]
  \item (a) Der Nutzer klickt und zieht mit der mittleren Maustaste in einem leeren Bereich des Sichtfeldes.
  \item (a) Das Sichtfeld verschiebt sich entgegen der Richtung, in welche die Maus bewegt wird.
\end{enumerate}
\textbf{Alternativen:}
\begin{enumerate}[nolistsep]
  \item (b) Eine Navigationstaste wird betätigt.
  \item (b) Das Sichtfeld bewegt sich um eine feste Länge (in Abhängigkeit vom Zoom-Grad) in die entsprechende Richtung.
\end{enumerate}
\begin{enumerate}[nolistsep]
  \item (c) Einer der beiden Scroll-Balken (horizontal/vertikal) wird bewegt.
  \item (c) Das Sichtfeld bewegt sich relativ zur Größe des gesamten Graphen um das gleiche Maß wie der Scroll-Balken zur Länge seiner Fahrbahn in die entsprechende Richtung.
\end{enumerate}

\fano{Zoom-Grad ändern}\label{fa:zoomen}
\textbf{Ziel:} Den Zoom-Grad des Sichtfeldes soll vergrößert bzw. verkleinert werden. \\
\textbf{Kategorie:} Navigation \\
\textbf{Vorbedingung:} Ein Graph wurde geladen. \\
\textbf{Nachbedingung (Erfolg):} Der Zoom-Grad hat sich geändert. \\
\textbf{Nachbedingung (Fehlschlag):} Der Zoom-Grad bleibt gleich, falls ein Maximum/Minimum erreicht wurde. Keine Fehlermeldung. \\
\textbf{Auslösende Ereignisse:}
\begin{enumerate}[nolistsep, label=(\alph*)]
  \item Der Nutzer dreht am Mausrad.
  \item Der Nutzer betätigt die zum Zoomen designierten Tasten. (siehe FAXXXX). %TODO: Referenz auf Tasten Auswahl funktion
\end{enumerate}
\textbf{Beschreibung:}
\begin{enumerate}[nolistsep]
  \item (a) Der Zoom-Grad ändert sich mit dem Drehen. Die Richtung kann sich abhängig vom Betriebssystem ändern.
  \item Falls das Maximum/Minimum erreicht wurde, wird dieses nicht überschritten.
\end{enumerate}
\textbf{Alternativen:}
\begin{enumerate}[nolistsep]
  \item (b) Der Zoom-Grad ändert sich um einen in Abhängigkeit zum derzeitigen Zoom-Grad festen Wert.
\end{enumerate}

\fano{Knoten selektieren}\label{fa:selekt_knoten}
\textbf{Ziel:} Ein oder mehrere Knoten sollen der Auswahl hinzugefügt werden. \\
\textbf{Kategorie:} Auswahl \\
\textbf{Vorbedingung:} Ein Graph wurde geladen. \\
\textbf{Nachbedingung (Erfolg):} Ein oder mehrere Knoten wurden der Auswahl hinzugefügt. \\
\textbf{Nachbedingung (Fehlschlag):} - \\
\textbf{Auslösende Ereignisse:}
\begin{enumerate}[nolistsep, label=(\alph*)]
  \item Der Nutzer klickt mit der linken Maustaste auf einen Knoten.
  \item Der Nutzer klickt mit der linken Maustaste in einen leeren Bereich des Sichtfeldes und zieht die Maus nach rechts unten.
  \item Der Nutzer klickt mit der linken Maustaste in einen leeren Bereich des Sichtfeldes und zieht die Maus nach links oben.
\end{enumerate}
\textbf{Beschreibung:}
\begin{enumerate}[nolistsep]
  \item Falls die Shift-Taste nicht gedrückt ist werden alle Knoten von der Auswahl entfernt.
  \item (a) Der Knoten wird der Auswahl hinzugefügt.
\end{enumerate}
\textbf{Alternativen:}
\begin{enumerate}[nolistsep]
  \setcounter{enumi}{1}
  \item (b) Es wird eine Auswahlbox vom Startpunkt bis zur jetztigen Position des Mauszeigers gezeichnet.
  \item (b) Der Nutzer lässt die linke Maustaste los, oder drückt eine andere Taste.
  \item (b) Alle Knoten, die in der Auswahlbox komplett enthalten, sind werden selektiert.
\end{enumerate}
\begin{enumerate}[nolistsep]
  \setcounter{enumi}{1}
  \item (c) Es wird eine Auswahlbox vom Startpunkt bis zur jetztigen Position des Mauszeigers gezeichnet.
  \item (c) Der Nutzer lässt die linke Maustaste los, oder drückt eine andere Taste.
  \item (c) Alle Knoten, die in der Auswahlbox enthalten sind oder von ihr geschnitten werden, werden selektiert.
\end{enumerate}

\fano{Export als Bilddatei}\label{fa:export_img}
\textbf{Ziel:} Der Graph soll in seiner aktuellen Darstellung als Bilddatei exportiert werden. \\
\textbf{Kategorie:} \gls{io} \\
\textbf{Vorbedingung:} Ein Graph wurde geladen.  \\
\textbf{Nachbedingung (Erfolg):} Es wurde eine Bilddatei an einer gewünschten, schreibbaren Stelle im Dateissystem erstellt, welche ein Abbild des derzeit angezeigten Graphens enthält. \\
\textbf{Nachbedingung (Fehlschlag):} Die Bilddatei konnte nicht erstellt werden. Es wird eine Fehlermeldung ausgegeben. \\
\textbf{Auslösendes Ereignis:}
Der Nutzer wählt die Exportfunktion im Menübalken aus.
\textbf{Beschreibung:}
\begin{enumerate}[nolistsep]
  \item Der Nutzer wählt die Exportfunktion im Menübalken unter Datei-->Export--><Dateiformat> aus. Wobei <Dateiformat> durch das gewünschte Bilddatei-Format zu ersetzen ist. Unterstützt werden mindestens das \gls{svg}- und \gls{jpg}-Format. %TODO: Referenz auf Dateiformate und Erweiterbarkeit durch Plugins
  \item Der Nutzer gibt über ein Dateisystem-Menü den gewünschten Pfad zum Abspeichern der Bilddatei an.
  \item Eventuell wird der Nutzer nach weiteren, zur Abspeicherung relevanten, Details abgefagt. (z.B. Stärke der Komprimierung bei JPEG)
  \item Es wird versucht die Bilddatei am ausgewählten Ort abzuspeichern.
\end{enumerate}
\textbf{Alternativen:} -

%\fano{<++>}\label{fa:<++>}
%\textbf{Ziel:} <++> \\
%\textbf{Kategorie:} <++> \\
%\textbf{Vorbedingung:} <++> \\
%\textbf{Nachbedingung (Erfolg):} <++> \\
%\textbf{Nachbedingung (Fehlschlag):} <++> \\
%\textbf{Auslösende Ereignisse:}
%\begin{enumerate}[nolistsep, label=(\alph*)]
%  \item <++>
%\end{enumerate}
%\textbf{Beschreibung:}
%\begin{enumerate}[nolistsep]
%  \item <++>
%\end{enumerate}
%\textbf{Alternativen:}
%\begin{enumerate}[nolistsep]
%  \item <++>
%\end{enumerate}
