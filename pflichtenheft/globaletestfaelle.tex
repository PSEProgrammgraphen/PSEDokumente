\chapter{Globale Testfälle}

\setcounter{tnr}{10}
\newcommand{\testno}{\ifnum\value{tnr}<10 00\else\ifnum\value{tnr}<100 0\fi\fi\arabic{tnr}\addtocounter{tnr}{10}}
\newcommand\test[2]{\namedlabel{test:#1}{\textbf{/T\testno/}}: & #2 \\ [1ex] }
\newcommand\etest[3]{\namedlabel{test:#1}{\textbf{/T\testno/}}: & #2 \\ & Erweiterung: #3 \\ [1ex] }

%TODO: Siehe im Tips Dokument für Beispiel für richtige Testfälle. Die Wichtigsten Funkt. anforderungen testen.
\begin{tabular}{lp{0.9\linewidth}}
  \test{start}{Das Programm wird vom Tester gestartet.
    Es wird überprüft, ob sich die GUI öffnet und, wie in \ref{ch:gui} beschrieben, dargestellt wird.}
  \etest{startcmd}{Das Programm wird vom Tester über die Kommandozeile geöffnet.
    Die Graphdatei wird als Argument mitübergeben.
    Wie in \ref{test:start} wird die korrekte Darstellung der GUI überprüft.
    Zusätzlich wird getestet, ob der(die) Graph(en) aus der Graphdatei korrekt geladen werden. \ref{fa:import}}
    {Alle möglichen validen Kombinationen von zusätzlichen Argumenten, beschrieben in \ref{sec:uicmd}, werden ausprobiert.
    Außerdem werden auch invalide Argumente übergeben und überprüft \ref{nfa:gracefulexit} eingehalten wird.}
  \test{import}{Der Tester wählt über die Menüleiste eine Import-Möglichkeit aus.
    Es wird überprüft ob, die GUI die Anfrage, wie in \ref{fa:import} beschrieben,
    behandelt und, ob die Graphen in der Graphdatei in der Graph-Übersicht gelistet werden.}
  \test{export}{Ein Graph ist in der Graphansicht geladen.
    Der Tester wählt über die Menüleiste eine Export-Möglichkeit.
    Es wird überprüft, ob die GUI die Anfrage, wie in \ref{fa:export_img} beschrieben,
    behandelt und ob die exportierte Bilddatei mit der in der Graphansicht dargestellten Visualisierung des Graphen übereinstimmt.}
  \test{layouten}{Der Tester wählt über die Menüleiste eine Layout-Möglichkeit aus.
    Es wird überprüft, ob das Layout auf den Graphen umgesetzt wird \ref{fa:layout}
    und die Zeitbedingungen \ref{nfa:berechzeit} eingehalten werden.
    Für den Fall, dass JOANA-Layout ausgewählt wurde,
    werden zusätzlich alle Nichtfunktionalen Anforderungen für das \nameref{sec:nfajoana} überprüft.}
  \test{navigation}{Der Tester führt alle in \ref{fa:zoom} \ref{fa:verschieben} beschriebene Navigationsaktionen aus
    und überprüft, ob das Produkt sich wie in den Anforderungen beschrieben verhält.}
  \test{filter}{Ein Graph ist in der Graphansicht geladen.
    Der Tester wählt eine Filter-Funktion aus der Menüleiste aus.
    Es wird überprüft, ob genau die Knoten ausgeblendet bzw. wieder eingeblendet werden, welche das Filterkriterium erfüllen.}
\end{tabular}
