\chapter{Globale Testfälle}\label{ch:test}

\newcounter{stnr}
\setcounter{tnr}{10}

\newcommand{\testno}{\ifnum\value{tnr}<10 00\else\ifnum\value{tnr}<100 0\fi\fi\arabic{tnr}}
\newcommand\test[2]{\namedlabel{test:#1}{\textbf{/T\testno/}}\addtocounter{tnr}{10}\setcounter{stnr}{1} \hspace*{0.003\linewidth} \textbf{#2} \\ \\}

\newcommand\subtest[3]{\textbf{\arabic{stnr}}\addtocounter{stnr}{1}. & \textbf{Stand:} & #1 \\ & \textbf{Aktion:} & #2 \\ & \textbf{Reaktion:} & #3 \\ [1ex] }

\test{importJOANA}{Import und Darstellung von einem JOANA Graphen} 
\begin{tabular}{llp{0.9\linewidth}}
	\subtest
		{Graph von Ansicht ist gestartet und es wird kein Graph dargestellt}
		{Den Menüeintrag Datei->Import auswählen}
		{Ein Dateiauswahldialog wird geöffnet.}
	\subtest
		{Eine von JOANA exportierte Datei wurde im Dateisystem ausgewählt}
		{Button zum Laden der ausgewählten Datei drücken}
		{Der Dateiauswahldialog schließt sich und ein Dialog zur Auswahl des Layoutalgorithmus öffnet sich.}
	\subtest
		{Der Dialog zur Auswahl des Layoutalgorithmus wird angezeigt}
		{Der JOANA-Algorithmus wird ausgewählt und Ok gedrückt}
		{Der Dialog schließt sich und der \gls{callgraph} der geladen Datei wird im passenden Layout (siehe \ref{fa:layout}) angezeigt.}
\end{tabular}
\\ \\ \\
\test{openmethod}{Öffnen eines JOANA-Methodengraphen}
\begin{tabular}{llp{0.9\linewidth}}
	\subtest
		{Ein JOANA-Graph wurde importiert, der \gls{callgraph} wird angezeigt (siehe Test \ref{test:importJOANA}).}
		{Doppelklick auf einen Callknoten in der Strukturansicht.}
		{Ein neuer Tab öffnet sich, in dem der \gls{methgraph} auf den der Knoten verweist in passendem Layout dargestellt wird.}
	\subtest
		{Ein \gls{methgraph} wird dargestellt.}
		{In der Tableiste über der Graphansicht den ursprünglichen \glspl{callgraph} aus Schritt 1. auswählen.}
		{Der \gls{callgraph} des JOANA-Imports aus Schritt 1. wird wieder angezeigt.}
	\subtest
		{Der \gls{callgraph} wird angezeigt.}
		{Rechte Maustaste auf einen Callknoten.}
		{Das Kontextmenü für den selektierten Knoten öffnet sich.}
	\subtest
		{Das Kontextmenü für den Knoten wird angezeigt.}
		{Den Menüeintrag Öffnen auswählen.}
		{Ein neuer Tab öffnet sich, oder bei selber Auswahl wie in Schritt 1. wird zu dem bereits offenen Tab gewechselt, in dem der \gls{methgraph} auf den der Knoten verweist in passendem Layout dargestellt wird.}
\end{tabular}
\\ \\ \\
\test{navigation}{Navigation und Zoom}
\begin{tabular}{llp{0.9\linewidth}}
	\subtest
		{Ein JOANA-Graph wurde importiert und einer der enthaltenen \glspl{methgraph} oder \glspl{callgraph} wird angezeigt (siehe Test \ref{test:importJOANA}, \ref{test:openmethod}).}
		{Den Menüeintrag Ansicht->Zoom->Erhöhen auswählen.}
		{Der Mauszeiger ändert sich zu einer Lupe mit einem Plus in der Linse.}
	\subtest
		{Der Mauszeiger zeigt eine Lupe und befindet sich über der Graphansicht.}
		{Linksklick innerhalb der Graphansicht, in die nähe eines \glspl{subgraph}}
		{Das vorherige Sichtifeld wurde, in Richtung der Knoten und Kanten die in der Nähe des Mauszeigers beim Klick waren, herangezoomt.}
	\subtest
		{Der angezeigte Graph wird leicht herangezoomt dargestellt.}
		{Mittlere Maustaste in einem leeren Bereich der Graphansicht drücken und halten und ziehen des Mauszeigers.}
		{Das Sichtfeld wird, wenn nicht der Rand des Graphen erreicht wurde entgegen der Richtung in die der Mauszeiger bewegt wird, verschoben.}
%	\subtest{herauszoomen}
%	\subtest{kollabieren}
%	\subtest{ausklappen}
\end{tabular}
\\ \\ \\
\test{select}{Selektieren mehrerer Knoten und Kanten}
\begin{tabular}{llp{0.9\linewidth}}
	\subtest
	{Ein JOANA-Graph wurde importiert und einer der enthaltenen \glspl{methgraph} oder \glspl{callgraph} wird angezeigt.}
	{Selektieren eines \glspl{subgraph}, durch ziehen oder einzelnem Auswählen.}
	{Die ausgewählten Knoten und Kanten sind selektiert und visuell leicht hervorgehoben. Unter der Strukturansicht werden Informationen und Statistiken zu den selektierten Knoten und Kanten angezeigt.}
\end{tabular}
\\ \\ \\
\test{constraint}{Constraint zu Knoten eines geladenen Graphen hinzufügen}
\begin{tabular}{llp{0.9\linewidth}}
	\subtest
		{Ein JOANA-Graph wurde importiert und einer der enthaltenen \glspl{methgraph} oder \glspl{callgraph} wird angezeigt (siehe Test \ref{test:importJOANA}, \ref{test:openmethod}). Mindestens zwei Knoten auf unterschiedlichen Ebenen wurden selektiert (siehe Test \ref{test:select})}
		{Rechte Maustaste auf einen der selektierten Knoten.}
		{Das Kontextmenü für die selektierten Knoten öffnet sich.}
	\subtest
		{Kontextmenü wird angezeigt.}
		{Den Menüeintrag Constraint hinzufügen auswählen.}
		{Ein Dialog öffnet sich.}
	\subtest
		{Der Dialog zum Hinzufügen neuer Constraints ist geöffnet.}
		{Auswählen des Constraints zum nebeneinander Darstellen der Knoten und bestätigen des Dialogs.}
		{Der Dialog schließt sich. Der Graph wird neu gelayoutet und die zuvor selektierten Knoten werden nebeneinander in der gleichen Zeile dargestellt.}
\end{tabular}
\\ \\ \\
\test{exportSVG}{Export von einem geladenen JOANA-Graphen als SVG}
\begin{tabular}{llp{0.9\linewidth}}
	\subtest
		{Ein JOANA-Graph wurde importiert und einer der enthaltenen \glspl{methgraph} oder \glspl{callgraph} wird angezeigt (siehe Test \ref{test:importJOANA}, \ref{test:openmethod}).}
		{Den Menüeintrag Datei->Export->\gls{svg} auswählen.}
		{Es öffnet sich ein Dateiauswahldialog.}
	\subtest
		{Speicherort und Name der zu exportierenden Datei wurde im Dialog angegeben.}
		{Button zum Exportieren drücken.}
		{Der Dateiauswahldialog wird geschlossen, eine \gls{svg}-Datei mit dem angegebenen Namen wurde am ausgewählten Ort erstellt und sie enthält die visuelle Repräsentation des gesamten Graphen wie in der Graphansicht dargestellt wurde.}
\end{tabular}
