%% LaTeX-Beamer template for KIT design
%% by Erik Burger, Christian Hammer
%% title picture by Klaus Krogmann
%%
%% version 2.1
%%
%% mostly compatible to KIT corporate design v2.0
%% http://intranet.kit.edu/gestaltungsrichtlinien.php
%%
%% Problems, bugs and comments to
%% burger@kit.edu

\documentclass[18pt]{beamer}
\usepackage[utf8]{inputenc}

%% SLIDE FORMAT

% use 'beamerthemekit' for standard 4:3 ratio
% for widescreen slides (16:9), use 'beamerthemekitwide'

\usepackage{templates/beamerthemekit}
% \usepackage{templates/beamerthemekitwide}

%% TITLE PICTURE

% if a custom picture is to be used on the title page, copy it into the 'logos'
% directory, in the line below, replace 'mypicture' with the
% filename (without extension) and uncomment the following line
% (picture proportions: 63 : 20 for standard, 169 : 40 for wide
% *.eps format if you use latex+dvips+ps2pdf,
% *.jpg/*.png/*.pdf if you use pdflatex)

%\titleimage{mypicture}

%% TITLE LOGO

% for a custom logo on the front page, copy your file into the 'logos'
% directory, insert the filename in the line below and uncomment it

%\titlelogo{mylogo}

% (*.eps format if you use latex+dvips+ps2pdf,
% *.jpg/*.png/*.pdf if you use pdflatex)

%% TikZ INTEGRATION

% use these packages for PCM symbols and UML classes
% \usepackage{templates/tikzkit}
% \usepackage{templates/tikzuml}

% the presentation starts here

\title[Graph von Ansicht]{Graph von Ansicht:\\ Visualisierung von Programmabhängigkeitsgraphen}
\subtitle{}
\author{Nicolas Boltz, Jonas Fehrenbach, Sven Kummetz, Jonas Meier, Lucas Steinmann}

\institute{}

% Bibliography

\bibliographystyle{plain}
%\usepackage[citestyle=authoryear,bibstyle=numeric,hyperref,backend=biber]{biblatex}
%\addbibresource{templates/example.bib}
%\bibhang1em

\begin{document}

% change the following line to "ngerman" for German style date and logos
\selectlanguage{ngerman}

%title page
\begin{frame}
\titlepage
\end{frame}

\author{Nicolas B., Jonas F., Sven K., Jonas M., Lucas S.}
%table of contents
\begin{frame}{Gliederung}
\tableofcontents
\end{frame}

\section{Einleitung/Zielbestimmung}
\begin{frame}{Ziele}
\begin{itemize}
\item Bullet point 1
\pause
\item Bullet point 2
\item \dots
\end{itemize}
\end{frame}

\section{Benutzerschnittstellen}
\begin{frame}{Benutzerschnittstellen}
\end{frame}

\section{Szenario}
\begin{frame}{Szenario}
\end{frame}

\section{Weitere Funktionen}
\begin{frame}{Funktionsübersicht}
\end{frame}

\section{Erweiterbarkeit}
\begin{frame}{Pluginschnittstellen}
\end{frame}
\appendix
\beginbackup

\begin{frame}[allowframebreaks]{References}
\bibliography{templates/example}
\end{frame}

\backupend

\end{document}
