\newdualentry{jdk}
  {JDK}
  {Java Development Kit}
  {Das Java Development Kit ist eine Entwicklungsumgebung für Java, sie enthält u.a. die Java-Bibliothek, einen Java-Compiler, einen Interpreter, einen Applet-Viewer und einen Debugger}

\newdualentry{jvm}
  {JVM}
  {Java Virtual Machine}
  {Die Java Virtual Machine ist eine Virtuelle Maschine, die es ermöglicht plattformunabhängige Java Programme auszuführen}

\newdualentry{gui}
  {GUI}
  {Graphical User Interface}
  {Ein Graphical User Interface (engl. für grafische Benutzeroberfläche) besteht aus Fenster, Menüs, Knöpfen und weiteren grafischen Elementen zur Steuerung und Darstellung von Inhalten eines Programmes.}

\newdualentry{jre}
  {JRE}
  {Java Runtime Environment}
  {Die JRE ist eine Laufzeitumgebung die zur Ausführung von Java-Programmen nötig ist. Sie enthält die Java Virtuelle Maschine, Java Kernklassen und Hilfsdateien}

\newdualentry{io}
  {I/O}
  {Input/Output}
  {Merkmale und Funktionen, die sich auf Eingabe bzw. Ausgabe von Daten mit einem Programm beziehen. (z.B. Laden bzw. Abspeichern von Dateien)}

\newdualentry{sdg}
  {SDG}
  {System Dependency Graph}
  {Ein System Dependency Graph (engl. für Systemabhänghigkeitsgraph) bildet eine Darstellungsmöglichkeit von prozedurübergreifenden Abhängigkeiten in einem Programm.}

\newpluraldualentry{pdg}
  {PDG}
  {Programmabhängigkeitsgraph}
  {In einem Program Dependency Graph (engl. für Programmabhängigkeitsgraph) werden Kontrollfluss- und Datenabhängigkeiten innerhalb einer Prozedur angezeigt.}
  {Programmabhängigkeitsgraphen}

\newglossaryentry{jfx}
{
  name=JavaFX,
  description={JavaFX ist eine Java Bibliothek, die ein Framework zur Erstellung von GUI-Schnittstellen für Java Anwendungen bereitstellt}
}

\newglossaryentry{svg}
{
  name=SVG,
  description={Scalable Vector Graphics (SVG) (engl. für skalierbare Vektorgrafik) ist ein XML-basiertes Dateiformat zur Darstellung von Vektorgrafiken}
}

\newglossaryentry{jpg}
{
  name=JPEG,
  description={JPEG ist ein weit verbreitetes Bilddatei-Format, welches mehrere Komprimierungsmethoden (u.a. verlustfreie Komprimierung) unterstützt}
}

\newglossaryentry{prefapi}
{
  name=Preferences API,
  description={Das java.util.prefs Paket bietet Funktionen an, welche Nutzereinstellungen und Daten persistent speichert. (\url{http://docs.oracle.com/javase/1.5.0/docs/guide/preferences/index.html})}
}

\newglossaryentry{graphml}
{
  name=GraphML,
  description={GraphML ist ein XML-basiertes Dateiformat welches benutzt wird, um Graphen und ihre Struktur zu speichern}
}

\newglossaryentry{sugiyama}
{
  name=Framework von Sugiyama,
  description={Das Framework von Sugiyama beschreibt eine Vorgehensweise, um einen Graphen hierarchisch darzustellen}
}

\newglossaryentry{bezier}
{
  name=Bézierkurven,
  description={Bézierkurven sind parametrisch modellierte Kurven, welche bei der mathematischen Beschreibung von Freiformkurven helfen}
}

\newglossaryentry{callgraph}
{
  name=Callgraph,
  description={Der Callgraph bezeichnet die Graphdarstellung eines Programms, bei welchem nur die Abhängigkeiten der Methoden dargestellt wird und nicht die Abhängigkeiten innerhalb einer Methode},
  plural=Callgraphen
}

\newglossaryentry{methgraph}
{
  name=Methodengraph,
  description={Der Methodengraph bezeichnet die Graphdarstellung einer einzigen Methode. Es werden nur der Steuerfluss und die Datenabhängigkeiten innerhalb einer einzigen Methode angezeigt},
  plural=Methodengraphen
}

\newglossaryentry{constraint}
{
  name=Constraint,
  description={Eine Einschränkung es Graph Layout Algorithmus, die nach Möglichkeit beachtet wird.},
  plural=Constraints
}

\newglossaryentry{joana}
{
  name=JOANA,
  description={JOANA ist ein Analyse Tool des KIT, welches Java Programme analysiert und eine Graphdatei zurückgibt}
}

\newglossaryentry{hyperkante}
{
  name=Hyperkante,
  description={Hyperkanten sind Kanten in einem Graphen, welche zwei oder mehr Knoten miteinander verbinden. Ein Graph der Hyperkanten enthält wird als Hypergraph bezeichnet.}
  plural=Hyperkanten
}

\newglossaryentry{subgraph}
{
  name=Subgraph,
  description={Der durch mindestens einen selektierten Knoten induzierte Graph},
  plural=Subgraphen
}

\newglossaryentry{sichtfeld}
{
	name=Sichtfeld,
	description={Das Sichtfeld beschreibt den Bildschirmausschnitt des Graphen, welcher der Nutzer momentan sieht},
  plural=Sichtfelder
}

\newglossaryentry{gpm}
{
  name=Graph Pattern Matching,
  description={Ein Vorgang der automatisch und nach Vorgabe Subgraphen in einem Graphen findet}
}

\newglossaryentry{pattern}
{
  name=Pattern,
  description={Eine abstrakte Beschreibung von Graphen.}
}

\newglossaryentry{gruppe}
{
  name=Gruppe,
  description={Die Menge von Knoten die in ihrem Gruppenattribut einen Eintrag identisch haben},
  plural=Gruppen
}

\newglossaryentry{arbeitsumgebung}
 {
   name=Arbeitsumgebung,
   description={Eine Arbeistsumgebung ist eine Reihe von Konfigurationen und Aktionen die auf gerade importierte Graphen angewendet werden können.
     Für Informationen zur Anwendung einer Arbeitsumgebung und Beispiele siehe \ref{fa:umgebung}},
   plural=Arbeitsumgebungen
 }
 
 \newglossaryentry{datenstrom}
 {
   name=Datenstrom,
   description={Ein Datenstrom ist ein Fluss von Daten. Ein wahlfreier Zugriff auf die Daten ist nur über die Pufferung beim Empfänger möglich. Daher müssen Datenströme meist sequentiell abgearbeitet werden. Datenströme werden in Netzwerken aber auch bei der Kooperation mehrerer Programme oder Programmkompnenten auf einem System eingesetzt. Letzteres ist der Fall bei der Kommunikation von \gls{io}-Daten zwischen Graph von Ansicht und Plugins.},
   plural=Datenströme
 }
 
\newglossaryentry{toplevel}
{
  name=''Top-Level''-Graph,
  description={Ein Top-Level Graph ist ein Graph, welcher nicht in einem geschachtelten Graphen enthalten ist. Beispielsweise ist der JOANA-Callgraph ein Top-Level Graph.}
}


%\newglossaryentry{<++>}
%{
%  name=<++>,
%  description={<++>}
%}

%\newdualentry{<++>}
%  {<++>}
%  {<++>}
%  {<++>}

%\newpluraldualentry{<++>}
%  {<++>}
%  {<++>}
%  {<++>}
%  {<++>} % Plural ausgeschrieben
