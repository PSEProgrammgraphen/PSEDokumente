\newdualentry{jdk}
  {JDK}
  {Java Development Kit}
  {Das Java Development Kit ist eine Entwicklungsumgebung für Java, sie enthält u.a. die Java-Bibliothek, einen Java-Compiler, einen Interpreter, einen Applet-Viewer und einen Debugger}

\newdualentry{jvm}
  {JVM}
  {Java Virtual Machine}
  {Die Java Virtual Machine ist eine Virtuelle Maschine, die es ermöglicht plattformunabhängige Java Programme auszuführen}

\newdualentry{jre}
  {JRE}
  {Java Runtime Environment}
  {Die JRE ist eine Laufzeitumgebung die zur Ausführung von Java-Programmen nötig ist. Sie enthält die Java Virtuelle Maschine, Java Kernklassen und Hilfsdateien}

\newdualentry{io}
  {I/O}
  {Input/Output}
  {Merkmale und Funktionen, die sich auf Eingabe bzw. Ausgabe von Daten mit einem Programm beziehen. (z.B. Laden bzw. Abspeichern von Dateien)}

\newglossaryentry{jfx}
{
  name=JavaFX,
  description={JavaFX ist eine Java Bibliothek die ein Framework zur Erstellung von GUI-Schnittstellen für Java Anwendungen bereitstellt}
}

\newglossaryentry{svg}
{
  name=SVG,
  description={Scalable Vector Graphics (SVG) (engl. für skalierbare Vektorgrafik) ist ein XML-basiertes Dateiformat zur Darstellung von Vektorgrafiken}
}

\newglossaryentry{jpg}
{
  name=JPEG,
  description={JPEG ist ein weit verbreitetes Bilddatei-Format, welches mehrere Komprimierungsmethoden (u.a. verlustfreie Komprimierung) unterstützt}
}

\newglossaryentry{prefapi}
{
  name=Preferences API,
  description={Das java.util.prefs Paket bietet Funktionen an welche Nutzereinstellungen und Daten persistent speichert. (\url{http://docs.oracle.com/javase/1.5.0/docs/guide/preferences/index.html})}
}

\newglossaryentry{graphml}
{
  name=GraphML,
  description={GraphML ist ein XML-basiertes Dateiformat welches benutzt wird um Graphen und ihre Struktur zu speichern}
}

\newglossaryentry{sugiyama}
{
  name=Framework von Sugiyama,
  description={Das Framework von Sugiyama beschreibt eine Vorgehensweise um einen Graphen hierarchisch darzustellen}
}

\newglossaryentry{bezier}
{
  name=Bézierkurven,
  description={Bézierkurven sind parametrisch modellierte Kurven, welche bei der mathematischen Beschreibung von Freiformkurven helfen}
}

\newglossaryentry{hyperkante}
{
  name=Hyperkante,
  description={Hyperkanten sind Kanten in einem Graphen, welche zwei oder mehr Knoten miteinander verbinden. Ein Graph der Hyperkanten enthält wird als Hypergraph bezeichnet.}
}

%\newglossaryentry{<++>}
%{
%  name=<++>,
%  description={<++>}
%}
