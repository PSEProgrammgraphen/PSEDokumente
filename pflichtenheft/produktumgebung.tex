\chapter{Produktumgebung}
\label{ch:umgebung}

\section{Software}\label{sec:software}
Das Produkt muss in folgenden Desktop-Systemen ausführbar und, wie im restlichen Dokument beschrieben, benutzbar sein:
\begin{itemize}
  \setlength\itemsep{0em}
  \item Linux Fedora 22/23 %TODO nochmal die Versionen in der ATIS nachsehen und abgleichen.
  \item Linux Ubuntu 15.10/16.04 LTS
  \item Windows 7 und höher
\end{itemize}
Zur Programmierung wird die Programmiersprache Java benutzt. Daher ist eine Installation der \gls{jre} 8+ zur Ausführung notwendig.

\section{Hardware}\label{sec:hardware}
Das Produkt ist als \gls{jfx}-Anwendung zur Ausführung auf Desktop-Systemen konzipiert.
Durch die Verwendung von Java ist das Produkt unabhängig von Details der unterliegenden Hardware, sofern diese in der Lage ist, die benötigte \gls{jre} (siehe \autoref{sec:software}) auszuführen.
Um die in \autoref{ch:nfa} beschriebenen nichtfunktionalen Anforderungen einhalten zu können, sind folgende Mindestanforderungen an das System, auf dem das Produkt ausgeführt werden soll, notwendig:

\begin{itemize}
  \setlength\itemsep{0em}
  \item Arbeitsspeicher: 4 GB
  \item Prozessor: Intel Core i5-4210U/FX 7600P (2,7 GHz)
  \item Festplattenspeicher: 200 MB 
  \item Display: 1280x960
\end{itemize}

\section{Bibliotheken}
Folgende Bibliotheken werden verwendet um die Programmierung des Produktes zu vereinfachen. \\

\begin{tabular}{ >{\bfseries}l p{0.6 \linewidth}}
  JavaFX & Diese Bibliothek wird verwendet um die \gls{gui} des Programmes zu erstellen, und auch um die Grundformen des Graphen zu zeichnen. \\
  Tinkerpop/Blueprints & Die Bibliothek unterstützt das importieren von \gls{graphml}-Dateien, durch einen bereits implementierten Parser. \\
\end{tabular}
