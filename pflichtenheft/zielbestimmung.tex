\chapter{Zielbestimmung}

Graph von Ansicht soll Programmgraphen von externen Analyse-Programmen visualisieren. Dafür wird eine bereits vorhandene Graphdatei importiert und ausgewertet. Der Benutzer kann verschiedene Constraints einstellen \ref{fa:constraints}, welche im weiteren Dokument näher erläutert werden.
%TODO: Verweis einfügen zu den konkreten Erläuterungen
Das Endergebnis soll eine übersichtliche Darstellung der Abhängigkeiten und des Steuerflusses eines Programmes zeigen, das dem Benutzer die Möglichkeit bietet, das Programm besser verstehen zu können.

\section{Pflichtkriterien}

\subsection{Graph von Ansicht (Graphviewer)}
  \begin{itemize}
    \item Navigation mittels Zoom und Verschieben (siehe \ref{fa:zoom} und \ref{fa:verschieben})
    \item Selektieren von einzelnen oder mehreren Knoten (siehe \ref{fa:selekt_knoten})
    \item Kollabieren und Ausklappen von \gls{subgraph}en % und Gruppen (wenn mir jemand erklärt was der unterschied ist kann ers wieder unkommentieren :).
%    \item Ausblendung selektierter Knoten (wie ist das gemeint? Wie kann man Knoten wieder einblenden?)
    \item Akzeptieren von Kommandozeilenargumenten zur Angabe von Graphdatei und Layoutalgorithmus für ein schnelles Starten (siehe \ref{sec:uicmd})
    \item Schnittstellen für Plugins in den Bereichen Import, Export, Layoutalgorithmen, Filter für Knoten- und Kantentypen und weitere Operationen auf einzelne Knoten und Kanten
    \item Informationsanzeige zu einzelnen Knoten und Kanten (siehe \ref{fa:infoanzeige})
    \item Anzeige von Statistiken über den Graphen und \gls{subgraph}en (siehe \ref{fa:statistik})
  \end{itemize}
  
\subsection{Plugins}
  \begin{itemize}
    \item Es gibt ein Pluginmanagement-System, welches externe Plugins laden und verwalten kann. (siehe \nameref{ch:plugschnitt})
    \item \gls{graphml}-Plugin welches den Import von generischen Graphen im \gls{graphml}-Format (beschrieben in \nameref{ch:daten}) unterstützt (siehe \ref{fa:importgraphml})
    \item \gls{svg}-Plugin, welches den Export der visualisierten Graphen im \gls{svg}-Format unterstützt (siehe \ref{fa:export_svg})
  \end{itemize}
  
  \subsubsection{JOANA-Plugin}
    \begin{itemize}
      \item Hierarchisches Layout mit dem \gls{sugiyama} (siehe \ref{fa:joanalayout})
      \item Filter für Knoten- und Kantentypen aus \gls{joana} (siehe \ref{fa:filter})
      \item Ein \gls{callgraph}-Layout, welches übersichtlich die Abhängigkeiten der Methoden darstellt (siehe \ref{fa:callview})
      \item Ein \gls{methgraph}-Layout, welches die Abhängigkeiten innerhalb einer Methode - mithilfe von vorgegebenen Constraints, (siehe \ref{fa:constraints}) - darstellt (siehe \ref{fa:methview} und \ref{fa:joanalayout})
    \end{itemize}
  

\section{Wunschkriterien}

\subsection{Graph von Ansicht (Graphviewer)}
  \begin{itemize}
    \item Muster Definition mittels "GraphRegex" %TODO: zugehörige FA Sven
    \item Layout Constraints, die vom Nutzer angepasst werden (siehe \ref{fa:constraints}) %TODO: zugehörige FA Sven
    \item Tastaturkürzel (evtl. benutzerdefiniert) (siehe \ref{fa:hotkey})
    \item Fortschrittsbalken bei der Berechnung des Layouts des Graphen (siehe \ref{fa:fortschritt})
    \item Eine Übersicht des angezeigten Graphen. Diese Übersicht besteht aus einem Bild des kompletten Graphen und zeigt an in welchem Teil des Graphen das aktuelle \gls{sichtfeld} liegt.  %TODO: zugehörige FA  Jonas F.
    \item Algorithmus zur Erreichbarkeit eines Knoten %TODO: zugehörige FA Sven
    \item Die Darstellung von Kanten kann geändert werden (\gls{bezier}, orthogonale Kanten oder direkte Kanten) %TODO: zugehörige FA Jonas F.
    \item Reload-Funktion  %TODO: zugehörige FA Lucas
    \item Rückgängig machen der letzten Änderung am Graphen %TODO: zugehörige FA Lucas
  \end{itemize}

\subsection{Plugins}
  \begin{itemize}
    \item weitere Export-Plugins im \gls{jpg}- und \gls{graphml}-Format %TODO Unterschied zur importierten GraphML-Datei
  \end{itemize}
  
\section{Abgrenzungskriterien}

\subsection{Graph von Ansicht (Graphviewer)}
  \begin{itemize}
    \item Das Produkt ist kein Graph-Editor und unterstützt deshalb die Manipulation des Graphen hinsichtlich seiner Struktur (z.B. Kanten entfernen, Knoten hinzufügen/löschen) nicht.
    \item Das \gls{gui} wird nicht von Grund auf neu entwickelt, es werden Bibliotheken verwendet, um die Entwicklung zu erleichtern.
    \item Das Zeichnen der Kanten und Knoten mit primitiven geometrischen Objekten wird mithilfe von Bibliotheken umgesetzt. %TODO: Referenz zu benutzten Bibliotheken
    \item Das Produkt ist kein Analysetool für Programme, sondern dient lediglich zur Visualisierung von bereits vorhandenen Graphdateien.
  \end{itemize}
\subsection{Plugins}
  \begin{itemize}
    \item Die Plugins können nur auf die in Kapitel \ref{ch:plugschnitt} beschriebenen Schnittstellen zugreifen
    \item Neue Plugins können nicht zur Laufzeit hinzugefügt werden, sondern müssen beim Programmstart vorhanden sein, um genutzt werden zu können.
  \end{itemize}
