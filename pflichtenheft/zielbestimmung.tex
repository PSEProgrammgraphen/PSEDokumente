\chapter{Zielbestimmung}

Graph von Ansicht soll Programmgraphen von externen Analyse-Programmen visualisieren. Dafür wird eine bereits vorhandene Graphdatei importiert und ausgewertet. Der Benutzer kann verschiedene Constraints einstellen, welche im weiteren Dokument näher erläutert werden.
%TODO: Verweis einfügen zu den konkreten Erläuterungen
Das Endergebnis soll eine übersichtliche Darstellung der Abhängigkeiten und des Steuerflusses eines Programmes zeigen, das dem Benutzer die Möglichkeit bietet, das Programm besser analysieren zu können.

\section{Pflichtkriterien}

\subsection{Graph von Ansicht (Graphviewer)}
  \begin{itemize}
    \item Navigation mittels Zoom und Verschieben \ref{fa:zoom} \ref{fa:verschieben}
    \item Selektieren von einzelnen oder mehreren Knoten \ref{fa:selekt_knoten}
    \item Kollabieren und Ausklappen von Teilgraphen % und Gruppen (wenn mir jemand erklärt was der unterschied ist kann ers wieder unkommentieren :).
    \item Ausblendung selektierter Knoten
    \item Akzeptieren von Kommandozeilenargumenten zur Angabe von Graphdatei und Layoutalgorithmus für ein schnelles Starten.
    \item Schnittstellen für Plugins anbieten in den Bereichen Import, Export, Layoutalgorithmen, Darstellungsoptionen, Filter für Knoten- und Kantentypen und weitere Operationen auf einzelne Knoten und Kanten
    \item Kanten werden durch \gls{bezier} dargestellt
    \item Informationsanzeige von Knoten und Kanten
  \end{itemize}
  
\subsection{Plugins}
  \begin{itemize}
    \item \gls{graphml}-Plugin welches den Import von generischen Graphen im \gls{graphml}-Format unterstützt (siehe \ref{ch:daten})
    \item \gls{svg}-Plugin welches den Export der visualisierten Graphen im \gls{svg}-Format unterstützt
  \end{itemize}
  
  \subsubsection{JOANA-Plugin}
    \begin{itemize}
      \item Hierarchisches Layout mit dem \gls{sugiyama}   
      \item Filter für Knoten- und Kantentypen aus \gls{joana} \ref{fa:filter}
      \item Ein \gls{callgraph}-Layout welcher übersichtlich die Abhängigkeiten der Methoden darstellt
      \item Ein \gls{methgraph}-Layout welcher die Abhängigkeiten innerhalb einer Methode - mithilfe von vorgegebenen Constraints, siehe \ref{ch:nfa} - darstellt 
    \end{itemize}
  

\section{Wunschkriterien}

\subsection{Graph von Ansicht (Graphviewer)}
  \begin{itemize}
    \item Muster Definition mittels "GraphRegex"
    \item Benutzerdefinierte Hotkeys
    \item Layout Constraints die vom Nutzer angepasst werden
    \item Fortschrittsbalken bei Berechnung der Visualisierung des Graphen mithilfe einer Zeitabschätzung
    \item Minimap bzw. Übersicht des Graphens
    \item Algorithmus zur Erreichbarkeit eines Knoten
    \item Die visuelle Darstellung von Kanten kann geändert werden (\gls{bezier}, orthogonale Kanten oder direkte Kanten)
  \end{itemize}

\subsection{Plugins}
  \begin{itemize}
    \item weitere Export-Plugins im \gls{jpg}- und \gls{graphml}-Format
  \end{itemize}
  
  \subsubsection{JOANA-Plugin}
  \begin{itemize}
    \item Zeitabschätzung des Layoutalgorithmus zur Darstellung des Fortschrittsbalken
  \end{itemize}


\section{Abgrenzungskriterien}

\subsection{Graph von Ansicht (Graphviewer)}
  \begin{itemize}
    \item Das Produkt ist kein Graph Editor und unterstützt nicht das manuelle Zeichnen/Hinzufügen von neuen Kanten und Knoten.
    \item Die GUI wird nicht von Grund auf neu entwickelt, es werden Bibliotheken verwendet, um die Entwicklung zu erleichtern.
    \item Das Darstellen von Kanten und Knoten selbst wird mithilfe von Bibliotheken umgesetzt.
    \item Das Produkt ist kein Analysetool für Programme, sondern dient lediglich zur Visualisierung von bereits vorhandenen Graphdateien.
  \end{itemize}
\subsection{Plugins}
  \begin{itemize}
    \item Die Plugins können nur auf die vorhandenen Schnittstellen zugreifen %TODO: verweis auf Plugins Kapitel
  \end{itemize}
