\chapter{Zielbestimmung}
Graph von Ansicht soll \glspl{pdg} von externen Analyse-Programmen visualisieren.
\glspl{pdg} können benutzt werden, um Daten- und Kontrollabhängigkeiten in Programmen deutlich zu machen.
Anwendungsbereiche (siehe \ref{ch:einsatz}) sind beispielsweise die Erkennung etwaiger Sicherheitslücken\cite{hammer09ijis} in Programmen
oder die Optimierung von Programmen beim Kompilieren\cite{Ferrante:1987:PDG:24039.24041}.\\
Um diese Graphen zu Visualisieren, muss eine vorhandene Graphdatei importiert werden.
Graph von Ansicht ist dann in der Lage die Elemente des Graphen übersichtlich anzuordnen (im weiteren Dokument ``layouten'' genannt) und in einer grafischen Oberfläche zu zeichnen.
%Der Benutzer kann verschiedene Constraints einstellen \ref{fa:constraints}, welche im weiteren Dokument näher erläutert werden.
%TODO: Verweis einfügen zu den konkreten Erläuterungen
Das Endergebnis soll eine übersichtliche Darstellung der Abhängigkeiten und des Steuerflusses eines Programmes zeigen, die dem Benutzer die Möglichkeit bietet, das Programm besser verstehen zu können.\\

\textit{Anmerkung zum Aufbau des Produktes:} Graph von Ansicht soll als Kernprogramm die Möglichkeit bieten,
mittels Plugins Unterstützung für weitere Arten von Graphen (z.B. DOM-Trees)
hinzuzufügen, ohne Anpassungen an dem Kernprogramm selbst machen zu müssen.
Die Schnittstellen für Plugins sind in Kapitel \ref{ch:plugschnitt} beschrieben.
Die notwendige Unterstützung für JOANA-PDGs, der Export in das \gls{svg}-Format, sowie der Import aus dem \gls{graphml}-Format werden auch über Plugins realisiert.
Da diese Plugins aber fest mit dem Kernprogramm ausgeliefert werden und eine Unterscheidung in Kernprogramm und Plugin für die Benutzung des
Programmes nicht relevant ist, wird von dieser Unterscheidung im Großteil des restlichen Dokumentes abgesehen und nur falls nötig gemacht.

\section{Pflichtkriterien}

\subsection{Allgemein}
  \begin{itemize}
    \item Hierarchisches Layout mit dem \gls{sugiyama} (siehe \ref{fa:layout})
    \item Ein \gls{callgraph}-Layout, welches übersichtlich die Abhängigkeiten der Methoden darstellt (siehe \ref{fa:layout})
    \item Ein \gls{methgraph}-Layout, welches die Abhängigkeiten innerhalb einer Methode - mithilfe von vorgegebenen Constraints (siehe \ref{fa:constraints}) - darstellt (siehe \ref{fa:layout})
    \item Kollabieren und Ausklappen von \glspl{subgraph} (siehe \ref{fa:kollabieren} und \ref{fa:ausklappen})
    \item Informationsanzeige zu einzelnen Knoten und Kanten (siehe \ref{fa:infoanzeige})
    \item Statistiken über den Graphen und \glspl{subgraph} (siehe \ref{fa:statistik})
    \item Filter für Knoten- und Kantentypen aus \gls{joana} (siehe \ref{fa:filter})
    \item Tabs für geöffnete Graphen (siehe \ref{fa:graphwechsel})
    \item Akzeptieren von Kommandozeilenargumenten zur Angabe von Graphdatei und Layoutalgorithmus für ein schnelles Starten (siehe \autoref{sec:uicmd})
  \end{itemize}

\subsection{Input/Output}
  \begin{itemize}
    \item Import von generischen Graphen im \gls{graphml}-Format (beschrieben in \nameref{ch:daten}) (siehe \ref{fa:import})
    \item Export der visualisierten Graphen im \gls{svg}-Format (siehe \ref{fa:export_img})
  \end{itemize}

\subsection{Steuerung}
  \begin{itemize}
    \item Navigation mittels Zoom und Verschieben (siehe \ref{fa:zoom} und \ref{fa:verschieben})
    \item Selektieren von einzelnen oder mehreren Knoten (siehe \ref{fa:selekt_knoten})
  \end{itemize}

\subsection{Plugins}
  \begin{itemize}
    \item Schnittstellen für Plugins in den Bereichen Import, Export, Layoutalgorithmen, Filter für Knoten- und Kantentypen und weitere Operationen auf einzelne Knoten und Kanten (siehe \autoref{ch:plugschnitt})
    \item Es gibt ein Pluginmanagement-System, welches externe Plugins laden und verwalten kann. (siehe \autoref{ch:plugschnitt})
  \end{itemize}

\section{Wunschkriterien}

\subsection{Allgemein}
  \begin{itemize}
    \item Automatisierte \gls{subgraph} Findung mittels \gls{gpm} (siehe \ref{fa:gpm})
    \item Layout Constraints, die vom Nutzer angepasst werden (siehe \ref{fa:constraints}) %TODO: zugehörige FA Sven
    \item Fortschrittsbalken bei der Berechnung des Layouts des Graphen (siehe \ref{fa:fortschritt})
    \item Eine Übersicht des angezeigten Graphen (siehe \ref{fa:uebersicht}
    \item Algorithmus zur Erreichbarkeit eines Knoten (siehe \ref{fa:erreichbarkeit})
    \item Die Darstellung von Kanten kann geändert werden (siehe \ref{fa:darst-kanten})
    \item Reload-Funktion (siehe \ref{fa:reload})
  \end{itemize}

\subsection{Steuerung}
  \begin{itemize}
    \item Tastaturkürzel (evtl. benutzerdefiniert) (siehe \ref{fa:hotkey})
  \end{itemize}

\subsection{Input/Output}
  \begin{itemize}
    \item weitere Exportfunktionen für das \gls{jpg}- und das \gls{graphml}-Format (mit Koordinaten der Knoten und Kanten des aktuellen Layouts) (vgl. \ref{fa:export_img} )
  \end{itemize}
  
\section{Abgrenzungskriterien}

\subsection{Allgemein}
  \begin{itemize}
    \item Das Produkt ist kein Graph-Editor und unterstützt deshalb die Manipulation des Graphen hinsichtlich seiner Struktur (z.B. Kanten entfernen, Knoten hinzufügen/entfernen) nicht.
    \item Das \gls{gui} wird nicht von Grund auf neu entwickelt, es werden Bibliotheken verwendet, um die Entwicklung zu erleichtern.
    \item Das Zeichnen der Kanten und Knoten mit primitiven geometrischen Objekten wird mithilfe von Bibliotheken umgesetzt. %TODO: Referenz zu benutzten Bibliotheken
    \item Das Produkt ist kein Analysetool für Programme, sondern dient lediglich zur Visualisierung von bereits vorhandenen Graphdateien.
  \end{itemize}
\subsection{Plugins}
  \begin{itemize}
    \item Die Plugins können nur auf die in \autoref{ch:plugschnitt} beschriebenen Schnittstellen zugreifen
    \item Neue Plugins können nicht zur Laufzeit hinzugefügt werden, sondern müssen beim Programmstart vorhanden sein, um genutzt werden zu können.
  \end{itemize}
