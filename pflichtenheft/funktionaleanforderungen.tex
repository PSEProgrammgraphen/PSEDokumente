\chapter{Funktionale Anforderungen}
\label{ch:funktionen}


\setcounter{fanr}{10}
\newcommand{\fano}[1]{\subsubsection{#1}\addtocounter{fanr}{10}}
\newcommand{\subfano}[1]{\subsubsection{#1}\addtocounter{fanr}{1}}
\renewcommand\thesubsubsection{/FA\ifnum\value{fanr}<10 00\else\ifnum\value{fanr}<100 0\fi\fi\arabic{fanr}/}

\section{Graph von Ansicht}

\subsection{Allgemein}

%TODO Anzeige unterscheiden: direkt nach Import(nicht gelayouted), nach layout(gelayouted).
\fano{Anzeige von Graphen}\label{fa:graphen}
\textbf{Ziel:} Ein Graph soll in der Graphansicht angezeigt werden.\\
\textbf{Vorbedingung:} -\\
\textbf{Nachbedingung (Erfolg):} Der Graph ist in der Graphansicht sichtbar.\\
\textbf{Nachbedingung (Fehlschlag):} -\\
\textbf{Auslösende Ereignisse:}
\begin{enumerate}[nolistsep, label=(\alph*)]
  \item Eine Graphdatei wird importiert.
  \item Ein Graph wird aus der Strukturansicht ausgewählt.
\end{enumerate}
\textbf{Beschreibung:}\\
Importieren (a): Der in der Graphdatei zuerst beschriebene Graph wird in der Graphansicht angezeigt.\\
Auswahl (b): Der ausgewählte Graph wird in der Graphansicht angezeigt.

%TODO Definition in Glossar, bzw. nötig?
\fano{Anzeige von geschachtelten Graphen}\label{fa:hierarchgraph}
\textbf{Ziel:} Falls ein Graph einen geschachtelten Graphen enthält, soll dieser angezeigt werden.\\
\textbf{Vorbedingung:} Eine Graphdatei mit geschachtelten Graphen wurde importiert.\\
\textbf{Nachbedingung (Erfolg):} Der geschachtelte Graph wird in der Graphansicht angezeigt.\\
\textbf{Nachbedingung (Fehlschlag):} -\\
\textbf{Auslösende Ereignisse:}
\begin{enumerate}[nolistsep, label=(\alph*)]
  \item Der Benutzer wählt den geschachtelten Graphen aus der Strukturansicht aus.
  \item Ein Graph mit einem geschachtelten Graphen wird in der Graphansicht angezeigt.
  Der Benutzer wählt die Funktion über das Kontextmenü des Knotens, der den geschachtelten Graphen enthält, aus oder löst sie durch ein Tastenkürzel aus.
\end{enumerate}
\textbf{Beschreibung:}\\
Geschachtelte Graphen werden in der Graphansicht als Knoten dargestellt.
Wenn der Benutzer eines der Ereignisse auslöst, wird der Graph in der Graphansicht angezeigt.
Ein geschachtelter Graph wird in der Strukturansicht als Kind des Graphen dargestellt, der ihn enthält. Für ein Beispiel siehe hier:\\ %TODO: Referenz von GUI Bild mit Strukturansicht
Ein geschachtelter Graph wird in der Graphansicht wie ein ``Top-Level''-Graph behandelt. %TODO Top-Level in Glossar


\fano{Graphen layouten}\label{fa:layout}
\textbf{Ziel:} Ein Graph soll nach bestimmten Vorgaben bzw. einem bestimmten Muster angeordnet werden.\\
%TODO Muster beziehen sich vlt auf constrains/bzw. auf die parametriesierung möglicher anderer Layoutalgos
\textbf{Vorbedingung:} Ein Graph wurde in die Graphansicht geladen.\\
\textbf{Nachbedingung (Erfolg):} Allen Knoten wurden bestimmte Positionen zugeordnet, Kanten wurden feste Verläufe zugewiesen.\\
\textbf{Nachbedingung (Fehlschlag):} -\\
\textbf{Auslösende Ereignisse:}
%TODO wird ein popup vor dem layouten geöffnet
Der Benutzer wählt über die Menüleiste ein Layout aus.\\
%TODO keine generischen graphen werden unterstützt
\textbf{Beschreibung:}\\
Nach der Auswahl wird das Layout berechnet. Hierbei kann es zu Verzögerungen kommen. (siehe \ref{nfa:berechzeit})
Nach der Berechnung wird der Graph neugezeichnet. Layouts sind nicht in Graph von Ansicht enthalten und können durch Plugins hinzugefügt werden.
Dem Produkt werden über das JOANA-Plugin Layout zur Darstellung von, von \gls{joana} berechneten, \gls{pdg}s mitgeliefert.

%TODO wie wird der Subgraph gewählt? Wie werden Constrains gewählt? Dialog?
\fano{Constraints auswählen}\label{fa:constraints}
\textbf{Ziel:} Der Nutzer kann Constraints auswählen, die auf den \gls{subgraph}en beim layouten (siehe \ref{fa:layout}) verwendet werden.\\
\textbf{Vorbedingung:} Ein Graph wurde geladen.\\
\textbf{Nachbedingungen (Erfolg):} Es wurde ein neuer Constraint hinzugefügt.\\
\textbf{Nachbedingungen (Fehlschlag):} -\\
\textbf{Auslösende Ereignisse:} Der Benutzer klickt in der Hauptansicht auf Constraint hinzufügen.\\
\textbf{Beschreibung: } \\
Der Benutzer legt eine Anzahl von Constraints für einen \gls{subgraph}en aus, die für das layouten verwendet werden sollen (siehe \ref{fa:layout}).
Dem Produkt mitgeliefert werden folgende Constraints: %TODO: spezifizeren wo die funktionalität geliefert wird
\begin{itemize}[nolistsep]
  \item Positionierung
  \begin{itemize}[nolistsep]
    \item positioniere \textit{Subgraph 1} über \textit{Subgraph 2}
    \item positioniere \textit{Subgraph 1} unter \textit{Subgraph 2}
    \item positioniere \textit{Subgraph 1} links von \textit{Subgraph 2}
    \item positioniere \textit{Subgraph 1} rechts von \textit{Subgraph 2}
    \item positioniere \textit{Subgraph} an fester Position
  \end{itemize}
  \item Proximität
  \begin{itemize}[nolistsep]
    \item gruppiere \textit{Subgraph}
    \item trenne \textit{Subgraph}
  \end{itemize}
  \item Lage
  \begin{itemize}[nolistsep]
    \item \textit{Subgraph} in einer Reihe
    \item \textit{Subgraph} in einer Spalte
  \end{itemize}
\end{itemize}

\fano{Knoten und Kanten filtern}\label{fa:filter}
\textbf{Ziel:} Knoten und Kanten können bezüglich ihres Types gefiltert werden.\\
\textbf{Vorbedingung:} Ein Graph wurde geladen.\\
\textbf{Nachbedingung (Erfolg):} Es wurden Knoten und Kanten eines bestimmten Types ein-/ausgeblendet.\\
\textbf{Nachbedingung (Fehlschlag):} -\\
\textbf{Auslösende Ereignisse:}
Der Benutzer wählt die Filter-Funktion aus der Menüleiste aus.\\
\textbf{Beschreibung:}\\
Der Benutzer kann aus einer Reihe von Filtern auswählen.
Mitgeliefert werden über das JOANA-Plugin folgende Filter: % Referenz Plugin
\begin{itemize}[nolistsep]
  \item Kontrollfluss
  \item Kontrollabhängigkeit
  \item Datenabhängigkeit
  \item Heap-Abhängigkeiten
  \item Parameterstruktur
  \item Threadinterferenzen
\end{itemize}
Die Kanten werden entsprechend aus- oder eingeblendet.\\
\textbf{Alternative:}
Der Graph wird neu gezeichnet um kompakter zu werden, oder um Platz für neue Kanten zu schaffen.

\subsection{Input/Output}
\setcounter{fanr}{100}

%TODO Internes Format definieren und einsetzen.
\fano{Graph aus Datei importieren}\label{fa:import}
\textbf{Ziel:} Der/Die in der Graphdatei beschriebene Graph(en) soll(en) visuell dargestellt werden.\\
\textbf{Vorbedingung:} Der Benutzer hat eine unterstützte Graphdatei (siehe \ref{ch:daten}) in seinem Dateissystem, auf welche zugegriffen werden kann.\\
\textbf{Nachbedingung (Erfolg):} Die Graphen werden in der Strukturübersicht aufgelistet. Der zuerst beschriebene Graph wird dargestellt.\\
\textbf{Nachbedingung (Fehlschlag):}
Eine Fehlermeldung wird ausgegeben, dass die Datei nicht geöffnet werden konnte, bzw. dass die Datei keinen korrekten Graphen beschreibt.\\
\textbf{Auslösende Ereignisse:}
\begin{enumerate}[nolistsep, label=(\alph*)]
  \item Der Benutzer wählt die Import-Funktion über die Menüleiste aus, nachdem das Programm geöffnet wurde. %TODO: Referenz zu Menüleiste in GUI
  \item Der Benutzer ruft das Programm auf und übergibt den Pfad zur Graphdatei als Argument. (siehe \ref{sec:uicmd})
\end{enumerate}
\textbf{Beschreibung:}\\
Menüleiste (a):
Es öffnet sich ein Fenster, in dem der Benutzer den Typ des Graphen auswählt, welcher importiert werden soll. %TODO: Ref gui
Ein Dateiverzeichnis-Menü öffnet sich. %TODO: Dateiverzeichnis-Menü auch in GUI-Entwurf oder klar?
Der Benutzer wählt die zu importierende Datei aus.
Der in der Datei beschriebene Graph wird in der Graphansicht dargestellt.\\%TODO: % Referenz zur GUI (Graph-Fenster)
Kommandozeilenargument (b): Der Graph wird gleich nach dem Programmstart in der Graphansicht dargestellt. \\

%\begin{figure}[ht]
%  \centering
%  \includegraphics[scale=0.4]{resourcen/fa-import.1}
%  \caption{Import}
%  \label{fig:import}
%\end{figure} %TODO: Outdated

\fano{Export als Bilddatei}\label{fa:export_img}
\textbf{Ziel:} Der Graph soll in seiner aktuellen Darstellung als Bilddatei exportiert werden.\\
\textbf{Vorbedingung:} Ein Graph wurde in die Graphansicht geladen. \\
\textbf{Nachbedingung (Erfolg):} Es wurde eine Bilddatei an einer gewünschten, schreibbaren Stelle im Dateisystem erstellt, welche ein Abbild des derzeit angezeigten Graphen enthält.\\
\textbf{Nachbedingung (Fehlschlag):} Die Bilddatei konnte nicht erstellt werden. Es wird eine Fehlermeldung ausgegeben.\\
\textbf{Auslösendes Ereignis:}
Der Benutzer wählt die Exportfunktion aus der Menüleiste aus.\\
\textbf{Beschreibung:}\\
Der Benutzer wählt die Exportfunktion aus der Menüleiste aus.
Es erscheint ein Dateisystem-Menü, in welchem der Pfad zum Abspeichern der Bilddatei ausgewählt werden kann.\\
Export-Optionen werden durch Plugins bereitgestellt.
Mit dem Produkt wird ein \gls{svg}-Plugin zum Exportieren in das \gls{svg}-Dateiformat mitgeliefert (siehe \ref{fa:export_svg}).\\
Eventuell wird der Benutzer noch durch das Plugin nach weiteren, zur Abspeicherung relevanten, Details gefagt. (z.B. Stärke der Komprimierung bei JPEG)\\ %TODO: Durch Beispiel in SVG ersetzten
Es wird versucht, die Bilddatei am ausgewählten Ort abzuspeichern.
\begin{figure}[ht]
  \centering
  \includegraphics[scale=0.4]{resourcen/fa-export.1}
  \caption{Export}
  \label{fig:export}
\end{figure}

\subsection{Steuerung}
\setcounter{fanr}{200}

\fano{Sichtfeld verschieben}\label{fa:verschieben}
\textbf{Ziel:} Das \gls{sichtfeld} soll verschoben werden.\\
\textbf{Vorbedingung:} Ein Graph wurde in die Graphansicht geladen und das \gls{sichtfeld} deckt nicht alle Elemente des Graphen ab.\\
\textbf{Nachbedingung (Erfolg):}  Das \gls{sichtfeld} deckt nun den gewünschten Teil des Graphen ab.\\
\textbf{Nachbedingung (Fehlschlag):} Das \gls{sichtfeld} deckte einen Randabschnitt ab und es wurde versucht das \gls{sichtfeld} über den Rand hinaus zu bewegen.\\
Das \gls{sichtfeld} wird nicht über den Rand bewegt. Es wird keine Fehlermeldung ausgegeben.\\
\textbf{Auslösende Ereignisse:}
\begin{enumerate}[nolistsep, label=(\alph*)]
  \item Der Benutzer klickt und zieht mit der mittleren Maustaste in einem leeren Bereich des Graph-Fenster.
  \item Der Benutzer bewegt die Scroll-Balken am Rand des \gls{sichtfeld}es. % TODO: Referenzt auf GUI Entwurf, wo scrollbalken sichtbar sind
\end{enumerate}
\textbf{Beschreibung:}\\
Klicken und Ziehen (a): Das \gls{sichtfeld} bewegt sich entgegen der Richtung, in welche der Mauszeiger bewegt wird. Es entsteht somit ein intuitives Verschieben des Graphen.\\
Navigationstasten (b): Das \gls{sichtfeld} verschiebt sich um eine feste Länge (in Abhängigkeit des Zoom-Grades) in die entsprechende Richtung.\\
Scroll-Balken (c): Das \gls{sichtfeld} bewegt sich relativ zur Größe des gesamten Graphen um das gleiche Maß wie der Scroll-Balken zur Länge seiner Fahrbahn in die entsprechende Richtung (Horizontal/Vertikal).\\

\fano{Zoom-Grad ändern}\label{fa:zoom}
\textbf{Ziel:} Der Zoom-Grad des \gls{sichtfeld}es soll vergrößert bzw. verkleinert werden.\\
\textbf{Vorbedingung:} Ein Graph wurde in die Graphansicht geladen.\\
\textbf{Nachbedingung (Erfolg):} Der Zoom-Grad wurde angepasst.\\
\textbf{Nachbedingung (Fehlschlag):} Der Zoom-Grad bleibt gleich, falls ein Maximum/Minimum erreicht wurde. Es wird keine Fehlermeldung ausgegeben.\\
\textbf{Auslösende Ereignisse:} Der Benutzer dreht am Mausrad, während der Mauszeiger innerhalb der Graphansicht liegt.
\textbf{Beschreibung:}\\
Mausrad (a): Der Zoom-Grad ändert sich mit dem Drehen des Mausrades. Die Änderungsrichtung kann sich in Abhängigkeit vom Betriebssystem ändern.\\
Zoom-Tasten (b): Der Zoom-Grad ändert sich um einen, in Abhängigkeit zum derzeitigen Zoom-Grad, festen Wert.\\

%TODO Wie Deselektieren, shift/strg Selektieren mehrerer Knoten, welche Knoten werden beim Ziehen ausgewählt.
\fano{Knoten selektieren}\label{fa:selekt_knoten}
\textbf{Ziel:} Ein oder mehrere Knoten sollen der Auswahl hinzugefügt werden.\\
\textbf{Vorbedingung:} Ein Graph wurde in die Graphansicht geladen.\\
\textbf{Nachbedingung (Erfolg):} Ein oder mehrere Knoten wurden der Auswahl hinzugefügt.\\
\textbf{Nachbedingung (Fehlschlag):} -\\
\textbf{Auslösende Ereignisse:}
\begin{enumerate}[nolistsep, label=(\alph*)]
  \item Der Benutzer klickt mit der linken Maustaste auf einen Knoten.
  \item Der Benutzer klickt mit der linken Maustaste in einen leeren Bereich des \gls{sichtfeld}es und zieht den Mauszeiger.
\end{enumerate}
\textbf{Beschreibung:}\\
Der Benutzer kann Knoten mit der Maus zur Auswahl hinzufügen.
Auf eine Auswahl von Knoten können verschiedene Operationen ausgeführt werden. %TODO: Referenz auf Funktionen auf Knotenmengen/Subgraphen
Die derzeit ausgewählten Knoten werden graphisch hervorgehoben.\\

\fano{Wechsel zwischen Graphen}\label{fa:graphwechsel}
\textbf{Ziel:} Der in der Graphansicht angezeigte Graph wurde ausgetauscht. \\
\textbf{Vorbedingung:} Eine Graphdatei wurde geladen.\\
\textbf{Nachbedingung (Erfolg):} Der ausgewählte Graph wird in der Graphansicht angezeigt.\\
\textbf{Nachbedingung (Fehlschlag):} -\\
\textbf{Auslösende Ereignisse:}
Der Benutzer wählt über die Strukturansicht einen anderen Graphen aus.\\
\textbf{Beschreibung:}\\
Das Produkt unterstützt Graphdateien, welche mehr als einen Graphen beschreiben.
Mit dieser Funktion kann der Benutzer die Ansicht zwischen den Graphen wechseln.

\fano{Informationsanzeige zu einzelnen Knoten und Kanten}\label{fa:infoanzeige}
\textbf{Ziel:} Zu einem ausgewählten Knoten oder einer Kanten wird Information angezeigt.\\
\textbf{Vorbedingung:} Ein Graph wurde in die Graphansicht geladen.\\
\textbf{Nachbedingung (Erfolg):} In der Strukturansicht wird die Information zu dem ausgewählten Knoten oder der Kante angezeigt.\\
\textbf{Nachbedingung (Fehlschlag):} -\\
\textbf{Auslösende Ereignisse:} Der Benutzer wählt einen Knoten oder eine Kante aus.\\
\textbf{Beschreibung:}\\
Die Information eines über einen per Mausklick ausgewählten Knoten oder einer Kante wird in der Strukturansicht angezeigt. Diese Information ist abhängig von der importierten Graph-Datei (siehe \ref{fa:import}) und des ausgewählten Graphlayouts (siehe \ref{fa:layout}).

\fano{Statistiken zu Graphen}\label{fa:statistik}
\textbf{Ziel:} Zu ausgewählten Graphen und \gls{subgraph}en wird eine Statistik angezeigt.\\
\textbf{Vorbedingung:} Ein Graph wurde in die Graphansicht geladen.\\
\textbf{Nachbedingung (Erfolg):} In der Strukturansicht wird eine Statistik zu dem ausgewählten Graphen oder \gls{subgraph}en angezeigt.\\
\textbf{Nachbedingung (Fehlschlag):} -\\
\textbf{Auslösende Ereignisse:}
\begin{enumerate}[nolistsep, label=(\alph*)]
  \item Der Benutzer wählt einen Graphen oder \gls{subgraph}en über die Graphansicht aus.
  \item Die Auswahl erfolgt über die Strukturansicht.
\end{enumerate}
\textbf{Beschreibung:}\\
Die in der Strukturansicht angezeigte Statistik über einen Graphen oder \gls{subgraph}en enthält zum Beispiel Informationen über die Knoten- und Kantenzahl.


\fano{TODO: Abspeichern als Graph}\label{fa:speichern}
\fano{TODO: Import von vorher abgespeicherten Graphen}\label{fa:laden}

\section{\gls{graphml}-Plugin}
\setcounter{fanr}{300}

%TODO siehe datei importieren
\fano{GraphML importieren}\label{fa:importgraphml}
\textbf{Ziel:} Eine \gls{graphml}-Datei soll in das Programm zur Ansicht geladen werden.\\
\textbf{Vorbedingung:} Eine korrekte und unterstützte (siehe \nameref{ch:daten}) GraphML-Datei befindet sich im Dateisystem.\\
%TODO Daten zu Produktdaten
\textbf{Nachbedingung (Erfolg):} Die in der Datei beschriebenen Graphen werden an Graph von Ansicht zurückgegeben.\\
\textbf{Nachbedingung (Fehlschlag):} Falls die Datei nicht unterstützt wird oder Fehler enthält, wird eine Fehlermeldung an Graph von Ansicht zurückgegeben.\\
%TODO Hässlich
\textbf{Auslösende Ereignisse:} Beim Importieren wird ein Typ ausgewählt, welcher das \gls{graphml}-Plugin als Importer beschreibt.\\
\textbf{Beschreibung:}\\
Die GraphML-Datei wird eingelesen, in die interne Repräsentation von Graphen überführt und an Graph von Ansicht zur Darstellung zurückgegeben.

\section{\gls{svg}-Plugin}
\setcounter{fanr}{400}

\fano{Graph als \gls{svg}-Datei exportieren}\label{fa:export_svg}
\textbf{Ziel:} Der in der Graphansicht angezeigte Graph soll im \gls{svg}-Format abgespeichert werden.\\
\textbf{Vorbedingung:} Ein Graph wurde in die Graphansicht geladen.\\
\textbf{Nachbedingung (Erfolg):} Der Graph wurde abgespeichert.\\
%TODO nur IO fehler
\textbf{Nachbedingung (Fehlschlag):} Der Graph konnte nicht abgespeichert werden. Es wird eine Fehlermeldung an Graph von Ansicht zurückgegeben.\\
\textbf{Auslösende Ereignisse:} Die SVG-Exportoption wird aus dem Menü wie in \ref{fa:export_img} beschrieben ausgewählt.\\
\textbf{Beschreibung:}\\
Es wird versucht, die derzeitige Darstellung des Graphen in der Graphansicht in das SVG-Format zu überführen.

\section{\gls{joana}-Plugin}
\setcounter{fanr}{500}

\fano{Callgraph-View}\label{fa:callview}
\textbf{Ziel:} \gls{callgraph} in \gls{joana}-Graphdateien werden speziell dargestellt und behandelt.\\
\textbf{Vorbedingung:} -\\
%TODO unten beschrieben, direkte referenz auf beschreibung
\textbf{Nachbedingung (Erfolg):} Der Callgraph wird wie in der Beschreibung beschrieben dargestellt.\\
\textbf{Nachbedingung (Fehlschlag):} -\\
\textbf{Auslösende Ereignisse:} Eine Graphdatei wird als JOANA-Graph importiert.\\
\textbf{Beschreibung:}\\
Beim Importieren wird der Callgraph zuerst in der Graphansicht geöffnet.\\
Die Elemente des Graphen werden, wie in \ref{fa:joanaview} beschrieben, behandelt.
\textit{Weitere Eigenschaften folgen nach Absprache.}
%TODO: Nach Gespräch weitere Eigenschaften des Callgraphen auflisten. Gruppierung von Java-Methoden etc aus E-Mail

\fano{Methoden-View}\label{fa:methview}
\textbf{Ziel:} \gls{methgraph} in \gls{joana}-Graphdateien werden, um die Betrachtung als \gls{pdg} zu erleichtern, speziell behandelt und dargestellt.\\
\textbf{Vorbedingung:} -\\
%TODO wie unten beschrieben wieder direkte referenz auf beschreibung
\textbf{Nachbedingung (Erfolg):} Der Methodengraph wird wie in der Beschreibung beschrieben behandelt. \\
\textbf{Nachbedingung (Fehlschlag):} -\\
\textbf{Auslösende Ereignisse:} Eine Graphdatei wird als JOANA-Graph importiert.\\
\textbf{Beschreibung:}\\
Ein Methodengraph kann als \gls{subgraph} des \gls{callgraph}, wie in \ref{fa:hierarchgraph} beschrieben, geöffnet werden.
Er wird beim Laden in die Graphansicht automatisch durch das in \ref{fa:joanalayout} definierte Layout gelayoutet.
Die Elemente des Graphen werden, wie in \ref{fa:joanaview} beschrieben, behandelt.


%TODO die zwei anforderungen besser(neu) beschreiben/voneinander trennen.
\fano{\gls{joana}-Layout}\label{fa:joanalayout}
\textbf{Ziel:} Ein Graph soll zur Betrachtung als \gls{pdg} sinnvoll gelayoutet werden.\\
\textbf{Vorbedingung:} Ein Graph wurde in die Graphansicht geladen.\\
\textbf{Nachbedingung (Erfolg):} Der Graph wurde wie unten beschrieben gelayoutet.\\
\textbf{Nachbedingung (Fehlschlag):} -\\
\textbf{Auslösende Ereignisse:}
\begin{enumerate}[nolistsep, label=(\alph*)]
  \item Ein \gls{methgraph} wird zum ersten mal in die Graphansicht geladen (siehe \ref{fa:methview}).
  \item Über die Menüleiste wird das JOANA-Layout als Layout ausgewählt (siehe \ref{fa:layout}).
\end{enumerate}
\textbf{Beschreibung:}\\
Das JOANA-Layout ist eine hierarchisches Layout zur Ansicht von PDGs.
Diese Layouts muss alle Anforderungen beschrieben in \ref{sec:nfajoana} erfüllen.

\fano{Darstellung von \gls{joana}-Graphen}\label{fa:joanaview}
\textbf{Ziel:} Elemente in JOANA-Graphen bekommen basierend auf ihrem Typ eine spezielle Darstellung.\\
\textbf{Vorbedingung:} Ein Graph wurde in die Graphansicht geladen.\\
%TODO was ist unten?
\textbf{Nachbedingung (Erfolg):} Der Graph wurde wie unten beschrieben dargestellt.\\
\textbf{Nachbedingung (Fehlschlag):} -\\
\textbf{Auslösende Ereignisse:}
\begin{enumerate}[nolistsep, label=(\alph*)]
  \item Der Nutzer wählt über das Darstellungsmenü aus der Menüleiste die Darstellung als JOANA-Graphen aus.
  \item Der Graph wird als JOANA-Graph geladen
\end{enumerate}
\textbf{Beschreibung:}\\
Knoten und Kanten gleichen Types in Call- und Methodengraphen werden in gleichen Farben dargestellt.\\
Für Beispiele von Kantentypen siehe \ref{fa:filter}.\\
Das Zeichnen der Elemente wird nicht von dem JOANA-Plugin übernommen, sondern ist Teil von Graph von Ansicht. %TODO: Referenz auf Visualisierungsmöglichkeiten.
Es wird lediglich die Darstellung festgelegt. (siehe \ref{s:darstellung})


\fano{TODO: Definition des JOANA-Graphtypes}\label{fa:joanatyp}
\section{Wunsch-Funktionen}

\fano{Steuerung über Tastaturkürzel}\label{fa:hotkey}
\textbf{Ziel:} Häufig verwendete Funktionen können über Tastaturkürzel ausgeführt werden.\\
\textbf{Vorbedingung:} -\\
\textbf{Nachbedingung (Erfolg):} Die dem Tastaturkürzel zugeordnete Funktion wurde ausgeführt.\\
\textbf{Nachbedingung (Fehlschlag):} -\\
\textbf{Auslösende Ereignisse:}
Ein Benutzer aktiviert ein Tastenkürzel, welchem eine Funktion zugeordnet ist.\\
\textbf{Beschreibung:}\\
Häufig verwendete Funktionen (wie z.B. Navigation im Graphen) und Menüs sind Tastaturkürzel zugeordnet.
Die Tastenkürzel werden hinter dem Funktionsnamen im Menüleiste oder Kontextmenü angezeigt.

\fano{Fortschrittsbalken bei Berechnung}\label{fa:fortschritt}
\textbf{Ziel:} Während das Produkt das Graphlayout berechnet, wird ein Fortschrittsbalken angezeigt.\\
\textbf{Vorbedingung:} Ein Graph wurde in die Graphansicht geladen.\\
\textbf{Nachbedingung (Erfolg):} Während der Layoutberechnung wurde ein Fortschrittsbalken angezeigt.\\
\textbf{Nachbedingung (Fehlschlag):} -\\
\textbf{Auslösende Ereignisse:} Der Benutzer wählt über die Menüleiste ein Layout aus.\\
\textbf{Beschreibung:}\\
Der Fortschrittsbalken dient dazu dem Benutzer anzuzeigen, wie weit das Programm bisher mit dem Layouten des Graphen ist, und bietet ihm somit die Möglichkeit der Zeitabschätzung bei der Berechnung eines Graphlayoutes.

\fano{Eine Übersicht des angezeigten Graphen}\label{fa:uebersicht}
\textbf{Ziel:} Es soll ein kleines Bild des kompletten Graphen angezeigt werden \\
\textbf{Vorbedingung:} Ein Graph wurde in die Graphansicht geladen.\\
\textbf{Nachbedingung (Erfolg):} Die Übersicht wird korrekt (mit dem aktuell ausgewählten Layout) angezeigt \\
\textbf{Nachbedingung (Fehlschlag):} -\\
\textbf{Auslösende Ereignisse:}
\begin{enumerate}[nolistsep, label=(\alph*)]
  \item Ein Graph wird in die Graphansicht geladen.
  \item Ein Graph wird neu gelayoutet.
\end{enumerate}
\textbf{Beschreibung:}\\
Diese Übersicht besteht aus einem Bild des kompletten Graphen und zeigt an in welchem Teil des Graphen das aktuelle \gls{sichtfeld} liegt. Somit sieht der Benutzer, mit einem Blick, in welchem Teil des Graphen das \gls{sichtfeld} liegt und kann sich besser orientieren.

\fano{Änderung der Darstellung von Kanten}\label{fa:darst-kanten}
\textbf{Ziel:} Der Nutzer kann auswählen in welcher Form die Kanten des Graphen gezeichnet werden. \\
\textbf{Vorbedingung:} -\\
\textbf{Nachbedingung (Erfolg):} Die neu gezeichneten Graphen werden mit der ausgewählten Darstellung für Kanten gezeichnet \\
\textbf{Nachbedingung (Fehlschlag):} -\\
\textbf{Auslösende Ereignisse:} Der Nutzer wählt in den Einstellungen die gewünschte Darstellung der Kanten.
\textbf{Beschreibung:}\\
Es kann ausgewählt werden ob die Kanten als \gls{bezier}, orthogonale Kanten oder als direkte Kanten dargestellt werden. Alle Graphen welche neu gelayoutet werden, besitzen diese Darstellung der Kanten. Für bereits gelayoutete Graphen wird diese Einstellung nicht automatisch übernommen.

\fano{Reload-Funktion}\label{fa:reload}
\textbf{Ziel:} Änderungen, welche an der importierten Graphdatei vorgenommen wurden, können durch die Reload-Funktion übernommen werden.\\
\textbf{Vorbedingung:} Eine Graphdatei wurde importiert und ist weiterhin an der selben Stelle im Dateisystem vorhanden und zugreifbar.\\
\textbf{Nachbedingung (Erfolg):} Die aktuelle Version des Graphen aus der Datei wurde importiert.\\
\textbf{Nachbedingung (Fehlschlag):} Auf die Datei konnte nicht zugegriffen werden oder enthält ungültige Informationen. Keine Graphdatei ist geladen. Es wird ein Fehler ausgegeben.\\
\textbf{Auslösende Ereignisse:} Der Nutzer wählt die Reload-Funktion über die Menüleiste.\\
\textbf{Beschreibung:}\\
Die Reload-Funktion ermöglicht einen schnellen erneuten Import einer zuvor schon importierten Graphdatei.
Falls der derzeitig in der Graphansicht geladene Graph in der neuen Datei vorhanden ist, wird er aktualisiert in die Graphansicht geladen.
Layout und andere Darstellungsoption, wie etwaige aktivierte Knoten- und Kantenfilter werden übernommen.

%\fano{<++>}\label{fa:<++>}
%\textbf{Ziel:} <++>\\
%\textbf{Vorbedingung:} <++>\\
%\textbf{Nachbedingung (Erfolg):} <++>\\
%\textbf{Nachbedingung (Fehlschlag):} <++>\\
%\textbf{Auslösende Ereignisse:}
%\begin{enumerate}[nolistsep, label=(\alph*)]
%  \item <++>
%\end{enumerate}
%\textbf{Beschreibung:}
%\begin{enumerate}[nolistsep]
%  \item <++>
%\end{enumerate}
%\textbf{Alternativen:}
%\begin{enumerate}[nolistsep]
%  \item <++>
%\end{enumerate}


%etwas kürzer, falls keine enumerations verwendet werden ;)
%\fano{}\label{fa:}
%\textbf{Ziel:} \\
%\textbf{Vorbedingung:} \\
%\textbf{Nachbedingung (Erfolg):} \\
%\textbf{Nachbedingung (Fehlschlag):} \\
%\textbf{Auslösende Ereignisse:} \\
%\textbf{Beschreibung:}\\
%
