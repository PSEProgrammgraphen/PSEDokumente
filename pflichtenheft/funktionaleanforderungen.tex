\chapter{Funktionale Anforderungen}
\label{ch:funktionen}


\setcounter{fanr}{10}
\newcommand{\fano}[1]{\subsubsection{#1}\addtocounter{fanr}{10}}
\newcommand{\subfano}[1]{\subsubsection{#1}\addtocounter{fanr}{1}}
\renewcommand\thesubsubsection{/FA\ifnum\value{fanr}<10 00\else\ifnum\value{fanr}<100 0\fi\fi\arabic{fanr}/}

\section{Pflichtfunktionen}

\subsection{Allgemein}

\fano{Anzeige von Graphen}\label{fa:graphen}
\textbf{Ziel:} Ein Graph soll in der Graphansicht angezeigt werden.\\
\textbf{Vorbedingung:} -\\
\textbf{Nachbedingung (Erfolg):} Der Graph ist in der Graphansicht sichtbar.\\
\textbf{Nachbedingung (Fehlschlag):} -\\
\textbf{Auslösende Ereignisse:}
\begin{enumerate}[nolistsep, label=(\alph*)]
  \item Eine Graphdatei wird importiert.
  \item Ein Graph wird aus der Strukturansicht ausgewählt.
\end{enumerate}
\textbf{Beschreibung:}\\
Importieren (a): Der in der Graphdatei zuerst beschriebene Graph wird in der Graphansicht angezeigt.\\
Auswahl (b): Der ausgewählte Graph wird in der Graphansicht angezeigt.\\
Das Anzeigen eines Graphen beinhaltet das Zeichnen der Knoten an den ihnen zugeordneten Positionen und den Kanten
nach den ihnen zugeordneten Kurven.
Die Zuordnung der Positionen und Kurven erfolgt über die Layout-Funktion (siehe \ref{fa:layout}).\\
Das Erscheinungsbild von Knoten (z.B. Farbe und Größe) und Kanten (z.B. Farbe und Breite) kann nach Typ des
Knoten oder der Kante variieren, ist aber nicht vom Benutzer manuell anpassbar.
Für Beispiele von Kantentypen siehe \ref{fa:filter}. Beispiele für Knotentypen sind Entry-Knoten, Actual-in-Knoten, Actual-out-Knoten, etc..\\
Knoten und Kanten gleichen Types in Call- und Methodengraphen werden in gleichen Farben dargestellt.\\
Weitere Zuordungen von Knoten- und Kantentypen mit dem Erscheinungsbild der jeweiligen Elemente können von
Plugins hinzugefügt werden (siehe \ref{s:darstellung}).

%TODO Definition in Glossar, bzw. nötig?
\fano{Anzeige von geschachtelten Graphen}\label{fa:hierarchgraph}
\textbf{Ziel:} Falls ein Graph einen geschachtelten Graphen enthält, soll dieser angezeigt werden.\\
\textbf{Vorbedingung:} Eine Graphdatei mit geschachtelten Graphen wurde importiert.\\
\textbf{Nachbedingung (Erfolg):} Der geschachtelte Graph wird in der Graphansicht angezeigt.\\
\textbf{Nachbedingung (Fehlschlag):} -\\
\textbf{Auslösende Ereignisse:}
\begin{enumerate}[nolistsep, label=(\alph*)]
  \item Der Benutzer wählt den geschachtelten Graphen aus der Strukturansicht aus.
  \item Ein Graph mit einem geschachtelten Graphen wird in der Graphansicht angezeigt.
  Der Benutzer wählt die Funktion über das Kontextmenü des Knotens, der den geschachtelten Graphen enthält, aus oder löst sie durch ein Tastenkürzel aus.
\end{enumerate}
\textbf{Beschreibung:}\\
Geschachtelte Graphen werden in der Graphansicht als Knoten dargestellt.
Wenn der Benutzer eines der Ereignisse auslöst, wird der Graph in der Graphansicht angezeigt.
Ein geschachtelter Graph wird in der Strukturansicht als Kind des Graphen dargestellt, der ihn enthält. Für ein Beispiel siehe hier:\\ %TODO: Referenz von GUI Bild mit Strukturansicht
Ein geschachtelter Graph wird in der Graphansicht wie ein ``Top-Level''-Graph behandelt. %TODO Top-Level in Glossar
Ein Methodengraph kann als \gls{subgraph} des Callgraphen geöffnet werden.


\fano{Graphen layouten}\label{fa:layout}
\textbf{Ziel:} Ein Graph soll nach bestimmten Vorgaben bzw. einem bestimmten Muster angeordnet werden.\\
%TODO Muster beziehen sich vlt auf constrains/bzw. auf die parametriesierung möglicher anderer Layoutalgos
\textbf{Vorbedingung:} Ein Graph wurde in die Graphansicht geladen.\\
\textbf{Nachbedingung (Erfolg):} Allen Knoten wurden bestimmte Positionen zugeordnet, Kanten wurden feste Kurven zugewiesen.\\
\textbf{Nachbedingung (Fehlschlag):} -\\
\textbf{Auslösende Ereignisse:}
Der Benutzer wählt über die Menüleiste ein Layout aus.\\
\textbf{Beschreibung:}\\
Nach der Auswahl des Layouts öffnet sich ein Fenster zur Parametrisierung des Layout-Algorithmus.
Beispiele für Parameter sind die Ebenenanzahl oder die maximale Anzahl an Knoten auf jeder Ebene bei einem hierarchischen Layout.
Nachdem der Benutzer seine gewünschten Einstellungen bestätigt hat, wird das Layout berechnet. Hierbei kann es zu Verzögerungen kommen (siehe \ref{nfa:berechzeit}).
Nach der Berechnung wird der Graph neu gezeichnet.
Graph von Ansicht enthält zwei hierarchische Layouts:
\begin{itemize}
  \item \textbf{\gls{callgraph}-Layout:} Für das Layouten von \glspl{callgraph} gedacht. Beinhaltet folgende spezielle Regeln:
    \begin{itemize}
      \item Mehrere Instanzen der selben Java-Methode werden gruppiert.
    \end{itemize}
  \item \textbf{\gls{methgraph}-Layout:} Für das Layouten von \glspl{methgraph} gedacht. Beinhaltet folgende spezielle Regeln:
    \begin{itemize}
      \item Auf Layer 0 (dem obersten Layer) wird nur der Entry-Knoten der Methode platziert.
      \item Auf Layer 1 werden nur die Knoten der Parameter der Methode platziert.
      \item Zu einem Methodenaufruf gehörende Knoten (z.B. Actual-in und Actual-out) werden auf dem Layer direkt unter dem Layer des Call-Knoten platziert.
      \item Zu einem Methodenaufruf gehörende Knoten (Actual-in, Actual-out etc.) werden direkt nebeneinander und in der Reihenfolge der Parameter der Methode platziert.
      \item Der Call-Knoten wird in dem Bereich oberhalb der Parameter platziert.
      \item Knoten die einen Feldzugriff darstellen, werden immer nahe beieinander und in einem einheitlichen Muster platziert. %TODO: Beispielbilder aus E-Mail einfügen?
      \item Feldzugriffskanten werden senkrecht gezeichnet.
    \end{itemize}
\end{itemize}
Beide Layouts müssen alle in \ref{sec:nfajoana} beschriebenen Anforderungen erfüllen.
Weitere Layouts können durch Plugins hinzugefügt werden (siehe \ref{s:layoutalgo})

\fano{Constraints hinzufügen}\label{fa:constraints}
\textbf{Ziel:} Ein Layout-spezifisches Constraints kann hinzugefügt werden.\\
\textbf{Vorbedingung:} Ein Graph wurde in die Graphansicht geladen. Ein Layoutalgorithmus wurde angewendet.\\
\textbf{Nachbedingungen (Erfolg):} Es wurde ein neuer Constraint hinzugefügt.\\
\textbf{Nachbedingungen (Fehlschlag):} -\\
\textbf{Auslösende Ereignisse:} Der Benutzer klickt in der \ref{gui:layoutsetting} auf Constraint hinzufügen.\\
\textbf{Beschreibung: } \\
Der Benutzer legt eine Anzahl von Constraints für einen Layoutalgorithmus fest, die für das layouten verwendet werden sollen (siehe \ref{fa:layout}). Hierfür wählt er zunächst ein Constraint aus einer Menge an verfügbaren Constraints aus. Diese Constraints werden vom jeweiligen Layoutplugin geliefert und implementieren vom Produkt vorgegebene Interfaces. Vom Produkt werden folgende Interfaces vorgegeben:
\begin{itemize}[nolistsep]
  \item \textbf{Relative Positionierung} positioniert eine \gls{gruppe} relativ zu einer anderen \gls{gruppe}.
  \item \textbf{Absolute Positionierung} positioniert eine \gls{gruppe} absolut im Layout.
  \item \textbf{Proximität} positioniert die Elemente einer \gls{gruppe} in festgelegter Nähe zueinander.
  \item \textbf{Lage} positioniert die Elemente einer \gls{gruppe} in Layoutspezifischen Positionen.
  \item \textbf{Algorithmus anwenden} wendet beim Graphen layouten(siehe \ref{fa:layout}) einen, von diesem bereitgestellten Algorithmus, auf eine \gls{gruppe} an. 
\end{itemize}

\fano{Constraint bearbeiten}\label{fa:constraintsettings}
\textbf{Ziel:} Die Parameter eines Constraints können verändert werden. \\
\textbf{Vorbedingung:} Ein Graph wurde in Graphansicht geladen. Ein Layoutalgorithmus wurde ausgewählt. Ein Constraint wurde hinzugefügt. \\
\textbf{Nachbedingung (Erfolg):} Die Parameter eines Constraints wurden verändert. \\
\textbf{Nachbedingung (Fehlschlag):} Ein Parameter wurde auf einen fehlerhaften Wert gesetzt. Es wird eine Fehlermeldung ausgegeben. \\
\textbf{Auslösende Ereignisse:} Der Benutzer klickt auf die veränderlichen Parameter eines Constraints. \\
\textbf{Beschreibung:}\\
Für ein spezifisches Constraint können die Parameter, wie die \gls{gruppe} die das Constraint betrifft nach der Erstellung geändert werden.\\

\fano{Constraint löschen}\label{fa:deleteconstraint}
\textbf{Ziel:} Ein Constraint kann gelöscht werden. \\
\textbf{Vorbedingung:} Ein Graph wurde in die Graphansicht geladen. Ein Layoutalgorithmus wurde ausgewählt. Ein Constraint wurde hinzugefügt. \\
\textbf{Nachbedingung (Erfolg):} Das Constraint wurde gelöscht. \\
\textbf{Nachbedingung (Fehlschlag):} - \\
\textbf{Auslösende Ereignisse:} Der Benutzer klickt auf den Löschen Knopf eines Constraints. \\
\textbf{Beschreibung:}\\
Ein Constraint kann vom Nutzer gelöscht werden.\\

\fano{Knoten und Kanten filtern}\label{fa:filter}
\textbf{Ziel:} Knoten und Kanten können bezüglich ihres Types gefiltert werden.\\
\textbf{Vorbedingung:} Ein Graph wurde geladen.\\
\textbf{Nachbedingung (Erfolg):} Es wurden Knoten und Kanten eines bestimmten Types ein-/ausgeblendet.\\
\textbf{Nachbedingung (Fehlschlag):} -\\
\textbf{Auslösende Ereignisse:}
Der Benutzer wählt die Filter-Funktion aus der Menüleiste aus.\\
\textbf{Beschreibung:}\\
Der Benutzer kann aus einer Reihe von Filtern auswählen.\\
Folgende Filter für JOANA-Methodengraphen sind im Produkt enthalten:
\begin{itemize}[nolistsep]
  \item Kontrollfluss
  \item Kontrollabhängigkeit
  \item Datenabhängigkeit
  \item Heap-Abhängigkeiten
  \item Parameterstruktur
  \item Threadinterferenzen
\end{itemize}
Die Kanten werden entsprechend aus- oder eingeblendet.\\
Der Graph wird nicht automatisch neugezeichnet. Eine weitläufige Darstellung beim Ausblenden oder viele Kantenkreuzungen beim Einblenden können vorkommen.
Der Benutzer kann dann bei Bedarf den Layoutalgorithmus ein weiteres Mal ausführen (siehe \ref{fa:layout}) um das Layout auf die
veränderte Struktur der Darstellung des Graphen anzupassen.\\

\fano{Anwendung einer \gls{arbeitsumgebung}}\label{fa:umgebung}
\textbf{Ziel:} Durch automatisierte Einstellungen und Operationen werden dem Nutzer wiederkehrende Aufgaben nach dem Importieren einer Graphdatei abgenommen.\\
\textbf{Vorbedingung:} Eine Graphdatei wurde importiert (siehe \ref{fa:import})\\
\textbf{Nachbedingung (Erfolg):} Alle durch die Arbeitsumgebung automatisierten Aktionen wurden ausgeführt.\\
\textbf{Nachbedingung (Fehlschlag):} Der Nutzer wählt eine Graphdatei mit Graphen, für welche die Arbeitsumgebung nicht bestimmt ist. Es kann sein, dass manche Aktionen nicht ausgeführt werden können, bzw. manche Aktionen unerwünschte Auswirkungen auf die Visualisierung des Graphen haben.\\
\textbf{Auslösende Ereignisse:} Nach dem Import wählt der Nutzer eine Arbeitsumgebung aus einem Auswahlfenster aus.\\ %TODO: Ref GUI
\textbf{Beschreibung:}\\
Um die Zusammenarbeit von Plugins zu erleichtern und dem Nutzer zu ersparen Layout, Darstellung und weitere,
 Graph-spezifische Einstellung vornehmen zu müssen, wird das Konzept einer \gls{arbeitsumgebung} eingeführt:\\
Eine \gls{arbeitsumgebung} nimmt nach dem Importieren der Datei (siehe \ref{fa:import}) automatisierte Einstellungen vor.
Beispiele für automatisierte Einstellungen sind:
\begin{itemize}
  \item Das Auswählen eines Layouts für die Graphen.
    Ein Layout auswählen bedeutet nicht, dass der Graph automatisch gelayoutet wird.
    Das eigentlich Berechnen des Layouts wird erst beim ersten Öffnen des Graphen ausgeführt, um lange Verzögerungen
    beim Importieren von großen Dateien zu verhindern.
  \item Anpassungen von Attributen von Graphen und Graphelementen (z.B. Name, Farbe von Knoten, etc.)
\end{itemize}
Eine Umgebung für \gls{joana}-Graphen ist schon enthalten. Folgende Einstellung werden nach der Auswahl dieser Umgebung ausgeführt:
\begin{itemize}
  \item Für den \gls{callgraph} wird das Callgraph-Layout ausgewählt.
  \item Für alle \glspl{methgraph} wird das Methodengraph-Layout ausgewählt
  \item Der Name eines Graphen wird auf die Bezeichnung der zugehörigen Java-Methode mit Paket- und Klassennamen gesetzt.
    Diese werden aus den Attributen des Entry-Knoten bestimmt.
  \item Weitere Informationen (siehe \ref{fa:infoanzeige}) zu einem Methodengraphen werden beigefügt und formatiert. Alle Informationen werden aus Attributen der Knoten oder der Struktur des Graphen bestimmt (z.B. Parameter der Funktion).
\end{itemize}
Umgebungen sind immer gebunden an eine bestimmte Art von Graphen und können zu unerwünschten Ergebnissen führen, falls sie auf Graphen angewendet werden, für welche die Umgebung nicht gedacht ist.\\
Neue \glspl{arbeitsumgebung} können von Plugins registriert werden und werden dann auch in dem Auswahlfenster aufgelistet (siehe \ref{s:umgebung}).


\fano{Kollabieren von \glspl{subgraph}}\label{fa:kollabieren}
\textbf{Ziel:} \glspl{subgraph} können zur Übersicht eingeklappt (kollabiert) werden.\\
\textbf{Vorbedingung:} Ein Graph wurde in die Graphansicht geladen.\\
\textbf{Nachbedingung (Erfolg):} Der zuvor selektierte Subgraph wurde in einen Knoten kollabiert.\\
\textbf{Nachbedingung (Fehlschlag):} -\\
\textbf{Auslösende Ereignisse:} Der Nutzer hat einen \glspl{subgraph} selektiert (siehe \ref{fa:selekt_knoten}).
Über das Kontextmenü wählt er die Layout-Funktion für das Kollabieren von \glspl{subgraph}.\\
\textbf{Beschreibung:}\\
Alle Kanten mit genau einem Knoten, der nicht in dem \glspl{subgraph} liegt, werden als Kanten zwischen dem entstandenen Knoten und dem Knoten außerhalb der Menge gezeichnet.\\
Die Darstellung des neuen Knotens überschneidet keine andere Knoten.\\
Da sich aber beim Kollabieren von größeren \glspl{subgraph} das Bild des gesamten Graphen stark ändert, wird nicht garantiert,
dass es nicht zu überdurchschnittlich vielen Kantenkreuzungen oder langen Kanten kommt.\\
Der Benutzer kann dann bei Bedarf den Layoutalgorithmus ein weiteres Mal ausführen (siehe \ref{fa:layout}), um das Layout auf die
veränderte Struktur der Darstellung des Graphen anzupassen.\\
Ein kollabierter \gls{subgraph} kann wieder ausgeklappt werden (siehe \ref{fa:ausklappen}).


\fano{Ausklappen von kollabierten \glspl{subgraph}}\label{fa:ausklappen}
\textbf{Ziel:} Eingeklappte \glspl{subgraph} können bei Bedarf wieder ausgeklappt werden.\\
\textbf{Vorbedingung:} Ein Graph wurde in die Graphansicht geladen.\\
\textbf{Nachbedingung (Erfolg):} Der Knoten, der den Graphen dargestellt hat wird entfernt. Die im \glspl{subgraph} enthaltenen Knoten werden wie vor dem Kollabieren wieder eingezeichnet.\\
\textbf{Nachbedingung (Fehlschlag):} -\\
\textbf{Auslösende Ereignisse:} Der Nutzer wählt die Ausklapp-Funktion aus dem Kontextmenü eines Knotens, der einen eingeklappten Graphen repräsentiert.\\
\textbf{Beschreibung:}\\
Die ausgeklappten Knoten werden an der Position, die sie vor dem Kollabieren einnahmen,  wieder eingezeichnet.
Kanten zwischen den Knoten des \glspl{subgraph} werden wie vor dem Kollabieren eingezeichnet.
Kanten zwischen Knoten des \glspl{subgraph} und anderen Knoten werden nicht wie vor dem Kollabieren gezeichnet,
falls sich die Position des anderen Knotens seit dem Kollabieren geändert hat, z.B. wenn er kollabiert wurde oder der Graph neu gelayoutet wurde.
Nach dem Ausklappen kann es bei der Darstellung zu Überlagerungen kommen.
Ist das der Fall kann der Nutzer über die Layout-Funktion den Graphen neu layouten (siehe \ref{fa:layout}).

\subsection{Input/Output}
\setcounter{fanr}{100}

\fano{Graph aus Datei importieren}\label{fa:import}
\textbf{Ziel:} Die in der Graphdatei beschriebene Graphen sollen visuell dargestellt werden.\\
\textbf{Vorbedingung:} Der Benutzer hat eine unterstützte Graphdatei (siehe \ref{ch:daten}) in seinem Dateissystem, auf welche zugegriffen werden kann.\\
\textbf{Nachbedingung (Erfolg):} Die Graphen werden in der Strukturansicht aufgelistet. Der zuerst beschriebene Graph wird dargestellt.\\
\textbf{Nachbedingung (Fehlschlag):}
Eine Fehlermeldung wird ausgegeben, dass die Datei nicht geöffnet werden konnte, bzw. dass die Datei keinen korrekten Graphen beschreibt.\\
\textbf{Auslösende Ereignisse:}
\begin{enumerate}[nolistsep, label=(\alph*)]
  \item Der Benutzer wählt die Import-Funktion über die Menüleiste aus, nachdem das Programm geöffnet wurde. %TODO: Referenz zu Menüleiste in GUI
  \item Der Benutzer ruft das Programm auf und übergibt den Pfad zur Graphdatei als Argument (siehe \ref{sec:uicmd}).
\end{enumerate}
\textbf{Beschreibung:}\\
Es wird das \gls{graphml}-Format als Import-Format unterstützt. Für eine genauere Beschreibung der Unterstützung des \gls{graphml}-Formates siehe \ref{ch:daten}.\\
Falls eine Graphdatei schon geöffnet ist, wird diese geschlossen bevor die gerade importierte Datei geladen wird.
%Es öffnet sich ein Fenster, in dem der Benutzer den Typ des Graphen auswählt, welcher importiert werden soll. %TODO: Ref gui
Falls der Benutzer die Import-Funktion über die Menüleiste ausführt öffnet sich zuerst ein Dateiverzeichnis-Menü.
%TODO: Dateiverzeichnis-Menü auch in GUI-Entwurf oder klar?
Der Benutzer wählt die zu importierende Datei aus.\\
Falls die Datei beim Programmstart als Argument angegeben wurde startet der Import-Vorgang automatisch.
Die in der Graphdatei beschriebene Graphen werden in eine interne Repräsentatition, welche unabhängig vom importierten Dateiformat ist, überführt.%TODO: Referenz zu Grafik mit Repräsentationen
Die Graphen werden nach Abschluss des Imports, wie in \ref{fa:graphen} beschrieben, dargestellt.\\
Weitere Import-Optionen können durch Plugins bereitgestellt werden (siehe \ref{s:import}).

\begin{figure}[ht]
  \centering
  \includegraphics[scale=0.4]{resourcen/fa-import.1}
  \caption{Import}
  \label{fig:import}
\end{figure} %TODO: Outdated

\fano{Export als Bilddatei}\label{fa:export_img}
\textbf{Ziel:} Der Graph soll in seiner aktuellen Darstellung als Bilddatei exportiert werden.\\
\textbf{Vorbedingung:} Ein Graph wurde in die Graphansicht geladen. \\
\textbf{Nachbedingung (Erfolg):} Es wurde eine Bilddatei an einer gewünschten, schreibbaren Stelle im Dateisystem erstellt, welche ein Abbild des derzeit angezeigten Graphen enthält.\\
\textbf{Nachbedingung (Fehlschlag):} Die Bilddatei konnte wegen eines Fehlers beim Schreiben in die Datei nicht erstellt werden. Es wird eine Fehlermeldung ausgegeben.\\
\textbf{Auslösendes Ereignis:}
Der Benutzer wählt die Exportfunktion aus der Menüleiste aus.\\
\textbf{Beschreibung:}\\
Es wird das Exportieren von Graphen in das \gls{svg}-Dateiformat unterstützt.\\
Es erscheint ein Dateisystem-Menü, in welchem der Pfad zum Abspeichern der Bilddatei ausgewählt werden kann.\\
Eventuell wird der Benutzer noch nach weiteren, zur Abspeicherung relevanten, Details gefragt. (z.B. Stärke der Komprimierung bei JPEG)\\ %TODO: Durch Beispiel in SVG ersetzten
Es wird versucht, die Bilddatei am ausgewählten Ort abzuspeichern.
Weitere Export-Optionen können durch Plugins bereitgestellt werden (siehe \ref{s:export}).

\begin{figure}[ht]
  \centering
  \includegraphics[scale=0.4]{resourcen/fa-export.1}
  \caption{Export}
  \label{fig:export}
\end{figure}

\subsection{Steuerung}
\setcounter{fanr}{200}

\fano{Sichtfeld verschieben}\label{fa:verschieben}
\textbf{Ziel:} Das \gls{sichtfeld} soll verschoben werden.\\
\textbf{Vorbedingung:} Ein Graph wurde in die Graphansicht geladen und das \gls{sichtfeld} deckt nicht alle Elemente des Graphen ab.\\
\textbf{Nachbedingung (Erfolg):}  Das \gls{sichtfeld} deckt nun den gewünschten Teil des Graphen ab.\\
\textbf{Nachbedingung (Fehlschlag):} Das \gls{sichtfeld} deckte einen Randabschnitt der Zeichenfläche, auf der sich der Graph befindet, ab und es wurde versucht das \gls{sichtfeld} über den Rand hinaus zu bewegen.\\
Das \gls{sichtfeld} wird nicht über den Rand bewegt. Es wird keine Fehlermeldung ausgegeben.\\
\textbf{Auslösende Ereignisse:}
\begin{enumerate}[nolistsep, label=(\alph*)]
  \item Der Benutzer klickt und zieht mit der mittleren Maustaste in einem leeren Bereich der Graphansichansichtt.
  \item Der Benutzer bewegt die Scroll-Balken an dem rechten bzw. unteren Rand der Graphansicht. % TODO: Referenzt auf GUI Entwurf, wo scrollbalken sichtbar sind
\end{enumerate}
\textbf{Beschreibung:}\\
Klicken und Ziehen (a): Das \gls{sichtfeld} bewegt sich entgegen der Richtung, in welche der Mauszeiger bewegt wird. Es entsteht somit ein intuitives Verschieben des Graphen.\\
Scroll-Balken (b): Das \gls{sichtfeld} bewegt sich relativ zur Größe des gesamten Graphen um das gleiche Maß wie der Scroll-Balken zur Länge seiner Fahrbahn in die entsprechende Richtung (Horizontal/Vertikal).\\

\fano{Zoom-Grad ändern}\label{fa:zoom}
\textbf{Ziel:} Der Zoom-Grad des \gls{sichtfeld}es soll vergrößert bzw. verkleinert werden.\\
\textbf{Vorbedingung:} Ein Graph wurde in die Graphansicht geladen.\\
\textbf{Nachbedingung (Erfolg):} Der Zoom-Grad wurde angepasst.\\
\textbf{Nachbedingung (Fehlschlag):} Der Zoom-Grad bleibt gleich, falls ein Maximum/Minimum erreicht wurde. Es wird keine Fehlermeldung ausgegeben.\\
\textbf{Auslösende Ereignisse:} Der Benutzer drückt STRG und dreht am Mausrad, während der Mauszeiger innerhalb der Graphansicht liegt.\\
\textbf{Beschreibung:}\\
Der Zoom-Grad ändert sich mit dem Drehen des Mausrades. Die Änderungsrichtung kann sich in Abhängigkeit vom Betriebssystem ändern.\\

%TODO Wie Deselektieren, shift/strg Selektieren mehrerer Knoten, welche Knoten werden beim Ziehen ausgewählt.
\fano{Knoten selektieren}\label{fa:selekt_knoten}
\textbf{Ziel:} Ein oder mehrere Knoten sollen der Auswahl hinzugefügt werden.\\
\textbf{Vorbedingung:} Ein Graph wurde in die Graphansicht geladen.\\
\textbf{Nachbedingung (Erfolg):} Ein oder mehrere Knoten wurden der Auswahl hinzugefügt.\\
\textbf{Nachbedingung (Fehlschlag):} -\\
\textbf{Auslösende Ereignisse:}
\begin{enumerate}[nolistsep, label=(\alph*)]
  \item Der Benutzer klickt mit der linken Maustaste auf einen Knoten.
  \item Der Benutzer klickt mit der linken Maustaste in einen leeren Bereich des \gls{sichtfeld}es und zieht den Mauszeiger.
\end{enumerate}
\textbf{Beschreibung:}\\
Der Benutzer kann Knoten mit der Maus zur Auswahl hinzufügen. Durch klicken und ziehen mit der linken Maustaste werden die Knoten, die innerhalb der Auswahlfläche liegen, zur Auswahl hinzugefügt.
Auf eine Auswahl von Knoten können verschiedene Operationen ausgeführt werden. %TODO: Referenz auf Funktionen auf Knotenmengen/Subgraphen
Die derzeit ausgewählten Knoten werden graphisch hervorgehoben.\\

\fano{Knoten einer \gls{gruppe} hinzufügen}\label{fa:gruppe}
\textbf{Ziel:} Einer oder mehrere Knoten können zu einer \gls{gruppe} hinzugefügt werden \\
\textbf{Vorbedingung:} Ein Graph wurde in die Graphansicht geladen. Einer oder mehrere Knoten wurden selektiert. \\
\textbf{Nachbedingung (Erfolg):} Die ausgewählten Knoten wurden einer \gls{gruppe} hinzugefügt \\
\textbf{Nachbedingung (Fehlschlag):} - \\
\textbf{Auslösende Ereignisse:} Der Benutzer wählt im Kontextmenü aus, dass die Knoten zu einer \gls{gruppe} hinzugefügt werden sollen. \\
\textbf{Beschreibung:} Eine selektierte Menge von Knoten kann zu einer \gls{gruppe} hinzugefügt werden. Diese können dann beispielsweise für die Einstellung der Constraints als Parameter verwendet werden(siehe \ref{fa:constraintsettings}).\\

\fano{Gruppe löschen}\label{fa:deletegruppe}
\textbf{Ziel:} Eine \gls{gruppe} entfernen \\
\textbf{Vorbedingung:} Ein Graph wurde in die Graphansicht geladen. Es existiert eine \gls{gruppe}.\\
\textbf{Nachbedingung (Erfolg):} Die \gls{gruppe} wurde gelöscht\\
\textbf{Nachbedingung (Fehlschlag):} - \\
\textbf{Auslösende Ereignisse:} Der Benutzer wählt im Kontextmenü aus, dass die \gls{gruppe} gelöscht wird. \\
\textbf{Beschreibung:} Eine vorhandene \gls{gruppe} wird vom Benutzer gelöscht.\\

\fano{Wechsel zwischen Graphen}\label{fa:graphwechsel}
\textbf{Ziel:} Der in der Graphansicht angezeigte Graph wurde ausgetauscht. \\
\textbf{Vorbedingung:} Eine Graphdatei wurde geladen.\\
\textbf{Nachbedingung (Erfolg):} Der ausgewählte Graph wird in der Graphansicht angezeigt.\\
\textbf{Nachbedingung (Fehlschlag):} -\\
\textbf{Auslösende Ereignisse:}
\begin{enumerate}[nolistsep, label=(\alph*)]
  \item Der Benutzer wählt über die Strukturansicht einen anderen Graphen aus.
  \item Der Benutzer wählt über die Tabansicht einen bereits geöffneten Graphen %TODO: GUI Referenz
\end{enumerate}
\textbf{Beschreibung:}\\
Das Produkt unterstützt Graphdateien, welche mehr als einen Graphen beschreiben.
Mit dieser Funktion kann der Benutzer die Ansicht zwischen den Graphen wechseln.

\fano{Informationsanzeige zu einzelnen Knoten und Kanten}\label{fa:infoanzeige}
\textbf{Ziel:} Zu einem ausgewählten Knoten oder einer Kanten wird Information angezeigt.\\
\textbf{Vorbedingung:} Ein Graph wurde in die Graphansicht geladen.\\
\textbf{Nachbedingung (Erfolg):} In der Informationsansicht wird die Information zu dem ausgewählten Knoten oder der Kante angezeigt.\\
\textbf{Nachbedingung (Fehlschlag):} -\\
\textbf{Auslösende Ereignisse:} Der Benutzer wählt einen Knoten oder eine Kante aus.\\
\textbf{Beschreibung:}\\
Die Information eines über einen per Mausklick ausgewählten Knoten oder einer Kante wird in der Informationsansicht angezeigt. Diese Information ist abhängig von der importierten Graphdatei (siehe \ref{fa:import}) und des ausgewählten Graphlayouts (siehe \ref{fa:layout}).

\fano{Statistiken zu Graphen}\label{fa:statistik}
\textbf{Ziel:} Zu ausgewählten Graphen und \glspl{subgraph} wird eine Statistik angezeigt.\\
\textbf{Vorbedingung:} Ein Graph wurde in die Graphansicht geladen.\\
\textbf{Nachbedingung (Erfolg):} In der Informationsansicht wird eine Statistik zu dem ausgewählten Graphen oder \glspl{subgraph} angezeigt.\\
\textbf{Nachbedingung (Fehlschlag):} -\\
\textbf{Auslösende Ereignisse:}
\begin{enumerate}[nolistsep, label=(\alph*)]
  \item Der Benutzer wählt einen \glspl{subgraph} über die Graphansicht aus.
  \item Der Benutzer wählt einen Graphen über die Strukturansicht aus.
\end{enumerate}
\textbf{Beschreibung:}\\
Die in der Informationsansicht angezeigte Statistik über einen Graphen oder \glspl{subgraph} enthält zum Beispiel Informationen über die Knoten- und Kantenzahl.

\fano{Kontextmenü}\label{fa:kontextmenü}
\textbf{Ziel:} Dem Benutzer werden Funktion, welche auf seine derzeitige Auswahl ausführbar sind, zur Verfügung gestellt.\\
%TODO: Referenz zu GUI Entwurf mit sichtbaren Kontextmenü
\textbf{Vorbedingung:} -\\
\textbf{Nachbedingung (Erfolg):} Es erscheint ein Kontextmenü mit Funktionen, die auf die derzeit ausgewählten Elementen ausgeführt werden können.\\
\textbf{Nachbedingung (Fehlschlag):} -\\
\textbf{Auslösende Ereignisse:} Der Benutzer klickt mit der rechten Maustaste auf
\begin{itemize}[nolistsep]
  \item einen beliebiegen Knoten im derzeit angezeigten Graphen.
  \item einen Knoten in einer selektierten Knotenmenge im derzeit angezeigten Graphen.
  \item einen Namen eines Graphen in der Strukturansicht.
\end{itemize}
\textbf{Beschreibung:}\\
Das Kontextmenü bietet einen schnellen Zugriff auf Funktionen die auf eine Auswahl ausgeführt werden können.
Abhängig von den ausgewählten Elementen können sich die zur Verfügung stehenden Funktionen ändern.
Selektierte \glspl{subgraph} können über das Kontexmenü kollabiert werden (siehe \ref{fa:kollabieren}).
Ein kollabierter \gls{subgraph} kann durch das Kontextmenü des repräsentierenden Knoten wieder ausgeklappt werden (siehe \ref{fa:ausklappen}).
Plugins können Operationen auf Knoten, Teilgraphen und Graphen, wie in \ref{s:operationen} beschrieben, registrieren.
Diese Operationen werden dann im jeweiligen Kontextmenü dargestellt.


\fano{TODO: Abspeichern als Graph}\label{fa:speichern}

\fano{TODO: Definition des JOANA-Graphtypes}\label{fa:joanatyp}
\section{Wunsch-Funktionen}

\fano{Steuerung über Tastaturkürzel}\label{fa:hotkey}
\textbf{Ziel:} Häufig verwendete Funktionen können über Tastaturkürzel ausgeführt werden.\\
\textbf{Vorbedingung:} -\\
\textbf{Nachbedingung (Erfolg):} Die dem Tastaturkürzel zugeordnete Funktion wurde ausgeführt.\\
\textbf{Nachbedingung (Fehlschlag):} -\\
\textbf{Auslösende Ereignisse:}
Ein Benutzer aktiviert ein Tastenkürzel, welchem eine Funktion zugeordnet ist.\\
\textbf{Beschreibung:}\\
Häufig verwendete Funktionen (wie z.B. Navigation im Graphen) und Menüs sind Tastaturkürzel zugeordnet.
Die Tastenkürzel werden hinter dem Funktionsnamen im Menüleiste oder Kontextmenü angezeigt.

\fano{Fortschrittsbalken bei Berechnung}\label{fa:fortschritt}
\textbf{Ziel:} Während das Produkt das Graphlayout berechnet, wird ein Fortschrittsbalken angezeigt.\\
\textbf{Vorbedingung:} Ein Graph wurde in die Graphansicht geladen.\\
\textbf{Nachbedingung (Erfolg):} Während der Layoutberechnung wurde ein Fortschrittsbalken angezeigt.\\
\textbf{Nachbedingung (Fehlschlag):} -\\
\textbf{Auslösende Ereignisse:} Der Benutzer wählt über die Menüleiste ein Layout aus.\\
\textbf{Beschreibung:}\\
Der Fortschrittsbalken dient dazu dem Benutzer anzuzeigen, wie weit das Programm bisher mit dem Layouten des Graphen ist, und bietet ihm somit die Möglichkeit der Zeitabschätzung bei der Berechnung eines Graphlayoutes.

\fano{Eine Übersicht des angezeigten Graphen}\label{fa:uebersicht}
\textbf{Ziel:} Es soll ein kleines Bild des kompletten Graphen angezeigt werden \\
\textbf{Vorbedingung:} Ein Graph wurde in die Graphansicht geladen.\\
\textbf{Nachbedingung (Erfolg):} Die Übersicht wird korrekt (mit dem aktuell ausgewählten Layout) angezeigt \\
\textbf{Nachbedingung (Fehlschlag):} -\\
\textbf{Auslösende Ereignisse:}
\begin{enumerate}[nolistsep, label=(\alph*)]
  \item Ein Graph wird in die Graphansicht geladen.
  \item Ein Graph wird neu gelayoutet.
\end{enumerate}
\textbf{Beschreibung:}\\
Diese Übersicht besteht aus einem Bild des kompletten Graphen und zeigt an in welchem Teil des Graphen das aktuelle \gls{sichtfeld} liegt. Somit sieht der Benutzer, mit einem Blick, in welchem Teil des Graphen das \gls{sichtfeld} liegt und kann sich besser orientieren.

\fano{Änderung der Darstellung von Kanten}\label{fa:darst-kanten}
\textbf{Ziel:} Der Nutzer kann auswählen in welcher Form die Kanten des Graphen gezeichnet werden. \\
\textbf{Vorbedingung:} -\\
\textbf{Nachbedingung (Erfolg):} Die neu gezeichneten Graphen werden mit der ausgewählten Darstellung für Kanten gezeichnet \\
\textbf{Nachbedingung (Fehlschlag):} -\\
\textbf{Auslösende Ereignisse:} Der Nutzer wählt in den Einstellungen die gewünschte Darstellung der Kanten.\\
\textbf{Beschreibung:}\\
Es kann ausgewählt werden, ob die Kanten als \gls{bezier}, orthogonale Kanten oder als direkte Kanten dargestellt werden. Alle Graphen, welche neu gelayoutet werden, besitzen diese Darstellung der Kanten. Für bereits gelayoutete Graphen wird diese Einstellung nicht automatisch übernommen.

\fano{Reload-Funktion}\label{fa:reload}
\textbf{Ziel:} Änderungen, welche an der importierten Graphdatei vorgenommen wurden, können durch die Reload-Funktion übernommen werden.\\
\textbf{Vorbedingung:} Eine Graphdatei wurde importiert und ist weiterhin an der selben Stelle im Dateisystem vorhanden und zugreifbar.\\
\textbf{Nachbedingung (Erfolg):} Die aktuelle Version des Graphen aus der Datei wurde importiert.\\
\textbf{Nachbedingung (Fehlschlag):} Auf die Datei konnte nicht zugegriffen werden oder enthält ungültige Informationen. Keine Graphdatei ist geladen. Es wird ein Fehler ausgegeben.\\
\textbf{Auslösende Ereignisse:} Der Nutzer wählt die Reload-Funktion über die Menüleiste.\\
\textbf{Beschreibung:}\\
Die Reload-Funktion ermöglicht einen schnellen erneuten Import einer zuvor schon importierten Graphdatei.
Falls der derzeitig in der Graphansicht geladene Graph in der neuen Datei vorhanden ist, wird er aktualisiert in die Graphansicht geladen.
Layout und andere Darstellungsoption, wie etwaige aktivierte Knoten- und Kantenfilter werden übernommen.

\fano{Automatisierte Subgraph Auswahl}\label{fa:gpm}
\textbf{Ziel:} Die automatisierte Selektion von \glspl{subgraph} in einem Graphen \\
\textbf{Vorbedingung:} Ein Graph wurde in die Graphansicht geladen.\\
\textbf{Nachbedingung (Erfolg):} Es wurden alle \glspl{subgraph} ausgewählt, die mit dem \gls{pattern} übereinstimmen.\\
\textbf{Nachbedingung (Fehlschlag):} - \\
\textbf{Auslösende Ereignisse:} Der User erstellt ein Selektor\\
\textbf{Beschreibung:}\\
Die automatisierte Selektion von Subgraphen ermöglicht es automatisch Gruppen zu erstellen, die mit einem bestimmtem \gls{pattern} übereinstimmen. Dies kann beispielsweise genutzt werden, um in größeren Graphen schneller Constraints erstellen.\\

\fano{Testen von Erreichbarkeit von Knoten}\label{fa:erreichbarkeit}
\textbf{Ziel:} Der Benutzer kann überprüfen ob ein Knoten von einem anderen erreichbar ist. \\
\textbf{Vorbedingung:} Ein Graph wurde in die Graphansicht geladen. Ein Knoten wurde selektiert. \\
\textbf{Nachbedingung (Erfolg):} Der Pfad zwischen den Knoten wird selektiert. \\
\textbf{Nachbedingung (Fehlschlag):} Ausgabe: "kein Pfad gefunden." \\
\textbf{Auslösende Ereignisse:} Der User öffnet das Kontextmenü eines zweiten Knotens und wählt: "finde Pfad zu" aus.\\
\textbf{Beschreibung:}\\
Die Funktion zum Testen von Erreichbarkeit von Knoten ermöglicht es, Pfade zwischen Knoten zu betrachten oder sie zu einer Gruppe hinzu zu fügen.\\

%\fano{}\label{fa:}
%\textbf{Ziel:} \\
%\textbf{Vorbedingung:} \\
%\textbf{Nachbedingung (Erfolg):} \\
%\textbf{Nachbedingung (Fehlschlag):} \\
%\textbf{Auslösende Ereignisse:} \\
%\textbf{Beschreibung:}\\
%
