\chapter{Plugin-Schnittstellen}
\label{ch:plugschnitt}

%TODO Plugin Architektur erklären, vlt Graph von Whiteboard

\setcounter{psnr}{10}
\newcommand{\psno}{\ifnum\value{psnr}<10 00\else\ifnum\value{psnr}<100 0\fi\fi\arabic{psnr}}
\newcommand\ps[2]{\namedlabel{s:#1}{/S\psno/}\addtocounter{psnr}{10}: & #2 \\ [1ex] }

\begin{tabular}{lp{0.9\linewidth}}
  \ps{graphtyp}{\textit{Graphtypen-Schnittstelle:} Plugins sind in der Lage neue Graphtypen zu definieren. Alle von Plugins definierten Graphtypen sind dem Benutzer beim Importieren zur Auswahl gestellt. Das Plugin kann dann mit einem Graphtyp bestimmte Import-Möglichkeiten, Layoutalgorithmen und Darstellungsoptionen verbinden, sodass der Benutzer diese nicht bei jedem Öffnen eines Graphen neu festlegen muss und gleichzeitig eine Importfunktion für ein Datenformat von mehreren Plugins wiedervewendet werden können.}
  \ps{import}{\textit{Import-Schnittstelle:} Ein Plugin, welches auf diese Schnittstelle zugreift, kann einen neuen Datentyp festlegen, für welchen es eine Importfunktion implementiert. Dieser steht dann in der Importfunktion von Graph von Ansicht zur Verfügung. Das Plugin soll in der Lage sein, den neuen Datentyp zu parsen und an Graph von Ansicht weiterzuleiten.}
  \ps{export}{\textit{Export-Schnittstelle:} Graph von Ansicht sendet interne Daten über den gezeichneten Graphen an diese Schnittstelle. Das Plugin, welches auf diese Schnittstelle zugreift, soll in der Lage sein, aus diesen Daten einen neuen Datentyp zu erstellen und abzuspeichern. Der Datentyp erscheint dann auch bei dem Speicherdialog.}
  \ps{layoutalgo}{\textit{Layoutalgorithmen-Schnittstelle:} Diese Schnittstelle ermöglicht das Implementieren eines neuen Algorithmus, auf welchen dann beim Layouten eines Graphen zugegriffen werden kann. Das Plugin bekommt eine interne Repräsentation des Graphen und soll jedem Knoten eine Koordinate zuordnen. Außerdem sollen die Verläufe von Kanten festgelegt werden.}
  \ps{filter}{\textit{Filter-Schnittstelle:} Es können neue Filter für spezielle Knoten- und Kantentypen hinzugefügt werden, welche bei den Filteroptionen von Graph von Ansicht automatisch hinzugefügt werden. Diese Filter bestehen darin, dass sie Kanten oder Knoten nach bestimmten Eigenschaften ausblenden.}
  \ps{darstellung}{\textit{Darstellungs-Schnittstelle:} Es können einzelnen Knoten- und Kantentypen Eigenschaften zugeordnet werden, wie zum Beispiel die Farbe für das Layout, die Form oder zusätzliche Informationen, welche in der Informationsanzeige angezeigt werden.}
  \ps{operationen}{\textit{Operation-Schnittstelle für Knoten und Kanten:} Es können neue Operationen auf Knoten und Kanten ausgeführt werden, wie das Kollabieren von bereits definierten \gls{subgraph}en (z.B. Feldzugriffe bei JOANA).}
\end{tabular}
