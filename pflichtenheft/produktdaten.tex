\chapter{Produktdaten}\label{ch:daten}

\begin{itemize}
  \item Gesetzte Einstellungen (z.B. Speicherort der zuletzt exportierten Datei oder Größe des Fensters) werden mithilfe der \gls{prefapi} von Java gespeichert.
  \item Als Inputformat für die zu zeichnenden Graphen wird das \gls{graphml}-Format unterstützt.\\
    Das GraphML-Format ist so aufgebaut, dass nicht alle Konstrukte welche mit GraphML beschrieben werden können (wie z.B. \gls{hyperkante}),
    von einer Applikation unterstützt werden müssen.
    Nicht unterstützte Konstrukte können ignoriert werden, ohne strukturelle Informationen des restlichen Graphens zu verlieren.\\
    Wir listen hier die GraphML-Konstrukte, welche von dem Produkt unterstützt werden müssen und für welche eventuell eine Unterstützung später hinzugefügt wird, auf:
    \begin{itemize}
      \item Muss-Konstrukte: graph, node, edge, desc, key, data, default, hierarchische/geschachtelte Graphen  % Ist Port ein Musskriterium? Beim ersten Treffen wurde es erwähnt
      \item Kann-Konstrukte: hyperedge, port, endpoint, Multi-Graphen
    \end{itemize}
    Für den Fall das ein Kann-Konstrukt nicht unterstützt wird, wird es, wie in der GraphML Spezifikation vorgeschrieben, behandelt.
  \item Das Produkt wird den resultierenden Graphen als \gls{svg}-Format exportieren können.
\end{itemize}
