\chapter{Verwendete Entwurfsmuster}
\label{ch:entwurfsmuster}

\section{Parameterklassen Visitor Pattern}
In den Parameterklassen wird ein Visitor Pattern verwendet um bei Änderungen der Werte auch direkte Änderungen in der GUI zu erhalten und umgekehrt. Die zum textspezifischen Parameter passende Funktion des Visitors wird per Double Dispatch vom Parameter aufgerufen. In dieser Funktion kann der Visitor nun Änderungen am Parameter machen, oder ihn Auslesen. In der GUI ist ein solcher Visitor implementiert. Er liest mehrere Parameter aus und erstellt passend zu den Typen und Werten der Parameter einen Dialog, welcher genau Elemente beinhaltet, mit denen man die Werte der Parameter anpassen/editieren kann. Wurde ein Wert im Dialog geändert wird dieser im passenden Parameter gesetzt. Deshalb werden die Parameterklassen zum dynamischen Erstellen von Einstellungsdialogen für Layouts und/oder Constraints verwendet.

\section{Plugins}
Durch Plugin soll eine einfache Erweiterung des Programmes möglich gemacht werden. Dafür werden Schnittstellen zur Verfügung gestellt welche zu jeder Zeit erweitert werden können.
Es existieren Schnittstellen für:
\begin{itemize}
  \item Importer und Exporter um neue Dateiformate unterstützen zu können
  \item Layoutalgorithmen für verschiedene Graphrepräsentationen
  \item Graphrepräsentationen, welche mit einem \textbf{Workspace} implementiert werden
\end{itemize}

\section{Sugiyama Algorithmus Strategy Pattern}
Der \textbf{SugiyamaLayoutAlgorithm} basiert auf dem Strategy Pattern. Es gibt 5 Phasen/Algorithmen welche das Sugiyama Framework durchläuft.
Jede Phase entspricht einer Strategie, welche durch eine Schnittstelle definiert ist (z.B \textbf{ICrossMinimizer}).
Die Konkreten Strategien implementieren dann diese gegebene Schnittstelle. Dadurch können einzelne Phasen des SugiyamaLayoutAlgorithm leicht ausgetauscht werden.

\section{Graph Builder Builder Pattern}
Da viele unterschiedliche Graphtypen existieren können, wird für die Konstruktion eines speziellen Graphes ein Builder Pattern verwendet.
Dies hat den großen Vorteil, dass ein Importer nur eine Schnittstelle kennen muss, aber unterschiedliche Graphtypen importieren kann.
Dazu erhält der Importer einen GraphModelBuilder, welchem er die Attribute des Graphes übergibt. Der konkrete GraphModelBuilder konstruiert nun aus diesen Attributen die spezielle Graphrepräsentation.

\section{GUI-Klassen Model View Controller Pattern}
In der GUI von GAns gibt es verschiedene \textbf{Ansichten(Views)} welche auf verschiedenen \textbf{Datensätzen(Models)} darstellen und auf Benutzerinteraktionen unterschiedlich reagieren. \\
Die \textbf{StrukturView} arbeitet auf einem GraphModel-Object, bzw. dessen Liste von Graphen, welche beim Import von GAnsApplication gesetzt wird. Die meisten Interaktionen werden hier von der View selber behandelt. Da aber der Doppelklick auf ein Baumelement eine anzeigeübergreifende Aktion ist, wird dieser von der GAnsApplication bearbeitet. \\
Eine \textbf{GraphView} arbeitet auf einem Graph-Objekt aus dem graphmodel-Paket welches durch die GAnsApplication gesetzt wird. Die GraphView behandelt alle Benutzerinteraktionen, wie Zoomen und navigieren selbst und hat ein SelectionModel(GraphViewSelectionModel), welches Änderungen in der Selektion behandelt. \\
Die \textbf{InformationView} arbeitet auf dem SelectionModel der aktuell angezeigten GraphView. Ändert sich die Selektion wird die InformationView informiert und bekommt neue Daten gesetzt. Einen Controller gibt es nicht, da die TableView keine besonderen Benutzerinteraktionen benötigt.