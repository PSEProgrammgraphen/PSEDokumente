\section*{Class Hierarchy}{
\thispagestyle{empty}
\markboth{Class Hierarchy}{Class Hierarchy}
\addcontentsline{toc}{section}{Class Hierarchy}
\subsection*{Classes}
{\raggedright
\hspace{0.0cm} $\bullet$ java.lang.Object {\tiny \refdefined{java.lang.Object}} \\
\hspace{1.0cm} $\bullet$ gui.GraphViewEventHandler {\tiny \refdefined{gui.GraphViewEventHandler}} \\
\hspace{1.0cm} $\bullet$ javafx.application.Application {\tiny \refdefined{javafx.application.Application}} \\
\hspace{2.0cm} $\bullet$ gui.GAnsApplication {\tiny \refdefined{gui.GAnsApplication}} \\
\hspace{1.0cm} $\bullet$ javafx.scene.Node {\tiny \refdefined{javafx.scene.Node}} \\
\hspace{2.0cm} $\bullet$ javafx.scene.Parent {\tiny \refdefined{javafx.scene.Parent}} \\
\hspace{3.0cm} $\bullet$ javafx.scene.layout.Region {\tiny \refdefined{javafx.scene.layout.Region}} \\
\hspace{4.0cm} $\bullet$ javafx.scene.control.Control {\tiny \refdefined{javafx.scene.control.Control}} \\
\hspace{5.0cm} $\bullet$ javafx.scene.control.TableView {\tiny \refdefined{javafx.scene.control.TableView}} \\
\hspace{6.0cm} $\bullet$ gui.InformationView {\tiny \refdefined{gui.InformationView}} \\
\hspace{5.0cm} $\bullet$ javafx.scene.control.TreeView {\tiny \refdefined{javafx.scene.control.TreeView}} \\
\hspace{6.0cm} $\bullet$ gui.StructureView {\tiny \refdefined{gui.StructureView}} \\
\hspace{4.0cm} $\bullet$ javafx.scene.layout.Pane {\tiny \refdefined{javafx.scene.layout.Pane}} \\
\hspace{5.0cm} $\bullet$ gui.GraphView {\tiny \refdefined{gui.GraphView}} \\
\hspace{5.0cm} $\bullet$ javafx.scene.layout.StackPane {\tiny \refdefined{javafx.scene.layout.StackPane}} \\
\hspace{6.0cm} $\bullet$ gui.VertexShape {\tiny \refdefined{gui.VertexShape}} \\
\hspace{2.0cm} $\bullet$ javafx.scene.shape.Shape {\tiny \refdefined{javafx.scene.shape.Shape}} \\
\hspace{3.0cm} $\bullet$ javafx.scene.shape.Line {\tiny \refdefined{javafx.scene.shape.Line}} \\
\hspace{4.0cm} $\bullet$ gui.EdgeShape {\tiny \refdefined{gui.EdgeShape}} \\
}
}
\section{Package gui}{
\label{gui}\hskip -.05in
\hbox to \hsize{\textit{ Package Contents\hfil Page}}
\vskip .13in
\hbox{{\bf  Classes}}
\entityintro{EdgeShape}{gui.EdgeShape}{A visual representation of an edge with a text.}
\entityintro{GAnsApplication}{gui.GAnsApplication}{Main application of GAns.}
\entityintro{GraphView}{gui.GraphView}{A view used for showing and creating a graph in GAns.}
\entityintro{GraphViewEventHandler}{gui.GraphViewEventHandler}{GraphViewEventHandler provides listeners for making the \texttt{\small GraphView}{\small 
\refdefined{gui.GraphView}} draggable and zoomable.}
\entityintro{InformationView}{gui.InformationView}{The InformationView shows a given set of properties from the selected vertices in the \texttt{\small GraphView}{\small 
\refdefined{gui.GraphView}}.}
\entityintro{StructureView}{gui.StructureView}{The StructureView regulates the access and representation of the elements in the StructreView of GAns.}
\entityintro{VertexShape}{gui.VertexShape}{A visual representation of a vertex with a text inside of it.}
\vskip .1in
\vskip .1in
\subsection{\label{gui.EdgeShape}\index{EdgeShape}Class EdgeShape}{
\vskip .1in 
A visual representation of an edge with a text.\vskip .1in 
\subsubsection{Declaration}{
\begin{lstlisting}[frame=none]
public class EdgeShape
 extends javafx.scene.shape.Line\end{lstlisting}
\subsubsection{Constructors}{
\vskip -2em
\begin{itemize}
\item{ 
\index{EdgeShape(VertexShape, VertexShape)}
{\bf  EdgeShape}\\
\begin{lstlisting}[frame=none]
public EdgeShape(VertexShape vertex1,VertexShape vertex2)\end{lstlisting} %end signature
\begin{itemize}
\item{
{\bf  Description}

Constructor.
}
\item{
{\bf  Parameters}
  \begin{itemize}
   \item{
\texttt{vertex1} -- First of the two vertices that shall be connected.}
   \item{
\texttt{vertex2} -- Second of the two vertices that shall be connected.}
  \end{itemize}
}%end item
\end{itemize}
}%end item
\end{itemize}
}
\subsubsection{Methods}{
\vskip -2em
\begin{itemize}
\item{ 
\index{getText()}
{\bf  getText}\\
\begin{lstlisting}[frame=none]
public java.lang.String getText()\end{lstlisting} %end signature
\begin{itemize}
\item{
{\bf  Description}

Returns the text shown on the edge.
}
\item{{\bf  Returns} -- 
The text that is being displayed on the edge. 
}%end item
\end{itemize}
}%end item
\item{ 
\index{setColor(Color)}
{\bf  setColor}\\
\begin{lstlisting}[frame=none]
public void setColor(javafx.scene.paint.Color color)\end{lstlisting} %end signature
\begin{itemize}
\item{
{\bf  Description}

Sets the color of the edge
}
\item{
{\bf  Parameters}
  \begin{itemize}
   \item{
\texttt{color} -- The color the edge will be.}
  \end{itemize}
}%end item
\end{itemize}
}%end item
\item{ 
\index{setText(String)}
{\bf  setText}\\
\begin{lstlisting}[frame=none]
public void setText(java.lang.String text)\end{lstlisting} %end signature
\begin{itemize}
\item{
{\bf  Description}

Sets the text shown on the edge
}
\item{
{\bf  Parameters}
  \begin{itemize}
   \item{
\texttt{text} -- The text that is being displayed on the edge.}
  \end{itemize}
}%end item
\end{itemize}
}%end item
\end{itemize}
}
}
\subsection{\label{gui.GAnsApplication}\index{GAnsApplication}Class GAnsApplication}{
\vskip .1in 
Main application of GAns.\vskip .1in 
\subsubsection{Declaration}{
\begin{lstlisting}[frame=none]
public class GAnsApplication
 extends javafx.application.Application\end{lstlisting}
\subsubsection{Constructors}{
\vskip -2em
\begin{itemize}
\item{ 
\index{GAnsApplication()}
{\bf  GAnsApplication}\\
\begin{lstlisting}[frame=none]
public GAnsApplication()\end{lstlisting} %end signature
}%end item
\end{itemize}
}
\subsubsection{Methods}{
\vskip -2em
\begin{itemize}
\item{ 
\index{main(String\lbrack \rbrack )}
{\bf  main}\\
\begin{lstlisting}[frame=none]
public static void main(java.lang.String[] args)\end{lstlisting} %end signature
\begin{itemize}
\item{
{\bf  Description}

Main method.
}
\item{
{\bf  Parameters}
  \begin{itemize}
   \item{
\texttt{args} -- Arguments.}
  \end{itemize}
}%end item
\end{itemize}
}%end item
\item{ 
\index{start(Stage)}
{\bf  start}\\
\begin{lstlisting}[frame=none]
public abstract void start(javafx.stage.Stage arg0) throws java.lang.Exception\end{lstlisting} %end signature
}%end item
\end{itemize}
}
}
\subsection{\label{gui.GraphView}\index{GraphView}Class GraphView}{
\vskip .1in 
A view used for showing and creating a graph in GAns. It supports zooming and other general navigation features.\vskip .1in 
\subsubsection{Declaration}{
\begin{lstlisting}[frame=none]
public class GraphView
 extends javafx.scene.layout.Pane\end{lstlisting}
\subsubsection{Constructors}{
\vskip -2em
\begin{itemize}
\item{ 
\index{GraphView()}
{\bf  GraphView}\\
\begin{lstlisting}[frame=none]
public GraphView()\end{lstlisting} %end signature
\begin{itemize}
\item{
{\bf  Description}

Constructor.
}
\end{itemize}
}%end item
\end{itemize}
}
\subsubsection{Methods}{
\vskip -2em
\begin{itemize}
\item{ 
\index{addGrid()}
{\bf  addGrid}\\
\begin{lstlisting}[frame=none]
public void addGrid()\end{lstlisting} %end signature
\begin{itemize}
\item{
{\bf  Description}

Adds a grid to the GraphView, on which the dragging can be mapped.
}
\end{itemize}
}%end item
\item{ 
\index{addVertex(double, double, String)}
{\bf  addVertex}\\
\begin{lstlisting}[frame=none]
public void addVertex(double x,double y,java.lang.String text)\end{lstlisting} %end signature
\begin{itemize}
\item{
{\bf  Description}

Adds a single vertex in the GraphView.
}
\item{
{\bf  Parameters}
  \begin{itemize}
   \item{
\texttt{x} -- The x position in the view.}
   \item{
\texttt{y} -- The y position in the view.}
   \item{
\texttt{Text} -- The text of the vertex.}
  \end{itemize}
}%end item
\end{itemize}
}%end item
\item{ 
\index{getScale()}
{\bf  getScale}\\
\begin{lstlisting}[frame=none]
public double getScale()\end{lstlisting} %end signature
\begin{itemize}
\item{
{\bf  Description}

Returns the scale on which the GraphView currently is.
}
\item{{\bf  Returns} -- 
The scale of the GraphView. 
}%end item
\end{itemize}
}%end item
\item{ 
\index{setGraph()}
{\bf  setGraph}\\
\begin{lstlisting}[frame=none]
public void setGraph()\end{lstlisting} %end signature
\begin{itemize}
\item{
{\bf  Description}

Sets a graph. Every element in the graph will be generated and then shown.
}
\item{
{\bf  Parameters}
  \begin{itemize}
   \item{
\texttt{graph} -- The graph to be visualized in the view.}
  \end{itemize}
}%end item
\end{itemize}
}%end item
\item{ 
\index{setPivot(double, double)}
{\bf  setPivot}\\
\begin{lstlisting}[frame=none]
public void setPivot(double x,double y)\end{lstlisting} %end signature
\begin{itemize}
\item{
{\bf  Description}

Sets the pivot so the scrolling follows the mouse position on the GraphView.
}
\item{
{\bf  Parameters}
  \begin{itemize}
   \item{
\texttt{x} -- }
   \item{
\texttt{y} -- }
  \end{itemize}
}%end item
\end{itemize}
}%end item
\item{ 
\index{setScale(double)}
{\bf  setScale}\\
\begin{lstlisting}[frame=none]
public void setScale(double scale)\end{lstlisting} %end signature
\begin{itemize}
\item{
{\bf  Description}

Sets the scale of the GraphView.
}
\item{
{\bf  Parameters}
  \begin{itemize}
   \item{
\texttt{scale} -- The scale of the GraphView.}
  \end{itemize}
}%end item
\end{itemize}
}%end item
\end{itemize}
}
}
\subsection{\label{gui.GraphViewEventHandler}\index{GraphViewEventHandler}Class GraphViewEventHandler}{
\vskip .1in 
GraphViewEventHandler provides listeners for making the \texttt{\small GraphView}{\small 
\refdefined{gui.GraphView}} draggable and zoomable.\vskip .1in 
\subsubsection{Declaration}{
\begin{lstlisting}[frame=none]
public class GraphViewEventHandler
 extends java.lang.Object\end{lstlisting}
\subsubsection{Constructors}{
\vskip -2em
\begin{itemize}
\item{ 
\index{GraphViewEventHandler(GraphView)}
{\bf  GraphViewEventHandler}\\
\begin{lstlisting}[frame=none]
public GraphViewEventHandler(GraphView canvas)\end{lstlisting} %end signature
\begin{itemize}
\item{
{\bf  Description}

Constructor.
}
\item{
{\bf  Parameters}
  \begin{itemize}
   \item{
\texttt{canvas} -- The \texttt{\small GraphView}{\small 
\refdefined{gui.GraphView}} which later on will get the EventHandler set.}
  \end{itemize}
}%end item
\end{itemize}
}%end item
\end{itemize}
}
\subsubsection{Methods}{
\vskip -2em
\begin{itemize}
\item{ 
\index{getOnMouseDraggedEventHandler()}
{\bf  getOnMouseDraggedEventHandler}\\
\begin{lstlisting}[frame=none]
public javafx.event.EventHandler getOnMouseDraggedEventHandler()\end{lstlisting} %end signature
\begin{itemize}
\item{
{\bf  Description}

Returns an EventHandler which handles dragging inside the \texttt{\small GraphView}{\small 
\refdefined{gui.GraphView}}.
}
\item{{\bf  Returns} -- 
An EventHandler which handles dragging. 
}%end item
\end{itemize}
}%end item
\item{ 
\index{getOnMousePressedEventHandler()}
{\bf  getOnMousePressedEventHandler}\\
\begin{lstlisting}[frame=none]
public javafx.event.EventHandler getOnMousePressedEventHandler()\end{lstlisting} %end signature
\begin{itemize}
\item{
{\bf  Description}

Returns an EventHandler which handles pressing the mouse inside the \texttt{\small GraphView}{\small 
\refdefined{gui.GraphView}}.
}
\item{{\bf  Returns} -- 
An EventHandler which handles pressing the mouse. 
}%end item
\end{itemize}
}%end item
\item{ 
\index{getOnScrollEventHandler()}
{\bf  getOnScrollEventHandler}\\
\begin{lstlisting}[frame=none]
public javafx.event.EventHandler getOnScrollEventHandler()\end{lstlisting} %end signature
\begin{itemize}
\item{
{\bf  Description}

Returns an EventHandler which maps scrolling the mousewheel to zooming on the \texttt{\small GraphView}{\small 
\refdefined{gui.GraphView}}.
}
\item{{\bf  Returns} -- 
An EventHandler which maps scrolling the mousewheel to zooming. 
}%end item
\end{itemize}
}%end item
\end{itemize}
}
}
\subsection{\label{gui.InformationView}\index{InformationView}Class InformationView}{
\vskip .1in 
The InformationView shows a given set of properties from the selected vertices in the \texttt{\small GraphView}{\small 
\refdefined{gui.GraphView}}.\vskip .1in 
\subsubsection{Declaration}{
\begin{lstlisting}[frame=none]
public class InformationView
 extends javafx.scene.control.TableView\end{lstlisting}
\subsubsection{Constructors}{
\vskip -2em
\begin{itemize}
\item{ 
\index{InformationView()}
{\bf  InformationView}\\
\begin{lstlisting}[frame=none]
public InformationView()\end{lstlisting} %end signature
}%end item
\end{itemize}
}
\subsubsection{Methods}{
\vskip -2em
\begin{itemize}
\item{ 
\index{setInformations(ObservableList)}
{\bf  setInformations}\\
\begin{lstlisting}[frame=none]
public void setInformations(javafx.collections.ObservableList informations)\end{lstlisting} %end signature
\begin{itemize}
\item{
{\bf  Description}

Sets the properties which should be shown in the InformationView. The \texttt{\small GAnsProperty}{\small 
\refdefined{objectproperty.GAnsProperty}} are being processed in an internal factory, which automatically creates the tablecells. The function will be called whenever the selection of the \texttt{\small GraphView}{\small 
\refdefined{gui.GraphView}} changes.
}
\item{
{\bf  Parameters}
  \begin{itemize}
   \item{
\texttt{informations} -- List with \texttt{\small GAnsProperty}{\small 
\refdefined{objectproperty.GAnsProperty}}-elements which define the content of the InformationView}
  \end{itemize}
}%end item
\end{itemize}
}%end item
\end{itemize}
}
}
\subsection{\label{gui.StructureView}\index{StructureView}Class StructureView}{
\vskip .1in 
The StructureView regulates the access and representation of the elements in the StructreView of GAns.\vskip .1in 
\subsubsection{Declaration}{
\begin{lstlisting}[frame=none]
public class StructureView
 extends javafx.scene.control.TreeView\end{lstlisting}
\subsubsection{Constructors}{
\vskip -2em
\begin{itemize}
\item{ 
\index{StructureView()}
{\bf  StructureView}\\
\begin{lstlisting}[frame=none]
public StructureView()\end{lstlisting} %end signature
\begin{itemize}
\item{
{\bf  Description}

Constructor.
}
\end{itemize}
}%end item
\end{itemize}
}
\subsubsection{Methods}{
\vskip -2em
\begin{itemize}
\item{ 
\index{showTree()}
{\bf  showTree}\\
\begin{lstlisting}[frame=none]
public void showTree()\end{lstlisting} %end signature
\begin{itemize}
\item{
{\bf  Description}

Creates a tree like representation from a given graph and its subgraphs. Should be called, before calling other methods, because there could be a dummy root-node in the View.
}
\item{
{\bf  Parameters}
  \begin{itemize}
   \item{
\texttt{graph} -- The graph which should be represented.}
  \end{itemize}
}%end item
\end{itemize}
}%end item
\end{itemize}
}
}
\subsection{\label{gui.VertexShape}\index{VertexShape}Class VertexShape}{
\vskip .1in 
A visual representation of a vertex with a text inside of it.\vskip .1in 
\subsubsection{Declaration}{
\begin{lstlisting}[frame=none]
public class VertexShape
 extends javafx.scene.layout.StackPane\end{lstlisting}
\subsubsection{Constructors}{
\vskip -2em
\begin{itemize}
\item{ 
\index{VertexShape()}
{\bf  VertexShape}\\
\begin{lstlisting}[frame=none]
public VertexShape()\end{lstlisting} %end signature
\begin{itemize}
\item{
{\bf  Description}

Constructor
}
\end{itemize}
}%end item
\item{ 
\index{VertexShape(String)}
{\bf  VertexShape}\\
\begin{lstlisting}[frame=none]
public VertexShape(java.lang.String text)\end{lstlisting} %end signature
\begin{itemize}
\item{
{\bf  Description}

Constructor which directly sets the text.
}
\item{
{\bf  Parameters}
  \begin{itemize}
   \item{
\texttt{text} -- The text that will be displayed in the vertex.}
  \end{itemize}
}%end item
\end{itemize}
}%end item
\end{itemize}
}
\subsubsection{Methods}{
\vskip -2em
\begin{itemize}
\item{ 
\index{getText()}
{\bf  getText}\\
\begin{lstlisting}[frame=none]
public java.lang.String getText()\end{lstlisting} %end signature
\begin{itemize}
\item{
{\bf  Description}

Returns the text shown in the vertex.
}
\item{{\bf  Returns} -- 
The text that is being displayed in the vertex. 
}%end item
\end{itemize}
}%end item
\item{ 
\index{setColor(Color)}
{\bf  setColor}\\
\begin{lstlisting}[frame=none]
public void setColor(javafx.scene.paint.Color color)\end{lstlisting} %end signature
\begin{itemize}
\item{
{\bf  Description}

Sets the color of the vertex
}
\item{
{\bf  Parameters}
  \begin{itemize}
   \item{
\texttt{color} -- The color the vertex will be.}
  \end{itemize}
}%end item
\end{itemize}
}%end item
\item{ 
\index{setText(String)}
{\bf  setText}\\
\begin{lstlisting}[frame=none]
public void setText(java.lang.String text)\end{lstlisting} %end signature
\begin{itemize}
\item{
{\bf  Description}

Sets the text and adjusts the size of the rectangle to the size of the text.
}
\item{
{\bf  Parameters}
  \begin{itemize}
   \item{
\texttt{text} -- The text that will be displayed in the vertex.}
  \end{itemize}
}%end item
\end{itemize}
}%end item
\end{itemize}
}
}
}
\printindex
