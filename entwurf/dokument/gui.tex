\section{Package gui}{
\label{gui}\hskip -.05in
\hbox to \hsize{\textit{ Package Contents\hfil Page}}
\vskip .13in
\hbox{{\bf  Classes}}
\entityintro{EdgeShape}{gui.EdgeShape}{A visual representation of an edge with a text.}
\entityintro{GAnsApplication}{gui.GAnsApplication}{Main application of GAns.}
\entityintro{GAnsGraphElement}{gui.GAnsGraphElement}{A abstract class to generalize the visual elements that can be displayed in the \texttt{\small GraphView}{\small 
\refdefined{gui.GraphView}}.}
\entityintro{GraphView}{gui.GraphView}{A view used for showing and creating a graph in GAns.}
\entityintro{GraphViewEventHandler}{gui.GraphViewEventHandler}{GraphViewEventHandler provides listeners for making the \texttt{\small GraphView}{\small 
\refdefined{gui.GraphView}} draggable and zoomable.}
\entityintro{GraphViewGraphFactory}{gui.GraphViewGraphFactory}{The GraphViewGraphFactory generates the visual representation of a given \texttt{\small Graph}{\small 
\refdefined{graphmodel.Graph}} and gives access to the set \texttt{\small Graph}{\small 
\refdefined{graphmodel.Graph}}.}
\entityintro{GraphViewSelectionModel}{gui.GraphViewSelectionModel}{The selection model for the \texttt{\small GraphView}{\small 
\refdefined{gui.GraphView}}, that supports multiple selection of vertices and edges.}
\entityintro{InformationView}{gui.InformationView}{The InformationView shows a given set of properties from the selected vertices in the \texttt{\small GraphView}{\small 
\refdefined{gui.GraphView}}.}
\entityintro{ParameterDialogGenerator}{gui.ParameterDialogGenerator}{Generates a parameter dialog given a parent node and a set of parameters.}
\entityintro{StructureView}{gui.StructureView}{The StructureView regulates the access and representation of the elements in the StructureView of GAns.}
\entityintro{VertexShape}{gui.VertexShape}{A visual representation of a vertex with a text inside of it.}
\vskip .1in
\vskip .1in
\subsection{\label{gui.EdgeShape}\index{EdgeShape}Class EdgeShape}{
\vskip .1in 
A visual representation of an edge with a text.\vskip .1in 
\subsubsection{Declaration}{
\begin{lstlisting}[frame=none]
public class EdgeShape
 extends gui.GAnsGraphElement\end{lstlisting}
\subsubsection{Constructors}{
\vskip -2em
\begin{itemize}
\item{ 
\index{EdgeShape(VertexShape, VertexShape)}
{\bf  EdgeShape}\\
\begin{lstlisting}[frame=none]
public EdgeShape(VertexShape vertex1,VertexShape vertex2)\end{lstlisting} %end signature
\begin{itemize}
\item{
{\bf  Description}

Constructor.
}
\item{
{\bf  Parameters}
  \begin{itemize}
   \item{
\texttt{vertex1} -- First of the two vertices that shall be connected.}
   \item{
\texttt{vertex2} -- Second of the two vertices that shall be connected.}
  \end{itemize}
}%end item
\end{itemize}
}%end item
\item{ 
\index{EdgeShape(VertexShape, VertexShape, String)}
{\bf  EdgeShape}\\
\begin{lstlisting}[frame=none]
public EdgeShape(VertexShape vertex1,VertexShape vertex2,java.lang.String text)\end{lstlisting} %end signature
\begin{itemize}
\item{
{\bf  Description}

Constructor.
}
\item{
{\bf  Parameters}
  \begin{itemize}
   \item{
\texttt{vertex1} -- First of the two vertices that shall be connected.}
   \item{
\texttt{vertex2} -- Second of the two vertices that shall be connected.}
   \item{
\texttt{text} -- Text that will be displayed on the edge.}
  \end{itemize}
}%end item
\end{itemize}
}%end item
\end{itemize}
}
\subsubsection{Methods}{
\vskip -2em
\begin{itemize}
\item{ 
\index{getText()}
{\bf  getText}\\
\begin{lstlisting}[frame=none]
public abstract java.lang.String getText()\end{lstlisting} %end signature
\begin{itemize}
\item{
{\bf  Description copied from GAnsGraphElement{\small \refdefined{gui.GAnsGraphElement}} }

Returns the text shown on the element.
}
\item{{\bf  Returns} -- 
The text that is being displayed on the element. 
}%end item
\end{itemize}
}%end item
\item{ 
\index{setColor(Color)}
{\bf  setColor}\\
\begin{lstlisting}[frame=none]
public abstract void setColor(javafx.scene.paint.Color color)\end{lstlisting} %end signature
\begin{itemize}
\item{
{\bf  Description copied from GAnsGraphElement{\small \refdefined{gui.GAnsGraphElement}} }

Sets the color of the element.
}
\item{
{\bf  Parameters}
  \begin{itemize}
   \item{
\texttt{color} -- The color the element will be displayed in.}
  \end{itemize}
}%end item
\end{itemize}
}%end item
\item{ 
\index{setText(String)}
{\bf  setText}\\
\begin{lstlisting}[frame=none]
public abstract void setText(java.lang.String text)\end{lstlisting} %end signature
\begin{itemize}
\item{
{\bf  Description copied from GAnsGraphElement{\small \refdefined{gui.GAnsGraphElement}} }

Sets the text shown on the element.
}
\item{
{\bf  Parameters}
  \begin{itemize}
   \item{
\texttt{text} -- The text that will be displayed on the element.}
  \end{itemize}
}%end item
\end{itemize}
}%end item
\end{itemize}
}
}
\subsection{\label{gui.GAnsApplication}\index{GAnsApplication}Class GAnsApplication}{
\vskip .1in 
Main application of GAns.\vskip .1in 
\subsubsection{Declaration}{
\begin{lstlisting}[frame=none]
public class GAnsApplication
 extends javafx.application.Application\end{lstlisting}
\subsubsection{Constructors}{
\vskip -2em
\begin{itemize}
\item{ 
\index{GAnsApplication()}
{\bf  GAnsApplication}\\
\begin{lstlisting}[frame=none]
public GAnsApplication()\end{lstlisting} %end signature
}%end item
\end{itemize}
}
\subsubsection{Methods}{
\vskip -2em
\begin{itemize}
\item{ 
\index{main(String\lbrack \rbrack )}
{\bf  main}\\
\begin{lstlisting}[frame=none]
public static void main(java.lang.String[] args)\end{lstlisting} %end signature
\begin{itemize}
\item{
{\bf  Description}

Main method.
}
\item{
{\bf  Parameters}
  \begin{itemize}
   \item{
\texttt{args} -- Arguments.}
  \end{itemize}
}%end item
\end{itemize}
}%end item
\item{ 
\index{start(Stage)}
{\bf  start}\\
\begin{lstlisting}[frame=none]
public abstract void start(javafx.stage.Stage arg0) throws java.lang.Exception\end{lstlisting} %end signature
}%end item
\end{itemize}
}
}
\subsection{\label{gui.GAnsGraphElement}\index{GAnsGraphElement}Class GAnsGraphElement}{
\vskip .1in 
A abstract class to generalize the visual elements that can be displayed in the \texttt{\small GraphView}{\small 
\refdefined{gui.GraphView}}.\vskip .1in 
\subsubsection{Declaration}{
\begin{lstlisting}[frame=none]
public abstract class GAnsGraphElement
 extends javafx.scene.layout.StackPane\end{lstlisting}
\subsubsection{All known subclasses}{VertexShape\small{\refdefined{gui.VertexShape}}, EdgeShape\small{\refdefined{gui.EdgeShape}}}
\subsubsection{Constructors}{
\vskip -2em
\begin{itemize}
\item{ 
\index{GAnsGraphElement()}
{\bf  GAnsGraphElement}\\
\begin{lstlisting}[frame=none]
public GAnsGraphElement()\end{lstlisting} %end signature
}%end item
\end{itemize}
}
\subsubsection{Methods}{
\vskip -2em
\begin{itemize}
\item{ 
\index{getText()}
{\bf  getText}\\
\begin{lstlisting}[frame=none]
public abstract java.lang.String getText()\end{lstlisting} %end signature
\begin{itemize}
\item{
{\bf  Description}

Returns the text shown on the element.
}
\item{{\bf  Returns} -- 
The text that is being displayed on the element. 
}%end item
\end{itemize}
}%end item
\item{ 
\index{setColor(Color)}
{\bf  setColor}\\
\begin{lstlisting}[frame=none]
public abstract void setColor(javafx.scene.paint.Color color)\end{lstlisting} %end signature
\begin{itemize}
\item{
{\bf  Description}

Sets the color of the element.
}
\item{
{\bf  Parameters}
  \begin{itemize}
   \item{
\texttt{color} -- The color the element will be displayed in.}
  \end{itemize}
}%end item
\end{itemize}
}%end item
\item{ 
\index{setText(String)}
{\bf  setText}\\
\begin{lstlisting}[frame=none]
public abstract void setText(java.lang.String text)\end{lstlisting} %end signature
\begin{itemize}
\item{
{\bf  Description}

Sets the text shown on the element.
}
\item{
{\bf  Parameters}
  \begin{itemize}
   \item{
\texttt{text} -- The text that will be displayed on the element.}
  \end{itemize}
}%end item
\end{itemize}
}%end item
\end{itemize}
}
}
\subsection{\label{gui.GraphView}\index{GraphView}Class GraphView}{
\vskip .1in 
A view used for showing and creating a graph in GAns. It supports zooming and other general navigation features.\vskip .1in 
\subsubsection{Declaration}{
\begin{lstlisting}[frame=none]
public class GraphView
 extends javafx.scene.layout.Pane\end{lstlisting}
\subsubsection{Constructors}{
\vskip -2em
\begin{itemize}
\item{ 
\index{GraphView()}
{\bf  GraphView}\\
\begin{lstlisting}[frame=none]
public GraphView()\end{lstlisting} %end signature
\begin{itemize}
\item{
{\bf  Description}

Constructor.
}
\end{itemize}
}%end item
\end{itemize}
}
\subsubsection{Methods}{
\vskip -2em
\begin{itemize}
\item{ 
\index{addGrid()}
{\bf  addGrid}\\
\begin{lstlisting}[frame=none]
public void addGrid()\end{lstlisting} %end signature
\begin{itemize}
\item{
{\bf  Description}

Adds a grid to the GraphView, on which the dragging can be mapped.
}
\end{itemize}
}%end item
\item{ 
\index{getFactory()}
{\bf  getFactory}\\
\begin{lstlisting}[frame=none]
public GraphViewGraphFactory getFactory()\end{lstlisting} %end signature
\begin{itemize}
\item{
{\bf  Description}

Returns the current \texttt{\small GraphViewGraphFactory}{\small 
\refdefined{gui.GraphViewGraphFactory}} from the view.
}
\item{{\bf  Returns} -- 
The current \texttt{\small GraphViewGraphFactory}{\small 
\refdefined{gui.GraphViewGraphFactory}}. 
}%end item
\end{itemize}
}%end item
\item{ 
\index{getScale()}
{\bf  getScale}\\
\begin{lstlisting}[frame=none]
public double getScale()\end{lstlisting} %end signature
\begin{itemize}
\item{
{\bf  Description}

Returns the scale on which the GraphView currently is.
}
\item{{\bf  Returns} -- 
The scale of the GraphView. 
}%end item
\end{itemize}
}%end item
\item{ 
\index{getSelectionModel()}
{\bf  getSelectionModel}\\
\begin{lstlisting}[frame=none]
public GraphViewSelectionModel getSelectionModel()\end{lstlisting} %end signature
\begin{itemize}
\item{
{\bf  Description}

Returns the selection model of the GraphView.
}
\item{{\bf  Returns} -- 
The selection model of the GraphView. 
}%end item
\end{itemize}
}%end item
\item{ 
\index{setGraph(Graph)}
{\bf  setGraph}\\
\begin{lstlisting}[frame=none]
public void setGraph(graphmodel.Graph graph)\end{lstlisting} %end signature
\begin{itemize}
\item{
{\bf  Description}

Sets a graph. Every element in the graph will be generated and then shown.
}
\item{
{\bf  Parameters}
  \begin{itemize}
   \item{
\texttt{graph} -- The graph to be visualized in the view.}
  \end{itemize}
}%end item
\end{itemize}
}%end item
\item{ 
\index{setPivot(double, double)}
{\bf  setPivot}\\
\begin{lstlisting}[frame=none]
public void setPivot(double x,double y)\end{lstlisting} %end signature
\begin{itemize}
\item{
{\bf  Description}

Sets the pivot so the scrolling follows the mouse position on the GraphView.
}
\item{
{\bf  Parameters}
  \begin{itemize}
   \item{
\texttt{x} -- The x coordinate of the pivot.}
   \item{
\texttt{y} -- The y coordinate of the pivot.}
  \end{itemize}
}%end item
\end{itemize}
}%end item
\item{ 
\index{setScale(double)}
{\bf  setScale}\\
\begin{lstlisting}[frame=none]
public void setScale(double scale)\end{lstlisting} %end signature
\begin{itemize}
\item{
{\bf  Description}

Sets the scale of the GraphView.
}
\item{
{\bf  Parameters}
  \begin{itemize}
   \item{
\texttt{scale} -- The scale of the GraphView.}
  \end{itemize}
}%end item
\end{itemize}
}%end item
\item{ 
\index{setSelectionModel(GraphViewSelectionModel)}
{\bf  setSelectionModel}\\
\begin{lstlisting}[frame=none]
public void setSelectionModel(GraphViewSelectionModel selectionModel)\end{lstlisting} %end signature
\begin{itemize}
\item{
{\bf  Description}

Sets the selection model for the GraphView.
}
\item{
{\bf  Parameters}
  \begin{itemize}
   \item{
\texttt{selectionModel} -- The selection model for the GraphView.}
  \end{itemize}
}%end item
\end{itemize}
}%end item
\item{ 
\index{updateGraph()}
{\bf  updateGraph}\\
\begin{lstlisting}[frame=none]
public void updateGraph()\end{lstlisting} %end signature
\begin{itemize}
\item{
{\bf  Description}

Updates the shown graph.
}
\end{itemize}
}%end item
\end{itemize}
}
}
\subsection{\label{gui.GraphViewEventHandler}\index{GraphViewEventHandler}Class GraphViewEventHandler}{
\vskip .1in 
GraphViewEventHandler provides listeners for making the \texttt{\small GraphView}{\small 
\refdefined{gui.GraphView}} draggable and zoomable.\vskip .1in 
\subsubsection{Declaration}{
\begin{lstlisting}[frame=none]
public class GraphViewEventHandler
 extends java.lang.Object\end{lstlisting}
\subsubsection{Constructors}{
\vskip -2em
\begin{itemize}
\item{ 
\index{GraphViewEventHandler(GraphView)}
{\bf  GraphViewEventHandler}\\
\begin{lstlisting}[frame=none]
public GraphViewEventHandler(GraphView canvas)\end{lstlisting} %end signature
\begin{itemize}
\item{
{\bf  Description}

Constructor.
}
\item{
{\bf  Parameters}
  \begin{itemize}
   \item{
\texttt{canvas} -- The \texttt{\small GraphView}{\small 
\refdefined{gui.GraphView}} which later on will get the EventHandler set.}
  \end{itemize}
}%end item
\end{itemize}
}%end item
\end{itemize}
}
\subsubsection{Methods}{
\vskip -2em
\begin{itemize}
\item{ 
\index{getOnMouseDraggedEventHandler()}
{\bf  getOnMouseDraggedEventHandler}\\
\begin{lstlisting}[frame=none]
public javafx.event.EventHandler getOnMouseDraggedEventHandler()\end{lstlisting} %end signature
\begin{itemize}
\item{
{\bf  Description}

Returns an EventHandler which handles dragging inside the \texttt{\small GraphView}{\small 
\refdefined{gui.GraphView}}.
}
\item{{\bf  Returns} -- 
An EventHandler which handles dragging. 
}%end item
\end{itemize}
}%end item
\item{ 
\index{getOnMousePressedEventHandler()}
{\bf  getOnMousePressedEventHandler}\\
\begin{lstlisting}[frame=none]
public javafx.event.EventHandler getOnMousePressedEventHandler()\end{lstlisting} %end signature
\begin{itemize}
\item{
{\bf  Description}

Returns an EventHandler which handles pressing the mouse inside the \texttt{\small GraphView}{\small 
\refdefined{gui.GraphView}}.
}
\item{{\bf  Returns} -- 
An EventHandler which handles pressing the mouse. 
}%end item
\end{itemize}
}%end item
\item{ 
\index{getOnScrollEventHandler()}
{\bf  getOnScrollEventHandler}\\
\begin{lstlisting}[frame=none]
public javafx.event.EventHandler getOnScrollEventHandler()\end{lstlisting} %end signature
\begin{itemize}
\item{
{\bf  Description}

Returns an EventHandler which maps scrolling the mousewheel to zooming on the \texttt{\small GraphView}{\small 
\refdefined{gui.GraphView}}.
}
\item{{\bf  Returns} -- 
An EventHandler which maps scrolling the mousewheel to zooming. 
}%end item
\end{itemize}
}%end item
\end{itemize}
}
}
\subsection{\label{gui.GraphViewGraphFactory}\index{GraphViewGraphFactory}Class GraphViewGraphFactory}{
\vskip .1in 
The GraphViewGraphFactory generates the visual representation of a given \texttt{\small Graph}{\small 
\refdefined{graphmodel.Graph}} and gives access to the set \texttt{\small Graph}{\small 
\refdefined{graphmodel.Graph}}.\vskip .1in 
\subsubsection{Declaration}{
\begin{lstlisting}[frame=none]
public class GraphViewGraphFactory
 extends java.lang.Object\end{lstlisting}
\subsubsection{Constructors}{
\vskip -2em
\begin{itemize}
\item{ 
\index{GraphViewGraphFactory(Graph)}
{\bf  GraphViewGraphFactory}\\
\begin{lstlisting}[frame=none]
public GraphViewGraphFactory(graphmodel.Graph graph)\end{lstlisting} %end signature
\begin{itemize}
\item{
{\bf  Description}

Constructor. Sets the graph and generates the vertices and edges for visualization.
}
\item{
{\bf  Parameters}
  \begin{itemize}
   \item{
\texttt{graph} -- The graph data that will be shown.}
  \end{itemize}
}%end item
\end{itemize}
}%end item
\end{itemize}
}
\subsubsection{Methods}{
\vskip -2em
\begin{itemize}
\item{ 
\index{getEdgeFromShape(GAnsGraphElement)}
{\bf  getEdgeFromShape}\\
\begin{lstlisting}[frame=none]
public graphmodel.Edge getEdgeFromShape(GAnsGraphElement shape)\end{lstlisting} %end signature
\begin{itemize}
\item{
{\bf  Description}

Returns the edge element from the graph model that is being represented by the shape. Can be null if an \texttt{\small VertexShape}{\small 
\refdefined{gui.VertexShape}} is passed.
}
\item{
{\bf  Parameters}
  \begin{itemize}
   \item{
\texttt{shape} -- The shape that represents the edge.}
  \end{itemize}
}%end item
\item{{\bf  Returns} -- 
The Edge being represented by the passed shape. 
}%end item
\end{itemize}
}%end item
\item{ 
\index{getGraphicalElements()}
{\bf  getGraphicalElements}\\
\begin{lstlisting}[frame=none]
public java.util.LinkedList getGraphicalElements()\end{lstlisting} %end signature
\begin{itemize}
\item{
{\bf  Description}

Returns all graphical elements that have been generated by the factory.
}
\item{{\bf  Returns} -- 
All graphical elements generated by the factory. 
}%end item
\end{itemize}
}%end item
\item{ 
\index{getSizeOfVertex(String)}
{\bf  getSizeOfVertex}\\
\begin{lstlisting}[frame=none]
public static javafx.util.Pair getSizeOfVertex(java.lang.String text)\end{lstlisting} %end signature
\begin{itemize}
\item{
{\bf  Description}

Calculates and returns the size of a vertex with the given text.
}
\item{
{\bf  Parameters}
  \begin{itemize}
   \item{
\texttt{text} -- The text which size the vertex depends on.}
  \end{itemize}
}%end item
\item{{\bf  Returns} -- 
A Pair of width and height of the vertex. 
}%end item
\end{itemize}
}%end item
\item{ 
\index{getVertexFromShape(GAnsGraphElement)}
{\bf  getVertexFromShape}\\
\begin{lstlisting}[frame=none]
public graphmodel.Vertex getVertexFromShape(GAnsGraphElement shape)\end{lstlisting} %end signature
\begin{itemize}
\item{
{\bf  Description}

Returns the vertex element from the graph model that is being represented by the shape. Can be null if an \texttt{\small EdgeShape}{\small 
\refdefined{gui.EdgeShape}} is passed.
}
\item{
{\bf  Parameters}
  \begin{itemize}
   \item{
\texttt{shape} -- The shape that represents the vertex.}
  \end{itemize}
}%end item
\item{{\bf  Returns} -- 
The Vertex being represented by the passed shape. 
}%end item
\end{itemize}
}%end item
\end{itemize}
}
}
\subsection{\label{gui.GraphViewSelectionModel}\index{GraphViewSelectionModel}Class GraphViewSelectionModel}{
\vskip .1in 
The selection model for the \texttt{\small GraphView}{\small 
\refdefined{gui.GraphView}}, that supports multiple selection of vertices and edges.\vskip .1in 
\subsubsection{Declaration}{
\begin{lstlisting}[frame=none]
public class GraphViewSelectionModel
 extends javafx.scene.control.MultipleSelectionModel\end{lstlisting}
\subsubsection{Constructors}{
\vskip -2em
\begin{itemize}
\item{ 
\index{GraphViewSelectionModel()}
{\bf  GraphViewSelectionModel}\\
\begin{lstlisting}[frame=none]
public GraphViewSelectionModel()\end{lstlisting} %end signature
}%end item
\end{itemize}
}
\subsubsection{Methods}{
\vskip -2em
\begin{itemize}
\item{ 
\index{clearAndSelect(int)}
{\bf  clearAndSelect}\\
\begin{lstlisting}[frame=none]
public abstract void clearAndSelect(int arg0)\end{lstlisting} %end signature
}%end item
\item{ 
\index{clearSelection()}
{\bf  clearSelection}\\
\begin{lstlisting}[frame=none]
public abstract void clearSelection()\end{lstlisting} %end signature
}%end item
\item{ 
\index{clearSelection(int)}
{\bf  clearSelection}\\
\begin{lstlisting}[frame=none]
public abstract void clearSelection(int arg0)\end{lstlisting} %end signature
}%end item
\item{ 
\index{getSelectedIndices()}
{\bf  getSelectedIndices}\\
\begin{lstlisting}[frame=none]
public abstract javafx.collections.ObservableList getSelectedIndices()\end{lstlisting} %end signature
}%end item
\item{ 
\index{getSelectedItems()}
{\bf  getSelectedItems}\\
\begin{lstlisting}[frame=none]
public abstract javafx.collections.ObservableList getSelectedItems()\end{lstlisting} %end signature
}%end item
\item{ 
\index{getSelectedItemsProperties()}
{\bf  getSelectedItemsProperties}\\
\begin{lstlisting}[frame=none]
public javafx.collections.ObservableList getSelectedItemsProperties()\end{lstlisting} %end signature
\begin{itemize}
\item{
{\bf  Description}

Returns the \texttt{\small GAnsProperty}{\small 
\refdefined{objectproperty.GAnsProperty}} of all selected items.
}
\item{{\bf  Returns} -- 
A list with all the \texttt{\small GAnsProperty}{\small 
\refdefined{objectproperty.GAnsProperty}} of all selected items. 
}%end item
\end{itemize}
}%end item
\item{ 
\index{isEmpty()}
{\bf  isEmpty}\\
\begin{lstlisting}[frame=none]
public abstract boolean isEmpty()\end{lstlisting} %end signature
}%end item
\item{ 
\index{isSelected(int)}
{\bf  isSelected}\\
\begin{lstlisting}[frame=none]
public abstract boolean isSelected(int arg0)\end{lstlisting} %end signature
}%end item
\item{ 
\index{select(GAnsGraphElement)}
{\bf  select}\\
\begin{lstlisting}[frame=none]
public void select(GAnsGraphElement obj)\end{lstlisting} %end signature
}%end item
\item{ 
\index{select(int)}
{\bf  select}\\
\begin{lstlisting}[frame=none]
public abstract void select(int arg0)\end{lstlisting} %end signature
}%end item
\item{ 
\index{selectAll()}
{\bf  selectAll}\\
\begin{lstlisting}[frame=none]
public abstract void selectAll()\end{lstlisting} %end signature
}%end item
\item{ 
\index{selectFirst()}
{\bf  selectFirst}\\
\begin{lstlisting}[frame=none]
public abstract void selectFirst()\end{lstlisting} %end signature
}%end item
\item{ 
\index{selectIndices(int, int\lbrack \rbrack )}
{\bf  selectIndices}\\
\begin{lstlisting}[frame=none]
public abstract void selectIndices(int arg0,int[] arg1)\end{lstlisting} %end signature
}%end item
\item{ 
\index{selectLast()}
{\bf  selectLast}\\
\begin{lstlisting}[frame=none]
public abstract void selectLast()\end{lstlisting} %end signature
}%end item
\item{ 
\index{selectNext()}
{\bf  selectNext}\\
\begin{lstlisting}[frame=none]
public abstract void selectNext()\end{lstlisting} %end signature
}%end item
\item{ 
\index{selectPrevious()}
{\bf  selectPrevious}\\
\begin{lstlisting}[frame=none]
public abstract void selectPrevious()\end{lstlisting} %end signature
}%end item
\end{itemize}
}
}
\subsection{\label{gui.InformationView}\index{InformationView}Class InformationView}{
\vskip .1in 
The InformationView shows a given set of properties from the selected vertices in the \texttt{\small GraphView}{\small 
\refdefined{gui.GraphView}}.\vskip .1in 
\subsubsection{Declaration}{
\begin{lstlisting}[frame=none]
public class InformationView
 extends javafx.scene.control.TableView\end{lstlisting}
\subsubsection{Constructors}{
\vskip -2em
\begin{itemize}
\item{ 
\index{InformationView()}
{\bf  InformationView}\\
\begin{lstlisting}[frame=none]
public InformationView()\end{lstlisting} %end signature
}%end item
\end{itemize}
}
\subsubsection{Methods}{
\vskip -2em
\begin{itemize}
\item{ 
\index{setInformations(ObservableList)}
{\bf  setInformations}\\
\begin{lstlisting}[frame=none]
public void setInformations(javafx.collections.ObservableList informations)\end{lstlisting} %end signature
\begin{itemize}
\item{
{\bf  Description}

Sets the properties which should be shown in the InformationView. The \texttt{\small GAnsProperty}{\small 
\refdefined{objectproperty.GAnsProperty}} are being processed in an internal factory, which automatically creates the tablecells. The function will be called whenever the selection of the \texttt{\small GraphView}{\small 
\refdefined{gui.GraphView}} changes.
}
\item{
{\bf  Parameters}
  \begin{itemize}
   \item{
\texttt{informations} -- List with \texttt{\small GAnsProperty}{\small 
\refdefined{objectproperty.GAnsProperty}} elements which define the content of the InformationView}
  \end{itemize}
}%end item
\end{itemize}
}%end item
\end{itemize}
}
}
\subsection{\label{gui.ParameterDialogGenerator}\index{ParameterDialogGenerator}Class ParameterDialogGenerator}{
\vskip .1in 
Generates a parameter dialog given a parent node and a set of parameters.\vskip .1in 
\subsubsection{Declaration}{
\begin{lstlisting}[frame=none]
public class ParameterDialogGenerator
 extends parameter.ParameterVisitor\end{lstlisting}
\subsubsection{Constructors}{
\vskip -2em
\begin{itemize}
\item{ 
\index{ParameterDialogGenerator(GridPane, Settings)}
{\bf  ParameterDialogGenerator}\\
\begin{lstlisting}[frame=none]
public ParameterDialogGenerator(javafx.scene.layout.GridPane parent,parameter.Settings settings)\end{lstlisting} %end signature
\begin{itemize}
\item{
{\bf  Description}

Constructs a new ParameterDialogGenerator and sets the parent, where all parameter GUI-Elements are placed in afterwards.
}
\end{itemize}
}%end item
\end{itemize}
}
\subsubsection{Methods}{
\vskip -2em
\begin{itemize}
\item{ 
\index{visit(IntegerParameter)}
{\bf  visit}\\
\begin{lstlisting}[frame=none]
public abstract void visit(parameter.IntegerParameter parameter)\end{lstlisting} %end signature
\begin{itemize}
\item{
{\bf  Description copied from parameter.ParameterVisitor{\small \refdefined{parameter.ParameterVisitor}} }

Visits the specified parameter and performs some by the subclass chosen actions on it.
}
\item{
{\bf  Parameters}
  \begin{itemize}
   \item{
\texttt{parameter} -- The parameter to visit}
  \end{itemize}
}%end item
\end{itemize}
}%end item
\item{ 
\index{visit(MultipleChoiceParameter)}
{\bf  visit}\\
\begin{lstlisting}[frame=none]
public abstract void visit(parameter.MultipleChoiceParameter parameter)\end{lstlisting} %end signature
\begin{itemize}
\item{
{\bf  Description copied from parameter.ParameterVisitor{\small \refdefined{parameter.ParameterVisitor}} }

Visits the specified parameter and performs some by the subclass chosen actions on it.
}
\item{
{\bf  Parameters}
  \begin{itemize}
   \item{
\texttt{parameter} -- The parameter to visit}
  \end{itemize}
}%end item
\end{itemize}
}%end item
\item{ 
\index{visit(StringParameter)}
{\bf  visit}\\
\begin{lstlisting}[frame=none]
public abstract void visit(parameter.StringParameter parameter)\end{lstlisting} %end signature
\begin{itemize}
\item{
{\bf  Description copied from parameter.ParameterVisitor{\small \refdefined{parameter.ParameterVisitor}} }

Visits the specified parameter and performs some by the subclass chosen actions on it.
}
\item{
{\bf  Parameters}
  \begin{itemize}
   \item{
\texttt{parameter} -- The parameter to visit}
  \end{itemize}
}%end item
\end{itemize}
}%end item
\end{itemize}
}
}
\subsection{\label{gui.StructureView}\index{StructureView}Class StructureView}{
\vskip .1in 
The StructureView regulates the access and representation of the elements in the StructureView of GAns.\vskip .1in 
\subsubsection{Declaration}{
\begin{lstlisting}[frame=none]
public class StructureView
 extends javafx.scene.control.TreeView\end{lstlisting}
\subsubsection{Constructors}{
\vskip -2em
\begin{itemize}
\item{ 
\index{StructureView()}
{\bf  StructureView}\\
\begin{lstlisting}[frame=none]
public StructureView()\end{lstlisting} %end signature
\begin{itemize}
\item{
{\bf  Description}

Constructor.
}
\end{itemize}
}%end item
\end{itemize}
}
\subsubsection{Methods}{
\vskip -2em
\begin{itemize}
\item{ 
\index{getIdOfSelectedItem()}
{\bf  getIdOfSelectedItem}\\
\begin{lstlisting}[frame=none]
public java.lang.String getIdOfSelectedItem()\end{lstlisting} %end signature
\begin{itemize}
\item{
{\bf  Description}

Returns the id of the selected graph.
}
\item{{\bf  Returns} -- 
The id of the selected graph. 
}%end item
\end{itemize}
}%end item
\item{ 
\index{showTree(Graph)}
{\bf  showTree}\\
\begin{lstlisting}[frame=none]
public void showTree(graphmodel.Graph graph)\end{lstlisting} %end signature
\begin{itemize}
\item{
{\bf  Description}

Creates a tree like representation from a given graph and its subgraphs. Should be called, before calling other methods, because there could be a dummy root-node in the View.
}
\item{
{\bf  Parameters}
  \begin{itemize}
   \item{
\texttt{graph} -- The graph which should be represented.}
  \end{itemize}
}%end item
\end{itemize}
}%end item
\end{itemize}
}
}
\subsection{\label{gui.VertexShape}\index{VertexShape}Class VertexShape}{
\vskip .1in 
A visual representation of a vertex with a text inside of it.\vskip .1in 
\subsubsection{Declaration}{
\begin{lstlisting}[frame=none]
public class VertexShape
 extends gui.GAnsGraphElement\end{lstlisting}
\subsubsection{Constructors}{
\vskip -2em
\begin{itemize}
\item{ 
\index{VertexShape()}
{\bf  VertexShape}\\
\begin{lstlisting}[frame=none]
public VertexShape()\end{lstlisting} %end signature
\begin{itemize}
\item{
{\bf  Description}

Constructor
}
\end{itemize}
}%end item
\item{ 
\index{VertexShape(String)}
{\bf  VertexShape}\\
\begin{lstlisting}[frame=none]
public VertexShape(java.lang.String text)\end{lstlisting} %end signature
\begin{itemize}
\item{
{\bf  Description}

Constructor which directly sets the text.
}
\item{
{\bf  Parameters}
  \begin{itemize}
   \item{
\texttt{text} -- The text that will be displayed in the vertex.}
  \end{itemize}
}%end item
\end{itemize}
}%end item
\end{itemize}
}
\subsubsection{Methods}{
\vskip -2em
\begin{itemize}
\item{ 
\index{getText()}
{\bf  getText}\\
\begin{lstlisting}[frame=none]
public abstract java.lang.String getText()\end{lstlisting} %end signature
\begin{itemize}
\item{
{\bf  Description copied from GAnsGraphElement{\small \refdefined{gui.GAnsGraphElement}} }

Returns the text shown on the element.
}
\item{{\bf  Returns} -- 
The text that is being displayed on the element. 
}%end item
\end{itemize}
}%end item
\item{ 
\index{setColor(Color)}
{\bf  setColor}\\
\begin{lstlisting}[frame=none]
public abstract void setColor(javafx.scene.paint.Color color)\end{lstlisting} %end signature
\begin{itemize}
\item{
{\bf  Description copied from GAnsGraphElement{\small \refdefined{gui.GAnsGraphElement}} }

Sets the color of the element.
}
\item{
{\bf  Parameters}
  \begin{itemize}
   \item{
\texttt{color} -- The color the element will be displayed in.}
  \end{itemize}
}%end item
\end{itemize}
}%end item
\item{ 
\index{setText(String)}
{\bf  setText}\\
\begin{lstlisting}[frame=none]
public abstract void setText(java.lang.String text)\end{lstlisting} %end signature
\begin{itemize}
\item{
{\bf  Description copied from GAnsGraphElement{\small \refdefined{gui.GAnsGraphElement}} }

Sets the text shown on the element.
}
\item{
{\bf  Parameters}
  \begin{itemize}
   \item{
\texttt{text} -- The text that will be displayed on the element.}
  \end{itemize}
}%end item
\end{itemize}
}%end item
\end{itemize}
}
}
}
\printindex
