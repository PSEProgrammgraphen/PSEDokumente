\section{Package joana}{
\label{joana}\hskip -.05in
\hbox to \hsize{\textit{ Package Contents\hfil Page}}
\vskip .13in
\hbox{{\bf  Classes}}
\entityintro{CallGraph}{joana.CallGraph}{This is a specified graph representation for the Callgraph in Joana.}
\entityintro{CallGraphBuilder}{joana.CallGraphBuilder}{The CallGraphBuilder implements an \texttt{\small IGraphBuilder}{\small 
\refdefined{graphmodel.IGraphBuilder}} and builds one \texttt{\small CallGraph}{\small 
\refdefined{joana.CallGraph}}.}
\entityintro{CallGraphLayout}{joana.CallGraphLayout}{}
\entityintro{CallGraphLayoutOption}{joana.CallGraphLayoutOption}{A \texttt{\small LayoutOption}{\small 
\refdefined{plugin.LayoutOption}} which is specific for \texttt{\small CallGraph}{\small 
\refdefined{joana.CallGraph}}.}
\entityintro{FieldAccess}{joana.FieldAccess}{This specifies the vertex representation of FieldAccesses in a MethodGraph It contains a \texttt{\small FieldAccessGraph}.}
\entityintro{FieldAccessGraph}{joana.FieldAccessGraph}{A \texttt{\small JoanaGraph}{\small 
\refdefined{joana.JoanaGraph}} which specifies a \texttt{\small FieldAccess}{\small 
\refdefined{joana.FieldAccess}} in a \texttt{\small JoanaGraph}{\small 
\refdefined{joana.JoanaGraph}}}
\entityintro{FieldAccessLayout}{joana.FieldAccessLayout}{The FieldAccessLayout applies its layout to a \texttt{\small FieldAccess}{\small 
\refdefined{joana.FieldAccess}}.}
\entityintro{JoanaEdge}{joana.JoanaEdge}{A Joana specific Edge.}
\entityintro{JoanaEdgeBuilder}{joana.JoanaEdgeBuilder}{The JoanaEdgeBuilder is a \texttt{\small IEdgeBuilder}{\small 
\refdefined{graphmodel.IEdgeBuilder}}, specifically for building \texttt{\small JoanaEdge}{\small 
\refdefined{joana.JoanaEdge}}.}
\entityintro{JoanaGraph}{joana.JoanaGraph}{An abstract superclass for all JOANA specific graphs.}
\entityintro{JoanaGraphModel}{joana.JoanaGraphModel}{A Joana specific \texttt{\small GraphModel}{\small 
\refdefined{graphmodel.GraphModel}}.}
\entityintro{JoanaGraphModelBuilder}{joana.JoanaGraphModelBuilder}{The JoanaGraphModelBuilder implements the \texttt{\small IGraphModelBuilder}{\small 
\refdefined{graphmodel.IGraphModelBuilder}} and creates a \texttt{\small JoanaGraphModel}{\small 
\refdefined{joana.JoanaGraphModel}}.}
\entityintro{JoanaPlugin}{joana.JoanaPlugin}{A plugin for GAns that supports the creation and visualization of Joana system dependence graphs.}
\entityintro{JoanaPlugin.CallGraphLayoutRegister}{joana.JoanaPlugin.CallGraphLayoutRegister}{}
\entityintro{JoanaPlugin.MethodGraphLayoutRegister}{joana.JoanaPlugin.MethodGraphLayoutRegister}{}
\entityintro{JoanaVertex}{joana.JoanaVertex}{A Joana specific Vertex.}
\entityintro{JoanaVertexBuilder}{joana.JoanaVertexBuilder}{The JoanaVertexBuilder implements an \texttt{\small IVertexBuilder}{\small 
\refdefined{graphmodel.IVertexBuilder}} and creates a \texttt{\small JoanaVertex}{\small 
\refdefined{joana.JoanaVertex}}.}
\entityintro{JoanaWorkspace}{joana.JoanaWorkspace}{The \texttt{\small JoanaWorkspace}{\small 
\refdefined{joana.JoanaWorkspace}} is the workspace for Joana graphs.}
\entityintro{MethodGraph}{joana.MethodGraph}{This is a specific graph representation for a MethodGraph in JOANA}
\entityintro{MethodGraphBuilder}{joana.MethodGraphBuilder}{The MethodGraphBuilder is a \texttt{\small IGraphBuilder}{\small 
\refdefined{graphmodel.IGraphBuilder}}, specifically for building \texttt{\small MethodGraph}{\small 
\refdefined{joana.MethodGraph}}.}
\entityintro{MethodGraphLayout}{joana.MethodGraphLayout}{Implements hierarchical layout with layers for \texttt{\small MethodGraph}{\small 
\refdefined{joana.MethodGraph}}.}
\entityintro{MethodGraphLayoutOption}{joana.MethodGraphLayoutOption}{A \texttt{\small LayoutOption}{\small 
\refdefined{plugin.LayoutOption}} which is specific for \texttt{\small MethodGraph}{\small 
\refdefined{joana.MethodGraph}}.}
\vskip .1in
\vskip .1in
\subsection{\label{joana.CallGraph}\index{CallGraph}Class CallGraph}{
\vskip .1in 
This is a specified graph representation for the Callgraph in Joana.\vskip .1in 
\subsubsection{Declaration}{
\begin{lstlisting}[frame=none]
public class CallGraph
 extends joana.JoanaGraph\end{lstlisting}
\subsubsection{Constructors}{
\vskip -2em
\begin{itemize}
\item{ 
\index{CallGraph()}
{\bf  CallGraph}\\
\begin{lstlisting}[frame=none]
public CallGraph()\end{lstlisting} %end signature
}%end item
\end{itemize}
}
\subsubsection{Methods}{
\vskip -2em
\begin{itemize}
\item{ 
\index{collapse(Set)}
{\bf  collapse}\\
\begin{lstlisting}[frame=none]
graphmodel.CompoundVertex collapse(java.util.Set subset)\end{lstlisting} %end signature
\begin{itemize}
\item{
{\bf  Description copied from graphmodel.Viewable{\small \refdefined{graphmodel.Viewable}} }

Collapses a set of vertices in one compound vertex. The collapsed vertices can be expanded back into their previous state with \texttt{\small expand(CompoundVertex)}.
}
\item{
{\bf  Parameters}
  \begin{itemize}
   \item{
\texttt{subset} -- the subset to collapse}
  \end{itemize}
}%end item
\item{{\bf  Returns} -- 
the resulting collapsed vertex 
}%end item
\end{itemize}
}%end item
\item{ 
\index{expand(CompoundVertex)}
{\bf  expand}\\
\begin{lstlisting}[frame=none]
java.util.Set expand(graphmodel.CompoundVertex vertex)\end{lstlisting} %end signature
\begin{itemize}
\item{
{\bf  Description copied from graphmodel.Viewable{\small \refdefined{graphmodel.Viewable}} }

Expands a collapsed vertex into its substituted set of vertices The vertices will be added back to the set of vertices of this graph. The compound vertex will be removed from the set of vertices. All to the compound vertex incident edges, will be resolved back into an edge between the vertices it connected before the collapse.
}
\item{
{\bf  Parameters}
  \begin{itemize}
   \item{
\texttt{vertex} -- the collapsed vertex to expand}
  \end{itemize}
}%end item
\item{{\bf  Returns} -- 
the set of vertices which was substituted by the collapsed vertex 
}%end item
\end{itemize}
}%end item
\item{ 
\index{getLayerWidth(int)}
{\bf  getLayerWidth}\\
\begin{lstlisting}[frame=none]
public int getLayerWidth(int layerN)\end{lstlisting} %end signature
}%end item
\item{ 
\index{getSubgraphs()}
{\bf  getSubgraphs}\\
\begin{lstlisting}[frame=none]
public java.util.List getSubgraphs()\end{lstlisting} %end signature
}%end item
\item{ 
\index{isCompound(Vertex)}
{\bf  isCompound}\\
\begin{lstlisting}[frame=none]
boolean isCompound(graphmodel.Vertex vertex)\end{lstlisting} %end signature
\begin{itemize}
\item{
{\bf  Description copied from graphmodel.Viewable{\small \refdefined{graphmodel.Viewable}} }

Returns true if the specified vertex is a compound vertex
}
\item{
{\bf  Parameters}
  \begin{itemize}
   \item{
\texttt{vertex} -- the vertex to check}
  \end{itemize}
}%end item
\item{{\bf  Returns} -- 
true if the vertex is a compound, false otherwise 
}%end item
\end{itemize}
}%end item
\end{itemize}
}
}
\subsection{\label{joana.CallGraphBuilder}\index{CallGraphBuilder}Class CallGraphBuilder}{
\vskip .1in 
The CallGraphBuilder implements an \texttt{\small IGraphBuilder}{\small 
\refdefined{graphmodel.IGraphBuilder}} and builds one \texttt{\small CallGraph}{\small 
\refdefined{joana.CallGraph}}.\vskip .1in 
\subsubsection{Declaration}{
\begin{lstlisting}[frame=none]
public class CallGraphBuilder
 extends java.lang.Object implements graphmodel.IGraphBuilder\end{lstlisting}
\subsubsection{Constructors}{
\vskip -2em
\begin{itemize}
\item{ 
\index{CallGraphBuilder()}
{\bf  CallGraphBuilder}\\
\begin{lstlisting}[frame=none]
public CallGraphBuilder()\end{lstlisting} %end signature
}%end item
\end{itemize}
}
\subsubsection{Methods}{
\vskip -2em
\begin{itemize}
\item{ 
\index{build()}
{\bf  build}\\
\begin{lstlisting}[frame=none]
graphmodel.Graph build()\end{lstlisting} %end signature
\begin{itemize}
\item{
{\bf  Description copied from graphmodel.IGraphBuilder{\small \refdefined{graphmodel.IGraphBuilder}} }

Builds a graph from the given settings and returns it.
}
\item{{\bf  Returns} -- 
The graph that is being build by the IGraphBuilder. 
}%end item
\end{itemize}
}%end item
\item{ 
\index{getEdgeBuilder()}
{\bf  getEdgeBuilder}\\
\begin{lstlisting}[frame=none]
graphmodel.IEdgeBuilder getEdgeBuilder()\end{lstlisting} %end signature
\begin{itemize}
\item{
{\bf  Description copied from graphmodel.IGraphBuilder{\small \refdefined{graphmodel.IGraphBuilder}} }

Returns the EdgeBuilder which is specified for this graph.
}
\item{{\bf  Returns} -- 
The \texttt{\small IEdgeBuilder}{\small 
\refdefined{graphmodel.IEdgeBuilder}} which is specified for this graph. 
}%end item
\end{itemize}
}%end item
\item{ 
\index{getVertexBuilder(String)}
{\bf  getVertexBuilder}\\
\begin{lstlisting}[frame=none]
graphmodel.IVertexBuilder getVertexBuilder(java.lang.String vertexID)\end{lstlisting} %end signature
\begin{itemize}
\item{
{\bf  Description copied from graphmodel.IGraphBuilder{\small \refdefined{graphmodel.IGraphBuilder}} }

Returns the VertexBuilder which is specified for this graph.
}
\item{
{\bf  Parameters}
  \begin{itemize}
   \item{
\texttt{vertexID} -- The id of the vertex which associated IVertexBuilder will be returned.}
  \end{itemize}
}%end item
\item{{\bf  Returns} -- 
The \texttt{\small IVertexBuilder}{\small 
\refdefined{graphmodel.IVertexBuilder}} which is specified for this graph. 
}%end item
\end{itemize}
}%end item
\end{itemize}
}
}
\subsection{\label{joana.CallGraphLayout}\index{CallGraphLayout}Class CallGraphLayout}{
\vskip .1in 
\subsubsection{Declaration}{
\begin{lstlisting}[frame=none]
public class CallGraphLayout
 extends java.lang.Object implements sugiyama.LayeredLayoutAlgorithm\end{lstlisting}
\subsubsection{Constructors}{
\vskip -2em
\begin{itemize}
\item{ 
\index{CallGraphLayout()}
{\bf  CallGraphLayout}\\
\begin{lstlisting}[frame=none]
public CallGraphLayout()\end{lstlisting} %end signature
}%end item
\end{itemize}
}
\subsubsection{Methods}{
\vskip -2em
\begin{itemize}
\item{ 
\index{getSettings()}
{\bf  getSettings}\\
\begin{lstlisting}[frame=none]
public parameter.Settings getSettings()\end{lstlisting} %end signature
}%end item
\item{ 
\index{layout(CallGraph)}
{\bf  layout}\\
\begin{lstlisting}[frame=none]
public void layout(CallGraph graph)\end{lstlisting} %end signature
}%end item
\item{ 
\index{layoutLayeredGraph(LayeredGraph)}
{\bf  layoutLayeredGraph}\\
\begin{lstlisting}[frame=none]
void layoutLayeredGraph(graphmodel.LayeredGraph graph)\end{lstlisting} %end signature
\begin{itemize}
\item{
{\bf  Description copied from sugiyama.LayeredLayoutAlgorithm{\small \refdefined{sugiyama.LayeredLayoutAlgorithm}} }

Applies its layout to a graph as in \texttt{\small layout(G graph)} but keeps the notion of layers. The algorithm will assign every vertex a coordinate and every edge a path. Additionally every vertex will be assigned a position in a layer in the LayeredGraph. A possible application is drawing of recursive graphs.
}
\item{
{\bf  Parameters}
  \begin{itemize}
   \item{
\texttt{graph} -- the graph to apply the layout to}
  \end{itemize}
}%end item
\end{itemize}
}%end item
\end{itemize}
}
}
\subsection{\label{joana.CallGraphLayoutOption}\index{CallGraphLayoutOption}Class CallGraphLayoutOption}{
\vskip .1in 
A \texttt{\small LayoutOption}{\small 
\refdefined{plugin.LayoutOption}} which is specific for \texttt{\small CallGraph}{\small 
\refdefined{joana.CallGraph}}.\vskip .1in 
\subsubsection{Declaration}{
\begin{lstlisting}[frame=none]
public abstract class CallGraphLayoutOption
 extends plugin.LayoutOption\end{lstlisting}
\subsubsection{Constructors}{
\vskip -2em
\begin{itemize}
\item{ 
\index{CallGraphLayoutOption()}
{\bf  CallGraphLayoutOption}\\
\begin{lstlisting}[frame=none]
public CallGraphLayoutOption()\end{lstlisting} %end signature
}%end item
\end{itemize}
}
\subsubsection{Methods}{
\vskip -2em
\begin{itemize}
\item{ 
\index{applyLayout()}
{\bf  applyLayout}\\
\begin{lstlisting}[frame=none]
public abstract void applyLayout()\end{lstlisting} %end signature
\begin{itemize}
\item{
{\bf  Description copied from plugin.LayoutOption{\small \refdefined{plugin.LayoutOption}} }

This should execute the layout on the graph, which should be specified on construction, or in beforehand. The settings, which are accessible over \texttt{\small getSettings()} will be used to instantiate the LayoutAlgorithm.
}
\end{itemize}
}%end item
\item{ 
\index{chooseLayout()}
{\bf  chooseLayout}\\
\begin{lstlisting}[frame=none]
public abstract void chooseLayout()\end{lstlisting} %end signature
\begin{itemize}
\item{
{\bf  Description copied from plugin.LayoutOption{\small \refdefined{plugin.LayoutOption}} }

Called when this layout option is chosen. This allows the layout option to prepare the actual LayoutAlgorithm.
}
\end{itemize}
}%end item
\item{ 
\index{setGraph(CallGraph)}
{\bf  setGraph}\\
\begin{lstlisting}[frame=none]
public void setGraph(CallGraph graph)\end{lstlisting} %end signature
\begin{itemize}
\item{
{\bf  Description}

Sets the \texttt{\small CallGraph}{\small 
\refdefined{joana.CallGraph}} that will be the target of the CallGraphLayoutOption.
}
\item{
{\bf  Parameters}
  \begin{itemize}
   \item{
\texttt{graph} -- The \texttt{\small CallGraph}{\small 
\refdefined{joana.CallGraph}} that will be the target of this CallGraphLayoutOption.}
  \end{itemize}
}%end item
\end{itemize}
}%end item
\item{ 
\index{setLayout(LayoutAlgorithm)}
{\bf  setLayout}\\
\begin{lstlisting}[frame=none]
public void setLayout(plugin.LayoutAlgorithm layout)\end{lstlisting} %end signature
\begin{itemize}
\item{
{\bf  Description}

Sets the LayoutAlgorithm that will be used to layout the set graph.
}
\item{
{\bf  Parameters}
  \begin{itemize}
   \item{
\texttt{layout} -- The LayoutAlgorithm that will be used to layout the set graph.}
  \end{itemize}
}%end item
\end{itemize}
}%end item
\end{itemize}
}
}
\subsection{\label{joana.FieldAccess}\index{FieldAccess}Class FieldAccess}{
\vskip .1in 
This specifies the vertex representation of FieldAccesses in a MethodGraph It contains a \texttt{\small FieldAccessGraph}.\vskip .1in 
\subsubsection{Declaration}{
\begin{lstlisting}[frame=none]
public class FieldAccess
 extends joana.JoanaVertex implements graphmodel.CompoundVertex\end{lstlisting}
\subsubsection{Constructors}{
\vskip -2em
\begin{itemize}
\item{ 
\index{FieldAccess(FieldAccessGraph)}
{\bf  FieldAccess}\\
\begin{lstlisting}[frame=none]
public FieldAccess(FieldAccessGraph graph)\end{lstlisting} %end signature
\begin{itemize}
\item{
{\bf  Description}

Constructor.
}
\item{
{\bf  Parameters}
  \begin{itemize}
   \item{
\texttt{graph} -- The FieldAccessGraph that will be set in the FieldAccess.}
  \end{itemize}
}%end item
\end{itemize}
}%end item
\end{itemize}
}
\subsubsection{Methods}{
\vskip -2em
\begin{itemize}
\item{ 
\index{addToFastGraphAccessor(FastGraphAccessor)}
{\bf  addToFastGraphAccessor}\\
\begin{lstlisting}[frame=none]
void addToFastGraphAccessor(graphmodel.FastGraphAccessor fga)\end{lstlisting} %end signature
\begin{itemize}
\item{
{\bf  Description copied from graphmodel.Vertex{\small \refdefined{graphmodel.Vertex}} }

Adds the vertex to a \texttt{\small FastGraphAccessor}{\small 
\refdefined{graphmodel.FastGraphAccessor}}.
}
\item{
{\bf  Parameters}
  \begin{itemize}
   \item{
\texttt{fga} -- The \texttt{\small FastGraphAccessor}{\small 
\refdefined{graphmodel.FastGraphAccessor}} to whom this vertex will be added.}
  \end{itemize}
}%end item
\end{itemize}
}%end item
\item{ 
\index{getConnectedVertex(Edge)}
{\bf  getConnectedVertex}\\
\begin{lstlisting}[frame=none]
graphmodel.Vertex getConnectedVertex(graphmodel.Edge edge)\end{lstlisting} %end signature
\begin{itemize}
\item{
{\bf  Description copied from graphmodel.CompoundVertex{\small \refdefined{graphmodel.CompoundVertex}} }

Returns the connected vertex contained in the compound vertex for a given edge, where one end point of the edge has to be this vertex.
}
\item{
{\bf  Parameters}
  \begin{itemize}
   \item{
\texttt{edge} -- the edge to get the vertex to}
  \end{itemize}
}%end item
\item{{\bf  Returns} -- 
the connected vertex if the edge is valid 
}%end item
\end{itemize}
}%end item
\item{ 
\index{getGraph()}
{\bf  getGraph}\\
\begin{lstlisting}[frame=none]
graphmodel.Graph getGraph()\end{lstlisting} %end signature
\begin{itemize}
\item{
{\bf  Description copied from graphmodel.CompoundVertex{\small \refdefined{graphmodel.CompoundVertex}} }

Returns the graph contained in the vertex.
}
\item{{\bf  Returns} -- 
the graph contained in the vertex. 
}%end item
\end{itemize}
}%end item
\end{itemize}
}
}
\subsection{\label{joana.FieldAccessGraph}\index{FieldAccessGraph}Class FieldAccessGraph}{
\vskip .1in 
A \texttt{\small JoanaGraph}{\small 
\refdefined{joana.JoanaGraph}} which specifies a \texttt{\small FieldAccess}{\small 
\refdefined{joana.FieldAccess}} in a \texttt{\small JoanaGraph}{\small 
\refdefined{joana.JoanaGraph}}\vskip .1in 
\subsubsection{Declaration}{
\begin{lstlisting}[frame=none]
public class FieldAccessGraph
 extends joana.JoanaGraph\end{lstlisting}
\subsubsection{Constructors}{
\vskip -2em
\begin{itemize}
\item{ 
\index{FieldAccessGraph()}
{\bf  FieldAccessGraph}\\
\begin{lstlisting}[frame=none]
public FieldAccessGraph()\end{lstlisting} %end signature
}%end item
\end{itemize}
}
\subsubsection{Methods}{
\vskip -2em
\begin{itemize}
\item{ 
\index{getLayerWidth(int)}
{\bf  getLayerWidth}\\
\begin{lstlisting}[frame=none]
public int getLayerWidth(int layerN)\end{lstlisting} %end signature
}%end item
\item{ 
\index{getSubgraphs()}
{\bf  getSubgraphs}\\
\begin{lstlisting}[frame=none]
public java.util.List getSubgraphs()\end{lstlisting} %end signature
}%end item
\end{itemize}
}
}
\subsection{\label{joana.FieldAccessLayout}\index{FieldAccessLayout}Class FieldAccessLayout}{
\vskip .1in 
The FieldAccessLayout applies its layout to a \texttt{\small FieldAccess}{\small 
\refdefined{joana.FieldAccess}}.\vskip .1in 
\subsubsection{Declaration}{
\begin{lstlisting}[frame=none]
public class FieldAccessLayout
 extends java.lang.Object implements sugiyama.LayeredLayoutAlgorithm\end{lstlisting}
\subsubsection{Constructors}{
\vskip -2em
\begin{itemize}
\item{ 
\index{FieldAccessLayout()}
{\bf  FieldAccessLayout}\\
\begin{lstlisting}[frame=none]
public FieldAccessLayout()\end{lstlisting} %end signature
}%end item
\end{itemize}
}
\subsubsection{Methods}{
\vskip -2em
\begin{itemize}
\item{ 
\index{getSettings()}
{\bf  getSettings}\\
\begin{lstlisting}[frame=none]
public parameter.Settings getSettings()\end{lstlisting} %end signature
}%end item
\item{ 
\index{layout(FieldAccessGraph)}
{\bf  layout}\\
\begin{lstlisting}[frame=none]
public void layout(FieldAccessGraph graph)\end{lstlisting} %end signature
}%end item
\item{ 
\index{layoutLayeredGraph(LayeredGraph)}
{\bf  layoutLayeredGraph}\\
\begin{lstlisting}[frame=none]
void layoutLayeredGraph(graphmodel.LayeredGraph graph)\end{lstlisting} %end signature
\begin{itemize}
\item{
{\bf  Description copied from sugiyama.LayeredLayoutAlgorithm{\small \refdefined{sugiyama.LayeredLayoutAlgorithm}} }

Applies its layout to a graph as in \texttt{\small layout(G graph)} but keeps the notion of layers. The algorithm will assign every vertex a coordinate and every edge a path. Additionally every vertex will be assigned a position in a layer in the LayeredGraph. A possible application is drawing of recursive graphs.
}
\item{
{\bf  Parameters}
  \begin{itemize}
   \item{
\texttt{graph} -- the graph to apply the layout to}
  \end{itemize}
}%end item
\end{itemize}
}%end item
\end{itemize}
}
}
\subsection{\label{joana.JoanaEdge}\index{JoanaEdge}Class JoanaEdge}{
\vskip .1in 
A Joana specific Edge. It contains parameters which are only used/usefull in \texttt{\small JoanaGraph}{\small 
\refdefined{joana.JoanaGraph}}.\vskip .1in 
\subsubsection{Declaration}{
\begin{lstlisting}[frame=none]
public class JoanaEdge
 extends graphmodel.DirectedEdge\end{lstlisting}
\subsubsection{Constructors}{
\vskip -2em
\begin{itemize}
\item{ 
\index{JoanaEdge()}
{\bf  JoanaEdge}\\
\begin{lstlisting}[frame=none]
public JoanaEdge()\end{lstlisting} %end signature
}%end item
\end{itemize}
}
\subsubsection{Methods}{
\vskip -2em
\begin{itemize}
\item{ 
\index{getEdgeKind()}
{\bf  getEdgeKind}\\
\begin{lstlisting}[frame=none]
public java.lang.String getEdgeKind()\end{lstlisting} %end signature
\begin{itemize}
\item{
{\bf  Description}

Returns the edgeKind of the JoanaEdge.
}
\item{{\bf  Returns} -- 
The edgeKind of the JoanaEdge. 
}%end item
\end{itemize}
}%end item
\end{itemize}
}
}
\subsection{\label{joana.JoanaEdgeBuilder}\index{JoanaEdgeBuilder}Class JoanaEdgeBuilder}{
\vskip .1in 
The JoanaEdgeBuilder is a \texttt{\small IEdgeBuilder}{\small 
\refdefined{graphmodel.IEdgeBuilder}}, specifically for building \texttt{\small JoanaEdge}{\small 
\refdefined{joana.JoanaEdge}}.\vskip .1in 
\subsubsection{Declaration}{
\begin{lstlisting}[frame=none]
public class JoanaEdgeBuilder
 extends java.lang.Object implements graphmodel.IEdgeBuilder\end{lstlisting}
\subsubsection{Constructors}{
\vskip -2em
\begin{itemize}
\item{ 
\index{JoanaEdgeBuilder()}
{\bf  JoanaEdgeBuilder}\\
\begin{lstlisting}[frame=none]
public JoanaEdgeBuilder()\end{lstlisting} %end signature
}%end item
\end{itemize}
}
\subsubsection{Methods}{
\vskip -2em
\begin{itemize}
\item{ 
\index{addData(String, String)}
{\bf  addData}\\
\begin{lstlisting}[frame=none]
void addData(java.lang.String keyname,java.lang.String value)\end{lstlisting} %end signature
\begin{itemize}
\item{
{\bf  Description copied from graphmodel.IEdgeBuilder{\small \refdefined{graphmodel.IEdgeBuilder}} }

Adds additional data to this edge. The specific EdgeBuilder implementation needs to decide how to save the value for given edge type.
}
\item{
{\bf  Parameters}
  \begin{itemize}
   \item{
\texttt{keyname} -- }
   \item{
\texttt{value} -- }
  \end{itemize}
}%end item
\end{itemize}
}%end item
\item{ 
\index{newEdge(String, String)}
{\bf  newEdge}\\
\begin{lstlisting}[frame=none]
void newEdge(java.lang.String source,java.lang.String target)\end{lstlisting} %end signature
\begin{itemize}
\item{
{\bf  Description copied from graphmodel.IEdgeBuilder{\small \refdefined{graphmodel.IEdgeBuilder}} }

Sets source and target vertices of the edge build by this.
}
\item{
{\bf  Parameters}
  \begin{itemize}
   \item{
\texttt{source} -- String represantation of the source vertex}
   \item{
\texttt{target} -- String represantation of the target vertex}
  \end{itemize}
}%end item
\end{itemize}
}%end item
\item{ 
\index{setDirection(String)}
{\bf  setDirection}\\
\begin{lstlisting}[frame=none]
void setDirection(java.lang.String direction)\end{lstlisting} %end signature
\begin{itemize}
\item{
{\bf  Description copied from graphmodel.IEdgeBuilder{\small \refdefined{graphmodel.IEdgeBuilder}} }

Sets the direction of the edge build by this.
}
\item{
{\bf  Parameters}
  \begin{itemize}
   \item{
\texttt{direction} -- String representation of the direction. Can be one of}
  \end{itemize}
}%end item
\end{itemize}
}%end item
\item{ 
\index{setID(String)}
{\bf  setID}\\
\begin{lstlisting}[frame=none]
void setID(java.lang.String id)\end{lstlisting} %end signature
\begin{itemize}
\item{
{\bf  Description copied from graphmodel.IEdgeBuilder{\small \refdefined{graphmodel.IEdgeBuilder}} }

Sets the ID of the edge build by this.
}
\item{
{\bf  Parameters}
  \begin{itemize}
   \item{
\texttt{id} -- value to which the id is set}
  \end{itemize}
}%end item
\end{itemize}
}%end item
\end{itemize}
}
}
\subsection{\label{joana.JoanaGraph}\index{JoanaGraph}Class JoanaGraph}{
\vskip .1in 
An abstract superclass for all JOANA specific graphs.\vskip .1in 
\subsubsection{Declaration}{
\begin{lstlisting}[frame=none]
public abstract class JoanaGraph
 extends graphmodel.DefaultDirectedGraph implements graphmodel.LayeredGraph\end{lstlisting}
\subsubsection{All known subclasses}{MethodGraph\small{\refdefined{joana.MethodGraph}}, FieldAccessGraph\small{\refdefined{joana.FieldAccessGraph}}, CallGraph\small{\refdefined{joana.CallGraph}}}
\subsubsection{Constructors}{
\vskip -2em
\begin{itemize}
\item{ 
\index{JoanaGraph()}
{\bf  JoanaGraph}\\
\begin{lstlisting}[frame=none]
public JoanaGraph()\end{lstlisting} %end signature
}%end item
\end{itemize}
}
\subsubsection{Methods}{
\vskip -2em
\begin{itemize}
\item{ 
\index{getHeight()}
{\bf  getHeight}\\
\begin{lstlisting}[frame=none]
int getHeight()\end{lstlisting} %end signature
\begin{itemize}
\item{
{\bf  Description copied from graphmodel.LayeredGraph{\small \refdefined{graphmodel.LayeredGraph}} }

Returns the height, i.e. the number of layers.
}
\item{{\bf  Returns} -- 
the height 
}%end item
\end{itemize}
}%end item
\item{ 
\index{getLayer(int)}
{\bf  getLayer}\\
\begin{lstlisting}[frame=none]
java.util.List getLayer(int layerNum)\end{lstlisting} %end signature
\begin{itemize}
\item{
{\bf  Description copied from graphmodel.LayeredGraph{\small \refdefined{graphmodel.LayeredGraph}} }

Get all vertices from a certain layer.
}
\item{
{\bf  Parameters}
  \begin{itemize}
   \item{
\texttt{layerN} -- the index of the layer}
  \end{itemize}
}%end item
\item{{\bf  Returns} -- 
a list of all vertices which are on this layer 
}%end item
\end{itemize}
}%end item
\item{ 
\index{getLayer(JoanaVertex)}
{\bf  getLayer}\\
\begin{lstlisting}[frame=none]
public int getLayer(JoanaVertex vertex)\end{lstlisting} %end signature
}%end item
\item{ 
\index{getLayerCount()}
{\bf  getLayerCount}\\
\begin{lstlisting}[frame=none]
int getLayerCount()\end{lstlisting} %end signature
\begin{itemize}
\item{
{\bf  Description copied from graphmodel.LayeredGraph{\small \refdefined{graphmodel.LayeredGraph}} }

Get the amount of layers.
}
\item{{\bf  Returns} -- 
the amount of layers that contain at least one vertex 
}%end item
\end{itemize}
}%end item
\item{ 
\index{getLayers()}
{\bf  getLayers}\\
\begin{lstlisting}[frame=none]
java.util.List getLayers()\end{lstlisting} %end signature
\begin{itemize}
\item{
{\bf  Description copied from graphmodel.LayeredGraph{\small \refdefined{graphmodel.LayeredGraph}} }

Get all layers that contain vertices.
}
\item{{\bf  Returns} -- 
a list of lists of vertices which are on this layer 
}%end item
\end{itemize}
}%end item
\item{ 
\index{getMaxWidth()}
{\bf  getMaxWidth}\\
\begin{lstlisting}[frame=none]
int getMaxWidth()\end{lstlisting} %end signature
\begin{itemize}
\item{
{\bf  Description copied from graphmodel.LayeredGraph{\small \refdefined{graphmodel.LayeredGraph}} }

Returns the width of the widest layer, i.e. the number of vertices the layer with the most vertices contains.
}
\item{{\bf  Returns} -- 
the maximum width 
}%end item
\end{itemize}
}%end item
\item{ 
\index{getVertexCount(int)}
{\bf  getVertexCount}\\
\begin{lstlisting}[frame=none]
int getVertexCount(int layerNum)\end{lstlisting} %end signature
\begin{itemize}
\item{
{\bf  Description copied from graphmodel.LayeredGraph{\small \refdefined{graphmodel.LayeredGraph}} }

Get the number of vertices which are on a certain layer
}
\item{
{\bf  Parameters}
  \begin{itemize}
   \item{
\texttt{layerNum} -- the layer number to get the vertex count from}
  \end{itemize}
}%end item
\item{{\bf  Returns} -- 
the number of vertices which are on this layer 
}%end item
\end{itemize}
}%end item
\end{itemize}
}
}
\subsection{\label{joana.JoanaGraphModel}\index{JoanaGraphModel}Class JoanaGraphModel}{
\vskip .1in 
A Joana specific \texttt{\small GraphModel}{\small 
\refdefined{graphmodel.GraphModel}}. It can only contain \texttt{\small MethodGraph}{\small 
\refdefined{joana.MethodGraph}} and \texttt{\small CallGraph}{\small 
\refdefined{joana.CallGraph}}.\vskip .1in 
\subsubsection{Declaration}{
\begin{lstlisting}[frame=none]
public class JoanaGraphModel
 extends graphmodel.GraphModel\end{lstlisting}
\subsubsection{Constructors}{
\vskip -2em
\begin{itemize}
\item{ 
\index{JoanaGraphModel()}
{\bf  JoanaGraphModel}\\
\begin{lstlisting}[frame=none]
public JoanaGraphModel()\end{lstlisting} %end signature
}%end item
\end{itemize}
}
\subsubsection{Methods}{
\vskip -2em
\begin{itemize}
\item{ 
\index{getCallGraph()}
{\bf  getCallGraph}\\
\begin{lstlisting}[frame=none]
public CallGraph getCallGraph()\end{lstlisting} %end signature
\begin{itemize}
\item{
{\bf  Description}

Returns all \texttt{\small CallGraph}{\small 
\refdefined{joana.CallGraph}} contained in the JoanaGraphModel.
}
\item{{\bf  Returns} -- 
A list of all the \texttt{\small CallGraph}{\small 
\refdefined{joana.CallGraph}} contained in the JoanaGraphModel. 
}%end item
\end{itemize}
}%end item
\item{ 
\index{getGraphs()}
{\bf  getGraphs}\\
\begin{lstlisting}[frame=none]
public abstract java.util.List getGraphs()\end{lstlisting} %end signature
\begin{itemize}
\item{
{\bf  Description copied from graphmodel.GraphModel{\small \refdefined{graphmodel.GraphModel}} }

Returns all \texttt{\small Graph}{\small 
\refdefined{graphmodel.Graph}} contained in the GraphModel.
}
\item{{\bf  Returns} -- 
A list of all the \texttt{\small Graph}{\small 
\refdefined{graphmodel.Graph}} contained in the GraphModel. 
}%end item
\end{itemize}
}%end item
\item{ 
\index{getMethodGraphs()}
{\bf  getMethodGraphs}\\
\begin{lstlisting}[frame=none]
public java.util.List getMethodGraphs()\end{lstlisting} %end signature
\begin{itemize}
\item{
{\bf  Description}

Returns all \texttt{\small MethodGraph}{\small 
\refdefined{joana.MethodGraph}} contained in the JoanaGraphModel.
}
\item{{\bf  Returns} -- 
A list of all the \texttt{\small MethodGraph}{\small 
\refdefined{joana.MethodGraph}} contained in the JoanaGraphModel. 
}%end item
\end{itemize}
}%end item
\item{ 
\index{setCallGraph(CallGraph)}
{\bf  setCallGraph}\\
\begin{lstlisting}[frame=none]
public void setCallGraph(CallGraph callgraph)\end{lstlisting} %end signature
\begin{itemize}
\item{
{\bf  Description}

Sets the \texttt{\small CallGraph}{\small 
\refdefined{joana.CallGraph}} in the JoanaGraphModel.
}
\item{
{\bf  Parameters}
  \begin{itemize}
   \item{
\texttt{callgraph} -- The \texttt{\small CallGraph}{\small 
\refdefined{joana.CallGraph}} that will be set in the JoanaGraphModel.}
  \end{itemize}
}%end item
\end{itemize}
}%end item
\item{ 
\index{setMethodGraphs(List)}
{\bf  setMethodGraphs}\\
\begin{lstlisting}[frame=none]
public void setMethodGraphs(java.util.List methodgraphs)\end{lstlisting} %end signature
\begin{itemize}
\item{
{\bf  Description}

Sets the \texttt{\small MethodGraph}{\small 
\refdefined{joana.MethodGraph}} objects in the JoanaGraphModel.
}
\item{
{\bf  Parameters}
  \begin{itemize}
   \item{
\texttt{methodgraphs} -- The \texttt{\small MethodGraph}{\small 
\refdefined{joana.MethodGraph}} objects that will be set in the JoanaGraphModel.}
  \end{itemize}
}%end item
\end{itemize}
}%end item
\end{itemize}
}
}
\subsection{\label{joana.JoanaGraphModelBuilder}\index{JoanaGraphModelBuilder}Class JoanaGraphModelBuilder}{
\vskip .1in 
The JoanaGraphModelBuilder implements the \texttt{\small IGraphModelBuilder}{\small 
\refdefined{graphmodel.IGraphModelBuilder}} and creates a \texttt{\small JoanaGraphModel}{\small 
\refdefined{joana.JoanaGraphModel}}.\vskip .1in 
\subsubsection{Declaration}{
\begin{lstlisting}[frame=none]
public class JoanaGraphModelBuilder
 extends java.lang.Object implements graphmodel.IGraphModelBuilder\end{lstlisting}
\subsubsection{Constructors}{
\vskip -2em
\begin{itemize}
\item{ 
\index{JoanaGraphModelBuilder()}
{\bf  JoanaGraphModelBuilder}\\
\begin{lstlisting}[frame=none]
public JoanaGraphModelBuilder()\end{lstlisting} %end signature
}%end item
\end{itemize}
}
\subsubsection{Methods}{
\vskip -2em
\begin{itemize}
\item{ 
\index{build()}
{\bf  build}\\
\begin{lstlisting}[frame=none]
graphmodel.GraphModel build()\end{lstlisting} %end signature
\begin{itemize}
\item{
{\bf  Description copied from graphmodel.IGraphModelBuilder{\small \refdefined{graphmodel.IGraphModelBuilder}} }

Builds a graphmodel from the given settings and returns it.
}
\item{{\bf  Returns} -- 
The \texttt{\small GraphModel}{\small 
\refdefined{graphmodel.GraphModel}} that is being build by the IGraphModelBuilder. 
}%end item
\end{itemize}
}%end item
\item{ 
\index{getGraphBuilder(String)}
{\bf  getGraphBuilder}\\
\begin{lstlisting}[frame=none]
graphmodel.IGraphBuilder getGraphBuilder(java.lang.String graphID)\end{lstlisting} %end signature
\begin{itemize}
\item{
{\bf  Description copied from graphmodel.IGraphModelBuilder{\small \refdefined{graphmodel.IGraphModelBuilder}} }

Returns a specific \texttt{\small IGraphBuilder}{\small 
\refdefined{graphmodel.IGraphBuilder}} for a graph, which belongs to the \texttt{\small GraphModel}{\small 
\refdefined{graphmodel.GraphModel}}.
}
\item{
{\bf  Parameters}
  \begin{itemize}
   \item{
\texttt{graphID} -- The id of the graph which associated \texttt{\small IGraphBuilder}{\small 
\refdefined{graphmodel.IGraphBuilder}} will be returned.}
  \end{itemize}
}%end item
\item{{\bf  Returns} -- 
The IGraphBuilder of the graph which is referenced over the graphID. 
}%end item
\end{itemize}
}%end item
\end{itemize}
}
}
\subsection{\label{joana.JoanaPlugin}\index{JoanaPlugin}Class JoanaPlugin}{
\vskip .1in 
A plugin for GAns that supports the creation and visualization of Joana system dependence graphs.\vskip .1in 
\subsubsection{Declaration}{
\begin{lstlisting}[frame=none]
public class JoanaPlugin
 extends java.lang.Object implements plugin.Plugin\end{lstlisting}
\subsubsection{Constructors}{
\vskip -2em
\begin{itemize}
\item{ 
\index{JoanaPlugin()}
{\bf  JoanaPlugin}\\
\begin{lstlisting}[frame=none]
public JoanaPlugin()\end{lstlisting} %end signature
\begin{itemize}
\item{
{\bf  Description}

Constructor. The constructor is called by the ServiceLoader.
}
\end{itemize}
}%end item
\end{itemize}
}
\subsubsection{Methods}{
\vskip -2em
\begin{itemize}
\item{ 
\index{getCallGraphLayoutRegister()}
{\bf  getCallGraphLayoutRegister}\\
\begin{lstlisting}[frame=none]
public static JoanaPlugin.CallGraphLayoutRegister getCallGraphLayoutRegister()\end{lstlisting} %end signature
}%end item
\item{ 
\index{getEdgeFilter()}
{\bf  getEdgeFilter}\\
\begin{lstlisting}[frame=none]
java.util.List getEdgeFilter()\end{lstlisting} %end signature
\begin{itemize}
\item{
{\bf  Description copied from plugin.Plugin{\small \refdefined{plugin.Plugin}} }

Returns all by the plugin provided \texttt{\small EdgeFilter}{\small 
\refdefined{plugin.EdgeFilter}}. If none are provided returns \texttt{\small null} or an empty list.
}
\item{{\bf  Returns} -- 
the list of provided edge filter 
}%end item
\end{itemize}
}%end item
\item{ 
\index{getMethodGraphLayoutRegister()}
{\bf  getMethodGraphLayoutRegister}\\
\begin{lstlisting}[frame=none]
public static JoanaPlugin.MethodGraphLayoutRegister getMethodGraphLayoutRegister()\end{lstlisting} %end signature
}%end item
\item{ 
\index{getName()}
{\bf  getName}\\
\begin{lstlisting}[frame=none]
java.lang.String getName()\end{lstlisting} %end signature
\begin{itemize}
\item{
{\bf  Description copied from plugin.Plugin{\small \refdefined{plugin.Plugin}} }

Returns the name of the plugin. Uniqueness can't be assumed.
}
\item{{\bf  Returns} -- 
the name of the plugin 
}%end item
\end{itemize}
}%end item
\item{ 
\index{getVertexFilter()}
{\bf  getVertexFilter}\\
\begin{lstlisting}[frame=none]
java.util.List getVertexFilter()\end{lstlisting} %end signature
\begin{itemize}
\item{
{\bf  Description copied from plugin.Plugin{\small \refdefined{plugin.Plugin}} }

Returns all by the plugin provided \texttt{\small VertexFilter}{\small 
\refdefined{plugin.VertexFilter}}. If none are provided returns \texttt{\small null} or an empty list.
}
\item{{\bf  Returns} -- 
the list of provided vertex filter 
}%end item
\end{itemize}
}%end item
\item{ 
\index{getWorkspaceOptions()}
{\bf  getWorkspaceOptions}\\
\begin{lstlisting}[frame=none]
java.util.List getWorkspaceOptions()\end{lstlisting} %end signature
\begin{itemize}
\item{
{\bf  Description copied from plugin.Plugin{\small \refdefined{plugin.Plugin}} }

Returns all provided by the plugin \texttt{\small WorkspaceOption}{\small 
\refdefined{plugin.WorkspaceOption}}. If none are provided returns \texttt{\small null} or an empty list.
}
\item{{\bf  Returns} -- 
The list of provided workspace options 
}%end item
\end{itemize}
}%end item
\item{ 
\index{load()}
{\bf  load}\\
\begin{lstlisting}[frame=none]
void load()\end{lstlisting} %end signature
\begin{itemize}
\item{
{\bf  Description copied from plugin.Plugin{\small \refdefined{plugin.Plugin}} }

Called after all plugins have been constructed. "Inter-Plugin" communication, like registering of layouts for graphs in other plugins should be executed in here.
}
\end{itemize}
}%end item
\end{itemize}
}
}
\subsection{\label{joana.JoanaPlugin.CallGraphLayoutRegister}\index{JoanaPlugin.CallGraphLayoutRegister}Class JoanaPlugin.CallGraphLayoutRegister}{
\vskip .1in 
\subsubsection{Declaration}{
\begin{lstlisting}[frame=none]
public static class JoanaPlugin.CallGraphLayoutRegister
 extends java.lang.Object implements plugin.LayoutRegister\end{lstlisting}
\subsubsection{Constructors}{
\vskip -2em
\begin{itemize}
\item{ 
\index{CallGraphLayoutRegister()}
{\bf  CallGraphLayoutRegister}\\
\begin{lstlisting}[frame=none]
public CallGraphLayoutRegister()\end{lstlisting} %end signature
}%end item
\end{itemize}
}
\subsubsection{Methods}{
\vskip -2em
\begin{itemize}
\item{ 
\index{addLayoutOption(CallGraphLayoutOption)}
{\bf  addLayoutOption}\\
\begin{lstlisting}[frame=none]
public void addLayoutOption(CallGraphLayoutOption option)\end{lstlisting} %end signature
}%end item
\item{ 
\index{getLayoutOptions()}
{\bf  getLayoutOptions}\\
\begin{lstlisting}[frame=none]
java.util.List getLayoutOptions()\end{lstlisting} %end signature
}%end item
\end{itemize}
}
}
\subsection{\label{joana.JoanaPlugin.MethodGraphLayoutRegister}\index{JoanaPlugin.MethodGraphLayoutRegister}Class JoanaPlugin.MethodGraphLayoutRegister}{
\vskip .1in 
\subsubsection{Declaration}{
\begin{lstlisting}[frame=none]
public static class JoanaPlugin.MethodGraphLayoutRegister
 extends java.lang.Object implements plugin.LayoutRegister\end{lstlisting}
\subsubsection{Constructors}{
\vskip -2em
\begin{itemize}
\item{ 
\index{MethodGraphLayoutRegister()}
{\bf  MethodGraphLayoutRegister}\\
\begin{lstlisting}[frame=none]
public MethodGraphLayoutRegister()\end{lstlisting} %end signature
}%end item
\end{itemize}
}
\subsubsection{Methods}{
\vskip -2em
\begin{itemize}
\item{ 
\index{addLayoutOption(MethodGraphLayoutOption)}
{\bf  addLayoutOption}\\
\begin{lstlisting}[frame=none]
public void addLayoutOption(MethodGraphLayoutOption option)\end{lstlisting} %end signature
}%end item
\item{ 
\index{getLayoutOptions()}
{\bf  getLayoutOptions}\\
\begin{lstlisting}[frame=none]
java.util.List getLayoutOptions()\end{lstlisting} %end signature
}%end item
\end{itemize}
}
}
\subsection{\label{joana.JoanaVertex}\index{JoanaVertex}Class JoanaVertex}{
\vskip .1in 
A Joana specific Vertex. It contains parameters which are only used/useful for Joana.\vskip .1in 
\subsubsection{Declaration}{
\begin{lstlisting}[frame=none]
public class JoanaVertex
 extends graphmodel.DefaultVertex\end{lstlisting}
\subsubsection{All known subclasses}{FieldAccess\small{\refdefined{joana.FieldAccess}}}
\subsubsection{Constructors}{
\vskip -2em
\begin{itemize}
\item{ 
\index{JoanaVertex()}
{\bf  JoanaVertex}\\
\begin{lstlisting}[frame=none]
public JoanaVertex()\end{lstlisting} %end signature
}%end item
\end{itemize}
}
\subsubsection{Methods}{
\vskip -2em
\begin{itemize}
\item{ 
\index{getNodeBCIndex()}
{\bf  getNodeBCIndex}\\
\begin{lstlisting}[frame=none]
public java.lang.Integer getNodeBCIndex()\end{lstlisting} %end signature
\begin{itemize}
\item{
{\bf  Description}

Returns the nodeBCIndex of the JoanaVertex.
}
\item{{\bf  Returns} -- 
The nodeBCIndex of the JoanaVertex. 
}%end item
\end{itemize}
}%end item
\item{ 
\index{getNodeBcName()}
{\bf  getNodeBcName}\\
\begin{lstlisting}[frame=none]
public java.lang.String getNodeBcName()\end{lstlisting} %end signature
\begin{itemize}
\item{
{\bf  Description}

Returns the nodeBcName of the JoanaVertex.
}
\item{{\bf  Returns} -- 
The nodeBcName of the JoanaVertex. 
}%end item
\end{itemize}
}%end item
\item{ 
\index{getNodeEc()}
{\bf  getNodeEc}\\
\begin{lstlisting}[frame=none]
public java.lang.Integer getNodeEc()\end{lstlisting} %end signature
\begin{itemize}
\item{
{\bf  Description}

Returns the nodeEc of the JoanaVertex.
}
\item{{\bf  Returns} -- 
The nodeEc of the JoanaVertex. 
}%end item
\end{itemize}
}%end item
\item{ 
\index{getNodeEr()}
{\bf  getNodeEr}\\
\begin{lstlisting}[frame=none]
public java.lang.Integer getNodeEr()\end{lstlisting} %end signature
\begin{itemize}
\item{
{\bf  Description}

Returns the nodeEr of the JoanaVertex.
}
\item{{\bf  Returns} -- 
The nodeEr of the JoanaVertex. 
}%end item
\end{itemize}
}%end item
\item{ 
\index{getNodeKind()}
{\bf  getNodeKind}\\
\begin{lstlisting}[frame=none]
public java.lang.String getNodeKind()\end{lstlisting} %end signature
\begin{itemize}
\item{
{\bf  Description}

Returns the nodeKind of the JoanaVertex.
}
\item{{\bf  Returns} -- 
The nodeKind of the JoanaVertex. 
}%end item
\end{itemize}
}%end item
\item{ 
\index{getNodeOperation()}
{\bf  getNodeOperation}\\
\begin{lstlisting}[frame=none]
public java.lang.String getNodeOperation()\end{lstlisting} %end signature
\begin{itemize}
\item{
{\bf  Description}

Returns the nodeOperation of the JoanaVertex.
}
\item{{\bf  Returns} -- 
The nodeOperation of the JoanaVertex. 
}%end item
\end{itemize}
}%end item
\item{ 
\index{getNodeProc()}
{\bf  getNodeProc}\\
\begin{lstlisting}[frame=none]
public java.lang.Integer getNodeProc()\end{lstlisting} %end signature
\begin{itemize}
\item{
{\bf  Description}

Returns the nodeProc of the JoanaVertex.
}
\item{{\bf  Returns} -- 
The nodeProc of the JoanaVertex. 
}%end item
\end{itemize}
}%end item
\item{ 
\index{getNodeSc()}
{\bf  getNodeSc}\\
\begin{lstlisting}[frame=none]
public java.lang.Integer getNodeSc()\end{lstlisting} %end signature
\begin{itemize}
\item{
{\bf  Description}

Returns the nodeSc of the JoanaVertex.
}
\item{{\bf  Returns} -- 
The nodeSc of the JoanaVertex. 
}%end item
\end{itemize}
}%end item
\item{ 
\index{getNodeSource()}
{\bf  getNodeSource}\\
\begin{lstlisting}[frame=none]
public java.lang.String getNodeSource()\end{lstlisting} %end signature
\begin{itemize}
\item{
{\bf  Description}

Returns the nodeSource of the JoanaVertex.
}
\item{{\bf  Returns} -- 
The nodeSource of the JoanaVertex. 
}%end item
\end{itemize}
}%end item
\item{ 
\index{getNodeSr()}
{\bf  getNodeSr}\\
\begin{lstlisting}[frame=none]
public java.lang.Integer getNodeSr()\end{lstlisting} %end signature
\begin{itemize}
\item{
{\bf  Description}

Returns the nodeSr of the JoanaVertex.
}
\item{{\bf  Returns} -- 
The nodeSr of the JoanaVertex. 
}%end item
\end{itemize}
}%end item
\end{itemize}
}
}
\subsection{\label{joana.JoanaVertexBuilder}\index{JoanaVertexBuilder}Class JoanaVertexBuilder}{
\vskip .1in 
The JoanaVertexBuilder implements an \texttt{\small IVertexBuilder}{\small 
\refdefined{graphmodel.IVertexBuilder}} and creates a \texttt{\small JoanaVertex}{\small 
\refdefined{joana.JoanaVertex}}.\vskip .1in 
\subsubsection{Declaration}{
\begin{lstlisting}[frame=none]
public class JoanaVertexBuilder
 extends java.lang.Object implements graphmodel.IVertexBuilder\end{lstlisting}
\subsubsection{Constructors}{
\vskip -2em
\begin{itemize}
\item{ 
\index{JoanaVertexBuilder()}
{\bf  JoanaVertexBuilder}\\
\begin{lstlisting}[frame=none]
public JoanaVertexBuilder()\end{lstlisting} %end signature
}%end item
\end{itemize}
}
\subsubsection{Methods}{
\vskip -2em
\begin{itemize}
\item{ 
\index{addData(String, String)}
{\bf  addData}\\
\begin{lstlisting}[frame=none]
void addData(java.lang.String value,java.lang.String keyname)\end{lstlisting} %end signature
\begin{itemize}
\item{
{\bf  Description copied from graphmodel.IVertexBuilder{\small \refdefined{graphmodel.IVertexBuilder}} }

Add Data to this Vertex. The vertexBuilder decides which kind of data it is and where to save in the concrete Vertex.
}
\item{
{\bf  Parameters}
  \begin{itemize}
   \item{
\texttt{value} -- }
   \item{
\texttt{keyname} -- }
  \end{itemize}
}%end item
\end{itemize}
}%end item
\item{ 
\index{build()}
{\bf  build}\\
\begin{lstlisting}[frame=none]
graphmodel.Vertex build()\end{lstlisting} %end signature
\begin{itemize}
\item{
{\bf  Description copied from graphmodel.IVertexBuilder{\small \refdefined{graphmodel.IVertexBuilder}} }

This method builds the concrete Vertex and returns it.
}
\item{{\bf  Returns} -- 
Vertex 
}%end item
\end{itemize}
}%end item
\item{ 
\index{getGraphBuilder(String)}
{\bf  getGraphBuilder}\\
\begin{lstlisting}[frame=none]
graphmodel.IGraphBuilder getGraphBuilder(java.lang.String graphID)\end{lstlisting} %end signature
\begin{itemize}
\item{
{\bf  Description copied from graphmodel.IVertexBuilder{\small \refdefined{graphmodel.IVertexBuilder}} }

This method returns an specific GraphBuilder. This method is used to implement nested Graphs.
}
\item{
{\bf  Parameters}
  \begin{itemize}
   \item{
\texttt{graphID} -- }
  \end{itemize}
}%end item
\item{{\bf  Returns} -- 
 
}%end item
\end{itemize}
}%end item
\end{itemize}
}
}
\subsection{\label{joana.JoanaWorkspace}\index{JoanaWorkspace}Class JoanaWorkspace}{
\vskip .1in 
The \texttt{\small JoanaWorkspace}{\small 
\refdefined{joana.JoanaWorkspace}} is the workspace for Joana graphs. It is used to define parameters, provides an \texttt{\small IGraphModelBuilder}{\small 
\refdefined{graphmodel.IGraphModelBuilder}} and contains a \texttt{\small JoanaGraphModel}{\small 
\refdefined{joana.JoanaGraphModel}}.\vskip .1in 
\subsubsection{Declaration}{
\begin{lstlisting}[frame=none]
public class JoanaWorkspace
 extends java.lang.Object implements plugin.Workspace\end{lstlisting}
\subsubsection{Constructors}{
\vskip -2em
\begin{itemize}
\item{ 
\index{JoanaWorkspace()}
{\bf  JoanaWorkspace}\\
\begin{lstlisting}[frame=none]
public JoanaWorkspace()\end{lstlisting} %end signature
}%end item
\end{itemize}
}
\subsubsection{Methods}{
\vskip -2em
\begin{itemize}
\item{ 
\index{getGraphModel()}
{\bf  getGraphModel}\\
\begin{lstlisting}[frame=none]
graphmodel.GraphModel getGraphModel()\end{lstlisting} %end signature
\begin{itemize}
\item{
{\bf  Description copied from plugin.Workspace{\small \refdefined{plugin.Workspace}} }

Returns the \texttt{\small GraphModel}{\small 
\refdefined{graphmodel.GraphModel}} stored in the workspace.
}
\item{{\bf  Returns} -- 
the graph model 
}%end item
\end{itemize}
}%end item
\item{ 
\index{getGraphModelBuilder()}
{\bf  getGraphModelBuilder}\\
\begin{lstlisting}[frame=none]
graphmodel.IGraphModelBuilder getGraphModelBuilder()\end{lstlisting} %end signature
\begin{itemize}
\item{
{\bf  Description copied from plugin.Workspace{\small \refdefined{plugin.Workspace}} }

Returns a builder to build a graph model in this workspace.
}
\item{{\bf  Returns} -- 
the builder 
}%end item
\end{itemize}
}%end item
\item{ 
\index{getSettings()}
{\bf  getSettings}\\
\begin{lstlisting}[frame=none]
parameter.Settings getSettings()\end{lstlisting} %end signature
\begin{itemize}
\item{
{\bf  Description copied from plugin.Workspace{\small \refdefined{plugin.Workspace}} }

Returns a set of parameters to initialize this workspace. When the settings have been adjusted, the client has to call \texttt{\small initialize()}. To initialize the workspace with the settings.
}
\item{{\bf  Returns} -- 
the settings 
}%end item
\end{itemize}
}%end item
\item{ 
\index{initialize()}
{\bf  initialize}\\
\begin{lstlisting}[frame=none]
void initialize()\end{lstlisting} %end signature
\begin{itemize}
\item{
{\bf  Description copied from plugin.Workspace{\small \refdefined{plugin.Workspace}} }

Initializes this workspace with the settings if they have not been adjusted. If the settings have not been adjusted, default values will be used.
}
\end{itemize}
}%end item
\end{itemize}
}
}
\subsection{\label{joana.MethodGraph}\index{MethodGraph}Class MethodGraph}{
\vskip .1in 
This is a specific graph representation for a MethodGraph in JOANA\vskip .1in 
\subsubsection{Declaration}{
\begin{lstlisting}[frame=none]
public class MethodGraph
 extends joana.JoanaGraph\end{lstlisting}
\subsubsection{Constructors}{
\vskip -2em
\begin{itemize}
\item{ 
\index{MethodGraph()}
{\bf  MethodGraph}\\
\begin{lstlisting}[frame=none]
public MethodGraph()\end{lstlisting} %end signature
}%end item
\end{itemize}
}
\subsubsection{Methods}{
\vskip -2em
\begin{itemize}
\item{ 
\index{collapse(Set)}
{\bf  collapse}\\
\begin{lstlisting}[frame=none]
graphmodel.CompoundVertex collapse(java.util.Set subset)\end{lstlisting} %end signature
\begin{itemize}
\item{
{\bf  Description copied from graphmodel.Viewable{\small \refdefined{graphmodel.Viewable}} }

Collapses a set of vertices in one compound vertex. The collapsed vertices can be expanded back into their previous state with \texttt{\small expand(CompoundVertex)}.
}
\item{
{\bf  Parameters}
  \begin{itemize}
   \item{
\texttt{subset} -- the subset to collapse}
  \end{itemize}
}%end item
\item{{\bf  Returns} -- 
the resulting collapsed vertex 
}%end item
\end{itemize}
}%end item
\item{ 
\index{expand(CompoundVertex)}
{\bf  expand}\\
\begin{lstlisting}[frame=none]
java.util.Set expand(graphmodel.CompoundVertex vertex)\end{lstlisting} %end signature
\begin{itemize}
\item{
{\bf  Description copied from graphmodel.Viewable{\small \refdefined{graphmodel.Viewable}} }

Expands a collapsed vertex into its substituted set of vertices The vertices will be added back to the set of vertices of this graph. The compound vertex will be removed from the set of vertices. All to the compound vertex incident edges, will be resolved back into an edge between the vertices it connected before the collapse.
}
\item{
{\bf  Parameters}
  \begin{itemize}
   \item{
\texttt{vertex} -- the collapsed vertex to expand}
  \end{itemize}
}%end item
\item{{\bf  Returns} -- 
the set of vertices which was substituted by the collapsed vertex 
}%end item
\end{itemize}
}%end item
\item{ 
\index{getEntryVertex()}
{\bf  getEntryVertex}\\
\begin{lstlisting}[frame=none]
public JoanaVertex getEntryVertex()\end{lstlisting} %end signature
\begin{itemize}
\item{
{\bf  Description}

Returns the entry vertex of a method.
}
\item{{\bf  Returns} -- 
The entry vertex of a method. 
}%end item
\end{itemize}
}%end item
\item{ 
\index{getFieldAccesses()}
{\bf  getFieldAccesses}\\
\begin{lstlisting}[frame=none]
public java.util.List getFieldAccesses()\end{lstlisting} %end signature
\begin{itemize}
\item{
{\bf  Description}

Returns a list of all FieldAcesses in the MethodGraph.
}
\item{{\bf  Returns} -- 
A list of all FieldAcesses in the MethodGraph. 
}%end item
\end{itemize}
}%end item
\item{ 
\index{getLayerWidth(int)}
{\bf  getLayerWidth}\\
\begin{lstlisting}[frame=none]
public int getLayerWidth(int layerN)\end{lstlisting} %end signature
}%end item
\item{ 
\index{getMethodCalls()}
{\bf  getMethodCalls}\\
\begin{lstlisting}[frame=none]
public java.util.List getMethodCalls()\end{lstlisting} %end signature
\begin{itemize}
\item{
{\bf  Description}

Returns a list of all MethodCall in the MethodGraph.
}
\item{{\bf  Returns} -- 
A list of all MethodCall in the MethodGraph. 
}%end item
\end{itemize}
}%end item
\item{ 
\index{getSubgraphs()}
{\bf  getSubgraphs}\\
\begin{lstlisting}[frame=none]
public java.util.List getSubgraphs()\end{lstlisting} %end signature
}%end item
\item{ 
\index{isCompound(Vertex)}
{\bf  isCompound}\\
\begin{lstlisting}[frame=none]
boolean isCompound(graphmodel.Vertex vertex)\end{lstlisting} %end signature
\begin{itemize}
\item{
{\bf  Description copied from graphmodel.Viewable{\small \refdefined{graphmodel.Viewable}} }

Returns true if the specified vertex is a compound vertex
}
\item{
{\bf  Parameters}
  \begin{itemize}
   \item{
\texttt{vertex} -- the vertex to check}
  \end{itemize}
}%end item
\item{{\bf  Returns} -- 
true if the vertex is a compound, false otherwise 
}%end item
\end{itemize}
}%end item
\item{ 
\index{setRegister(LayoutRegister)}
{\bf  setRegister}\\
\begin{lstlisting}[frame=none]
protected static void setRegister(plugin.LayoutRegister register)\end{lstlisting} %end signature
\begin{itemize}
\item{
{\bf  Description}

Sets the \texttt{\small LayoutRegister}{\small 
\refdefined{plugin.LayoutRegister}}, which stores the available \texttt{\small LayoutOption}{\small 
\refdefined{plugin.LayoutOption}} for all method graphs statically.
}
\item{
{\bf  Parameters}
  \begin{itemize}
   \item{
\texttt{register} -- The \texttt{\small LayoutRegister}{\small 
\refdefined{plugin.LayoutRegister}} that will be set.}
  \end{itemize}
}%end item
\end{itemize}
}%end item
\end{itemize}
}
}
\subsection{\label{joana.MethodGraphBuilder}\index{MethodGraphBuilder}Class MethodGraphBuilder}{
\vskip .1in 
The MethodGraphBuilder is a \texttt{\small IGraphBuilder}{\small 
\refdefined{graphmodel.IGraphBuilder}}, specifically for building \texttt{\small MethodGraph}{\small 
\refdefined{joana.MethodGraph}}.\vskip .1in 
\subsubsection{Declaration}{
\begin{lstlisting}[frame=none]
public class MethodGraphBuilder
 extends java.lang.Object implements graphmodel.IGraphBuilder\end{lstlisting}
\subsubsection{Constructors}{
\vskip -2em
\begin{itemize}
\item{ 
\index{MethodGraphBuilder()}
{\bf  MethodGraphBuilder}\\
\begin{lstlisting}[frame=none]
public MethodGraphBuilder()\end{lstlisting} %end signature
}%end item
\end{itemize}
}
\subsubsection{Methods}{
\vskip -2em
\begin{itemize}
\item{ 
\index{build()}
{\bf  build}\\
\begin{lstlisting}[frame=none]
graphmodel.Graph build()\end{lstlisting} %end signature
\begin{itemize}
\item{
{\bf  Description copied from graphmodel.IGraphBuilder{\small \refdefined{graphmodel.IGraphBuilder}} }

Builds a graph from the given settings and returns it.
}
\item{{\bf  Returns} -- 
The graph that is being build by the IGraphBuilder. 
}%end item
\end{itemize}
}%end item
\item{ 
\index{getEdgeBuilder()}
{\bf  getEdgeBuilder}\\
\begin{lstlisting}[frame=none]
graphmodel.IEdgeBuilder getEdgeBuilder()\end{lstlisting} %end signature
\begin{itemize}
\item{
{\bf  Description copied from graphmodel.IGraphBuilder{\small \refdefined{graphmodel.IGraphBuilder}} }

Returns the EdgeBuilder which is specified for this graph.
}
\item{{\bf  Returns} -- 
The \texttt{\small IEdgeBuilder}{\small 
\refdefined{graphmodel.IEdgeBuilder}} which is specified for this graph. 
}%end item
\end{itemize}
}%end item
\item{ 
\index{getVertexBuilder(String)}
{\bf  getVertexBuilder}\\
\begin{lstlisting}[frame=none]
graphmodel.IVertexBuilder getVertexBuilder(java.lang.String vertexID)\end{lstlisting} %end signature
\begin{itemize}
\item{
{\bf  Description copied from graphmodel.IGraphBuilder{\small \refdefined{graphmodel.IGraphBuilder}} }

Returns the VertexBuilder which is specified for this graph.
}
\item{
{\bf  Parameters}
  \begin{itemize}
   \item{
\texttt{vertexID} -- The id of the vertex which associated IVertexBuilder will be returned.}
  \end{itemize}
}%end item
\item{{\bf  Returns} -- 
The \texttt{\small IVertexBuilder}{\small 
\refdefined{graphmodel.IVertexBuilder}} which is specified for this graph. 
}%end item
\end{itemize}
}%end item
\end{itemize}
}
}
\subsection{\label{joana.MethodGraphLayout}\index{MethodGraphLayout}Class MethodGraphLayout}{
\vskip .1in 
Implements hierarchical layout with layers for \texttt{\small MethodGraph}{\small 
\refdefined{joana.MethodGraph}}. This graph contains field access subgraphs.\vskip .1in 
\subsubsection{Declaration}{
\begin{lstlisting}[frame=none]
public class MethodGraphLayout
 extends java.lang.Object implements sugiyama.LayeredLayoutAlgorithm\end{lstlisting}
\subsubsection{Constructors}{
\vskip -2em
\begin{itemize}
\item{ 
\index{MethodGraphLayout()}
{\bf  MethodGraphLayout}\\
\begin{lstlisting}[frame=none]
public MethodGraphLayout()\end{lstlisting} %end signature
}%end item
\end{itemize}
}
\subsubsection{Methods}{
\vskip -2em
\begin{itemize}
\item{ 
\index{getSettings()}
{\bf  getSettings}\\
\begin{lstlisting}[frame=none]
public parameter.Settings getSettings()\end{lstlisting} %end signature
}%end item
\item{ 
\index{layout(MethodGraph)}
{\bf  layout}\\
\begin{lstlisting}[frame=none]
public void layout(MethodGraph graph)\end{lstlisting} %end signature
\begin{itemize}
\item{
{\bf  Description}

Layouts a single \texttt{\small MethodGraph}{\small 
\refdefined{joana.MethodGraph}} with the configured settings.
}
\item{
{\bf  Parameters}
  \begin{itemize}
   \item{
\texttt{graph} -- The \texttt{\small MethodGraph}{\small 
\refdefined{joana.MethodGraph}} to layout.}
  \end{itemize}
}%end item
\end{itemize}
}%end item
\item{ 
\index{layoutLayeredGraph(LayeredGraph)}
{\bf  layoutLayeredGraph}\\
\begin{lstlisting}[frame=none]
void layoutLayeredGraph(graphmodel.LayeredGraph graph)\end{lstlisting} %end signature
\begin{itemize}
\item{
{\bf  Description copied from sugiyama.LayeredLayoutAlgorithm{\small \refdefined{sugiyama.LayeredLayoutAlgorithm}} }

Applies its layout to a graph as in \texttt{\small layout(G graph)} but keeps the notion of layers. The algorithm will assign every vertex a coordinate and every edge a path. Additionally every vertex will be assigned a position in a layer in the LayeredGraph. A possible application is drawing of recursive graphs.
}
\item{
{\bf  Parameters}
  \begin{itemize}
   \item{
\texttt{graph} -- the graph to apply the layout to}
  \end{itemize}
}%end item
\end{itemize}
}%end item
\end{itemize}
}
}
\subsection{\label{joana.MethodGraphLayoutOption}\index{MethodGraphLayoutOption}Class MethodGraphLayoutOption}{
\vskip .1in 
A \texttt{\small LayoutOption}{\small 
\refdefined{plugin.LayoutOption}} which is specific for \texttt{\small MethodGraph}{\small 
\refdefined{joana.MethodGraph}}.\vskip .1in 
\subsubsection{Declaration}{
\begin{lstlisting}[frame=none]
public abstract class MethodGraphLayoutOption
 extends plugin.LayoutOption\end{lstlisting}
\subsubsection{Constructors}{
\vskip -2em
\begin{itemize}
\item{ 
\index{MethodGraphLayoutOption()}
{\bf  MethodGraphLayoutOption}\\
\begin{lstlisting}[frame=none]
public MethodGraphLayoutOption()\end{lstlisting} %end signature
}%end item
\end{itemize}
}
\subsubsection{Methods}{
\vskip -2em
\begin{itemize}
\item{ 
\index{applyLayout()}
{\bf  applyLayout}\\
\begin{lstlisting}[frame=none]
public abstract void applyLayout()\end{lstlisting} %end signature
\begin{itemize}
\item{
{\bf  Description copied from plugin.LayoutOption{\small \refdefined{plugin.LayoutOption}} }

This should execute the layout on the graph, which should be specified on construction, or in beforehand. The settings, which are accessible over \texttt{\small getSettings()} will be used to instantiate the LayoutAlgorithm.
}
\end{itemize}
}%end item
\item{ 
\index{chooseLayout()}
{\bf  chooseLayout}\\
\begin{lstlisting}[frame=none]
public abstract void chooseLayout()\end{lstlisting} %end signature
\begin{itemize}
\item{
{\bf  Description copied from plugin.LayoutOption{\small \refdefined{plugin.LayoutOption}} }

Called when this layout option is chosen. This allows the layout option to prepare the actual LayoutAlgorithm.
}
\end{itemize}
}%end item
\item{ 
\index{setGraph(MethodGraph)}
{\bf  setGraph}\\
\begin{lstlisting}[frame=none]
public void setGraph(MethodGraph graph)\end{lstlisting} %end signature
\begin{itemize}
\item{
{\bf  Description}

Sets the \texttt{\small MethodGraph}{\small 
\refdefined{joana.MethodGraph}} that will be the target of the CallGraphLayoutOption.
}
\item{
{\bf  Parameters}
  \begin{itemize}
   \item{
\texttt{graph} -- The \texttt{\small MethodGraph}{\small 
\refdefined{joana.MethodGraph}} that will be the target of the MethodGraphLayoutOption.}
  \end{itemize}
}%end item
\end{itemize}
}%end item
\item{ 
\index{setLayout(LayoutAlgorithm)}
{\bf  setLayout}\\
\begin{lstlisting}[frame=none]
public void setLayout(plugin.LayoutAlgorithm layout)\end{lstlisting} %end signature
\begin{itemize}
\item{
{\bf  Description}

Sets the LayoutAlgorithm that will be used to layout the set graph.
}
\item{
{\bf  Parameters}
  \begin{itemize}
   \item{
\texttt{layout} -- The LayoutAlgorithm that will be used to layout the set graph.}
  \end{itemize}
}%end item
\end{itemize}
}%end item
\end{itemize}
}
}
}
\printindex
