\chapter{Einleitung}
\label{ch:einleitung}

Das Programm Graph von Ansicht soll, wie bereits im Pflichtenheft erläutert, der Visualisierung von Graphen dienen. Hierzu wurden im besagten Dokument Kriterien erarbeitet, welche zum erreichen dieses Ziels, notwendigerweise umgesetzt werden sollen. Die Erarbeitung dieser Pflichtkriterien, in Form eines Entwurfs, ist Ziel dieses Heftes. Im weiteren wird aber auch versucht die Umsetzung der Wunschkriterienmöglichst früh zu beachten und durch einen allgemeineren Entwurf, nach Möglichkeit, nicht zu verhindern. \\
Zur Umsetzung der genannten Kriterien wurden diese zunächst getrennt voneinander betrachtet, danach wurden die gröberen Designstrukturen die für diese relevant sind etabliert. So soll das bestimmte Ziel der Erweiterbarkeit (siehe Pflichtenheft 1.1.4) durch eine Pluginstruktur ermöglicht werden. Hierfür ist es notwendig zu bestimmen welche Bereiche des Programms erweiterbar sein sollen. Das Pflichtkriterium sieht hier vor, dass "`Schnittstellen für Plugins in den Bereichen Import, Export, Layoutalgorithmen, Filter für Knoten- und Kantentypen und weitere Operationen auf einzelne Knoten und Kanten"' existieren, welche in Form von Interfaces in dem Paket plugin zur Implementierung zur verfügung gestellt werden. Auch Teil dieses Pakets ist der PluginManager, welche die Plugins verwaltet und bei dem sie ihre Funktionalität registrieren, sowie ein Interface Plugin, welches dann erweitert und geladen werden soll.\\
Die in Input/Output (siehe Pflichtenheft 1.1.2) sowie Steuerung (siehe Pflichtenheft 1.1.4) gestellten Anforderungen legen eine Trennung zwischen Datenmodel und Anzeige nahe, um auf der einen Seite verschiedene Graphtypen zu unterstützen, die auf der anderen Seite wiederum nicht der GUI bekannt sein sollten, da diese auch von Plugins gestellt werden können. Diese Trennung erfolgt in Form der Pakete graphmodel und gui. Das graphmodel beinhaltet grundlegende Interfaces, welche dann von spezifischen graphen, wie Beispielsweise dem in diesem Paket liegendem DefaultGraph, implementiert werden. Das Paket gui wiederum enthält die Anzeigen logik sowie die Kontrolle dieser Anzeige und baut stark auf javafx. Insgesammt wird so eine MVC-artige Struktur aufgebaut, was wiederum der Kapselung zugute kommt.\\
Ein Paket objectproperty dient der weiteren Verknüpfung zwischen Model und View, hier wird ein von javafx bereitgestelltes Visitor Entwurfsmuster verwendet, um die Daten in der Anzeige bei Veränderung anzupassen.\\
Zudem werden die Plugins Sugiyama, Joana und graphml importer als Pakete modeliert.\\
Das Sugiyama Plugin liefert eine allgemeine, erweiterbare Implementierung des Sugiyama-Frameworks. Es stellt Interfaces bereit die von anderen Plugins genutzt werden können um ihre graphen mit diesem Plugin zu layouten. Zusätzlich werden die einzelnen Phasen des Frameworks in Form eines Strategy Patterns austauschbar gemacht, sodass man, beispielsweise an Stelle der heuristischen Methoden, optimale Lösungen implementieren könnte.\\
Das Joana Plugin liefert Algorithmen zur spezifischen Bearbeitung von Joana Graphtypen, die auch hier modeliert werden. Es stützt sich stark auf das Sugiyama Plugin, welches es erweitert. Hier werden auch joana spezifische Constraints sowie Filter geliefert\\
Der GraphML Importer implementiert das Importer Interface aus plugin und soll graphml-Dateien importieren können.\\
Allgemeine Kriterien(siehe 1.1.1), wie das bereitstellen von Tabs für verschiedene Graphen (siehe /FA260/), werden häufig schon von javafx ermöglicht.
