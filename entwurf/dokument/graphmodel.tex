\section{Package graphmodel}{
\label{graphmodel}\hskip -.05in
\hbox to \hsize{\textit{ Package Contents\hfil Page}}
\vskip .13in
\hbox{{\bf  Interfaces}}
\entityintro{CompoundVertex}{graphmodel.CompoundVertex}{A interface of a vertex that contains an entire subgraph.}
\entityintro{DirectedGraph}{graphmodel.DirectedGraph}{A \texttt{\small DirectedGraph}{\small 
\refdefined{graphmodel.DirectedGraph}} is a specific Graph which contains just \texttt{\small DirectedEdge}{\small 
\refdefined{graphmodel.DirectedEdge}} as edges}
\entityintro{Edge}{graphmodel.Edge}{This edge interface specifies an edge.}
\entityintro{Graph}{graphmodel.Graph}{This graph interface specifies a graph.}
\entityintro{IEdgeBuilder}{graphmodel.IEdgeBuilder}{An abstract interface, which is used to build one edge.}
\entityintro{IGraphBuilder}{graphmodel.IGraphBuilder}{An abstract interface, which is used to build a graph.}
\entityintro{IGraphModelBuilder}{graphmodel.IGraphModelBuilder}{An abstract interface, which is used to build a graphmodel.}
\entityintro{IVertexBuilder}{graphmodel.IVertexBuilder}{An abstract interface, which is used to build one vertex.}
\entityintro{LayeredGraph}{graphmodel.LayeredGraph}{A DirectedGraph which in addition to coordinates saves the relative position of all vertices in a layered structure.}
\entityintro{Vertex}{graphmodel.Vertex}{This vertex interface specifies a vertex.}
\entityintro{Viewable}{graphmodel.Viewable}{Adds methods for manipulation and observation of graphs.}
\entityintro{ViewableGraph}{graphmodel.ViewableGraph}{The base graph accessed by the UI.}
\vskip .13in
\hbox{{\bf  Classes}}
\entityintro{DefaultDirectedGraph}{graphmodel.DefaultDirectedGraph}{A \texttt{\small DefaultDirectedGraph}{\small 
\refdefined{graphmodel.DefaultDirectedGraph}} is a specific Graph which only contains \texttt{\small DirectedEdge}{\small 
\refdefined{graphmodel.DirectedEdge}} as edges.}
\entityintro{DefaultVertex}{graphmodel.DefaultVertex}{This is an DefaultVertex, which has basic functions and is provided by the main application.}
\entityintro{DirectedEdge}{graphmodel.DirectedEdge}{A \texttt{\small DirectedEdge}{\small 
\refdefined{graphmodel.DirectedEdge}} is an edge that has one source and one target vertex.}
\entityintro{DirectedGraphLayoutOption}{graphmodel.DirectedGraphLayoutOption}{A \texttt{\small LayoutOption}{\small 
\refdefined{plugin.LayoutOption}} which is specific for \texttt{\small DirectedGraph}{\small 
\refdefined{graphmodel.DirectedGraph}}.}
\entityintro{DirectedGraphLayoutRegister}{graphmodel.DirectedGraphLayoutRegister}{A \texttt{\small LayoutRegister}{\small 
\refdefined{plugin.LayoutRegister}} which is specific for \texttt{\small DirectedGraphLayoutOption}{\small 
\refdefined{graphmodel.DirectedGraphLayoutOption}}.}
\entityintro{EdgePath}{graphmodel.EdgePath}{An abstract super class for edge paths.}
\entityintro{EdgePath.Point}{graphmodel.EdgePath.Point}{This class is a standard immutable 2D Vector with integer values as it's components.}
\entityintro{FastGraphAccessor}{graphmodel.FastGraphAccessor}{This class provides a fast lookup of \texttt{\small Vertex}{\small 
\refdefined{graphmodel.Vertex}} and \texttt{\small Edge}{\small 
\refdefined{graphmodel.Edge}} for a given Attribute value pair without traversing a \texttt{\small Graph}{\small 
\refdefined{graphmodel.Graph}}.}
\entityintro{GraphModel}{graphmodel.GraphModel}{A GraphModel contains one or more graphs.}
\entityintro{Group}{graphmodel.Group}{This class allows to collect an amount of vertices.}
\entityintro{OrthogonalEdgePath}{graphmodel.OrthogonalEdgePath}{An orthogonal edge path used as standard graphical edge representation.}
\entityintro{SerializedEdge}{graphmodel.SerializedEdge}{A serialized version of a \texttt{\small Edge}{\small 
\refdefined{graphmodel.Edge}}.}
\entityintro{SerializedGraph}{graphmodel.SerializedGraph}{A serialized version of a \texttt{\small Graph}{\small 
\refdefined{graphmodel.Graph}}.}
\entityintro{SerializedVertex}{graphmodel.SerializedVertex}{A serialized version of a \texttt{\small Vertex}{\small 
\refdefined{graphmodel.Vertex}}.}
\vskip .1in
\vskip .1in
\subsection{\label{graphmodel.CompoundVertex}\index{CompoundVertex@\textit{ CompoundVertex}}Interface CompoundVertex}{
\vskip .1in 
A interface of a vertex that contains an entire subgraph.\vskip .1in 
\subsubsection{Declaration}{
\begin{lstlisting}[frame=none]
public interface CompoundVertex
 extends Vertex\end{lstlisting}
\subsubsection{Methods}{
\vskip -2em
\begin{itemize}
\item{ 
\index{getConnectedVertex(Edge)}
{\bf  getConnectedVertex}\\
\begin{lstlisting}[frame=none]
Vertex getConnectedVertex(Edge edge)\end{lstlisting} %end signature
\begin{itemize}
\item{
{\bf  Description}

Returns the connected vertex contained in the compound vertex for a given edge, where one end point of the edge has to be this vertex.
}
\item{
{\bf  Parameters}
  \begin{itemize}
   \item{
\texttt{edge} -- the edge to get the vertex to}
  \end{itemize}
}%end item
\item{{\bf  Returns} -- 
the connected vertex if the edge is valid 
}%end item
\end{itemize}
}%end item
\item{ 
\index{getGraph()}
{\bf  getGraph}\\
\begin{lstlisting}[frame=none]
Graph getGraph()\end{lstlisting} %end signature
\begin{itemize}
\item{
{\bf  Description}

Returns the graph contained in the vertex.
}
\item{{\bf  Returns} -- 
the graph contained in the vertex. 
}%end item
\end{itemize}
}%end item
\end{itemize}
}
}
\subsection{\label{graphmodel.DirectedGraph}\index{DirectedGraph@\textit{ DirectedGraph}}Interface DirectedGraph}{
\vskip .1in 
A \texttt{\small DirectedGraph}{\small 
\refdefined{graphmodel.DirectedGraph}} is a specific Graph which contains just \texttt{\small DirectedEdge}{\small 
\refdefined{graphmodel.DirectedEdge}} as edges\vskip .1in 
\subsubsection{Declaration}{
\begin{lstlisting}[frame=none]
public interface DirectedGraph
 extends Graph\end{lstlisting}
\subsubsection{All known subinterfaces}{LayeredGraph\small{\refdefined{graphmodel.LayeredGraph}}, DefaultDirectedGraph\small{\refdefined{graphmodel.DefaultDirectedGraph}}}
\subsubsection{All classes known to implement interface}{DefaultDirectedGraph\small{\refdefined{graphmodel.DefaultDirectedGraph}}}
\subsubsection{Methods}{
\vskip -2em
\begin{itemize}
\item{ 
\index{incomingEdgesOf(V)}
{\bf  incomingEdgesOf}\\
\begin{lstlisting}[frame=none]
java.util.Set incomingEdgesOf(Vertex vertex)\end{lstlisting} %end signature
\begin{itemize}
\item{
{\bf  Description}

Returns a set of all incoming edges of a vertex.
}
\item{
{\bf  Parameters}
  \begin{itemize}
   \item{
\texttt{vertex} -- Vertex whose incoming edges will be returned.}
  \end{itemize}
}%end item
\item{{\bf  Returns} -- 
The edges coming in the supplied vertex. 
}%end item
\end{itemize}
}%end item
\item{ 
\index{indegreeOf(V)}
{\bf  indegreeOf}\\
\begin{lstlisting}[frame=none]
java.lang.Integer indegreeOf(Vertex vertex)\end{lstlisting} %end signature
\begin{itemize}
\item{
{\bf  Description}

Returns the indegree of a vertex of the graph.
}
\item{
{\bf  Parameters}
  \begin{itemize}
   \item{
\texttt{vertex} -- Vertex whose indegree will be returned.}
  \end{itemize}
}%end item
\item{{\bf  Returns} -- 
The number of edges going into the supplied vertex. 
}%end item
\end{itemize}
}%end item
\item{ 
\index{outdegreeOf(V)}
{\bf  outdegreeOf}\\
\begin{lstlisting}[frame=none]
java.lang.Integer outdegreeOf(Vertex vertex)\end{lstlisting} %end signature
\begin{itemize}
\item{
{\bf  Description}

Returns the outdegree of a vertex of the graph.
}
\item{
{\bf  Parameters}
  \begin{itemize}
   \item{
\texttt{vertex} -- Vertex whose outdegree will be returned.}
  \end{itemize}
}%end item
\item{{\bf  Returns} -- 
The number of edges going out of the supplied vertex. 
}%end item
\end{itemize}
}%end item
\item{ 
\index{outgoingEdgesOf(V)}
{\bf  outgoingEdgesOf}\\
\begin{lstlisting}[frame=none]
java.util.Set outgoingEdgesOf(Vertex vertex)\end{lstlisting} %end signature
\begin{itemize}
\item{
{\bf  Description}

Returns a set of all outgoing edges of a vertex.
}
\item{
{\bf  Parameters}
  \begin{itemize}
   \item{
\texttt{vertex} -- Vertex whose outgoing edges will be returned.}
  \end{itemize}
}%end item
\item{{\bf  Returns} -- 
The edges going out of the supplied vertex. 
}%end item
\end{itemize}
}%end item
\end{itemize}
}
}
\subsection{\label{graphmodel.Edge}\index{Edge@\textit{ Edge}}Interface Edge}{
\vskip .1in 
This edge interface specifies an edge. An edge contains two vertices, an ID, a name and a label.\vskip .1in 
\subsubsection{Declaration}{
\begin{lstlisting}[frame=none]
public interface Edge
\end{lstlisting}
\subsubsection{All known subinterfaces}{SerializedEdge\small{\refdefined{graphmodel.SerializedEdge}}, DirectedEdge\small{\refdefined{graphmodel.DirectedEdge}}}
\subsubsection{All classes known to implement interface}{SerializedEdge\small{\refdefined{graphmodel.SerializedEdge}}, DirectedEdge\small{\refdefined{graphmodel.DirectedEdge}}}
\subsubsection{Methods}{
\vskip -2em
\begin{itemize}
\item{ 
\index{addToFastGraphAccessor(FastGraphAccessor)}
{\bf  addToFastGraphAccessor}\\
\begin{lstlisting}[frame=none]
void addToFastGraphAccessor(FastGraphAccessor fga)\end{lstlisting} %end signature
\begin{itemize}
\item{
{\bf  Description}

Adds the edge to a \texttt{\small FastGraphAccessor}{\small 
\refdefined{graphmodel.FastGraphAccessor}}.
}
\item{
{\bf  Parameters}
  \begin{itemize}
   \item{
\texttt{fga} -- The \texttt{\small FastGraphAccessor}{\small 
\refdefined{graphmodel.FastGraphAccessor}} to whom this edge will be added.}
  \end{itemize}
}%end item
\end{itemize}
}%end item
\item{ 
\index{getID()}
{\bf  getID}\\
\begin{lstlisting}[frame=none]
java.lang.Integer getID()\end{lstlisting} %end signature
\begin{itemize}
\item{
{\bf  Description}

Returns the ID of the edge.
}
\item{{\bf  Returns} -- 
The id of the edge. 
}%end item
\end{itemize}
}%end item
\item{ 
\index{getLabel()}
{\bf  getLabel}\\
\begin{lstlisting}[frame=none]
java.lang.String getLabel()\end{lstlisting} %end signature
\begin{itemize}
\item{
{\bf  Description}

Returns the label of the edge.
}
\item{{\bf  Returns} -- 
The label of the edge. 
}%end item
\end{itemize}
}%end item
\item{ 
\index{getName()}
{\bf  getName}\\
\begin{lstlisting}[frame=none]
java.lang.String getName()\end{lstlisting} %end signature
\begin{itemize}
\item{
{\bf  Description}

Returns the name of the edge.
}
\item{{\bf  Returns} -- 
The name of the edge. 
}%end item
\end{itemize}
}%end item
\item{ 
\index{getPath()}
{\bf  getPath}\\
\begin{lstlisting}[frame=none]
EdgePath getPath()\end{lstlisting} %end signature
\begin{itemize}
\item{
{\bf  Description}

Returns the \texttt{\small EdgePath}{\small 
\refdefined{graphmodel.EdgePath}} of the edge. The edge path is attached to the edge and cannot be replaced.
}
\item{{\bf  Returns} -- 
the edge path 
}%end item
\end{itemize}
}%end item
\item{ 
\index{getVertices()}
{\bf  getVertices}\\
\begin{lstlisting}[frame=none]
java.util.List getVertices()\end{lstlisting} %end signature
\begin{itemize}
\item{
{\bf  Description}

Returns the vertices connected with this edge.
}
\item{{\bf  Returns} -- 
The vertices connected with the edge. 
}%end item
\end{itemize}
}%end item
\item{ 
\index{serialize(List)}
{\bf  serialize}\\
\begin{lstlisting}[frame=none]
SerializedEdge serialize(java.util.List attributes)\end{lstlisting} %end signature
\begin{itemize}
\item{
{\bf  Description}

Returns a \texttt{\small SerializedEdge}{\small 
\refdefined{graphmodel.SerializedEdge}} representation of the edge.
}
\item{
{\bf  Parameters}
  \begin{itemize}
   \item{
\texttt{attributes} -- The attributes that have to be serialized.}
  \end{itemize}
}%end item
\item{{\bf  Returns} -- 
The \texttt{\small SerializedEdge}{\small 
\refdefined{graphmodel.SerializedEdge}} representation of the edge. 
}%end item
\end{itemize}
}%end item
\end{itemize}
}
}
\subsection{\label{graphmodel.Graph}\index{Graph@\textit{ Graph}}Interface Graph}{
\vskip .1in 
This graph interface specifies a graph. A graph contains edges and vertices.\vskip .1in 
\subsubsection{Declaration}{
\begin{lstlisting}[frame=none]
public interface Graph
\end{lstlisting}
\subsubsection{All known subinterfaces}{ViewableGraph\small{\refdefined{graphmodel.ViewableGraph}}, SerializedGraph\small{\refdefined{graphmodel.SerializedGraph}}, LayeredGraph\small{\refdefined{graphmodel.LayeredGraph}}, DirectedGraph\small{\refdefined{graphmodel.DirectedGraph}}, DefaultDirectedGraph\small{\refdefined{graphmodel.DefaultDirectedGraph}}}
\subsubsection{All classes known to implement interface}{SerializedGraph\small{\refdefined{graphmodel.SerializedGraph}}}
\subsubsection{Methods}{
\vskip -2em
\begin{itemize}
\item{ 
\index{addToFastGraphAccessor(FastGraphAccessor)}
{\bf  addToFastGraphAccessor}\\
\begin{lstlisting}[frame=none]
void addToFastGraphAccessor(FastGraphAccessor fga)\end{lstlisting} %end signature
\begin{itemize}
\item{
{\bf  Description}

Adds the graph to a \texttt{\small FastGraphAccessor}{\small 
\refdefined{graphmodel.FastGraphAccessor}}.
}
\item{
{\bf  Parameters}
  \begin{itemize}
   \item{
\texttt{fga} -- the \texttt{\small FastGraphAccessor}{\small 
\refdefined{graphmodel.FastGraphAccessor}} to whom this graph will be added.}
  \end{itemize}
}%end item
\end{itemize}
}%end item
\item{ 
\index{edgesOf(V)}
{\bf  edgesOf}\\
\begin{lstlisting}[frame=none]
java.util.Set edgesOf(Vertex vertex)\end{lstlisting} %end signature
\begin{itemize}
\item{
{\bf  Description}

Returns a list of all edges of a vertex.
}
\item{
{\bf  Parameters}
  \begin{itemize}
   \item{
\texttt{vertex} -- the vertex which edges will be returned.}
  \end{itemize}
}%end item
\item{{\bf  Returns} -- 
All edges which are connected with the supplied vertex. 
}%end item
\end{itemize}
}%end item
\item{ 
\index{getEdgeSet()}
{\bf  getEdgeSet}\\
\begin{lstlisting}[frame=none]
java.util.Set getEdgeSet()\end{lstlisting} %end signature
\begin{itemize}
\item{
{\bf  Description}

Returns all edges of the graph.
}
\item{{\bf  Returns} -- 
A set of all edges of the graph. 
}%end item
\end{itemize}
}%end item
\item{ 
\index{getFastGraphAccessor()}
{\bf  getFastGraphAccessor}\\
\begin{lstlisting}[frame=none]
FastGraphAccessor getFastGraphAccessor()\end{lstlisting} %end signature
\begin{itemize}
\item{
{\bf  Description}

Returns the FastGraphAccessor of this Graph.
}
\item{{\bf  Returns} -- 
the FastGraphAccessor of this Graph 
}%end item
\end{itemize}
}%end item
\item{ 
\index{getID()}
{\bf  getID}\\
\begin{lstlisting}[frame=none]
java.lang.Integer getID()\end{lstlisting} %end signature
\begin{itemize}
\item{
{\bf  Description}

Returns the ID of the graph.
}
\item{{\bf  Returns} -- 
The id of the graph. 
}%end item
\end{itemize}
}%end item
\item{ 
\index{getName()}
{\bf  getName}\\
\begin{lstlisting}[frame=none]
java.lang.String getName()\end{lstlisting} %end signature
\begin{itemize}
\item{
{\bf  Description}

Returns the name of the Graph.
}
\item{{\bf  Returns} -- 
The name of the graph. 
}%end item
\end{itemize}
}%end item
\item{ 
\index{getVertexSet()}
{\bf  getVertexSet}\\
\begin{lstlisting}[frame=none]
java.util.Set getVertexSet()\end{lstlisting} %end signature
\begin{itemize}
\item{
{\bf  Description}

Returns all vertices of the graph.
}
\item{{\bf  Returns} -- 
A set of all vertices of the graph. 
}%end item
\end{itemize}
}%end item
\end{itemize}
}
}
\subsection{\label{graphmodel.IEdgeBuilder}\index{IEdgeBuilder@\textit{ IEdgeBuilder}}Interface IEdgeBuilder}{
\vskip .1in 
An abstract interface, which is used to build one edge.\vskip .1in 
\subsubsection{Declaration}{
\begin{lstlisting}[frame=none]
public interface IEdgeBuilder
\end{lstlisting}
\subsubsection{Methods}{
\vskip -2em
\begin{itemize}
\item{ 
\index{addData(String, String)}
{\bf  addData}\\
\begin{lstlisting}[frame=none]
void addData(java.lang.String keyname,java.lang.String value)\end{lstlisting} %end signature
\begin{itemize}
\item{
{\bf  Description}

Adds additional data to this edge. The specific EdgeBuilder implementation needs to decide how to save the value for given edge type.
}
\item{
{\bf  Parameters}
  \begin{itemize}
   \item{
\texttt{keyname} -- Name of the attribute}
   \item{
\texttt{value} -- Value of the attribute}
  \end{itemize}
}%end item
\end{itemize}
}%end item
\item{ 
\index{build()}
{\bf  build}\\
\begin{lstlisting}[frame=none]
Edge build()\end{lstlisting} %end signature
\begin{itemize}
\item{
{\bf  Description}

. Builds an Edge with the given Data and returns it.
}
\item{{\bf  Returns} -- 
The Edge that is being build by the IEdgeBuilder 
}%end item
\end{itemize}
}%end item
\item{ 
\index{newEdge(String, String)}
{\bf  newEdge}\\
\begin{lstlisting}[frame=none]
void newEdge(java.lang.String source,java.lang.String target)\end{lstlisting} %end signature
\begin{itemize}
\item{
{\bf  Description}

Sets source and target vertices of the edge build by this.
}
\item{
{\bf  Parameters}
  \begin{itemize}
   \item{
\texttt{source} -- String representation of the source vertex as ID}
   \item{
\texttt{target} -- String representation of the target vertex as ID}
  \end{itemize}
}%end item
\end{itemize}
}%end item
\item{ 
\index{setDirection(String)}
{\bf  setDirection}\\
\begin{lstlisting}[frame=none]
void setDirection(java.lang.String direction)\end{lstlisting} %end signature
\begin{itemize}
\item{
{\bf  Description}

Sets the direction of the edge build by this.
}
\item{
{\bf  Parameters}
  \begin{itemize}
   \item{
\texttt{direction} -- String representation of the direction. Can be one of}
  \end{itemize}
}%end item
\end{itemize}
}%end item
\item{ 
\index{setID(String)}
{\bf  setID}\\
\begin{lstlisting}[frame=none]
void setID(java.lang.String id)\end{lstlisting} %end signature
\begin{itemize}
\item{
{\bf  Description}

Sets the ID of the edge build by this.
}
\item{
{\bf  Parameters}
  \begin{itemize}
   \item{
\texttt{id} -- value to which the id is set}
  \end{itemize}
}%end item
\end{itemize}
}%end item
\end{itemize}
}
}
\subsection{\label{graphmodel.IGraphBuilder}\index{IGraphBuilder@\textit{ IGraphBuilder}}Interface IGraphBuilder}{
\vskip .1in 
An abstract interface, which is used to build a graph.\vskip .1in 
\subsubsection{Declaration}{
\begin{lstlisting}[frame=none]
public interface IGraphBuilder
\end{lstlisting}
\subsubsection{Methods}{
\vskip -2em
\begin{itemize}
\item{ 
\index{build()}
{\bf  build}\\
\begin{lstlisting}[frame=none]
Graph build()\end{lstlisting} %end signature
\begin{itemize}
\item{
{\bf  Description}

Builds a graph from the given settings and returns it.
}
\item{{\bf  Returns} -- 
The graph that is being build by the IGraphBuilder. 
}%end item
\end{itemize}
}%end item
\item{ 
\index{getEdgeBuilder()}
{\bf  getEdgeBuilder}\\
\begin{lstlisting}[frame=none]
IEdgeBuilder getEdgeBuilder()\end{lstlisting} %end signature
\begin{itemize}
\item{
{\bf  Description}

Returns the EdgeBuilder which is specified for this graph.
}
\item{{\bf  Returns} -- 
The \texttt{\small IEdgeBuilder}{\small 
\refdefined{graphmodel.IEdgeBuilder}} which is specified for this graph. 
}%end item
\end{itemize}
}%end item
\item{ 
\index{getVertexBuilder(String)}
{\bf  getVertexBuilder}\\
\begin{lstlisting}[frame=none]
IVertexBuilder getVertexBuilder(java.lang.String vertexID)\end{lstlisting} %end signature
\begin{itemize}
\item{
{\bf  Description}

Returns the VertexBuilder which is specified for this graph.
}
\item{
{\bf  Parameters}
  \begin{itemize}
   \item{
\texttt{vertexID} -- The id of the vertex which associated IVertexBuilder will be returned.}
  \end{itemize}
}%end item
\item{{\bf  Returns} -- 
The \texttt{\small IVertexBuilder}{\small 
\refdefined{graphmodel.IVertexBuilder}} which is specified for this graph. 
}%end item
\end{itemize}
}%end item
\end{itemize}
}
}
\subsection{\label{graphmodel.IGraphModelBuilder}\index{IGraphModelBuilder@\textit{ IGraphModelBuilder}}Interface IGraphModelBuilder}{
\vskip .1in 
An abstract interface, which is used to build a graphmodel. This class is based on the Builder Pattern.\vskip .1in 
\subsubsection{Declaration}{
\begin{lstlisting}[frame=none]
public interface IGraphModelBuilder
\end{lstlisting}
\subsubsection{Methods}{
\vskip -2em
\begin{itemize}
\item{ 
\index{build()}
{\bf  build}\\
\begin{lstlisting}[frame=none]
GraphModel build()\end{lstlisting} %end signature
\begin{itemize}
\item{
{\bf  Description}

Builds a graphmodel from the given settings and returns it.
}
\item{{\bf  Returns} -- 
The \texttt{\small GraphModel}{\small 
\refdefined{graphmodel.GraphModel}} that is being build by the IGraphModelBuilder. 
}%end item
\end{itemize}
}%end item
\item{ 
\index{getGraphBuilder(String)}
{\bf  getGraphBuilder}\\
\begin{lstlisting}[frame=none]
IGraphBuilder getGraphBuilder(java.lang.String graphID)\end{lstlisting} %end signature
\begin{itemize}
\item{
{\bf  Description}

Returns a specific \texttt{\small IGraphBuilder}{\small 
\refdefined{graphmodel.IGraphBuilder}} for a graph, which belongs to the \texttt{\small GraphModel}{\small 
\refdefined{graphmodel.GraphModel}}.
}
\item{
{\bf  Parameters}
  \begin{itemize}
   \item{
\texttt{graphID} -- The id of the graph which associated \texttt{\small IGraphBuilder}{\small 
\refdefined{graphmodel.IGraphBuilder}} will be returned.}
  \end{itemize}
}%end item
\item{{\bf  Returns} -- 
The IGraphBuilder of the graph which is referenced over the graphID. 
}%end item
\end{itemize}
}%end item
\end{itemize}
}
}
\subsection{\label{graphmodel.IVertexBuilder}\index{IVertexBuilder@\textit{ IVertexBuilder}}Interface IVertexBuilder}{
\vskip .1in 
An abstract interface, which is used to build one vertex.\vskip .1in 
\subsubsection{Declaration}{
\begin{lstlisting}[frame=none]
public interface IVertexBuilder
\end{lstlisting}
\subsubsection{Methods}{
\vskip -2em
\begin{itemize}
\item{ 
\index{addData(String, String)}
{\bf  addData}\\
\begin{lstlisting}[frame=none]
void addData(java.lang.String keyname,java.lang.String value)\end{lstlisting} %end signature
\begin{itemize}
\item{
{\bf  Description}

Add Data to this Vertex. The IVertexBuilder needs to parse the data and add it to the edge
}
\item{
{\bf  Parameters}
  \begin{itemize}
   \item{
\texttt{keyname} -- Name of the attribute which is added}
   \item{
\texttt{value} -- Value of the attribute}
  \end{itemize}
}%end item
\end{itemize}
}%end item
\item{ 
\index{build()}
{\bf  build}\\
\begin{lstlisting}[frame=none]
Vertex build()\end{lstlisting} %end signature
\begin{itemize}
\item{
{\bf  Description}

. Builds a Vertex with the given Data and returns it.
}
\item{{\bf  Returns} -- 
The Vertex that is being build by the IVertexBuilder 
}%end item
\end{itemize}
}%end item
\item{ 
\index{getGraphBuilder(String)}
{\bf  getGraphBuilder}\\
\begin{lstlisting}[frame=none]
IGraphBuilder getGraphBuilder(java.lang.String graphID)\end{lstlisting} %end signature
\begin{itemize}
\item{
{\bf  Description}

This method returns an specific GraphBuilder. This method is used to implement nested Graphs.
}
\item{
{\bf  Parameters}
  \begin{itemize}
   \item{
\texttt{graphID} -- The id of the graph which associated \texttt{\small IGraphBuilder}{\small 
\refdefined{graphmodel.IGraphBuilder}} will be returned.}
  \end{itemize}
}%end item
\item{{\bf  Returns} -- 
The IGraphBuilder of the graph which is referenced over the graphID. 
}%end item
\end{itemize}
}%end item
\item{ 
\index{setID(String)}
{\bf  setID}\\
\begin{lstlisting}[frame=none]
void setID(java.lang.String id)\end{lstlisting} %end signature
\begin{itemize}
\item{
{\bf  Description}

Sets the ID of the vertex build by this.
}
\item{
{\bf  Parameters}
  \begin{itemize}
   \item{
\texttt{id} -- value to which the id is set}
  \end{itemize}
}%end item
\end{itemize}
}%end item
\end{itemize}
}
}
\subsection{\label{graphmodel.LayeredGraph}\index{LayeredGraph@\textit{ LayeredGraph}}Interface LayeredGraph}{
\vskip .1in 
A DirectedGraph which in addition to coordinates saves the relative position of all vertices in a layered structure. Every vertex is in a layer. Every layer is sorted so that every node has zero to two horizontal neighbors.\vskip .1in 
\subsubsection{Declaration}{
\begin{lstlisting}[frame=none]
public interface LayeredGraph
 extends DirectedGraph\end{lstlisting}
\subsubsection{Methods}{
\vskip -2em
\begin{itemize}
\item{ 
\index{getHeight()}
{\bf  getHeight}\\
\begin{lstlisting}[frame=none]
int getHeight()\end{lstlisting} %end signature
\begin{itemize}
\item{
{\bf  Description}

Returns the height, i.e. the number of layers.
}
\item{{\bf  Returns} -- 
the height 
}%end item
\end{itemize}
}%end item
\item{ 
\index{getLayer(int)}
{\bf  getLayer}\\
\begin{lstlisting}[frame=none]
java.util.List getLayer(int layerNum)\end{lstlisting} %end signature
\begin{itemize}
\item{
{\bf  Description}

Get all vertices from a certain layer.
}
\item{
{\bf  Parameters}
  \begin{itemize}
   \item{
\texttt{layerNum} -- the index of the layer}
  \end{itemize}
}%end item
\item{{\bf  Returns} -- 
a list of all vertices which are on this layer 
}%end item
\end{itemize}
}%end item
\item{ 
\index{getLayer(V)}
{\bf  getLayer}\\
\begin{lstlisting}[frame=none]
int getLayer(Vertex vertex)\end{lstlisting} %end signature
\begin{itemize}
\item{
{\bf  Description}

Get the layer from the vertex
}
\item{
{\bf  Parameters}
  \begin{itemize}
   \item{
\texttt{vertex} -- the vertex to get its layer from}
  \end{itemize}
}%end item
\item{{\bf  Returns} -- 
the layer number from this vertex 
}%end item
\end{itemize}
}%end item
\item{ 
\index{getLayerCount()}
{\bf  getLayerCount}\\
\begin{lstlisting}[frame=none]
int getLayerCount()\end{lstlisting} %end signature
\begin{itemize}
\item{
{\bf  Description}

Get the amount of layers.
}
\item{{\bf  Returns} -- 
the amount of layers that contain at least one vertex 
}%end item
\end{itemize}
}%end item
\item{ 
\index{getLayers()}
{\bf  getLayers}\\
\begin{lstlisting}[frame=none]
java.util.List getLayers()\end{lstlisting} %end signature
\begin{itemize}
\item{
{\bf  Description}

Get all layers that contain vertices.
}
\item{{\bf  Returns} -- 
a list of lists of vertices which are on this layer 
}%end item
\end{itemize}
}%end item
\item{ 
\index{getLayerWidth(int)}
{\bf  getLayerWidth}\\
\begin{lstlisting}[frame=none]
int getLayerWidth(int layerN)\end{lstlisting} %end signature
\begin{itemize}
\item{
{\bf  Description}

Returns the width of the layer specified by its index, i.e. the number of vertices in the layer.
}
\item{
{\bf  Parameters}
  \begin{itemize}
   \item{
\texttt{layerN} -- the index of the layer}
  \end{itemize}
}%end item
\item{{\bf  Returns} -- 
the width of the layer 
}%end item
\end{itemize}
}%end item
\item{ 
\index{getMaxWidth()}
{\bf  getMaxWidth}\\
\begin{lstlisting}[frame=none]
int getMaxWidth()\end{lstlisting} %end signature
\begin{itemize}
\item{
{\bf  Description}

Returns the width of the widest layer, i.e. the number of vertices the layer with the most vertices contains.
}
\item{{\bf  Returns} -- 
the maximum width 
}%end item
\end{itemize}
}%end item
\item{ 
\index{getSubgraphs()}
{\bf  getSubgraphs}\\
\begin{lstlisting}[frame=none]
java.util.List getSubgraphs()\end{lstlisting} %end signature
\begin{itemize}
\item{
{\bf  Description}

Returns all subgraphs contained in this graph. All subgraphs of layered graphs have to be layered graphs with equal parameters themselves.
}
\item{{\bf  Returns} -- 
subgraphs in this layered graph 
}%end item
\end{itemize}
}%end item
\item{ 
\index{getVertexCount(int)}
{\bf  getVertexCount}\\
\begin{lstlisting}[frame=none]
int getVertexCount(int layerNum)\end{lstlisting} %end signature
\begin{itemize}
\item{
{\bf  Description}

Get the number of vertices which are on a certain layer
}
\item{
{\bf  Parameters}
  \begin{itemize}
   \item{
\texttt{layerNum} -- the layer number to get the vertex count from}
  \end{itemize}
}%end item
\item{{\bf  Returns} -- 
the number of vertices which are on this layer 
}%end item
\end{itemize}
}%end item
\end{itemize}
}
}
\subsection{\label{graphmodel.Vertex}\index{Vertex@\textit{ Vertex}}Interface Vertex}{
\vskip .1in 
This vertex interface specifies a vertex. Every vertex contains an ID, a name and a label. The ID of a vertex is unique.\vskip .1in 
\subsubsection{Declaration}{
\begin{lstlisting}[frame=none]
public interface Vertex
\end{lstlisting}
\subsubsection{All known subinterfaces}{SerializedVertex\small{\refdefined{graphmodel.SerializedVertex}}, DefaultVertex\small{\refdefined{graphmodel.DefaultVertex}}, CompoundVertex\small{\refdefined{graphmodel.CompoundVertex}}}
\subsubsection{All classes known to implement interface}{SerializedVertex\small{\refdefined{graphmodel.SerializedVertex}}, DefaultVertex\small{\refdefined{graphmodel.DefaultVertex}}}
\subsubsection{Methods}{
\vskip -2em
\begin{itemize}
\item{ 
\index{addToFastGraphAccessor(FastGraphAccessor)}
{\bf  addToFastGraphAccessor}\\
\begin{lstlisting}[frame=none]
void addToFastGraphAccessor(FastGraphAccessor fga)\end{lstlisting} %end signature
\begin{itemize}
\item{
{\bf  Description}

Adds the vertex to a \texttt{\small FastGraphAccessor}{\small 
\refdefined{graphmodel.FastGraphAccessor}}.
}
\item{
{\bf  Parameters}
  \begin{itemize}
   \item{
\texttt{fga} -- The \texttt{\small FastGraphAccessor}{\small 
\refdefined{graphmodel.FastGraphAccessor}} to whom this vertex will be added.}
  \end{itemize}
}%end item
\end{itemize}
}%end item
\item{ 
\index{getID()}
{\bf  getID}\\
\begin{lstlisting}[frame=none]
java.lang.Integer getID()\end{lstlisting} %end signature
\begin{itemize}
\item{
{\bf  Description}

Returns the ID of the vertex. Every vertex in one graph has a unique ID.
}
\item{{\bf  Returns} -- 
The ID of the vertex. 
}%end item
\end{itemize}
}%end item
\item{ 
\index{getLabel()}
{\bf  getLabel}\\
\begin{lstlisting}[frame=none]
java.lang.String getLabel()\end{lstlisting} %end signature
\begin{itemize}
\item{
{\bf  Description}

Returns the label of the vertex, that will be shown in the GUI. The label can be an empty string.
}
\item{{\bf  Returns} -- 
The label of the vertex 
}%end item
\end{itemize}
}%end item
\item{ 
\index{getName()}
{\bf  getName}\\
\begin{lstlisting}[frame=none]
java.lang.String getName()\end{lstlisting} %end signature
\begin{itemize}
\item{
{\bf  Description}

Returns the name of the vertex. A descriptive name of the vertex. Multiple vertices with equal name in one graph are allowed. Therefore don't use this as identifier, instead use \texttt{\small getID()}.
}
\item{{\bf  Returns} -- 
The name of the vertex. 
}%end item
\end{itemize}
}%end item
\item{ 
\index{getX()}
{\bf  getX}\\
\begin{lstlisting}[frame=none]
int getX()\end{lstlisting} %end signature
\begin{itemize}
\item{
{\bf  Description}

Returns the X-coordinate of the vertex.
}
\item{{\bf  Returns} -- 
The X-coordinate of this vertex. 
}%end item
\end{itemize}
}%end item
\item{ 
\index{getY()}
{\bf  getY}\\
\begin{lstlisting}[frame=none]
int getY()\end{lstlisting} %end signature
\begin{itemize}
\item{
{\bf  Description}

Returns the Y-coordinate of the vertex.
}
\item{{\bf  Returns} -- 
The Y-coordinate of the vertex. 
}%end item
\end{itemize}
}%end item
\item{ 
\index{serialize(List)}
{\bf  serialize}\\
\begin{lstlisting}[frame=none]
SerializedVertex serialize(java.util.List attributes)\end{lstlisting} %end signature
\begin{itemize}
\item{
{\bf  Description}

Returns a \texttt{\small SerializedVertex}{\small 
\refdefined{graphmodel.SerializedVertex}} representation of the graph.
}
\item{
{\bf  Parameters}
  \begin{itemize}
   \item{
\texttt{attributes} -- The attributes that have to be serialized.}
  \end{itemize}
}%end item
\item{{\bf  Returns} -- 
The \texttt{\small SerializedVertex}{\small 
\refdefined{graphmodel.SerializedVertex}} representation of the graph. 
}%end item
\end{itemize}
}%end item
\end{itemize}
}
}
\subsection{\label{graphmodel.Viewable}\index{Viewable@\textit{ Viewable}}Interface Viewable}{
\vskip .1in 
Adds methods for manipulation and observation of graphs. This interface differentiates between domain specific graphs, which can be viewed (children of ViewableGraph), and utility graphs like \texttt{\small SerializedGraph}{\small 
\refdefined{graphmodel.SerializedGraph}} and SugiyamaGraph\vskip .1in 
\subsubsection{Declaration}{
\begin{lstlisting}[frame=none]
public interface Viewable
\end{lstlisting}
\subsubsection{All known subinterfaces}{ViewableGraph\small{\refdefined{graphmodel.ViewableGraph}}, DefaultDirectedGraph\small{\refdefined{graphmodel.DefaultDirectedGraph}}}
\subsubsection{Methods}{
\vskip -2em
\begin{itemize}
\item{ 
\index{collapse(Set)}
{\bf  collapse}\\
\begin{lstlisting}[frame=none]
CompoundVertex collapse(java.util.Set subset)\end{lstlisting} %end signature
\begin{itemize}
\item{
{\bf  Description}

Collapses a set of vertices in one compound vertex. The collapsed vertices can be expanded back into their previous state with \texttt{\small expand(CompoundVertex)}.
}
\item{
{\bf  Parameters}
  \begin{itemize}
   \item{
\texttt{subset} -- the subset to collapse}
  \end{itemize}
}%end item
\item{{\bf  Returns} -- 
the resulting collapsed vertex 
}%end item
\end{itemize}
}%end item
\item{ 
\index{expand(CompoundVertex)}
{\bf  expand}\\
\begin{lstlisting}[frame=none]
java.util.Set expand(CompoundVertex vertex)\end{lstlisting} %end signature
\begin{itemize}
\item{
{\bf  Description}

Expands a collapsed vertex into its substituted set of vertices The vertices will be added back to the set of vertices of this graph. The compound vertex will be removed from the set of vertices. All to the compound vertex incident edges, will be resolved back into an edge between the vertices it connected before the collapse.
}
\item{
{\bf  Parameters}
  \begin{itemize}
   \item{
\texttt{vertex} -- the collapsed vertex to expand}
  \end{itemize}
}%end item
\item{{\bf  Returns} -- 
the set of vertices which was substituted by the collapsed vertex 
}%end item
\end{itemize}
}%end item
\item{ 
\index{getDefaultLayout()}
{\bf  getDefaultLayout}\\
\begin{lstlisting}[frame=none]
plugin.LayoutOption getDefaultLayout()\end{lstlisting} %end signature
\begin{itemize}
\item{
{\bf  Description}

Returns the default layout for this graph. This can be called when to quickly get a suiting layout without having to decide between multiple options.
}
\item{{\bf  Returns} -- 
the default layout for this graph 
}%end item
\end{itemize}
}%end item
\item{ 
\index{getRegisteredLayouts()}
{\bf  getRegisteredLayouts}\\
\begin{lstlisting}[frame=none]
java.util.List getRegisteredLayouts()\end{lstlisting} %end signature
\begin{itemize}
\item{
{\bf  Description}

Returns a list of layouts which have been registered at the corresponding LayoutRegister for this graph type. The graph implementing this interface will be set as target of the LayoutOption.
}
\item{{\bf  Returns} -- 
A list of layouts which have been registered at the corresponding LayoutRegister for this graph type. 
}%end item
\end{itemize}
}%end item
\item{ 
\index{isCompound(Vertex)}
{\bf  isCompound}\\
\begin{lstlisting}[frame=none]
boolean isCompound(Vertex vertex)\end{lstlisting} %end signature
\begin{itemize}
\item{
{\bf  Description}

Returns true if the specified vertex is a compound vertex
}
\item{
{\bf  Parameters}
  \begin{itemize}
   \item{
\texttt{vertex} -- the vertex to check}
  \end{itemize}
}%end item
\item{{\bf  Returns} -- 
true if the vertex is a compound, false otherwise 
}%end item
\end{itemize}
}%end item
\item{ 
\index{serialize(List)}
{\bf  serialize}\\
\begin{lstlisting}[frame=none]
SerializedGraph serialize(java.util.List attributes)\end{lstlisting} %end signature
\begin{itemize}
\item{
{\bf  Description}

Returns a \texttt{\small SerializedGraph}{\small 
\refdefined{graphmodel.SerializedGraph}} representation of the graph.
}
\item{
{\bf  Parameters}
  \begin{itemize}
   \item{
\texttt{attributes} -- The attributes that have to be serialized.}
  \end{itemize}
}%end item
\item{{\bf  Returns} -- 
The \texttt{\small SerializedGraph}{\small 
\refdefined{graphmodel.SerializedGraph}} representation of the graph. 
}%end item
\end{itemize}
}%end item
\end{itemize}
}
}
\subsection{\label{graphmodel.ViewableGraph}\index{ViewableGraph@\textit{ ViewableGraph}}Interface ViewableGraph}{
\vskip .1in 
The base graph accessed by the UI.\vskip .1in 
\subsubsection{Declaration}{
\begin{lstlisting}[frame=none]
public interface ViewableGraph
 extends Viewable, Graph\end{lstlisting}
\subsubsection{All known subinterfaces}{DefaultDirectedGraph\small{\refdefined{graphmodel.DefaultDirectedGraph}}}
\subsubsection{All classes known to implement interface}{DefaultDirectedGraph\small{\refdefined{graphmodel.DefaultDirectedGraph}}}
}
\subsection{\label{graphmodel.DefaultDirectedGraph}\index{DefaultDirectedGraph}Class DefaultDirectedGraph}{
\vskip .1in 
A \texttt{\small DefaultDirectedGraph}{\small 
\refdefined{graphmodel.DefaultDirectedGraph}} is a specific Graph which only contains \texttt{\small DirectedEdge}{\small 
\refdefined{graphmodel.DirectedEdge}} as edges.\vskip .1in 
\subsubsection{Declaration}{
\begin{lstlisting}[frame=none]
public class DefaultDirectedGraph
 extends java.lang.Object implements DirectedGraph, ViewableGraph\end{lstlisting}
\subsubsection{Constructors}{
\vskip -2em
\begin{itemize}
\item{ 
\index{DefaultDirectedGraph()}
{\bf  DefaultDirectedGraph}\\
\begin{lstlisting}[frame=none]
public DefaultDirectedGraph()\end{lstlisting} %end signature
}%end item
\end{itemize}
}
\subsubsection{Methods}{
\vskip -2em
\begin{itemize}
\item{ 
\index{addToFastGraphAccessor(FastGraphAccessor)}
{\bf  addToFastGraphAccessor}\\
\begin{lstlisting}[frame=none]
public void addToFastGraphAccessor(FastGraphAccessor fga)\end{lstlisting} %end signature
}%end item
\item{ 
\index{collapse(Set)}
{\bf  collapse}\\
\begin{lstlisting}[frame=none]
public CompoundVertex collapse(java.util.Set subset)\end{lstlisting} %end signature
}%end item
\item{ 
\index{edgesOf(V)}
{\bf  edgesOf}\\
\begin{lstlisting}[frame=none]
public java.util.Set edgesOf(Vertex vertex)\end{lstlisting} %end signature
}%end item
\item{ 
\index{expand(CompoundVertex)}
{\bf  expand}\\
\begin{lstlisting}[frame=none]
public java.util.Set expand(CompoundVertex vertex)\end{lstlisting} %end signature
}%end item
\item{ 
\index{getDefaultLayout()}
{\bf  getDefaultLayout}\\
\begin{lstlisting}[frame=none]
public plugin.LayoutOption getDefaultLayout()\end{lstlisting} %end signature
}%end item
\item{ 
\index{getEdgeSet()}
{\bf  getEdgeSet}\\
\begin{lstlisting}[frame=none]
public java.util.Set getEdgeSet()\end{lstlisting} %end signature
}%end item
\item{ 
\index{getFastGraphAccessor()}
{\bf  getFastGraphAccessor}\\
\begin{lstlisting}[frame=none]
public FastGraphAccessor getFastGraphAccessor()\end{lstlisting} %end signature
}%end item
\item{ 
\index{getID()}
{\bf  getID}\\
\begin{lstlisting}[frame=none]
public java.lang.Integer getID()\end{lstlisting} %end signature
}%end item
\item{ 
\index{getName()}
{\bf  getName}\\
\begin{lstlisting}[frame=none]
public java.lang.String getName()\end{lstlisting} %end signature
}%end item
\item{ 
\index{getRegisteredLayouts()}
{\bf  getRegisteredLayouts}\\
\begin{lstlisting}[frame=none]
public java.util.List getRegisteredLayouts()\end{lstlisting} %end signature
}%end item
\item{ 
\index{getSource(E)}
{\bf  getSource}\\
\begin{lstlisting}[frame=none]
public Vertex getSource(DirectedEdge edge)\end{lstlisting} %end signature
\begin{itemize}
\item{
{\bf  Description}

Returns the source of a edge of the graph.
}
\item{
{\bf  Parameters}
  \begin{itemize}
   \item{
\texttt{edge} -- A edge which is contained in the graph.}
  \end{itemize}
}%end item
\item{{\bf  Returns} -- 
The vertex which the edge is coming from. 
}%end item
\end{itemize}
}%end item
\item{ 
\index{getVertexSet()}
{\bf  getVertexSet}\\
\begin{lstlisting}[frame=none]
public java.util.Set getVertexSet()\end{lstlisting} %end signature
}%end item
\item{ 
\index{incomingEdgesOf(V)}
{\bf  incomingEdgesOf}\\
\begin{lstlisting}[frame=none]
java.util.Set incomingEdgesOf(Vertex vertex)\end{lstlisting} %end signature
\begin{itemize}
\item{
{\bf  Description copied from DirectedGraph{\small \refdefined{graphmodel.DirectedGraph}} }

Returns a set of all incoming edges of a vertex.
}
\item{
{\bf  Parameters}
  \begin{itemize}
   \item{
\texttt{vertex} -- Vertex whose incoming edges will be returned.}
  \end{itemize}
}%end item
\item{{\bf  Returns} -- 
The edges coming in the supplied vertex. 
}%end item
\end{itemize}
}%end item
\item{ 
\index{indegreeOf(V)}
{\bf  indegreeOf}\\
\begin{lstlisting}[frame=none]
java.lang.Integer indegreeOf(Vertex vertex)\end{lstlisting} %end signature
\begin{itemize}
\item{
{\bf  Description copied from DirectedGraph{\small \refdefined{graphmodel.DirectedGraph}} }

Returns the indegree of a vertex of the graph.
}
\item{
{\bf  Parameters}
  \begin{itemize}
   \item{
\texttt{vertex} -- Vertex whose indegree will be returned.}
  \end{itemize}
}%end item
\item{{\bf  Returns} -- 
The number of edges going into the supplied vertex. 
}%end item
\end{itemize}
}%end item
\item{ 
\index{isCompound(Vertex)}
{\bf  isCompound}\\
\begin{lstlisting}[frame=none]
public boolean isCompound(Vertex vertex)\end{lstlisting} %end signature
}%end item
\item{ 
\index{outdegreeOf(V)}
{\bf  outdegreeOf}\\
\begin{lstlisting}[frame=none]
java.lang.Integer outdegreeOf(Vertex vertex)\end{lstlisting} %end signature
\begin{itemize}
\item{
{\bf  Description copied from DirectedGraph{\small \refdefined{graphmodel.DirectedGraph}} }

Returns the outdegree of a vertex of the graph.
}
\item{
{\bf  Parameters}
  \begin{itemize}
   \item{
\texttt{vertex} -- Vertex whose outdegree will be returned.}
  \end{itemize}
}%end item
\item{{\bf  Returns} -- 
The number of edges going out of the supplied vertex. 
}%end item
\end{itemize}
}%end item
\item{ 
\index{outgoingEdgesOf(V)}
{\bf  outgoingEdgesOf}\\
\begin{lstlisting}[frame=none]
java.util.Set outgoingEdgesOf(Vertex vertex)\end{lstlisting} %end signature
\begin{itemize}
\item{
{\bf  Description copied from DirectedGraph{\small \refdefined{graphmodel.DirectedGraph}} }

Returns a set of all outgoing edges of a vertex.
}
\item{
{\bf  Parameters}
  \begin{itemize}
   \item{
\texttt{vertex} -- Vertex whose outgoing edges will be returned.}
  \end{itemize}
}%end item
\item{{\bf  Returns} -- 
The edges going out of the supplied vertex. 
}%end item
\end{itemize}
}%end item
\item{ 
\index{serialize(List)}
{\bf  serialize}\\
\begin{lstlisting}[frame=none]
public SerializedGraph serialize(java.util.List attributes)\end{lstlisting} %end signature
}%end item
\end{itemize}
}
}
\subsection{\label{graphmodel.DefaultVertex}\index{DefaultVertex}Class DefaultVertex}{
\vskip .1in 
This is an DefaultVertex, which has basic functions and is provided by the main application. This vertex can be derived by plugins which offer more functionality than the basic vertex.\vskip .1in 
\subsubsection{Declaration}{
\begin{lstlisting}[frame=none]
public class DefaultVertex
 extends java.lang.Object implements Vertex\end{lstlisting}
\subsubsection{Constructors}{
\vskip -2em
\begin{itemize}
\item{ 
\index{DefaultVertex()}
{\bf  DefaultVertex}\\
\begin{lstlisting}[frame=none]
public DefaultVertex()\end{lstlisting} %end signature
}%end item
\end{itemize}
}
\subsubsection{Methods}{
\vskip -2em
\begin{itemize}
\item{ 
\index{addToFastGraphAccessor(FastGraphAccessor)}
{\bf  addToFastGraphAccessor}\\
\begin{lstlisting}[frame=none]
void addToFastGraphAccessor(FastGraphAccessor fga)\end{lstlisting} %end signature
\begin{itemize}
\item{
{\bf  Description copied from Vertex{\small \refdefined{graphmodel.Vertex}} }

Adds the vertex to a \texttt{\small FastGraphAccessor}{\small 
\refdefined{graphmodel.FastGraphAccessor}}.
}
\item{
{\bf  Parameters}
  \begin{itemize}
   \item{
\texttt{fga} -- The \texttt{\small FastGraphAccessor}{\small 
\refdefined{graphmodel.FastGraphAccessor}} to whom this vertex will be added.}
  \end{itemize}
}%end item
\end{itemize}
}%end item
\item{ 
\index{getID()}
{\bf  getID}\\
\begin{lstlisting}[frame=none]
java.lang.Integer getID()\end{lstlisting} %end signature
\begin{itemize}
\item{
{\bf  Description copied from Vertex{\small \refdefined{graphmodel.Vertex}} }

Returns the ID of the vertex. Every vertex in one graph has a unique ID.
}
\item{{\bf  Returns} -- 
The ID of the vertex. 
}%end item
\end{itemize}
}%end item
\item{ 
\index{getLabel()}
{\bf  getLabel}\\
\begin{lstlisting}[frame=none]
java.lang.String getLabel()\end{lstlisting} %end signature
\begin{itemize}
\item{
{\bf  Description copied from Vertex{\small \refdefined{graphmodel.Vertex}} }

Returns the label of the vertex, that will be shown in the GUI. The label can be an empty string.
}
\item{{\bf  Returns} -- 
The label of the vertex 
}%end item
\end{itemize}
}%end item
\item{ 
\index{getName()}
{\bf  getName}\\
\begin{lstlisting}[frame=none]
java.lang.String getName()\end{lstlisting} %end signature
\begin{itemize}
\item{
{\bf  Description copied from Vertex{\small \refdefined{graphmodel.Vertex}} }

Returns the name of the vertex. A descriptive name of the vertex. Multiple vertices with equal name in one graph are allowed. Therefore don't use this as identifier, instead use \texttt{\small getID()}.
}
\item{{\bf  Returns} -- 
The name of the vertex. 
}%end item
\end{itemize}
}%end item
\item{ 
\index{getX()}
{\bf  getX}\\
\begin{lstlisting}[frame=none]
int getX()\end{lstlisting} %end signature
\begin{itemize}
\item{
{\bf  Description copied from Vertex{\small \refdefined{graphmodel.Vertex}} }

Returns the X-coordinate of the vertex.
}
\item{{\bf  Returns} -- 
The X-coordinate of this vertex. 
}%end item
\end{itemize}
}%end item
\item{ 
\index{getY()}
{\bf  getY}\\
\begin{lstlisting}[frame=none]
int getY()\end{lstlisting} %end signature
\begin{itemize}
\item{
{\bf  Description copied from Vertex{\small \refdefined{graphmodel.Vertex}} }

Returns the Y-coordinate of the vertex.
}
\item{{\bf  Returns} -- 
The Y-coordinate of the vertex. 
}%end item
\end{itemize}
}%end item
\item{ 
\index{serialize(List)}
{\bf  serialize}\\
\begin{lstlisting}[frame=none]
SerializedVertex serialize(java.util.List attributes)\end{lstlisting} %end signature
\begin{itemize}
\item{
{\bf  Description copied from Vertex{\small \refdefined{graphmodel.Vertex}} }

Returns a \texttt{\small SerializedVertex}{\small 
\refdefined{graphmodel.SerializedVertex}} representation of the graph.
}
\item{
{\bf  Parameters}
  \begin{itemize}
   \item{
\texttt{attributes} -- The attributes that have to be serialized.}
  \end{itemize}
}%end item
\item{{\bf  Returns} -- 
The \texttt{\small SerializedVertex}{\small 
\refdefined{graphmodel.SerializedVertex}} representation of the graph. 
}%end item
\end{itemize}
}%end item
\end{itemize}
}
}
\subsection{\label{graphmodel.DirectedEdge}\index{DirectedEdge}Class DirectedEdge}{
\vskip .1in 
A \texttt{\small DirectedEdge}{\small 
\refdefined{graphmodel.DirectedEdge}} is an edge that has one source and one target vertex. The direction of the edge is specified.\vskip .1in 
\subsubsection{Declaration}{
\begin{lstlisting}[frame=none]
public class DirectedEdge
 extends java.lang.Object implements Edge\end{lstlisting}
\subsubsection{Constructors}{
\vskip -2em
\begin{itemize}
\item{ 
\index{DirectedEdge()}
{\bf  DirectedEdge}\\
\begin{lstlisting}[frame=none]
public DirectedEdge()\end{lstlisting} %end signature
}%end item
\end{itemize}
}
\subsubsection{Methods}{
\vskip -2em
\begin{itemize}
\item{ 
\index{addToFastGraphAccessor(FastGraphAccessor)}
{\bf  addToFastGraphAccessor}\\
\begin{lstlisting}[frame=none]
void addToFastGraphAccessor(FastGraphAccessor fga)\end{lstlisting} %end signature
\begin{itemize}
\item{
{\bf  Description copied from Edge{\small \refdefined{graphmodel.Edge}} }

Adds the edge to a \texttt{\small FastGraphAccessor}{\small 
\refdefined{graphmodel.FastGraphAccessor}}.
}
\item{
{\bf  Parameters}
  \begin{itemize}
   \item{
\texttt{fga} -- The \texttt{\small FastGraphAccessor}{\small 
\refdefined{graphmodel.FastGraphAccessor}} to whom this edge will be added.}
  \end{itemize}
}%end item
\end{itemize}
}%end item
\item{ 
\index{getID()}
{\bf  getID}\\
\begin{lstlisting}[frame=none]
java.lang.Integer getID()\end{lstlisting} %end signature
\begin{itemize}
\item{
{\bf  Description copied from Edge{\small \refdefined{graphmodel.Edge}} }

Returns the ID of the edge.
}
\item{{\bf  Returns} -- 
The id of the edge. 
}%end item
\end{itemize}
}%end item
\item{ 
\index{getLabel()}
{\bf  getLabel}\\
\begin{lstlisting}[frame=none]
java.lang.String getLabel()\end{lstlisting} %end signature
\begin{itemize}
\item{
{\bf  Description copied from Edge{\small \refdefined{graphmodel.Edge}} }

Returns the label of the edge.
}
\item{{\bf  Returns} -- 
The label of the edge. 
}%end item
\end{itemize}
}%end item
\item{ 
\index{getName()}
{\bf  getName}\\
\begin{lstlisting}[frame=none]
java.lang.String getName()\end{lstlisting} %end signature
\begin{itemize}
\item{
{\bf  Description copied from Edge{\small \refdefined{graphmodel.Edge}} }

Returns the name of the edge.
}
\item{{\bf  Returns} -- 
The name of the edge. 
}%end item
\end{itemize}
}%end item
\item{ 
\index{getPath()}
{\bf  getPath}\\
\begin{lstlisting}[frame=none]
EdgePath getPath()\end{lstlisting} %end signature
\begin{itemize}
\item{
{\bf  Description copied from Edge{\small \refdefined{graphmodel.Edge}} }

Returns the \texttt{\small EdgePath}{\small 
\refdefined{graphmodel.EdgePath}} of the edge. The edge path is attached to the edge and cannot be replaced.
}
\item{{\bf  Returns} -- 
the edge path 
}%end item
\end{itemize}
}%end item
\item{ 
\index{getSource()}
{\bf  getSource}\\
\begin{lstlisting}[frame=none]
public Vertex getSource()\end{lstlisting} %end signature
\begin{itemize}
\item{
{\bf  Description}

Returns the source vertex of this directed edge.
}
\item{{\bf  Returns} -- 
The vertex the edge is coming from. 
}%end item
\end{itemize}
}%end item
\item{ 
\index{getTarget()}
{\bf  getTarget}\\
\begin{lstlisting}[frame=none]
public Vertex getTarget()\end{lstlisting} %end signature
\begin{itemize}
\item{
{\bf  Description}

Returns the target vertex of this edge.
}
\item{{\bf  Returns} -- 
The vertex the edge is pointing at/going to. 
}%end item
\end{itemize}
}%end item
\item{ 
\index{getVertices()}
{\bf  getVertices}\\
\begin{lstlisting}[frame=none]
java.util.List getVertices()\end{lstlisting} %end signature
\begin{itemize}
\item{
{\bf  Description copied from Edge{\small \refdefined{graphmodel.Edge}} }

Returns the vertices connected with this edge.
}
\item{{\bf  Returns} -- 
The vertices connected with the edge. 
}%end item
\end{itemize}
}%end item
\item{ 
\index{serialize(List)}
{\bf  serialize}\\
\begin{lstlisting}[frame=none]
SerializedEdge serialize(java.util.List attributes)\end{lstlisting} %end signature
\begin{itemize}
\item{
{\bf  Description copied from Edge{\small \refdefined{graphmodel.Edge}} }

Returns a \texttt{\small SerializedEdge}{\small 
\refdefined{graphmodel.SerializedEdge}} representation of the edge.
}
\item{
{\bf  Parameters}
  \begin{itemize}
   \item{
\texttt{attributes} -- The attributes that have to be serialized.}
  \end{itemize}
}%end item
\item{{\bf  Returns} -- 
The \texttt{\small SerializedEdge}{\small 
\refdefined{graphmodel.SerializedEdge}} representation of the edge. 
}%end item
\end{itemize}
}%end item
\end{itemize}
}
}
\subsection{\label{graphmodel.DirectedGraphLayoutOption}\index{DirectedGraphLayoutOption}Class DirectedGraphLayoutOption}{
\vskip .1in 
A \texttt{\small LayoutOption}{\small 
\refdefined{plugin.LayoutOption}} which is specific for \texttt{\small DirectedGraph}{\small 
\refdefined{graphmodel.DirectedGraph}}.\vskip .1in 
\subsubsection{Declaration}{
\begin{lstlisting}[frame=none]
public abstract class DirectedGraphLayoutOption
 extends plugin.LayoutOption\end{lstlisting}
\subsubsection{Constructors}{
\vskip -2em
\begin{itemize}
\item{ 
\index{DirectedGraphLayoutOption()}
{\bf  DirectedGraphLayoutOption}\\
\begin{lstlisting}[frame=none]
public DirectedGraphLayoutOption()\end{lstlisting} %end signature
}%end item
\end{itemize}
}
\subsubsection{Methods}{
\vskip -2em
\begin{itemize}
\item{ 
\index{setGraph(DirectedGraph)}
{\bf  setGraph}\\
\begin{lstlisting}[frame=none]
public void setGraph(DirectedGraph graph)\end{lstlisting} %end signature
\begin{itemize}
\item{
{\bf  Description}

Sets the graph that will be the target of the DirectedGraphLayoutOption.
}
\item{
{\bf  Parameters}
  \begin{itemize}
   \item{
\texttt{graph} -- The graph that will be the target of this DirectedGraphLayoutOption.}
  \end{itemize}
}%end item
\end{itemize}
}%end item
\end{itemize}
}
}
\subsection{\label{graphmodel.DirectedGraphLayoutRegister}\index{DirectedGraphLayoutRegister}Class DirectedGraphLayoutRegister}{
\vskip .1in 
A \texttt{\small LayoutRegister}{\small 
\refdefined{plugin.LayoutRegister}} which is specific for \texttt{\small DirectedGraphLayoutOption}{\small 
\refdefined{graphmodel.DirectedGraphLayoutOption}}.\vskip .1in 
\subsubsection{Declaration}{
\begin{lstlisting}[frame=none]
public class DirectedGraphLayoutRegister
 extends java.lang.Object implements plugin.LayoutRegister\end{lstlisting}
\subsubsection{Constructors}{
\vskip -2em
\begin{itemize}
\item{ 
\index{DirectedGraphLayoutRegister()}
{\bf  DirectedGraphLayoutRegister}\\
\begin{lstlisting}[frame=none]
public DirectedGraphLayoutRegister()\end{lstlisting} %end signature
}%end item
\end{itemize}
}
\subsubsection{Methods}{
\vskip -2em
\begin{itemize}
\item{ 
\index{addLayoutOption(DirectedGraphLayoutOption)}
{\bf  addLayoutOption}\\
\begin{lstlisting}[frame=none]
public void addLayoutOption(DirectedGraphLayoutOption option)\end{lstlisting} %end signature
}%end item
\item{ 
\index{getLayoutOptions()}
{\bf  getLayoutOptions}\\
\begin{lstlisting}[frame=none]
java.util.List getLayoutOptions()\end{lstlisting} %end signature
\begin{itemize}
\item{
{\bf  Description copied from plugin.LayoutRegister{\small \refdefined{plugin.LayoutRegister}} }

Returns all available layouts for the graph associated with this register.
}
\item{{\bf  Returns} -- 
the available layouts 
}%end item
\end{itemize}
}%end item
\end{itemize}
}
}
\subsection{\label{graphmodel.EdgePath}\index{EdgePath}Class EdgePath}{
\vskip .1in 
An abstract super class for edge paths. Contains basic information every edge path must provide.\vskip .1in 
\subsubsection{Declaration}{
\begin{lstlisting}[frame=none]
public abstract class EdgePath
 extends java.lang.Object\end{lstlisting}
\subsubsection{All known subclasses}{OrthogonalEdgePath\small{\refdefined{graphmodel.OrthogonalEdgePath}}}
\subsubsection{Constructors}{
\vskip -2em
\begin{itemize}
\item{ 
\index{EdgePath()}
{\bf  EdgePath}\\
\begin{lstlisting}[frame=none]
public EdgePath()\end{lstlisting} %end signature
}%end item
\end{itemize}
}
\subsubsection{Methods}{
\vskip -2em
\begin{itemize}
\item{ 
\index{getNodes()}
{\bf  getNodes}\\
\begin{lstlisting}[frame=none]
public abstract java.util.List getNodes()\end{lstlisting} %end signature
\begin{itemize}
\item{
{\bf  Description}

Returns all nodes the edge has to pass through. In the order it has to pass through them.
}
\item{{\bf  Returns} -- 
the list of nodes 
}%end item
\end{itemize}
}%end item
\item{ 
\index{getSegmentsCount()}
{\bf  getSegmentsCount}\\
\begin{lstlisting}[frame=none]
public abstract int getSegmentsCount()\end{lstlisting} %end signature
\begin{itemize}
\item{
{\bf  Description}

Returns out of how many segments the path consists.
}
\item{{\bf  Returns} -- 
the number of segments 
}%end item
\end{itemize}
}%end item
\end{itemize}
}
}
\subsection{\label{graphmodel.EdgePath.Point}\index{EdgePath.Point}Class EdgePath.Point}{
\vskip .1in 
This class is a standard immutable 2D Vector with integer values as it's components.\vskip .1in 
\subsubsection{Declaration}{
\begin{lstlisting}[frame=none]
public class EdgePath.Point
 extends java.lang.Object\end{lstlisting}
\subsubsection{Fields}{
\begin{itemize}
\item{
\index{x}
\label{graphmodel.EdgePath.Point.x}\texttt{public final int\ {\bf  x}}
}
\item{
\index{y}
\label{graphmodel.EdgePath.Point.y}\texttt{public final int\ {\bf  y}}
}
\end{itemize}
}
\subsubsection{Constructors}{
\vskip -2em
\begin{itemize}
\item{ 
\index{Point(int, int)}
{\bf  Point}\\
\begin{lstlisting}[frame=none]
public Point(int x,int y)\end{lstlisting} %end signature
}%end item
\end{itemize}
}
}
\subsection{\label{graphmodel.FastGraphAccessor}\index{FastGraphAccessor}Class FastGraphAccessor}{
\vskip .1in 
This class provides a fast lookup of \texttt{\small Vertex}{\small 
\refdefined{graphmodel.Vertex}} and \texttt{\small Edge}{\small 
\refdefined{graphmodel.Edge}} for a given Attribute value pair without traversing a \texttt{\small Graph}{\small 
\refdefined{graphmodel.Graph}}. FastGraphAccessor is a helper class for looking up all \texttt{\small Vertex}{\small 
\refdefined{graphmodel.Vertex}} and \texttt{\small Edge}{\small 
\refdefined{graphmodel.Edge}} that have a specific value for a specific attribute. To achieve this all elements of a Graph need to add their attributes and values to a FastGraphAccessor. These values are not linked to the origin values so the fastGraphAccessor needs to be updated after changes when needed for following steps. This should be done by reverting and adding the values again.\vskip .1in 
\subsubsection{Declaration}{
\begin{lstlisting}[frame=none]
public class FastGraphAccessor
 extends java.lang.Object\end{lstlisting}
\subsubsection{Constructors}{
\vskip -2em
\begin{itemize}
\item{ 
\index{FastGraphAccessor()}
{\bf  FastGraphAccessor}\\
\begin{lstlisting}[frame=none]
public FastGraphAccessor()\end{lstlisting} %end signature
}%end item
\end{itemize}
}
\subsubsection{Methods}{
\vskip -2em
\begin{itemize}
\item{ 
\index{addEdgeForAttribute(String, Edge, int)}
{\bf  addEdgeForAttribute}\\
\begin{lstlisting}[frame=none]
public void addEdgeForAttribute(java.lang.String name,Edge edge,int value)\end{lstlisting} %end signature
\begin{itemize}
\item{
{\bf  Description}

Adds an \texttt{\small Edge}{\small 
\refdefined{graphmodel.Edge}} for a given attribute with a given value.
}
\item{
{\bf  Parameters}
  \begin{itemize}
   \item{
\texttt{name} -- name of the attribute}
   \item{
\texttt{value} -- value of the attribute}
   \item{
\texttt{edge} -- edge that has this value for the given attribute}
  \end{itemize}
}%end item
\end{itemize}
}%end item
\item{ 
\index{addEdgeForAttribute(String, String, Edge)}
{\bf  addEdgeForAttribute}\\
\begin{lstlisting}[frame=none]
public void addEdgeForAttribute(java.lang.String name,java.lang.String value,Edge edge)\end{lstlisting} %end signature
\begin{itemize}
\item{
{\bf  Description}

Adds an \texttt{\small Edge}{\small 
\refdefined{graphmodel.Edge}} for a given attribute with a given value.
}
\item{
{\bf  Parameters}
  \begin{itemize}
   \item{
\texttt{name} -- name of the attribute}
   \item{
\texttt{value} -- value of the attribute}
   \item{
\texttt{edge} -- edge that has this value for the given attribute}
  \end{itemize}
}%end item
\end{itemize}
}%end item
\item{ 
\index{addVertexForAttribute(Vertex, String, int)}
{\bf  addVertexForAttribute}\\
\begin{lstlisting}[frame=none]
public void addVertexForAttribute(Vertex vertex,java.lang.String value,int name)\end{lstlisting} %end signature
\begin{itemize}
\item{
{\bf  Description}

adds an \texttt{\small Vertex}{\small 
\refdefined{graphmodel.Vertex}} for a given attribute with a given value
}
\item{
{\bf  Parameters}
  \begin{itemize}
   \item{
\texttt{name} -- name of the attribute}
   \item{
\texttt{value} -- value of the attribute}
   \item{
\texttt{vertex} -- vertex that has this value for the given attribute}
  \end{itemize}
}%end item
\end{itemize}
}%end item
\item{ 
\index{addVertexForAttribute(Vertex, String, String)}
{\bf  addVertexForAttribute}\\
\begin{lstlisting}[frame=none]
public void addVertexForAttribute(Vertex vertex,java.lang.String value,java.lang.String name)\end{lstlisting} %end signature
\begin{itemize}
\item{
{\bf  Description}

adds an \texttt{\small Vertex}{\small 
\refdefined{graphmodel.Vertex}} for a given attribute with a given value
}
\item{
{\bf  Parameters}
  \begin{itemize}
   \item{
\texttt{name} -- name of the attribute}
   \item{
\texttt{value} -- value of the attribute}
   \item{
\texttt{vertex} -- vertex that has this value for the given attribute}
  \end{itemize}
}%end item
\end{itemize}
}%end item
\item{ 
\index{getEdgesByAttribute(String, int)}
{\bf  getEdgesByAttribute}\\
\begin{lstlisting}[frame=none]
public java.util.List getEdgesByAttribute(java.lang.String name,int value)\end{lstlisting} %end signature
\begin{itemize}
\item{
{\bf  Description}

gets a \texttt{\small List}{\small 
\refdefined{java.util.List}} of \texttt{\small Edge}{\small 
\refdefined{graphmodel.Edge}} that contains all \texttt{\small Edge}{\small 
\refdefined{graphmodel.Edge}} that have the value for given attribute
}
\item{
{\bf  Parameters}
  \begin{itemize}
   \item{
\texttt{name} -- name of the attribute}
   \item{
\texttt{value} -- value of the attribute}
  \end{itemize}
}%end item
\item{{\bf  Returns} -- 
a \texttt{\small List}{\small 
\refdefined{java.util.List}} of \texttt{\small Edge}{\small 
\refdefined{graphmodel.Edge}} that has the value for given attribute 
}%end item
\end{itemize}
}%end item
\item{ 
\index{getEdgesByAttribute(String, String)}
{\bf  getEdgesByAttribute}\\
\begin{lstlisting}[frame=none]
public java.util.List getEdgesByAttribute(java.lang.String name,java.lang.String value)\end{lstlisting} %end signature
\begin{itemize}
\item{
{\bf  Description}

gets a List of \texttt{\small Edge}{\small 
\refdefined{graphmodel.Edge}} that contains all \texttt{\small Edge}{\small 
\refdefined{graphmodel.Edge}} that have the value for given attribute
}
\item{
{\bf  Parameters}
  \begin{itemize}
   \item{
\texttt{name} -- name of the attribute}
   \item{
\texttt{value} -- value of the attribute}
  \end{itemize}
}%end item
\item{{\bf  Returns} -- 
a \texttt{\small List}{\small 
\refdefined{java.util.List}} of \texttt{\small Edge}{\small 
\refdefined{graphmodel.Edge}} that has tthe value for given attribute 
}%end item
\end{itemize}
}%end item
\item{ 
\index{getVerticesByAttribute(String, int)}
{\bf  getVerticesByAttribute}\\
\begin{lstlisting}[frame=none]
public java.util.List getVerticesByAttribute(java.lang.String name,int value)\end{lstlisting} %end signature
\begin{itemize}
\item{
{\bf  Description}

gets a \texttt{\small List}{\small 
\refdefined{java.util.List}} of \texttt{\small Vertex}{\small 
\refdefined{graphmodel.Vertex}} that contains all \texttt{\small Vertex}{\small 
\refdefined{graphmodel.Vertex}} that have the value for given attribut
}
\item{
{\bf  Parameters}
  \begin{itemize}
   \item{
\texttt{name} -- name of the attribute}
   \item{
\texttt{value} -- value of the attribute}
  \end{itemize}
}%end item
\item{{\bf  Returns} -- 
a \texttt{\small List}{\small 
\refdefined{java.util.List}} of \texttt{\small Vertex}{\small 
\refdefined{graphmodel.Vertex}} that has the value for given attribute 
}%end item
\end{itemize}
}%end item
\item{ 
\index{getVerticesByAttribute(String, String)}
{\bf  getVerticesByAttribute}\\
\begin{lstlisting}[frame=none]
public java.util.List getVerticesByAttribute(java.lang.String name,java.lang.String value)\end{lstlisting} %end signature
\begin{itemize}
\item{
{\bf  Description}

gets a \texttt{\small List}{\small 
\refdefined{java.util.List}} of \texttt{\small Vertex}{\small 
\refdefined{graphmodel.Vertex}} that contains all \texttt{\small Vertex}{\small 
\refdefined{graphmodel.Vertex}} that have the value for given attribut
}
\item{
{\bf  Parameters}
  \begin{itemize}
   \item{
\texttt{name} -- name of the attribute}
   \item{
\texttt{value} -- value of the attribute}
  \end{itemize}
}%end item
\item{{\bf  Returns} -- 
a \texttt{\small List}{\small 
\refdefined{java.util.List}} of \texttt{\small Vertex}{\small 
\refdefined{graphmodel.Vertex}} that has the value for given attribute 
}%end item
\end{itemize}
}%end item
\item{ 
\index{reset()}
{\bf  reset}\\
\begin{lstlisting}[frame=none]
public void reset()\end{lstlisting} %end signature
\begin{itemize}
\item{
{\bf  Description}

Deletes all data in this FastGraphAccessor. After this step all the information needs to be readded to this. This is necessary when updating the FastGraphAccessor
}
\end{itemize}
}%end item
\end{itemize}
}
}
\subsection{\label{graphmodel.GraphModel}\index{GraphModel}Class GraphModel}{
\vskip .1in 
A GraphModel contains one or more graphs. It is used to save nested or hierarchical graphs in one class.\vskip .1in 
\subsubsection{Declaration}{
\begin{lstlisting}[frame=none]
public abstract class GraphModel
 extends java.lang.Object\end{lstlisting}
\subsubsection{Constructors}{
\vskip -2em
\begin{itemize}
\item{ 
\index{GraphModel()}
{\bf  GraphModel}\\
\begin{lstlisting}[frame=none]
public GraphModel()\end{lstlisting} %end signature
}%end item
\end{itemize}
}
\subsubsection{Methods}{
\vskip -2em
\begin{itemize}
\item{ 
\index{getGraphs()}
{\bf  getGraphs}\\
\begin{lstlisting}[frame=none]
public abstract java.util.List getGraphs()\end{lstlisting} %end signature
\begin{itemize}
\item{
{\bf  Description}

Returns all \texttt{\small Graph}{\small 
\refdefined{graphmodel.Graph}} contained in the GraphModel.
}
\item{{\bf  Returns} -- 
A list of all the \texttt{\small Graph}{\small 
\refdefined{graphmodel.Graph}} contained in the GraphModel. 
}%end item
\end{itemize}
}%end item
\end{itemize}
}
}
\subsection{\label{graphmodel.Group}\index{Group}Class Group}{
\vskip .1in 
This class allows to collect an amount of vertices.\vskip .1in 
\subsubsection{Declaration}{
\begin{lstlisting}[frame=none]
public class Group
 extends java.lang.Object\end{lstlisting}
\subsubsection{Constructors}{
\vskip -2em
\begin{itemize}
\item{ 
\index{Group()}
{\bf  Group}\\
\begin{lstlisting}[frame=none]
public Group()\end{lstlisting} %end signature
}%end item
\end{itemize}
}
\subsubsection{Methods}{
\vskip -2em
\begin{itemize}
\item{ 
\index{addVertex(V)}
{\bf  addVertex}\\
\begin{lstlisting}[frame=none]
public void addVertex(Vertex vertex)\end{lstlisting} %end signature
\begin{itemize}
\item{
{\bf  Description}

Adds the vertex to this Group.
}
\item{
{\bf  Parameters}
  \begin{itemize}
   \item{
\texttt{vertex} -- vertex to add to this group}
  \end{itemize}
}%end item
\end{itemize}
}%end item
\item{ 
\index{removeVertex(V)}
{\bf  removeVertex}\\
\begin{lstlisting}[frame=none]
public void removeVertex(Vertex vertex)\end{lstlisting} %end signature
\begin{itemize}
\item{
{\bf  Description}

Removes the vertex argument from this Group.
}
\item{
{\bf  Parameters}
  \begin{itemize}
   \item{
\texttt{vertex} -- vertex to remove from this Group}
  \end{itemize}
}%end item
\end{itemize}
}%end item
\end{itemize}
}
}
\subsection{\label{graphmodel.OrthogonalEdgePath}\index{OrthogonalEdgePath}Class OrthogonalEdgePath}{
\vskip .1in 
An orthogonal edge path used as standard graphical edge representation.\vskip .1in 
\subsubsection{Declaration}{
\begin{lstlisting}[frame=none]
public class OrthogonalEdgePath
 extends graphmodel.EdgePath\end{lstlisting}
\subsubsection{Constructors}{
\vskip -2em
\begin{itemize}
\item{ 
\index{OrthogonalEdgePath()}
{\bf  OrthogonalEdgePath}\\
\begin{lstlisting}[frame=none]
public OrthogonalEdgePath()\end{lstlisting} %end signature
}%end item
\end{itemize}
}
\subsubsection{Methods}{
\vskip -2em
\begin{itemize}
\item{ 
\index{getNodes()}
{\bf  getNodes}\\
\begin{lstlisting}[frame=none]
public abstract java.util.List getNodes()\end{lstlisting} %end signature
\begin{itemize}
\item{
{\bf  Description copied from EdgePath{\small \refdefined{graphmodel.EdgePath}} }

Returns all nodes the edge has to pass through. In the order it has to pass through them.
}
\item{{\bf  Returns} -- 
the list of nodes 
}%end item
\end{itemize}
}%end item
\item{ 
\index{getSegmentsCount()}
{\bf  getSegmentsCount}\\
\begin{lstlisting}[frame=none]
public abstract int getSegmentsCount()\end{lstlisting} %end signature
\begin{itemize}
\item{
{\bf  Description copied from EdgePath{\small \refdefined{graphmodel.EdgePath}} }

Returns out of how many segments the path consists.
}
\item{{\bf  Returns} -- 
the number of segments 
}%end item
\end{itemize}
}%end item
\end{itemize}
}
}
\subsection{\label{graphmodel.SerializedEdge}\index{SerializedEdge}Class SerializedEdge}{
\vskip .1in 
A serialized version of a \texttt{\small Edge}{\small 
\refdefined{graphmodel.Edge}}. It contains all attributes as a \texttt{\small List}{\small 
\refdefined{java.util.List}} of String to String entries which can be used by an Exporter to export a \texttt{\small Edge}{\small 
\refdefined{graphmodel.Edge}}. It is designed as an intermediate Step in the export workflow and should not be used for other purposes. Attributes in the \texttt{\small List}{\small 
\refdefined{java.util.List}} are not synchronized with attributes outside the \texttt{\small List}{\small 
\refdefined{java.util.List}}, and Attributes of SerializedEdge are not synchronized with the origin \texttt{\small Edge}{\small 
\refdefined{graphmodel.Edge}} attributes.\vskip .1in 
\subsubsection{Declaration}{
\begin{lstlisting}[frame=none]
public class SerializedEdge
 extends java.lang.Object implements Edge\end{lstlisting}
\subsubsection{Constructors}{
\vskip -2em
\begin{itemize}
\item{ 
\index{SerializedEdge()}
{\bf  SerializedEdge}\\
\begin{lstlisting}[frame=none]
public SerializedEdge()\end{lstlisting} %end signature
}%end item
\end{itemize}
}
\subsubsection{Methods}{
\vskip -2em
\begin{itemize}
\item{ 
\index{addToFastGraphAccessor(FastGraphAccessor)}
{\bf  addToFastGraphAccessor}\\
\begin{lstlisting}[frame=none]
void addToFastGraphAccessor(FastGraphAccessor fga)\end{lstlisting} %end signature
\begin{itemize}
\item{
{\bf  Description copied from Edge{\small \refdefined{graphmodel.Edge}} }

Adds the edge to a \texttt{\small FastGraphAccessor}{\small 
\refdefined{graphmodel.FastGraphAccessor}}.
}
\item{
{\bf  Parameters}
  \begin{itemize}
   \item{
\texttt{fga} -- The \texttt{\small FastGraphAccessor}{\small 
\refdefined{graphmodel.FastGraphAccessor}} to whom this edge will be added.}
  \end{itemize}
}%end item
\end{itemize}
}%end item
\item{ 
\index{getAttributes()}
{\bf  getAttributes}\\
\begin{lstlisting}[frame=none]
public java.util.Map getAttributes()\end{lstlisting} %end signature
\begin{itemize}
\item{
{\bf  Description}

Gets all serialized Attributes as a Map from String to String. This Map gets created when serializing a \texttt{\small Edge}{\small 
\refdefined{graphmodel.Edge}} and is returned on demand. This should only be used for exporting Edges since the attributes are not synchronized with the attributes of the unserialized \texttt{\small Edge}{\small 
\refdefined{graphmodel.Edge}}
}
\item{{\bf  Returns} -- 
The Map of serialized Attributes 
}%end item
\item{{\bf  See also}
  \begin{itemize}
\item{ \texttt{java.util.Map} {\small 
\refdefined{java.util.Map}}%end
}
  \end{itemize}
}%end item
\end{itemize}
}%end item
\item{ 
\index{getID()}
{\bf  getID}\\
\begin{lstlisting}[frame=none]
java.lang.Integer getID()\end{lstlisting} %end signature
\begin{itemize}
\item{
{\bf  Description copied from Edge{\small \refdefined{graphmodel.Edge}} }

Returns the ID of the edge.
}
\item{{\bf  Returns} -- 
The id of the edge. 
}%end item
\end{itemize}
}%end item
\item{ 
\index{getLabel()}
{\bf  getLabel}\\
\begin{lstlisting}[frame=none]
java.lang.String getLabel()\end{lstlisting} %end signature
\begin{itemize}
\item{
{\bf  Description copied from Edge{\small \refdefined{graphmodel.Edge}} }

Returns the label of the edge.
}
\item{{\bf  Returns} -- 
The label of the edge. 
}%end item
\end{itemize}
}%end item
\item{ 
\index{getName()}
{\bf  getName}\\
\begin{lstlisting}[frame=none]
java.lang.String getName()\end{lstlisting} %end signature
\begin{itemize}
\item{
{\bf  Description copied from Edge{\small \refdefined{graphmodel.Edge}} }

Returns the name of the edge.
}
\item{{\bf  Returns} -- 
The name of the edge. 
}%end item
\end{itemize}
}%end item
\item{ 
\index{getPath()}
{\bf  getPath}\\
\begin{lstlisting}[frame=none]
EdgePath getPath()\end{lstlisting} %end signature
\begin{itemize}
\item{
{\bf  Description copied from Edge{\small \refdefined{graphmodel.Edge}} }

Returns the \texttt{\small EdgePath}{\small 
\refdefined{graphmodel.EdgePath}} of the edge. The edge path is attached to the edge and cannot be replaced.
}
\item{{\bf  Returns} -- 
the edge path 
}%end item
\end{itemize}
}%end item
\item{ 
\index{getVertices()}
{\bf  getVertices}\\
\begin{lstlisting}[frame=none]
java.util.List getVertices()\end{lstlisting} %end signature
\begin{itemize}
\item{
{\bf  Description copied from Edge{\small \refdefined{graphmodel.Edge}} }

Returns the vertices connected with this edge.
}
\item{{\bf  Returns} -- 
The vertices connected with the edge. 
}%end item
\end{itemize}
}%end item
\item{ 
\index{serialize(List)}
{\bf  serialize}\\
\begin{lstlisting}[frame=none]
SerializedEdge serialize(java.util.List attributes)\end{lstlisting} %end signature
\begin{itemize}
\item{
{\bf  Description copied from Edge{\small \refdefined{graphmodel.Edge}} }

Returns a \texttt{\small SerializedEdge}{\small 
\refdefined{graphmodel.SerializedEdge}} representation of the edge.
}
\item{
{\bf  Parameters}
  \begin{itemize}
   \item{
\texttt{attributes} -- The attributes that have to be serialized.}
  \end{itemize}
}%end item
\item{{\bf  Returns} -- 
The \texttt{\small SerializedEdge}{\small 
\refdefined{graphmodel.SerializedEdge}} representation of the edge. 
}%end item
\end{itemize}
}%end item
\end{itemize}
}
}
\subsection{\label{graphmodel.SerializedGraph}\index{SerializedGraph}Class SerializedGraph}{
\vskip .1in 
A serialized version of a \texttt{\small Graph}{\small 
\refdefined{graphmodel.Graph}}. It contains all attributes as a \texttt{\small List}{\small 
\refdefined{java.util.List}} of String to String entries which can be used by an \texttt{\small Exporter}{\small 
\refdefined{plugin.Exporter}} to export a \texttt{\small Graph}{\small 
\refdefined{graphmodel.Graph}}. It is designed as an intermediate Step in the export workflow and should not be used for other purposes. Attributes in the \texttt{\small List}{\small 
\refdefined{java.util.List}} are not synchronized with attributes outside the \texttt{\small List}{\small 
\refdefined{java.util.List}}, and Attributes of SerializedGraph are not synchronized with the origin \texttt{\small Graph}{\small 
\refdefined{graphmodel.Graph}} attributes.\vskip .1in 
\subsubsection{Declaration}{
\begin{lstlisting}[frame=none]
public class SerializedGraph
 extends java.lang.Object implements Graph\end{lstlisting}
\subsubsection{Constructors}{
\vskip -2em
\begin{itemize}
\item{ 
\index{SerializedGraph()}
{\bf  SerializedGraph}\\
\begin{lstlisting}[frame=none]
public SerializedGraph()\end{lstlisting} %end signature
}%end item
\end{itemize}
}
\subsubsection{Methods}{
\vskip -2em
\begin{itemize}
\item{ 
\index{addToFastGraphAccessor(FastGraphAccessor)}
{\bf  addToFastGraphAccessor}\\
\begin{lstlisting}[frame=none]
void addToFastGraphAccessor(FastGraphAccessor fga)\end{lstlisting} %end signature
\begin{itemize}
\item{
{\bf  Description copied from Graph{\small \refdefined{graphmodel.Graph}} }

Adds the graph to a \texttt{\small FastGraphAccessor}{\small 
\refdefined{graphmodel.FastGraphAccessor}}.
}
\item{
{\bf  Parameters}
  \begin{itemize}
   \item{
\texttt{fga} -- the \texttt{\small FastGraphAccessor}{\small 
\refdefined{graphmodel.FastGraphAccessor}} to whom this graph will be added.}
  \end{itemize}
}%end item
\end{itemize}
}%end item
\item{ 
\index{edgesOf(V)}
{\bf  edgesOf}\\
\begin{lstlisting}[frame=none]
java.util.Set edgesOf(Vertex vertex)\end{lstlisting} %end signature
\begin{itemize}
\item{
{\bf  Description copied from Graph{\small \refdefined{graphmodel.Graph}} }

Returns a list of all edges of a vertex.
}
\item{
{\bf  Parameters}
  \begin{itemize}
   \item{
\texttt{vertex} -- the vertex which edges will be returned.}
  \end{itemize}
}%end item
\item{{\bf  Returns} -- 
All edges which are connected with the supplied vertex. 
}%end item
\end{itemize}
}%end item
\item{ 
\index{getAttributes()}
{\bf  getAttributes}\\
\begin{lstlisting}[frame=none]
public java.util.Map getAttributes()\end{lstlisting} %end signature
\begin{itemize}
\item{
{\bf  Description}

Gets all serialized Attributes as a Map from String to String. This Map gets created when serializing a \texttt{\small Graph}{\small 
\refdefined{graphmodel.Graph}} and is returned on demand. This should only be used for exporting Graphs since the attributes are not synchronized with the attributes of the unserialized \texttt{\small Graph}{\small 
\refdefined{graphmodel.Graph}}
}
\item{{\bf  Returns} -- 
The Map of serialized Attributes 
}%end item
\item{{\bf  See also}
  \begin{itemize}
\item{ \texttt{java.util.Map} {\small 
\refdefined{java.util.Map}}%end
}
  \end{itemize}
}%end item
\end{itemize}
}%end item
\item{ 
\index{getEdgeSet()}
{\bf  getEdgeSet}\\
\begin{lstlisting}[frame=none]
java.util.Set getEdgeSet()\end{lstlisting} %end signature
\begin{itemize}
\item{
{\bf  Description copied from Graph{\small \refdefined{graphmodel.Graph}} }

Returns all edges of the graph.
}
\item{{\bf  Returns} -- 
A set of all edges of the graph. 
}%end item
\end{itemize}
}%end item
\item{ 
\index{getFastGraphAccessor()}
{\bf  getFastGraphAccessor}\\
\begin{lstlisting}[frame=none]
FastGraphAccessor getFastGraphAccessor()\end{lstlisting} %end signature
\begin{itemize}
\item{
{\bf  Description copied from Graph{\small \refdefined{graphmodel.Graph}} }

Returns the FastGraphAccessor of this Graph.
}
\item{{\bf  Returns} -- 
the FastGraphAccessor of this Graph 
}%end item
\end{itemize}
}%end item
\item{ 
\index{getID()}
{\bf  getID}\\
\begin{lstlisting}[frame=none]
java.lang.Integer getID()\end{lstlisting} %end signature
\begin{itemize}
\item{
{\bf  Description copied from Graph{\small \refdefined{graphmodel.Graph}} }

Returns the ID of the graph.
}
\item{{\bf  Returns} -- 
The id of the graph. 
}%end item
\end{itemize}
}%end item
\item{ 
\index{getName()}
{\bf  getName}\\
\begin{lstlisting}[frame=none]
java.lang.String getName()\end{lstlisting} %end signature
\begin{itemize}
\item{
{\bf  Description copied from Graph{\small \refdefined{graphmodel.Graph}} }

Returns the name of the Graph.
}
\item{{\bf  Returns} -- 
The name of the graph. 
}%end item
\end{itemize}
}%end item
\item{ 
\index{getVertexSet()}
{\bf  getVertexSet}\\
\begin{lstlisting}[frame=none]
java.util.Set getVertexSet()\end{lstlisting} %end signature
\begin{itemize}
\item{
{\bf  Description copied from Graph{\small \refdefined{graphmodel.Graph}} }

Returns all vertices of the graph.
}
\item{{\bf  Returns} -- 
A set of all vertices of the graph. 
}%end item
\end{itemize}
}%end item
\end{itemize}
}
}
\subsection{\label{graphmodel.SerializedVertex}\index{SerializedVertex}Class SerializedVertex}{
\vskip .1in 
A serialized version of a \texttt{\small Vertex}{\small 
\refdefined{graphmodel.Vertex}}. It contains all attributes as a \texttt{\small List}{\small 
\refdefined{java.util.List}} of String to String entries which can be used by an \texttt{\small Exporter}{\small 
\refdefined{plugin.Exporter}} to export a \texttt{\small Vertex}{\small 
\refdefined{graphmodel.Vertex}}. It is designed as an intermediate Step in the export workflow and should not be used for other purposes. Attributes in the \texttt{\small List}{\small 
\refdefined{java.util.List}} are not synchronized with attributes outside the \texttt{\small List}{\small 
\refdefined{java.util.List}}, and Attributes of SerializedVertex are not synchronized with the origin \texttt{\small Vertex}{\small 
\refdefined{graphmodel.Vertex}} attributes.\vskip .1in 
\subsubsection{Declaration}{
\begin{lstlisting}[frame=none]
public class SerializedVertex
 extends java.lang.Object implements Vertex\end{lstlisting}
\subsubsection{Constructors}{
\vskip -2em
\begin{itemize}
\item{ 
\index{SerializedVertex(List)}
{\bf  SerializedVertex}\\
\begin{lstlisting}[frame=none]
public SerializedVertex(java.util.List attributes)\end{lstlisting} %end signature
\begin{itemize}
\item{
{\bf  Description}

creates
}
\item{
{\bf  Parameters}
  \begin{itemize}
   \item{
\texttt{attributes} -- }
  \end{itemize}
}%end item
\end{itemize}
}%end item
\end{itemize}
}
\subsubsection{Methods}{
\vskip -2em
\begin{itemize}
\item{ 
\index{addToFastGraphAccessor(FastGraphAccessor)}
{\bf  addToFastGraphAccessor}\\
\begin{lstlisting}[frame=none]
void addToFastGraphAccessor(FastGraphAccessor fga)\end{lstlisting} %end signature
\begin{itemize}
\item{
{\bf  Description copied from Vertex{\small \refdefined{graphmodel.Vertex}} }

Adds the vertex to a \texttt{\small FastGraphAccessor}{\small 
\refdefined{graphmodel.FastGraphAccessor}}.
}
\item{
{\bf  Parameters}
  \begin{itemize}
   \item{
\texttt{fga} -- The \texttt{\small FastGraphAccessor}{\small 
\refdefined{graphmodel.FastGraphAccessor}} to whom this vertex will be added.}
  \end{itemize}
}%end item
\end{itemize}
}%end item
\item{ 
\index{getAttributes()}
{\bf  getAttributes}\\
\begin{lstlisting}[frame=none]
public java.util.Map getAttributes()\end{lstlisting} %end signature
\begin{itemize}
\item{
{\bf  Description}

Gets all serialized Attributes as a Map from String to String. This Map gets created when serializing a \texttt{\small Vertex}{\small 
\refdefined{graphmodel.Vertex}} and is returned on demand. This should only be used for exporting Vertices since the attributes are not synchronized with the attributes of the unserialized \texttt{\small Vertex}{\small 
\refdefined{graphmodel.Vertex}}
}
\item{{\bf  Returns} -- 
The Map of serialized Attributes 
}%end item
\item{{\bf  See also}
  \begin{itemize}
\item{ \texttt{java.util.Map} {\small 
\refdefined{java.util.Map}}%end
}
  \end{itemize}
}%end item
\end{itemize}
}%end item
\item{ 
\index{getID()}
{\bf  getID}\\
\begin{lstlisting}[frame=none]
java.lang.Integer getID()\end{lstlisting} %end signature
\begin{itemize}
\item{
{\bf  Description copied from Vertex{\small \refdefined{graphmodel.Vertex}} }

Returns the ID of the vertex. Every vertex in one graph has a unique ID.
}
\item{{\bf  Returns} -- 
The ID of the vertex. 
}%end item
\end{itemize}
}%end item
\item{ 
\index{getLabel()}
{\bf  getLabel}\\
\begin{lstlisting}[frame=none]
java.lang.String getLabel()\end{lstlisting} %end signature
\begin{itemize}
\item{
{\bf  Description copied from Vertex{\small \refdefined{graphmodel.Vertex}} }

Returns the label of the vertex, that will be shown in the GUI. The label can be an empty string.
}
\item{{\bf  Returns} -- 
The label of the vertex 
}%end item
\end{itemize}
}%end item
\item{ 
\index{getName()}
{\bf  getName}\\
\begin{lstlisting}[frame=none]
java.lang.String getName()\end{lstlisting} %end signature
\begin{itemize}
\item{
{\bf  Description copied from Vertex{\small \refdefined{graphmodel.Vertex}} }

Returns the name of the vertex. A descriptive name of the vertex. Multiple vertices with equal name in one graph are allowed. Therefore don't use this as identifier, instead use \texttt{\small getID()}.
}
\item{{\bf  Returns} -- 
The name of the vertex. 
}%end item
\end{itemize}
}%end item
\item{ 
\index{getX()}
{\bf  getX}\\
\begin{lstlisting}[frame=none]
int getX()\end{lstlisting} %end signature
\begin{itemize}
\item{
{\bf  Description copied from Vertex{\small \refdefined{graphmodel.Vertex}} }

Returns the X-coordinate of the vertex.
}
\item{{\bf  Returns} -- 
The X-coordinate of this vertex. 
}%end item
\end{itemize}
}%end item
\item{ 
\index{getY()}
{\bf  getY}\\
\begin{lstlisting}[frame=none]
int getY()\end{lstlisting} %end signature
\begin{itemize}
\item{
{\bf  Description copied from Vertex{\small \refdefined{graphmodel.Vertex}} }

Returns the Y-coordinate of the vertex.
}
\item{{\bf  Returns} -- 
The Y-coordinate of the vertex. 
}%end item
\end{itemize}
}%end item
\item{ 
\index{serialize(List)}
{\bf  serialize}\\
\begin{lstlisting}[frame=none]
SerializedVertex serialize(java.util.List attributes)\end{lstlisting} %end signature
\begin{itemize}
\item{
{\bf  Description copied from Vertex{\small \refdefined{graphmodel.Vertex}} }

Returns a \texttt{\small SerializedVertex}{\small 
\refdefined{graphmodel.SerializedVertex}} representation of the graph.
}
\item{
{\bf  Parameters}
  \begin{itemize}
   \item{
\texttt{attributes} -- The attributes that have to be serialized.}
  \end{itemize}
}%end item
\item{{\bf  Returns} -- 
The \texttt{\small SerializedVertex}{\small 
\refdefined{graphmodel.SerializedVertex}} representation of the graph. 
}%end item
\end{itemize}
}%end item
\end{itemize}
}
}
}
\printindex
