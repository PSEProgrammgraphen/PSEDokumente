\section*{Class Hierarchy}{
\thispagestyle{empty}
\markboth{Class Hierarchy}{Class Hierarchy}
\addcontentsline{toc}{section}{Class Hierarchy}
\subsection*{Classes}
{\raggedright
\hspace{0.0cm} $\bullet$ java.lang.Object {\tiny \refdefined{java.lang.Object}} \\
\hspace{1.0cm} $\bullet$ graphmodel.AbstractEdgeBuilder {\tiny \refdefined{graphmodel.AbstractEdgeBuilder}} \\
\hspace{1.0cm} $\bullet$ graphmodel.AbstractGraphBuilder {\tiny \refdefined{graphmodel.AbstractGraphBuilder}} \\
\hspace{1.0cm} $\bullet$ graphmodel.AbstractGraphModelBuilder {\tiny \refdefined{graphmodel.AbstractGraphModelBuilder}} \\
\hspace{1.0cm} $\bullet$ graphmodel.AbstractVertexBuilder {\tiny \refdefined{graphmodel.AbstractVertexBuilder}} \\
\hspace{1.0cm} $\bullet$ graphmodel.DefaultVertex {\tiny \refdefined{graphmodel.DefaultVertex}} \\
\hspace{1.0cm} $\bullet$ graphmodel.DirectedEdge {\tiny \refdefined{graphmodel.DirectedEdge}} \\
\hspace{1.0cm} $\bullet$ graphmodel.DirectedGraph {\tiny \refdefined{graphmodel.DirectedGraph}} \\
\hspace{1.0cm} $\bullet$ graphmodel.DirectedGraphLayoutRegister {\tiny \refdefined{graphmodel.DirectedGraphLayoutRegister}} \\
\hspace{1.0cm} $\bullet$ graphmodel.FastGraphAccessor {\tiny \refdefined{graphmodel.FastGraphAccessor}} \\
\hspace{1.0cm} $\bullet$ graphmodel.Graph.E {\tiny \refdefined{graphmodel.Graph.E}} \\
\hspace{1.0cm} $\bullet$ graphmodel.Graph.V {\tiny \refdefined{graphmodel.Graph.V}} \\
\hspace{1.0cm} $\bullet$ graphmodel.GraphModel {\tiny \refdefined{graphmodel.GraphModel}} \\
}
\subsection*{Interfaces}
\hspace{0.0cm} $\bullet$ graphmodel.Edge {\tiny \refdefined{graphmodel.Edge}} \\
\hspace{0.0cm} $\bullet$ graphmodel.Graph {\tiny \refdefined{graphmodel.Graph}} \\
\hspace{0.0cm} $\bullet$ graphmodel.Layoutable {\tiny \refdefined{graphmodel.Layoutable}} \\
\hspace{0.0cm} $\bullet$ graphmodel.Vertex {\tiny \refdefined{graphmodel.Vertex}} \\
}
\section{Package graphmodel}{
\label{graphmodel}\hskip -.05in
\hbox to \hsize{\textit{ Package Contents\hfil Page}}
\vskip .13in
\hbox{{\bf  Interfaces}}
\entityintro{Edge}{graphmodel.Edge}{This edge interface specifies an edge.}
\entityintro{Graph}{graphmodel.Graph}{This graph interface specifies a graph.}
\entityintro{Layoutable}{graphmodel.Layoutable}{}
\entityintro{Vertex}{graphmodel.Vertex}{This vertex interface specifies an vertex.}
\vskip .13in
\hbox{{\bf  Classes}}
\entityintro{AbstractEdgeBuilder}{graphmodel.AbstractEdgeBuilder}{This is an abstract Interface which is used to build a concrete edge.}
\entityintro{AbstractGraphBuilder}{graphmodel.AbstractGraphBuilder}{This is an abstract interface which is used to build a concrete graph.}
\entityintro{AbstractGraphModelBuilder}{graphmodel.AbstractGraphModelBuilder}{This is an abstract interface, which builds a concrete graphmodel.}
\entityintro{AbstractVertexBuilder}{graphmodel.AbstractVertexBuilder}{This is an abstract VertexBuilder which is used to build a concrete Vertex.}
\entityintro{DefaultVertex}{graphmodel.DefaultVertex}{This is an DefaultVertex, which has basic functions and is provided by the mainapplication.}
\entityintro{DirectedEdge}{graphmodel.DirectedEdge}{A \texttt{\small DirectedEdge}{\small 
\refdefined{graphmodel.DirectedEdge}} is an edge htat has one source and one target vertex.}
\entityintro{DirectedGraph}{graphmodel.DirectedGraph}{A \texttt{\small DirectedGraph}{\small 
\refdefined{graphmodel.DirectedGraph}} is a specific Graph which contains just \texttt{\small DirectedEdge}{\small 
\refdefined{graphmodel.DirectedEdge}} as edges}
\entityintro{DirectedGraphLayoutRegister}{graphmodel.DirectedGraphLayoutRegister}{}
\entityintro{FastGraphAccessor}{graphmodel.FastGraphAccessor}{Created by Sven on 09.06.2016.}
\entityintro{Graph.E}{graphmodel.Graph.E}{}
\entityintro{Graph.V}{graphmodel.Graph.V}{}
\entityintro{GraphModel}{graphmodel.GraphModel}{A \texttt{\small GraphModel}{\small 
\refdefined{graphmodel.GraphModel}} contains one or more graphs.}
\vskip .1in
\vskip .1in
\subsection{\label{graphmodel.Edge}\index{Edge@\textit{ Edge}}Interface Edge}{
\vskip .1in 
This edge interface specifies an edge. An edge contains two vertices and needs an ID.\vskip .1in 
\subsubsection{Declaration}{
\begin{lstlisting}[frame=none]
public interface Edge
\end{lstlisting}
\subsubsection{All known subinterfaces}{DirectedEdge\small{\refdefined{graphmodel.DirectedEdge}}}
\subsubsection{All classes known to implement interface}{DirectedEdge\small{\refdefined{graphmodel.DirectedEdge}}}
\subsubsection{Method summary}{
\begin{verse}
{\bf getID()} Get the ID of this edge\\
{\bf getName()} Get the name of this Edge\\
{\bf getSource()} Get the vource vertex of this edge\\
{\bf getTarget()} Get the target vertex of this edge\\
\end{verse}
}
\subsubsection{Methods}{
\vskip -2em
\begin{itemize}
\item{ 
\index{getID()}
{\bf  getID}\\
\begin{lstlisting}[frame=none]
java.lang.Integer getID()\end{lstlisting} %end signature
\begin{itemize}
\item{
{\bf  Description}

Get the ID of this edge
}
\item{{\bf  Returns} -- 
 
}%end item
\end{itemize}
}%end item
\item{ 
\index{getName()}
{\bf  getName}\\
\begin{lstlisting}[frame=none]
java.lang.String getName()\end{lstlisting} %end signature
\begin{itemize}
\item{
{\bf  Description}

Get the name of this Edge
}
\item{{\bf  Returns} -- 
 
}%end item
\end{itemize}
}%end item
\item{ 
\index{getSource()}
{\bf  getSource}\\
\begin{lstlisting}[frame=none]
Graph.V getSource()\end{lstlisting} %end signature
\begin{itemize}
\item{
{\bf  Description}

Get the vource vertex of this edge
}
\item{{\bf  Returns} -- 
 
}%end item
\end{itemize}
}%end item
\item{ 
\index{getTarget()}
{\bf  getTarget}\\
\begin{lstlisting}[frame=none]
Graph.V getTarget()\end{lstlisting} %end signature
\begin{itemize}
\item{
{\bf  Description}

Get the target vertex of this edge
}
\item{{\bf  Returns} -- 
 
}%end item
\end{itemize}
}%end item
\end{itemize}
}
}
\subsection{\label{graphmodel.Graph}\index{Graph@\textit{ Graph}}Interface Graph}{
\vskip .1in 
This graph interface specifies a graph. A graph contains edges and vertices.\vskip .1in 
\subsubsection{Declaration}{
\begin{lstlisting}[frame=none]
public interface Graph
\end{lstlisting}
\subsubsection{All known subinterfaces}{DirectedGraph\small{\refdefined{graphmodel.DirectedGraph}}}
\subsubsection{All classes known to implement interface}{DirectedGraph\small{\refdefined{graphmodel.DirectedGraph}}}
\subsubsection{Method summary}{
\begin{verse}
{\bf addEdge()} Adds a new Edge to the graph\\
{\bf addEdge(Graph.E)} Adds a new Edge to the graph\\
{\bf addVertex(Graph.V)} Adds a new Vertex to the graph\\
{\bf edgesOf(Graph.V)} get a list of all edges of a vertex\\
{\bf getEdges()} Returns an list of edges.\\
{\bf getEdgeSet()} Get a set of all Edges in this graph\\
{\bf getSource(Graph.E)} Get the source vertex of an edge\\
{\bf getTarget(Graph.E)} Get the target vertex of an edge\\
{\bf getVertexSet()} Get a set of all vertices in this graph\\
{\bf setEdgeSet(Set)} Set a Set of edges\\
\end{verse}
}
\subsubsection{Methods}{
\vskip -2em
\begin{itemize}
\item{ 
\index{addEdge()}
{\bf  addEdge}\\
\begin{lstlisting}[frame=none]
void addEdge()\end{lstlisting} %end signature
\begin{itemize}
\item{
{\bf  Description}

Adds a new Edge to the graph
}
\end{itemize}
}%end item
\item{ 
\index{addEdge(Graph.E)}
{\bf  addEdge}\\
\begin{lstlisting}[frame=none]
void addEdge(Graph.E edge)\end{lstlisting} %end signature
\begin{itemize}
\item{
{\bf  Description}

Adds a new Edge to the graph
}
\item{
{\bf  Parameters}
  \begin{itemize}
   \item{
\texttt{edge} -- }
  \end{itemize}
}%end item
\end{itemize}
}%end item
\item{ 
\index{addVertex(Graph.V)}
{\bf  addVertex}\\
\begin{lstlisting}[frame=none]
void addVertex(Graph.V vertex)\end{lstlisting} %end signature
\begin{itemize}
\item{
{\bf  Description}

Adds a new Vertex to the graph
}
\item{
{\bf  Parameters}
  \begin{itemize}
   \item{
\texttt{vertex} -- }
  \end{itemize}
}%end item
\end{itemize}
}%end item
\item{ 
\index{edgesOf(Graph.V)}
{\bf  edgesOf}\\
\begin{lstlisting}[frame=none]
java.util.List edgesOf(Graph.V vertex)\end{lstlisting} %end signature
\begin{itemize}
\item{
{\bf  Description}

get a list of all edges of a vertex
}
\item{
{\bf  Parameters}
  \begin{itemize}
   \item{
\texttt{vertex} -- }
  \end{itemize}
}%end item
\item{{\bf  Returns} -- 
 
}%end item
\end{itemize}
}%end item
\item{ 
\index{getEdges()}
{\bf  getEdges}\\
\begin{lstlisting}[frame=none]
java.util.List getEdges()\end{lstlisting} %end signature
\begin{itemize}
\item{
{\bf  Description}

Returns an list of edges.
}
\item{{\bf  Returns} -- 
 
}%end item
\end{itemize}
}%end item
\item{ 
\index{getEdgeSet()}
{\bf  getEdgeSet}\\
\begin{lstlisting}[frame=none]
java.util.Set getEdgeSet()\end{lstlisting} %end signature
\begin{itemize}
\item{
{\bf  Description}

Get a set of all Edges in this graph
}
\item{{\bf  Returns} -- 
 
}%end item
\end{itemize}
}%end item
\item{ 
\index{getSource(Graph.E)}
{\bf  getSource}\\
\begin{lstlisting}[frame=none]
Graph.V getSource(Graph.E edge)\end{lstlisting} %end signature
\begin{itemize}
\item{
{\bf  Description}

Get the source vertex of an edge
}
\item{
{\bf  Parameters}
  \begin{itemize}
   \item{
\texttt{edge} -- }
  \end{itemize}
}%end item
\item{{\bf  Returns} -- 
 
}%end item
\end{itemize}
}%end item
\item{ 
\index{getTarget(Graph.E)}
{\bf  getTarget}\\
\begin{lstlisting}[frame=none]
Graph.V getTarget(Graph.E edge)\end{lstlisting} %end signature
\begin{itemize}
\item{
{\bf  Description}

Get the target vertex of an edge
}
\item{
{\bf  Parameters}
  \begin{itemize}
   \item{
\texttt{edge} -- }
  \end{itemize}
}%end item
\item{{\bf  Returns} -- 
 
}%end item
\end{itemize}
}%end item
\item{ 
\index{getVertexSet()}
{\bf  getVertexSet}\\
\begin{lstlisting}[frame=none]
java.util.Set getVertexSet()\end{lstlisting} %end signature
\begin{itemize}
\item{
{\bf  Description}

Get a set of all vertices in this graph
}
\item{{\bf  Returns} -- 
 
}%end item
\end{itemize}
}%end item
\item{ 
\index{setEdgeSet(Set)}
{\bf  setEdgeSet}\\
\begin{lstlisting}[frame=none]
void setEdgeSet(java.util.Set edge)\end{lstlisting} %end signature
\begin{itemize}
\item{
{\bf  Description}

Set a Set of edges
}
\item{
{\bf  Parameters}
  \begin{itemize}
   \item{
\texttt{edge} -- }
  \end{itemize}
}%end item
\end{itemize}
}%end item
\end{itemize}
}
}
\subsection{\label{graphmodel.Layoutable}\index{Layoutable@\textit{ Layoutable}}Interface Layoutable}{
\vskip .1in 
\subsubsection{Declaration}{
\begin{lstlisting}[frame=none]
public interface Layoutable
\end{lstlisting}
\subsubsection{All known subinterfaces}{DirectedGraph\small{\refdefined{graphmodel.DirectedGraph}}}
\subsubsection{All classes known to implement interface}{DirectedGraph\small{\refdefined{graphmodel.DirectedGraph}}}
\subsubsection{Method summary}{
\begin{verse}
{\bf applyLayout(LayoutAlgorithm)} \\
{\bf getRegisteredLayouts()} \\
\end{verse}
}
\subsubsection{Methods}{
\vskip -2em
\begin{itemize}
\item{ 
\index{applyLayout(LayoutAlgorithm)}
{\bf  applyLayout}\\
\begin{lstlisting}[frame=none]
void applyLayout(plugin.LayoutAlgorithm alg)\end{lstlisting} %end signature
}%end item
\item{ 
\index{getRegisteredLayouts()}
{\bf  getRegisteredLayouts}\\
\begin{lstlisting}[frame=none]
java.util.List getRegisteredLayouts()\end{lstlisting} %end signature
}%end item
\end{itemize}
}
}
\subsection{\label{graphmodel.Vertex}\index{Vertex@\textit{ Vertex}}Interface Vertex}{
\vskip .1in 
This vertex interface specifies an vertex. An vertex contains an ID and a name\vskip .1in 
\subsubsection{Declaration}{
\begin{lstlisting}[frame=none]
public interface Vertex
\end{lstlisting}
\subsubsection{All known subinterfaces}{DefaultVertex\small{\refdefined{graphmodel.DefaultVertex}}}
\subsubsection{All classes known to implement interface}{DefaultVertex\small{\refdefined{graphmodel.DefaultVertex}}}
\subsubsection{Method summary}{
\begin{verse}
{\bf getID()} Get the ID of the vertex\\
{\bf getName()} Get the name of the vertex\\
\end{verse}
}
\subsubsection{Methods}{
\vskip -2em
\begin{itemize}
\item{ 
\index{getID()}
{\bf  getID}\\
\begin{lstlisting}[frame=none]
java.lang.Integer getID()\end{lstlisting} %end signature
\begin{itemize}
\item{
{\bf  Description}

Get the ID of the vertex
}
\item{{\bf  Returns} -- 
 
}%end item
\end{itemize}
}%end item
\item{ 
\index{getName()}
{\bf  getName}\\
\begin{lstlisting}[frame=none]
java.lang.String getName()\end{lstlisting} %end signature
\begin{itemize}
\item{
{\bf  Description}

Get the name of the vertex
}
\item{{\bf  Returns} -- 
 
}%end item
\end{itemize}
}%end item
\end{itemize}
}
}
\subsection{\label{graphmodel.AbstractEdgeBuilder}\index{AbstractEdgeBuilder}Class AbstractEdgeBuilder}{
\vskip .1in 
This is an abstract Interface which is used to build a concrete edge.\vskip .1in 
\subsubsection{Declaration}{
\begin{lstlisting}[frame=none]
public abstract class AbstractEdgeBuilder
 extends java.lang.Object\end{lstlisting}
\subsubsection{Constructor summary}{
\begin{verse}
{\bf AbstractEdgeBuilder()} \\
\end{verse}
}
\subsubsection{Method summary}{
\begin{verse}
{\bf addData(String, String)} Add optional data to this edge.\\
{\bf newEdge(String, String)} set source and target vertices of this edge\\
{\bf setDirection(String)} set the direction of this edge\\
{\bf setID(String)} set the ID of this edge\\
\end{verse}
}
\subsubsection{Constructors}{
\vskip -2em
\begin{itemize}
\item{ 
\index{AbstractEdgeBuilder()}
{\bf  AbstractEdgeBuilder}\\
\begin{lstlisting}[frame=none]
public AbstractEdgeBuilder()\end{lstlisting} %end signature
}%end item
\end{itemize}
}
\subsubsection{Methods}{
\vskip -2em
\begin{itemize}
\item{ 
\index{addData(String, String)}
{\bf  addData}\\
\begin{lstlisting}[frame=none]
public abstract void addData(java.lang.String keyname,java.lang.String value)\end{lstlisting} %end signature
\begin{itemize}
\item{
{\bf  Description}

Add optional data to this edge. The EdgeBuilder needs to decide how to save the value.
}
\item{
{\bf  Parameters}
  \begin{itemize}
   \item{
\texttt{keyname} -- }
   \item{
\texttt{value} -- }
  \end{itemize}
}%end item
\end{itemize}
}%end item
\item{ 
\index{newEdge(String, String)}
{\bf  newEdge}\\
\begin{lstlisting}[frame=none]
public abstract void newEdge(java.lang.String source,java.lang.String target)\end{lstlisting} %end signature
\begin{itemize}
\item{
{\bf  Description}

set source and target vertices of this edge
}
\item{
{\bf  Parameters}
  \begin{itemize}
   \item{
\texttt{source} -- }
   \item{
\texttt{target} -- }
  \end{itemize}
}%end item
\end{itemize}
}%end item
\item{ 
\index{setDirection(String)}
{\bf  setDirection}\\
\begin{lstlisting}[frame=none]
public abstract void setDirection(java.lang.String direction)\end{lstlisting} %end signature
\begin{itemize}
\item{
{\bf  Description}

set the direction of this edge
}
\item{
{\bf  Parameters}
  \begin{itemize}
   \item{
\texttt{direction} -- }
  \end{itemize}
}%end item
\end{itemize}
}%end item
\item{ 
\index{setID(String)}
{\bf  setID}\\
\begin{lstlisting}[frame=none]
public abstract void setID(java.lang.String id)\end{lstlisting} %end signature
\begin{itemize}
\item{
{\bf  Description}

set the ID of this edge
}
\item{
{\bf  Parameters}
  \begin{itemize}
   \item{
\texttt{id} -- }
  \end{itemize}
}%end item
\end{itemize}
}%end item
\end{itemize}
}
}
\subsection{\label{graphmodel.AbstractGraphBuilder}\index{AbstractGraphBuilder}Class AbstractGraphBuilder}{
\vskip .1in 
This is an abstract interface which is used to build a concrete graph.\vskip .1in 
\subsubsection{Declaration}{
\begin{lstlisting}[frame=none]
public abstract class AbstractGraphBuilder
 extends java.lang.Object\end{lstlisting}
\subsubsection{Constructor summary}{
\begin{verse}
{\bf AbstractGraphBuilder()} \\
\end{verse}
}
\subsubsection{Method summary}{
\begin{verse}
{\bf build()} This method is called, when the buildingprocess of the graph is finished.\\
{\bf getEdgeBuilder()} Returns the EdgeBuilder which is specified for this graph.\\
{\bf getVertexBuilder(String)} Returns the VertexBuilder which is specified for this graph.\\
\end{verse}
}
\subsubsection{Constructors}{
\vskip -2em
\begin{itemize}
\item{ 
\index{AbstractGraphBuilder()}
{\bf  AbstractGraphBuilder}\\
\begin{lstlisting}[frame=none]
public AbstractGraphBuilder()\end{lstlisting} %end signature
}%end item
\end{itemize}
}
\subsubsection{Methods}{
\vskip -2em
\begin{itemize}
\item{ 
\index{build()}
{\bf  build}\\
\begin{lstlisting}[frame=none]
public Graph build()\end{lstlisting} %end signature
\begin{itemize}
\item{
{\bf  Description}

This method is called, when the buildingprocess of the graph is finished. Then it builds the graph and returns it.
}
\item{{\bf  Returns} -- 
Graph 
}%end item
\end{itemize}
}%end item
\item{ 
\index{getEdgeBuilder()}
{\bf  getEdgeBuilder}\\
\begin{lstlisting}[frame=none]
public abstract AbstractEdgeBuilder getEdgeBuilder()\end{lstlisting} %end signature
\begin{itemize}
\item{
{\bf  Description}

Returns the EdgeBuilder which is specified for this graph.
}
\item{{\bf  Returns} -- 
AbstractEdgeBuilder 
}%end item
\end{itemize}
}%end item
\item{ 
\index{getVertexBuilder(String)}
{\bf  getVertexBuilder}\\
\begin{lstlisting}[frame=none]
public abstract AbstractVertexBuilder getVertexBuilder(java.lang.String vertexID)\end{lstlisting} %end signature
\begin{itemize}
\item{
{\bf  Description}

Returns the VertexBuilder which is specified for this graph.
}
\item{
{\bf  Parameters}
  \begin{itemize}
   \item{
\texttt{vertexID} -- }
  \end{itemize}
}%end item
\item{{\bf  Returns} -- 
AbstractVertexBuilder 
}%end item
\end{itemize}
}%end item
\end{itemize}
}
}
\subsection{\label{graphmodel.AbstractGraphModelBuilder}\index{AbstractGraphModelBuilder}Class AbstractGraphModelBuilder}{
\vskip .1in 
This is an abstract interface, which builds a concrete graphmodel. This Class is based on the Builder Pattern.\vskip .1in 
\subsubsection{Declaration}{
\begin{lstlisting}[frame=none]
public abstract class AbstractGraphModelBuilder
 extends java.lang.Object\end{lstlisting}
\subsubsection{Constructor summary}{
\begin{verse}
{\bf AbstractGraphModelBuilder()} \\
\end{verse}
}
\subsubsection{Method summary}{
\begin{verse}
{\bf build()} This method is called, when the buildingprocess of the graphmodel is finished.\\
{\bf getGraphBuilder(String)} Returns a specific \texttt{\small AbstractGraphBuilder}{\small 
\refdefined{graphmodel.AbstractGraphBuilder}} which belongs to the graphmodel.\\
\end{verse}
}
\subsubsection{Constructors}{
\vskip -2em
\begin{itemize}
\item{ 
\index{AbstractGraphModelBuilder()}
{\bf  AbstractGraphModelBuilder}\\
\begin{lstlisting}[frame=none]
public AbstractGraphModelBuilder()\end{lstlisting} %end signature
}%end item
\end{itemize}
}
\subsubsection{Methods}{
\vskip -2em
\begin{itemize}
\item{ 
\index{build()}
{\bf  build}\\
\begin{lstlisting}[frame=none]
public GraphModel build()\end{lstlisting} %end signature
\begin{itemize}
\item{
{\bf  Description}

This method is called, when the buildingprocess of the graphmodel is finished. It returns the finished graphmodel
}
\item{{\bf  Returns} -- 
GraphModel 
}%end item
\end{itemize}
}%end item
\item{ 
\index{getGraphBuilder(String)}
{\bf  getGraphBuilder}\\
\begin{lstlisting}[frame=none]
public abstract AbstractGraphBuilder getGraphBuilder(java.lang.String graphID)\end{lstlisting} %end signature
\begin{itemize}
\item{
{\bf  Description}

Returns a specific \texttt{\small AbstractGraphBuilder}{\small 
\refdefined{graphmodel.AbstractGraphBuilder}} which belongs to the graphmodel. It should also decide which graph should be builded with the help of the graphID.
}
\item{
{\bf  Parameters}
  \begin{itemize}
   \item{
\texttt{graphID} -- }
  \end{itemize}
}%end item
\item{{\bf  Returns} -- 
 
}%end item
\end{itemize}
}%end item
\end{itemize}
}
}
\subsection{\label{graphmodel.AbstractVertexBuilder}\index{AbstractVertexBuilder}Class AbstractVertexBuilder}{
\vskip .1in 
This is an abstract VertexBuilder which is used to build a concrete Vertex.\vskip .1in 
\subsubsection{Declaration}{
\begin{lstlisting}[frame=none]
public abstract class AbstractVertexBuilder
 extends java.lang.Object\end{lstlisting}
\subsubsection{Constructor summary}{
\begin{verse}
{\bf AbstractVertexBuilder()} \\
\end{verse}
}
\subsubsection{Method summary}{
\begin{verse}
{\bf addData(String, String)} Add Data to this Vertex.\\
{\bf build()} This method builds the concrete Vertex and returns it.\\
{\bf getGraphBuilder(String)} This method returns an specific GraphBuilder.\\
\end{verse}
}
\subsubsection{Constructors}{
\vskip -2em
\begin{itemize}
\item{ 
\index{AbstractVertexBuilder()}
{\bf  AbstractVertexBuilder}\\
\begin{lstlisting}[frame=none]
public AbstractVertexBuilder()\end{lstlisting} %end signature
}%end item
\end{itemize}
}
\subsubsection{Methods}{
\vskip -2em
\begin{itemize}
\item{ 
\index{addData(String, String)}
{\bf  addData}\\
\begin{lstlisting}[frame=none]
public abstract void addData(java.lang.String value,java.lang.String keyname)\end{lstlisting} %end signature
\begin{itemize}
\item{
{\bf  Description}

Add Data to this Vertex. The vertexBuilder decides which kind of data it is and where to save in the concrete Vertex.
}
\item{
{\bf  Parameters}
  \begin{itemize}
   \item{
\texttt{value} -- }
   \item{
\texttt{keyname} -- }
  \end{itemize}
}%end item
\end{itemize}
}%end item
\item{ 
\index{build()}
{\bf  build}\\
\begin{lstlisting}[frame=none]
public Vertex build()\end{lstlisting} %end signature
\begin{itemize}
\item{
{\bf  Description}

This method builds the concrete Vertex and returns it.
}
\item{{\bf  Returns} -- 
Vertex 
}%end item
\end{itemize}
}%end item
\item{ 
\index{getGraphBuilder(String)}
{\bf  getGraphBuilder}\\
\begin{lstlisting}[frame=none]
public AbstractGraphBuilder getGraphBuilder(java.lang.String graphID)\end{lstlisting} %end signature
\begin{itemize}
\item{
{\bf  Description}

This method returns an specific GraphBuilder. This method is used to implement nested Graphs.
}
\item{
{\bf  Parameters}
  \begin{itemize}
   \item{
\texttt{graphID} -- }
  \end{itemize}
}%end item
\item{{\bf  Returns} -- 
 
}%end item
\end{itemize}
}%end item
\end{itemize}
}
}
\subsection{\label{graphmodel.DefaultVertex}\index{DefaultVertex}Class DefaultVertex}{
\vskip .1in 
This is an DefaultVertex, which has basic functions and is provided by the mainapplication.\vskip .1in 
\subsubsection{Declaration}{
\begin{lstlisting}[frame=none]
public class DefaultVertex
 extends java.lang.Object implements Vertex\end{lstlisting}
\subsubsection{Constructor summary}{
\begin{verse}
{\bf DefaultVertex()} \\
\end{verse}
}
\subsubsection{Method summary}{
\begin{verse}
{\bf getID()} \\
{\bf getName()} \\
\end{verse}
}
\subsubsection{Constructors}{
\vskip -2em
\begin{itemize}
\item{ 
\index{DefaultVertex()}
{\bf  DefaultVertex}\\
\begin{lstlisting}[frame=none]
public DefaultVertex()\end{lstlisting} %end signature
}%end item
\end{itemize}
}
\subsubsection{Methods}{
\vskip -2em
\begin{itemize}
\item{ 
\index{getID()}
{\bf  getID}\\
\begin{lstlisting}[frame=none]
java.lang.Integer getID()\end{lstlisting} %end signature
\begin{itemize}
\item{
{\bf  Description copied from Vertex{\small \refdefined{graphmodel.Vertex}} }

Get the ID of the vertex
}
\item{{\bf  Returns} -- 
 
}%end item
\end{itemize}
}%end item
\item{ 
\index{getName()}
{\bf  getName}\\
\begin{lstlisting}[frame=none]
java.lang.String getName()\end{lstlisting} %end signature
\begin{itemize}
\item{
{\bf  Description copied from Vertex{\small \refdefined{graphmodel.Vertex}} }

Get the name of the vertex
}
\item{{\bf  Returns} -- 
 
}%end item
\end{itemize}
}%end item
\end{itemize}
}
}
\subsection{\label{graphmodel.DirectedEdge}\index{DirectedEdge}Class DirectedEdge}{
\vskip .1in 
A \texttt{\small DirectedEdge}{\small 
\refdefined{graphmodel.DirectedEdge}} is an edge htat has one source and one target vertex. So the direction of the edge is specified.\vskip .1in 
\subsubsection{Declaration}{
\begin{lstlisting}[frame=none]
public class DirectedEdge
 extends java.lang.Object implements Edge\end{lstlisting}
\subsubsection{Constructor summary}{
\begin{verse}
{\bf DirectedEdge()} \\
\end{verse}
}
\subsubsection{Method summary}{
\begin{verse}
{\bf getID()} \\
{\bf getName()} \\
{\bf getSource()} \\
{\bf getTarget()} \\
\end{verse}
}
\subsubsection{Constructors}{
\vskip -2em
\begin{itemize}
\item{ 
\index{DirectedEdge()}
{\bf  DirectedEdge}\\
\begin{lstlisting}[frame=none]
public DirectedEdge()\end{lstlisting} %end signature
}%end item
\end{itemize}
}
\subsubsection{Methods}{
\vskip -2em
\begin{itemize}
\item{ 
\index{getID()}
{\bf  getID}\\
\begin{lstlisting}[frame=none]
java.lang.Integer getID()\end{lstlisting} %end signature
\begin{itemize}
\item{
{\bf  Description copied from Edge{\small \refdefined{graphmodel.Edge}} }

Get the ID of this edge
}
\item{{\bf  Returns} -- 
 
}%end item
\end{itemize}
}%end item
\item{ 
\index{getName()}
{\bf  getName}\\
\begin{lstlisting}[frame=none]
java.lang.String getName()\end{lstlisting} %end signature
\begin{itemize}
\item{
{\bf  Description copied from Edge{\small \refdefined{graphmodel.Edge}} }

Get the name of this Edge
}
\item{{\bf  Returns} -- 
 
}%end item
\end{itemize}
}%end item
\item{ 
\index{getSource()}
{\bf  getSource}\\
\begin{lstlisting}[frame=none]
Graph.V getSource()\end{lstlisting} %end signature
\begin{itemize}
\item{
{\bf  Description copied from Edge{\small \refdefined{graphmodel.Edge}} }

Get the vource vertex of this edge
}
\item{{\bf  Returns} -- 
 
}%end item
\end{itemize}
}%end item
\item{ 
\index{getTarget()}
{\bf  getTarget}\\
\begin{lstlisting}[frame=none]
Graph.V getTarget()\end{lstlisting} %end signature
\begin{itemize}
\item{
{\bf  Description copied from Edge{\small \refdefined{graphmodel.Edge}} }

Get the target vertex of this edge
}
\item{{\bf  Returns} -- 
 
}%end item
\end{itemize}
}%end item
\end{itemize}
}
}
\subsection{\label{graphmodel.DirectedGraph}\index{DirectedGraph}Class DirectedGraph}{
\vskip .1in 
A \texttt{\small DirectedGraph}{\small 
\refdefined{graphmodel.DirectedGraph}} is a specific Graph which contains just \texttt{\small DirectedEdge}{\small 
\refdefined{graphmodel.DirectedEdge}} as edges\vskip .1in 
\subsubsection{Declaration}{
\begin{lstlisting}[frame=none]
public class DirectedGraph
 extends java.lang.Object implements Graph, Layoutable\end{lstlisting}
\subsubsection{Constructor summary}{
\begin{verse}
{\bf DirectedGraph()} \\
\end{verse}
}
\subsubsection{Method summary}{
\begin{verse}
{\bf addEdge()} \\
{\bf addEdge(Graph.E)} \\
{\bf addVertex(Graph.V)} \\
{\bf applyLayout(LayoutAlgorithm)} \\
{\bf edgesOf(Graph.V)} \\
{\bf getEdges()} \\
{\bf getEdgeSet()} \\
{\bf getRegisteredLayouts()} \\
{\bf getSource(Graph.E)} \\
{\bf getTarget(Graph.E)} \\
{\bf getVertexSet()} \\
{\bf incomingEdgesOf(Graph.V)} Get a set of all incoming edges of a vertex\\
{\bf indegreeOf(Graph.V)} Get the indegree of a vertex\\
{\bf outcomingEdgesOf(Graph.V)} Get a set of all outcoming edges of a vertex\\
{\bf outdegreeOf(Graph.V)} Get the outdegree of a vertex\\
{\bf setEdgeSet(Set)} \\
\end{verse}
}
\subsubsection{Constructors}{
\vskip -2em
\begin{itemize}
\item{ 
\index{DirectedGraph()}
{\bf  DirectedGraph}\\
\begin{lstlisting}[frame=none]
public DirectedGraph()\end{lstlisting} %end signature
}%end item
\end{itemize}
}
\subsubsection{Methods}{
\vskip -2em
\begin{itemize}
\item{ 
\index{addEdge()}
{\bf  addEdge}\\
\begin{lstlisting}[frame=none]
void addEdge()\end{lstlisting} %end signature
\begin{itemize}
\item{
{\bf  Description copied from Graph{\small \refdefined{graphmodel.Graph}} }

Adds a new Edge to the graph
}
\end{itemize}
}%end item
\item{ 
\index{addEdge(Graph.E)}
{\bf  addEdge}\\
\begin{lstlisting}[frame=none]
public void addEdge(Graph.E edge)\end{lstlisting} %end signature
\begin{itemize}
\item{
{\bf  Parameters}
  \begin{itemize}
   \item{
\texttt{edge} -- }
  \end{itemize}
}%end item
\end{itemize}
}%end item
\item{ 
\index{addVertex(Graph.V)}
{\bf  addVertex}\\
\begin{lstlisting}[frame=none]
public void addVertex(Graph.V vertex)\end{lstlisting} %end signature
\begin{itemize}
\item{
{\bf  Parameters}
  \begin{itemize}
   \item{
\texttt{vertex} -- }
  \end{itemize}
}%end item
\end{itemize}
}%end item
\item{ 
\index{applyLayout(LayoutAlgorithm)}
{\bf  applyLayout}\\
\begin{lstlisting}[frame=none]
void applyLayout(plugin.LayoutAlgorithm alg)\end{lstlisting} %end signature
}%end item
\item{ 
\index{edgesOf(Graph.V)}
{\bf  edgesOf}\\
\begin{lstlisting}[frame=none]
public java.util.List edgesOf(Graph.V vertex)\end{lstlisting} %end signature
\begin{itemize}
\item{
{\bf  Parameters}
  \begin{itemize}
   \item{
\texttt{vertex} -- }
  \end{itemize}
}%end item
\item{{\bf  Returns} -- 
 
}%end item
\end{itemize}
}%end item
\item{ 
\index{getEdges()}
{\bf  getEdges}\\
\begin{lstlisting}[frame=none]
java.util.List getEdges()\end{lstlisting} %end signature
\begin{itemize}
\item{
{\bf  Description copied from Graph{\small \refdefined{graphmodel.Graph}} }

Returns an list of edges.
}
\item{{\bf  Returns} -- 
 
}%end item
\end{itemize}
}%end item
\item{ 
\index{getEdgeSet()}
{\bf  getEdgeSet}\\
\begin{lstlisting}[frame=none]
java.util.Set getEdgeSet()\end{lstlisting} %end signature
\begin{itemize}
\item{
{\bf  Description copied from Graph{\small \refdefined{graphmodel.Graph}} }

Get a set of all Edges in this graph
}
\item{{\bf  Returns} -- 
 
}%end item
\end{itemize}
}%end item
\item{ 
\index{getRegisteredLayouts()}
{\bf  getRegisteredLayouts}\\
\begin{lstlisting}[frame=none]
java.util.List getRegisteredLayouts()\end{lstlisting} %end signature
}%end item
\item{ 
\index{getSource(Graph.E)}
{\bf  getSource}\\
\begin{lstlisting}[frame=none]
public Graph.V getSource(Graph.E edge)\end{lstlisting} %end signature
\begin{itemize}
\item{
{\bf  Parameters}
  \begin{itemize}
   \item{
\texttt{edge} -- }
  \end{itemize}
}%end item
\item{{\bf  Returns} -- 
 
}%end item
\end{itemize}
}%end item
\item{ 
\index{getTarget(Graph.E)}
{\bf  getTarget}\\
\begin{lstlisting}[frame=none]
public Graph.V getTarget(Graph.E edge)\end{lstlisting} %end signature
\begin{itemize}
\item{
{\bf  Parameters}
  \begin{itemize}
   \item{
\texttt{edge} -- }
  \end{itemize}
}%end item
\item{{\bf  Returns} -- 
 
}%end item
\end{itemize}
}%end item
\item{ 
\index{getVertexSet()}
{\bf  getVertexSet}\\
\begin{lstlisting}[frame=none]
java.util.Set getVertexSet()\end{lstlisting} %end signature
\begin{itemize}
\item{
{\bf  Description copied from Graph{\small \refdefined{graphmodel.Graph}} }

Get a set of all vertices in this graph
}
\item{{\bf  Returns} -- 
 
}%end item
\end{itemize}
}%end item
\item{ 
\index{incomingEdgesOf(Graph.V)}
{\bf  incomingEdgesOf}\\
\begin{lstlisting}[frame=none]
public java.util.Set incomingEdgesOf(Graph.V vertex)\end{lstlisting} %end signature
\begin{itemize}
\item{
{\bf  Description}

Get a set of all incoming edges of a vertex
}
\item{
{\bf  Parameters}
  \begin{itemize}
   \item{
\texttt{vertex} -- }
  \end{itemize}
}%end item
\item{{\bf  Returns} -- 
 
}%end item
\end{itemize}
}%end item
\item{ 
\index{indegreeOf(Graph.V)}
{\bf  indegreeOf}\\
\begin{lstlisting}[frame=none]
public java.lang.Integer indegreeOf(Graph.V vertex)\end{lstlisting} %end signature
\begin{itemize}
\item{
{\bf  Description}

Get the indegree of a vertex
}
\item{
{\bf  Parameters}
  \begin{itemize}
   \item{
\texttt{vertex} -- }
  \end{itemize}
}%end item
\item{{\bf  Returns} -- 
 
}%end item
\end{itemize}
}%end item
\item{ 
\index{outcomingEdgesOf(Graph.V)}
{\bf  outcomingEdgesOf}\\
\begin{lstlisting}[frame=none]
public java.util.Set outcomingEdgesOf(Graph.V vertex)\end{lstlisting} %end signature
\begin{itemize}
\item{
{\bf  Description}

Get a set of all outcoming edges of a vertex
}
\item{
{\bf  Parameters}
  \begin{itemize}
   \item{
\texttt{vertex} -- }
  \end{itemize}
}%end item
\item{{\bf  Returns} -- 
 
}%end item
\end{itemize}
}%end item
\item{ 
\index{outdegreeOf(Graph.V)}
{\bf  outdegreeOf}\\
\begin{lstlisting}[frame=none]
public java.lang.Integer outdegreeOf(Graph.V vertex)\end{lstlisting} %end signature
\begin{itemize}
\item{
{\bf  Description}

Get the outdegree of a vertex
}
\item{
{\bf  Parameters}
  \begin{itemize}
   \item{
\texttt{vertex} -- }
  \end{itemize}
}%end item
\item{{\bf  Returns} -- 
 
}%end item
\end{itemize}
}%end item
\item{ 
\index{setEdgeSet(Set)}
{\bf  setEdgeSet}\\
\begin{lstlisting}[frame=none]
public void setEdgeSet(java.util.Set edge)\end{lstlisting} %end signature
\begin{itemize}
\item{
{\bf  Parameters}
  \begin{itemize}
   \item{
\texttt{edge} -- }
  \end{itemize}
}%end item
\end{itemize}
}%end item
\end{itemize}
}
}
\subsection{\label{graphmodel.DirectedGraphLayoutRegister}\index{DirectedGraphLayoutRegister}Class DirectedGraphLayoutRegister}{
\vskip .1in 
\subsubsection{Declaration}{
\begin{lstlisting}[frame=none]
public class DirectedGraphLayoutRegister
 extends java.lang.Object implements plugin.LayoutRegister\end{lstlisting}
\subsubsection{Constructor summary}{
\begin{verse}
{\bf DirectedGraphLayoutRegister()} \\
\end{verse}
}
\subsubsection{Method summary}{
\begin{verse}
{\bf addLayoutOption(LayoutOption)} \\
{\bf getLayoutOptions()} \\
\end{verse}
}
\subsubsection{Constructors}{
\vskip -2em
\begin{itemize}
\item{ 
\index{DirectedGraphLayoutRegister()}
{\bf  DirectedGraphLayoutRegister}\\
\begin{lstlisting}[frame=none]
public DirectedGraphLayoutRegister()\end{lstlisting} %end signature
}%end item
\end{itemize}
}
\subsubsection{Methods}{
\vskip -2em
\begin{itemize}
\item{ 
\index{addLayoutOption(LayoutOption)}
{\bf  addLayoutOption}\\
\begin{lstlisting}[frame=none]
void addLayoutOption(plugin.LayoutOption option)\end{lstlisting} %end signature
}%end item
\item{ 
\index{getLayoutOptions()}
{\bf  getLayoutOptions}\\
\begin{lstlisting}[frame=none]
java.util.List getLayoutOptions()\end{lstlisting} %end signature
}%end item
\end{itemize}
}
}
\subsection{\label{graphmodel.FastGraphAccessor}\index{FastGraphAccessor}Class FastGraphAccessor}{
\vskip .1in 
Created by Sven on 09.06.2016.\vskip .1in 
\subsubsection{Declaration}{
\begin{lstlisting}[frame=none]
public class FastGraphAccessor
 extends java.lang.Object\end{lstlisting}
\subsubsection{Constructor summary}{
\begin{verse}
{\bf FastGraphAccessor()} \\
\end{verse}
}
\subsubsection{Method summary}{
\begin{verse}
{\bf addEdgeAttribute(String)} \\
{\bf addEdgeForAttribute(String, Edge, String)} \\
{\bf addVertexAttribute(String)} \\
{\bf addVertexForAttribute(Vertex, String, String)} \\
{\bf getEdgesByAttribute(String, String)} \\
{\bf getEdgeValues()} Getter of edgeValues\\
{\bf getGraph()} Getter of graph\\
{\bf getVertexValues()} Getter of vertexValues\\
{\bf getVerticesByAttribute(String, String)} \\
{\bf setEdgeValues(Map)} Setter of edgeValues\\
{\bf setGraph(Graph)} Setter of graph\\
{\bf setVertexValues(Map)} Setter of vertexValues\\
{\bf update()} \\
\end{verse}
}
\subsubsection{Constructors}{
\vskip -2em
\begin{itemize}
\item{ 
\index{FastGraphAccessor()}
{\bf  FastGraphAccessor}\\
\begin{lstlisting}[frame=none]
public FastGraphAccessor()\end{lstlisting} %end signature
}%end item
\end{itemize}
}
\subsubsection{Methods}{
\vskip -2em
\begin{itemize}
\item{ 
\index{addEdgeAttribute(String)}
{\bf  addEdgeAttribute}\\
\begin{lstlisting}[frame=none]
public void addEdgeAttribute(java.lang.String name)\end{lstlisting} %end signature
\begin{itemize}
\item{
{\bf  Parameters}
  \begin{itemize}
   \item{
\texttt{name} -- }
  \end{itemize}
}%end item
\end{itemize}
}%end item
\item{ 
\index{addEdgeForAttribute(String, Edge, String)}
{\bf  addEdgeForAttribute}\\
\begin{lstlisting}[frame=none]
public void addEdgeForAttribute(java.lang.String name,Edge edge,java.lang.String value)\end{lstlisting} %end signature
\begin{itemize}
\item{
{\bf  Parameters}
  \begin{itemize}
   \item{
\texttt{name} -- }
   \item{
\texttt{edge} -- }
   \item{
\texttt{value} -- }
  \end{itemize}
}%end item
\end{itemize}
}%end item
\item{ 
\index{addVertexAttribute(String)}
{\bf  addVertexAttribute}\\
\begin{lstlisting}[frame=none]
public void addVertexAttribute(java.lang.String name)\end{lstlisting} %end signature
\begin{itemize}
\item{
{\bf  Parameters}
  \begin{itemize}
   \item{
\texttt{name} -- }
  \end{itemize}
}%end item
\end{itemize}
}%end item
\item{ 
\index{addVertexForAttribute(Vertex, String, String)}
{\bf  addVertexForAttribute}\\
\begin{lstlisting}[frame=none]
public void addVertexForAttribute(Vertex vertex,java.lang.String value,java.lang.String name)\end{lstlisting} %end signature
\begin{itemize}
\item{
{\bf  Parameters}
  \begin{itemize}
   \item{
\texttt{vertex} -- }
   \item{
\texttt{value} -- }
   \item{
\texttt{name} -- }
  \end{itemize}
}%end item
\end{itemize}
}%end item
\item{ 
\index{getEdgesByAttribute(String, String)}
{\bf  getEdgesByAttribute}\\
\begin{lstlisting}[frame=none]
public java.util.List getEdgesByAttribute(java.lang.String value,java.lang.String name)\end{lstlisting} %end signature
\begin{itemize}
\item{
{\bf  Parameters}
  \begin{itemize}
   \item{
\texttt{value} -- }
   \item{
\texttt{name} -- }
  \end{itemize}
}%end item
\item{{\bf  Returns} -- 
 
}%end item
\end{itemize}
}%end item
\item{ 
\index{getEdgeValues()}
{\bf  getEdgeValues}\\
\begin{lstlisting}[frame=none]
public java.util.Map getEdgeValues()\end{lstlisting} %end signature
\begin{itemize}
\item{
{\bf  Description}

Getter of edgeValues
}
\end{itemize}
}%end item
\item{ 
\index{getGraph()}
{\bf  getGraph}\\
\begin{lstlisting}[frame=none]
public Graph getGraph()\end{lstlisting} %end signature
\begin{itemize}
\item{
{\bf  Description}

Getter of graph
}
\end{itemize}
}%end item
\item{ 
\index{getVertexValues()}
{\bf  getVertexValues}\\
\begin{lstlisting}[frame=none]
public java.util.Map getVertexValues()\end{lstlisting} %end signature
\begin{itemize}
\item{
{\bf  Description}

Getter of vertexValues
}
\end{itemize}
}%end item
\item{ 
\index{getVerticesByAttribute(String, String)}
{\bf  getVerticesByAttribute}\\
\begin{lstlisting}[frame=none]
public java.util.List getVerticesByAttribute(java.lang.String name,java.lang.String value)\end{lstlisting} %end signature
\begin{itemize}
\item{
{\bf  Parameters}
  \begin{itemize}
   \item{
\texttt{name} -- }
   \item{
\texttt{value} -- }
  \end{itemize}
}%end item
\item{{\bf  Returns} -- 
 
}%end item
\end{itemize}
}%end item
\item{ 
\index{setEdgeValues(Map)}
{\bf  setEdgeValues}\\
\begin{lstlisting}[frame=none]
public void setEdgeValues(java.util.Map edgeValues)\end{lstlisting} %end signature
\begin{itemize}
\item{
{\bf  Description}

Setter of edgeValues
}
\end{itemize}
}%end item
\item{ 
\index{setGraph(Graph)}
{\bf  setGraph}\\
\begin{lstlisting}[frame=none]
public void setGraph(Graph graph)\end{lstlisting} %end signature
\begin{itemize}
\item{
{\bf  Description}

Setter of graph
}
\end{itemize}
}%end item
\item{ 
\index{setVertexValues(Map)}
{\bf  setVertexValues}\\
\begin{lstlisting}[frame=none]
public void setVertexValues(java.util.Map vertexValues)\end{lstlisting} %end signature
\begin{itemize}
\item{
{\bf  Description}

Setter of vertexValues
}
\end{itemize}
}%end item
\item{ 
\index{update()}
{\bf  update}\\
\begin{lstlisting}[frame=none]
public void update()\end{lstlisting} %end signature
}%end item
\end{itemize}
}
}
\subsection{\label{graphmodel.Graph.E}\index{Graph.E}Class Graph.E}{
\vskip .1in 
\subsubsection{Declaration}{
\begin{lstlisting}[frame=none]
public static class Graph.E
 extends java.lang.Object\end{lstlisting}
\subsubsection{Constructor summary}{
\begin{verse}
{\bf E()} \\
\end{verse}
}
\subsubsection{Constructors}{
\vskip -2em
\begin{itemize}
\item{ 
\index{E()}
{\bf  E}\\
\begin{lstlisting}[frame=none]
public E()\end{lstlisting} %end signature
}%end item
\end{itemize}
}
}
\subsection{\label{graphmodel.Graph.V}\index{Graph.V}Class Graph.V}{
\vskip .1in 
\subsubsection{Declaration}{
\begin{lstlisting}[frame=none]
public static class Graph.V
 extends java.lang.Object\end{lstlisting}
\subsubsection{Constructor summary}{
\begin{verse}
{\bf V()} \\
\end{verse}
}
\subsubsection{Constructors}{
\vskip -2em
\begin{itemize}
\item{ 
\index{V()}
{\bf  V}\\
\begin{lstlisting}[frame=none]
public V()\end{lstlisting} %end signature
}%end item
\end{itemize}
}
}
\subsection{\label{graphmodel.GraphModel}\index{GraphModel}Class GraphModel}{
\vskip .1in 
A \texttt{\small GraphModel}{\small 
\refdefined{graphmodel.GraphModel}} contains one or more graphs. It is used to save nested or hierachical graphs in one class\vskip .1in 
\subsubsection{Declaration}{
\begin{lstlisting}[frame=none]
public abstract class GraphModel
 extends java.lang.Object\end{lstlisting}
\subsubsection{Constructor summary}{
\begin{verse}
{\bf GraphModel()} \\
\end{verse}
}
\subsubsection{Method summary}{
\begin{verse}
{\bf getGraphs()} Getter of graph\\
\end{verse}
}
\subsubsection{Constructors}{
\vskip -2em
\begin{itemize}
\item{ 
\index{GraphModel()}
{\bf  GraphModel}\\
\begin{lstlisting}[frame=none]
public GraphModel()\end{lstlisting} %end signature
}%end item
\end{itemize}
}
\subsubsection{Methods}{
\vskip -2em
\begin{itemize}
\item{ 
\index{getGraphs()}
{\bf  getGraphs}\\
\begin{lstlisting}[frame=none]
public abstract java.util.List getGraphs()\end{lstlisting} %end signature
\begin{itemize}
\item{
{\bf  Description}

Getter of graph
}
\item{{\bf  Returns} -- 
 
}%end item
\end{itemize}
}%end item
\end{itemize}
}
}
}
\printindex
