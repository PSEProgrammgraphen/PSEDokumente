\section{Package graphmodel}{
\label{graphmodel}\hskip -.05in
\hbox to \hsize{\textit{ Package Contents\hfil Page}}
\vskip .13in
\hbox{{\bf  Interfaces}}
\entityintro{CompoundVertex}{graphmodel.CompoundVertex}{}
\entityintro{DirectedGraph}{graphmodel.DirectedGraph}{A \texttt{\small DirectedGraph}{\small 
\refdefined{graphmodel.DirectedGraph}} is a specific Graph which contains just \texttt{\small DirectedEdge}{\small 
\refdefined{graphmodel.DirectedEdge}} as edges}
\entityintro{Edge}{graphmodel.Edge}{This edge interface specifies an edge.}
\entityintro{Graph}{graphmodel.Graph}{This graph interface specifies a graph.}
\entityintro{IEdgeBuilder}{graphmodel.IEdgeBuilder}{This is an abstract Interface which is used to build a concrete edge.}
\entityintro{IGraphBuilder}{graphmodel.IGraphBuilder}{This is an abstract interface which is used to build a concrete graph.}
\entityintro{IGraphModelBuilder}{graphmodel.IGraphModelBuilder}{This is an abstract interface, which builds a concrete graphmodel.}
\entityintro{IVertexBuilder}{graphmodel.IVertexBuilder}{This is an abstract VertexBuilder which is used to build a concrete Vertex.}
\entityintro{Vertex}{graphmodel.Vertex}{This vertex interface specifies a vertex.}
\vskip .13in
\hbox{{\bf  Classes}}
\entityintro{DefaultDirectedGraph}{graphmodel.DefaultDirectedGraph}{A \texttt{\small DefaultDirectedGraph}{\small 
\refdefined{graphmodel.DefaultDirectedGraph}} is a specific Graph which contains just \texttt{\small DirectedEdge}{\small 
\refdefined{graphmodel.DirectedEdge}} as edges}
\entityintro{DefaultVertex}{graphmodel.DefaultVertex}{This is an DefaultVertex, which has basic functions and is provided by the mainapplication.}
\entityintro{DirectedEdge}{graphmodel.DirectedEdge}{A \texttt{\small DirectedEdge}{\small 
\refdefined{graphmodel.DirectedEdge}} is an edge that has one source and one target vertex.}
\entityintro{DirectedGraphLayoutOption}{graphmodel.DirectedGraphLayoutOption}{}
\entityintro{DirectedGraphLayoutRegister}{graphmodel.DirectedGraphLayoutRegister}{}
\entityintro{FastGraphAccessor}{graphmodel.FastGraphAccessor}{}
\entityintro{GraphModel}{graphmodel.GraphModel}{A \texttt{\small GraphModel}{\small 
\refdefined{graphmodel.GraphModel}} contains one or more graphs.}
\entityintro{SerializedEdge}{graphmodel.SerializedEdge}{A \texttt{\small DirectedEdge}{\small 
\refdefined{graphmodel.DirectedEdge}} is an edge that has one source and one target vertex.}
\entityintro{SerializedGraph}{graphmodel.SerializedGraph}{A \texttt{\small SerializedGraph}{\small 
\refdefined{graphmodel.SerializedGraph}} is a specific Graph that contains all informations as Strings}
\entityintro{SerializedVertex}{graphmodel.SerializedVertex}{This vertex interface specifies an vertex.}
\vskip .1in
\vskip .1in
\subsection{\label{graphmodel.CompoundVertex}\index{CompoundVertex@\textit{ CompoundVertex}}Interface CompoundVertex}{
\vskip .1in 
\subsubsection{Declaration}{
\begin{lstlisting}[frame=none]
public interface CompoundVertex
 extends Vertex\end{lstlisting}
\subsubsection{Methods}{
\vskip -2em
\begin{itemize}
\item{ 
\index{getGraph()}
{\bf  getGraph}\\
\begin{lstlisting}[frame=none]
Graph getGraph()\end{lstlisting} %end signature
}%end item
\end{itemize}
}
}
\subsection{\label{graphmodel.DirectedGraph}\index{DirectedGraph@\textit{ DirectedGraph}}Interface DirectedGraph}{
\vskip .1in 
A \texttt{\small DirectedGraph}{\small 
\refdefined{graphmodel.DirectedGraph}} is a specific Graph which contains just \texttt{\small DirectedEdge}{\small 
\refdefined{graphmodel.DirectedEdge}} as edges\vskip .1in 
\subsubsection{Declaration}{
\begin{lstlisting}[frame=none]
public interface DirectedGraph
 extends Graph\end{lstlisting}
\subsubsection{All known subinterfaces}{DefaultDirectedGraph\small{\refdefined{graphmodel.DefaultDirectedGraph}}}
\subsubsection{All classes known to implement interface}{DefaultDirectedGraph\small{\refdefined{graphmodel.DefaultDirectedGraph}}}
\subsubsection{Methods}{
\vskip -2em
\begin{itemize}
\item{ 
\index{incomingEdgesOf(V)}
{\bf  incomingEdgesOf}\\
\begin{lstlisting}[frame=none]
java.util.Set incomingEdgesOf(Vertex vertex)\end{lstlisting} %end signature
\begin{itemize}
\item{
{\bf  Description}

Get a set of all incoming edges of a vertex
}
\item{
{\bf  Parameters}
  \begin{itemize}
   \item{
\texttt{vertex} -- }
  \end{itemize}
}%end item
\item{{\bf  Returns} -- 
 
}%end item
\end{itemize}
}%end item
\item{ 
\index{indegreeOf(V)}
{\bf  indegreeOf}\\
\begin{lstlisting}[frame=none]
java.lang.Integer indegreeOf(Vertex vertex)\end{lstlisting} %end signature
\begin{itemize}
\item{
{\bf  Description}

Get the indegree of a vertex
}
\item{
{\bf  Parameters}
  \begin{itemize}
   \item{
\texttt{vertex} -- }
  \end{itemize}
}%end item
\item{{\bf  Returns} -- 
 
}%end item
\end{itemize}
}%end item
\item{ 
\index{outcomingEdgesOf(V)}
{\bf  outcomingEdgesOf}\\
\begin{lstlisting}[frame=none]
java.util.Set outcomingEdgesOf(Vertex vertex)\end{lstlisting} %end signature
\begin{itemize}
\item{
{\bf  Description}

Get a set of all outcoming edges of a vertex
}
\item{
{\bf  Parameters}
  \begin{itemize}
   \item{
\texttt{vertex} -- }
  \end{itemize}
}%end item
\item{{\bf  Returns} -- 
 
}%end item
\end{itemize}
}%end item
\item{ 
\index{outdegreeOf(V)}
{\bf  outdegreeOf}\\
\begin{lstlisting}[frame=none]
java.lang.Integer outdegreeOf(Vertex vertex)\end{lstlisting} %end signature
\begin{itemize}
\item{
{\bf  Description}

Get the outdegree of a vertex
}
\item{
{\bf  Parameters}
  \begin{itemize}
   \item{
\texttt{vertex} -- }
  \end{itemize}
}%end item
\item{{\bf  Returns} -- 
 
}%end item
\end{itemize}
}%end item
\end{itemize}
}
}
\subsection{\label{graphmodel.Edge}\index{Edge@\textit{ Edge}}Interface Edge}{
\vskip .1in 
This edge interface specifies an edge. An edge contains two vertices, an ID, a name and a label.\vskip .1in 
\subsubsection{Declaration}{
\begin{lstlisting}[frame=none]
public interface Edge
\end{lstlisting}
\subsubsection{All known subinterfaces}{SerializedEdge\small{\refdefined{graphmodel.SerializedEdge}}, DirectedEdge\small{\refdefined{graphmodel.DirectedEdge}}}
\subsubsection{All classes known to implement interface}{SerializedEdge\small{\refdefined{graphmodel.SerializedEdge}}, DirectedEdge\small{\refdefined{graphmodel.DirectedEdge}}}
\subsubsection{Methods}{
\vskip -2em
\begin{itemize}
\item{ 
\index{addToFastGraphAccessor(FastGraphAccessor)}
{\bf  addToFastGraphAccessor}\\
\begin{lstlisting}[frame=none]
void addToFastGraphAccessor(FastGraphAccessor fga)\end{lstlisting} %end signature
\begin{itemize}
\item{
{\bf  Description}

Adds Values to FastGraphAccessor.
}
\item{
{\bf  Parameters}
  \begin{itemize}
   \item{
\texttt{fga} -- }
  \end{itemize}
}%end item
\end{itemize}
}%end item
\item{ 
\index{getID()}
{\bf  getID}\\
\begin{lstlisting}[frame=none]
java.lang.Integer getID()\end{lstlisting} %end signature
\begin{itemize}
\item{
{\bf  Description}

Returns the ID of the edge.
}
\item{{\bf  Returns} -- 
id of the edge 
}%end item
\end{itemize}
}%end item
\item{ 
\index{getLabel()}
{\bf  getLabel}\\
\begin{lstlisting}[frame=none]
java.lang.String getLabel()\end{lstlisting} %end signature
\begin{itemize}
\item{
{\bf  Description}

Returns the label of the edge.
}
\item{{\bf  Returns} -- 
label of the edge 
}%end item
\end{itemize}
}%end item
\item{ 
\index{getName()}
{\bf  getName}\\
\begin{lstlisting}[frame=none]
java.lang.String getName()\end{lstlisting} %end signature
\begin{itemize}
\item{
{\bf  Description}

Returns the name of the edge.
}
\item{{\bf  Returns} -- 
name of the edge 
}%end item
\end{itemize}
}%end item
\item{ 
\index{getVertices()}
{\bf  getVertices}\\
\begin{lstlisting}[frame=none]
java.util.List getVertices()\end{lstlisting} %end signature
\begin{itemize}
\item{
{\bf  Description}

Returns the vertices connected with this edge.
}
\item{{\bf  Returns} -- 
the vertices connected with the edge 
}%end item
\end{itemize}
}%end item
\item{ 
\index{serialize()}
{\bf  serialize}\\
\begin{lstlisting}[frame=none]
SerializedEdge serialize()\end{lstlisting} %end signature
\begin{itemize}
\item{{\bf  Returns} -- 
 
}%end item
\end{itemize}
}%end item
\end{itemize}
}
}
\subsection{\label{graphmodel.Graph}\index{Graph@\textit{ Graph}}Interface Graph}{
\vskip .1in 
This graph interface specifies a graph. A graph contains edges and vertices.\vskip .1in 
\subsubsection{Declaration}{
\begin{lstlisting}[frame=none]
public interface Graph
\end{lstlisting}
\subsubsection{All known subinterfaces}{SerializedGraph\small{\refdefined{graphmodel.SerializedGraph}}, DirectedGraph\small{\refdefined{graphmodel.DirectedGraph}}, DefaultDirectedGraph\small{\refdefined{graphmodel.DefaultDirectedGraph}}}
\subsubsection{All classes known to implement interface}{SerializedGraph\small{\refdefined{graphmodel.SerializedGraph}}}
\subsubsection{Methods}{
\vskip -2em
\begin{itemize}
\item{ 
\index{addToFastGraphAccessor(FastGraphAccessor)}
{\bf  addToFastGraphAccessor}\\
\begin{lstlisting}[frame=none]
void addToFastGraphAccessor(FastGraphAccessor fga)\end{lstlisting} %end signature
\begin{itemize}
\item{
{\bf  Description}

adds this graph to a fastGraphAccessor
}
\item{
{\bf  Parameters}
  \begin{itemize}
   \item{
\texttt{vertex} -- }
  \end{itemize}
}%end item
\item{{\bf  Returns} -- 
 
}%end item
\end{itemize}
}%end item
\item{ 
\index{edgesOf(V)}
{\bf  edgesOf}\\
\begin{lstlisting}[frame=none]
java.util.Set edgesOf(Vertex vertex)\end{lstlisting} %end signature
\begin{itemize}
\item{
{\bf  Description}

get a list of all edges of a vertex
}
\item{
{\bf  Parameters}
  \begin{itemize}
   \item{
\texttt{vertex} -- }
  \end{itemize}
}%end item
\item{{\bf  Returns} -- 
 
}%end item
\end{itemize}
}%end item
\item{ 
\index{getEdgeSet()}
{\bf  getEdgeSet}\\
\begin{lstlisting}[frame=none]
java.util.Set getEdgeSet()\end{lstlisting} %end signature
\begin{itemize}
\item{
{\bf  Description}

Get a set of all edges in this graph
}
\item{{\bf  Returns} -- 
 
}%end item
\end{itemize}
}%end item
\item{ 
\index{getID()}
{\bf  getID}\\
\begin{lstlisting}[frame=none]
java.lang.Integer getID()\end{lstlisting} %end signature
\begin{itemize}
\item{
{\bf  Description}

Returns the ID of the graph.
}
\item{{\bf  Returns} -- 
id of the graph 
}%end item
\end{itemize}
}%end item
\item{ 
\index{getName()}
{\bf  getName}\\
\begin{lstlisting}[frame=none]
java.lang.String getName()\end{lstlisting} %end signature
\begin{itemize}
\item{
{\bf  Description}

Returns the name of the Graph.
}
\item{{\bf  Returns} -- 
name of the graph 
}%end item
\end{itemize}
}%end item
\item{ 
\index{getRegisteredLayouts()}
{\bf  getRegisteredLayouts}\\
\begin{lstlisting}[frame=none]
java.util.List getRegisteredLayouts()\end{lstlisting} %end signature
\begin{itemize}
\item{
{\bf  Description}

Returns a list of layouts which have been registered at the corresponding LayoutRegister for this graph type. The graph implementing this interface will be set as target of the LayoutOption.
}
\end{itemize}
}%end item
\item{ 
\index{getVertexSet()}
{\bf  getVertexSet}\\
\begin{lstlisting}[frame=none]
java.util.Set getVertexSet()\end{lstlisting} %end signature
\begin{itemize}
\item{
{\bf  Description}

Get a set of all vertices in this graph
}
\item{{\bf  Returns} -- 
 
}%end item
\end{itemize}
}%end item
\item{ 
\index{serialize()}
{\bf  serialize}\\
\begin{lstlisting}[frame=none]
SerializedGraph serialize()\end{lstlisting} %end signature
}%end item
\end{itemize}
}
}
\subsection{\label{graphmodel.IEdgeBuilder}\index{IEdgeBuilder@\textit{ IEdgeBuilder}}Interface IEdgeBuilder}{
\vskip .1in 
This is an abstract Interface which is used to build a concrete edge.\vskip .1in 
\subsubsection{Declaration}{
\begin{lstlisting}[frame=none]
public interface IEdgeBuilder
\end{lstlisting}
\subsubsection{Methods}{
\vskip -2em
\begin{itemize}
\item{ 
\index{addData(String, String)}
{\bf  addData}\\
\begin{lstlisting}[frame=none]
void addData(java.lang.String keyname,java.lang.String value)\end{lstlisting} %end signature
\begin{itemize}
\item{
{\bf  Description}

Adds additional data to this edge. The specific EdgeBuilder implementation needs to decide how to save the value for given edge type.
}
\item{
{\bf  Parameters}
  \begin{itemize}
   \item{
\texttt{keyname} -- }
   \item{
\texttt{value} -- }
  \end{itemize}
}%end item
\end{itemize}
}%end item
\item{ 
\index{newEdge(String, String)}
{\bf  newEdge}\\
\begin{lstlisting}[frame=none]
void newEdge(java.lang.String source,java.lang.String target)\end{lstlisting} %end signature
\begin{itemize}
\item{
{\bf  Description}

sets source and target vertices of the edge build by this.
}
\item{
{\bf  Parameters}
  \begin{itemize}
   \item{
\texttt{source} -- String represantation of the source vertex}
   \item{
\texttt{target} -- String represantation of the target vertex}
  \end{itemize}
}%end item
\end{itemize}
}%end item
\item{ 
\index{setDirection(String)}
{\bf  setDirection}\\
\begin{lstlisting}[frame=none]
void setDirection(java.lang.String direction)\end{lstlisting} %end signature
\begin{itemize}
\item{
{\bf  Description}

sets the direction of the edge build by this.
}
\item{
{\bf  Parameters}
  \begin{itemize}
   \item{
\texttt{direction} -- String representation of the direction. Can be one of}
  \end{itemize}
}%end item
\end{itemize}
}%end item
\item{ 
\index{setID(String)}
{\bf  setID}\\
\begin{lstlisting}[frame=none]
void setID(java.lang.String id)\end{lstlisting} %end signature
\begin{itemize}
\item{
{\bf  Description}

sets the ID of the edge build by this.
}
\item{
{\bf  Parameters}
  \begin{itemize}
   \item{
\texttt{id} -- value to which the id is set}
  \end{itemize}
}%end item
\end{itemize}
}%end item
\end{itemize}
}
}
\subsection{\label{graphmodel.IGraphBuilder}\index{IGraphBuilder@\textit{ IGraphBuilder}}Interface IGraphBuilder}{
\vskip .1in 
This is an abstract interface which is used to build a concrete graph.\vskip .1in 
\subsubsection{Declaration}{
\begin{lstlisting}[frame=none]
public interface IGraphBuilder
\end{lstlisting}
\subsubsection{Methods}{
\vskip -2em
\begin{itemize}
\item{ 
\index{build()}
{\bf  build}\\
\begin{lstlisting}[frame=none]
Graph build()\end{lstlisting} %end signature
\begin{itemize}
\item{
{\bf  Description}

This method is called, when the buildingprocess of the graph is finished. Then it builds the graph and returns it.
}
\item{{\bf  Returns} -- 
Graph 
}%end item
\end{itemize}
}%end item
\item{ 
\index{getEdgeBuilder()}
{\bf  getEdgeBuilder}\\
\begin{lstlisting}[frame=none]
IEdgeBuilder getEdgeBuilder()\end{lstlisting} %end signature
\begin{itemize}
\item{
{\bf  Description}

Returns the EdgeBuilder which is specified for this graph.
}
\item{{\bf  Returns} -- 
AbstractEdgeBuilder 
}%end item
\end{itemize}
}%end item
\item{ 
\index{getVertexBuilder(String)}
{\bf  getVertexBuilder}\\
\begin{lstlisting}[frame=none]
IVertexBuilder getVertexBuilder(java.lang.String vertexID)\end{lstlisting} %end signature
\begin{itemize}
\item{
{\bf  Description}

Returns the VertexBuilder which is specified for this graph.
}
\item{
{\bf  Parameters}
  \begin{itemize}
   \item{
\texttt{vertexID} -- }
  \end{itemize}
}%end item
\item{{\bf  Returns} -- 
AbstractVertexBuilder 
}%end item
\end{itemize}
}%end item
\end{itemize}
}
}
\subsection{\label{graphmodel.IGraphModelBuilder}\index{IGraphModelBuilder@\textit{ IGraphModelBuilder}}Interface IGraphModelBuilder}{
\vskip .1in 
This is an abstract interface, which builds a concrete graphmodel. This Class is based on the Builder Pattern.\vskip .1in 
\subsubsection{Declaration}{
\begin{lstlisting}[frame=none]
public interface IGraphModelBuilder
\end{lstlisting}
\subsubsection{Methods}{
\vskip -2em
\begin{itemize}
\item{ 
\index{build()}
{\bf  build}\\
\begin{lstlisting}[frame=none]
GraphModel build()\end{lstlisting} %end signature
\begin{itemize}
\item{
{\bf  Description}

This method is called, when the buildingprocess of the graphmodel is finished. It returns the finished graphmodel
}
\item{{\bf  Returns} -- 
GraphModel 
}%end item
\end{itemize}
}%end item
\item{ 
\index{getGraphBuilder(String)}
{\bf  getGraphBuilder}\\
\begin{lstlisting}[frame=none]
IGraphBuilder getGraphBuilder(java.lang.String graphID)\end{lstlisting} %end signature
\begin{itemize}
\item{
{\bf  Description}

Returns a specific AbstractGraphBuilder which belongs to the graphmodel. It should also decide which graph should be builded with the help of the graphID.
}
\item{
{\bf  Parameters}
  \begin{itemize}
   \item{
\texttt{graphID} -- }
  \end{itemize}
}%end item
\item{{\bf  Returns} -- 
 
}%end item
\end{itemize}
}%end item
\end{itemize}
}
}
\subsection{\label{graphmodel.IVertexBuilder}\index{IVertexBuilder@\textit{ IVertexBuilder}}Interface IVertexBuilder}{
\vskip .1in 
This is an abstract VertexBuilder which is used to build a concrete Vertex.\vskip .1in 
\subsubsection{Declaration}{
\begin{lstlisting}[frame=none]
public interface IVertexBuilder
\end{lstlisting}
\subsubsection{Methods}{
\vskip -2em
\begin{itemize}
\item{ 
\index{addData(String, String)}
{\bf  addData}\\
\begin{lstlisting}[frame=none]
void addData(java.lang.String value,java.lang.String keyname)\end{lstlisting} %end signature
\begin{itemize}
\item{
{\bf  Description}

Add Data to this Vertex. The vertexBuilder decides which kind of data it is and where to save in the concrete Vertex.
}
\item{
{\bf  Parameters}
  \begin{itemize}
   \item{
\texttt{value} -- }
   \item{
\texttt{keyname} -- }
  \end{itemize}
}%end item
\end{itemize}
}%end item
\item{ 
\index{build()}
{\bf  build}\\
\begin{lstlisting}[frame=none]
Vertex build()\end{lstlisting} %end signature
\begin{itemize}
\item{
{\bf  Description}

This method builds the concrete Vertex and returns it.
}
\item{{\bf  Returns} -- 
Vertex 
}%end item
\end{itemize}
}%end item
\item{ 
\index{getGraphBuilder(String)}
{\bf  getGraphBuilder}\\
\begin{lstlisting}[frame=none]
IGraphBuilder getGraphBuilder(java.lang.String graphID)\end{lstlisting} %end signature
\begin{itemize}
\item{
{\bf  Description}

This method returns an specific GraphBuilder. This method is used to implement nested Graphs.
}
\item{
{\bf  Parameters}
  \begin{itemize}
   \item{
\texttt{graphID} -- }
  \end{itemize}
}%end item
\item{{\bf  Returns} -- 
 
}%end item
\end{itemize}
}%end item
\end{itemize}
}
}
\subsection{\label{graphmodel.Vertex}\index{Vertex@\textit{ Vertex}}Interface Vertex}{
\vskip .1in 
This vertex interface specifies a vertex. Every vertex contains an ID, a name and a label. The ID of a vertex is unique.\vskip .1in 
\subsubsection{Declaration}{
\begin{lstlisting}[frame=none]
public interface Vertex
\end{lstlisting}
\subsubsection{All known subinterfaces}{SerializedVertex\small{\refdefined{graphmodel.SerializedVertex}}, DefaultVertex\small{\refdefined{graphmodel.DefaultVertex}}, CompoundVertex\small{\refdefined{graphmodel.CompoundVertex}}}
\subsubsection{All classes known to implement interface}{SerializedVertex\small{\refdefined{graphmodel.SerializedVertex}}, DefaultVertex\small{\refdefined{graphmodel.DefaultVertex}}}
\subsubsection{Methods}{
\vskip -2em
\begin{itemize}
\item{ 
\index{addToFastGraphAccessor(FastGraphAccessor)}
{\bf  addToFastGraphAccessor}\\
\begin{lstlisting}[frame=none]
void addToFastGraphAccessor(FastGraphAccessor fga)\end{lstlisting} %end signature
\begin{itemize}
\item{
{\bf  Description}

Adds Values to FastGraphAccessor
}
\end{itemize}
}%end item
\item{ 
\index{getID()}
{\bf  getID}\\
\begin{lstlisting}[frame=none]
java.lang.Integer getID()\end{lstlisting} %end signature
\begin{itemize}
\item{
{\bf  Description}

Returns the ID of the vertex. Every vertex in one graph has a unique ID.
}
\item{{\bf  Returns} -- 
the ID of the vertex. 
}%end item
\end{itemize}
}%end item
\item{ 
\index{getLabel()}
{\bf  getLabel}\\
\begin{lstlisting}[frame=none]
java.lang.String getLabel()\end{lstlisting} %end signature
\begin{itemize}
\item{
{\bf  Description}

Returns the label of the vertex, that will be shown in the GUI. The label can be an empty string.
}
\item{{\bf  Returns} -- 
the label of the vertex 
}%end item
\end{itemize}
}%end item
\item{ 
\index{getName()}
{\bf  getName}\\
\begin{lstlisting}[frame=none]
java.lang.String getName()\end{lstlisting} %end signature
\begin{itemize}
\item{
{\bf  Description}

Returns the name of the vertex. A descriptive name of the vertex. Multiple vertices with equal name in one graph are allowed. Therefore don't use this as identifier, instead use \texttt{\small getID()}.
}
\item{{\bf  Returns} -- 
the name of the vertex. 
}%end item
\end{itemize}
}%end item
\item{ 
\index{getX()}
{\bf  getX}\\
\begin{lstlisting}[frame=none]
int getX()\end{lstlisting} %end signature
\begin{itemize}
\item{
{\bf  Description}

Returns the X-coordinate of the vertex.
}
\item{{\bf  Returns} -- 
the X-coordinate of this vertex. 
}%end item
\end{itemize}
}%end item
\item{ 
\index{getY()}
{\bf  getY}\\
\begin{lstlisting}[frame=none]
int getY()\end{lstlisting} %end signature
\begin{itemize}
\item{
{\bf  Description}

Returns the Y-coordinate of the vertex.
}
\item{{\bf  Returns} -- 
the Y-coordinate of the vertex. 
}%end item
\end{itemize}
}%end item
\item{ 
\index{serialize()}
{\bf  serialize}\\
\begin{lstlisting}[frame=none]
SerializedVertex serialize()\end{lstlisting} %end signature
}%end item
\end{itemize}
}
}
\subsection{\label{graphmodel.DefaultDirectedGraph}\index{DefaultDirectedGraph}Class DefaultDirectedGraph}{
\vskip .1in 
A \texttt{\small DefaultDirectedGraph}{\small 
\refdefined{graphmodel.DefaultDirectedGraph}} is a specific Graph which contains just \texttt{\small DirectedEdge}{\small 
\refdefined{graphmodel.DirectedEdge}} as edges\vskip .1in 
\subsubsection{Declaration}{
\begin{lstlisting}[frame=none]
public class DefaultDirectedGraph
 extends java.lang.Object implements DirectedGraph\end{lstlisting}
\subsubsection{Constructors}{
\vskip -2em
\begin{itemize}
\item{ 
\index{DefaultDirectedGraph()}
{\bf  DefaultDirectedGraph}\\
\begin{lstlisting}[frame=none]
public DefaultDirectedGraph()\end{lstlisting} %end signature
}%end item
\end{itemize}
}
\subsubsection{Methods}{
\vskip -2em
\begin{itemize}
\item{ 
\index{addToFastGraphAccessor(FastGraphAccessor)}
{\bf  addToFastGraphAccessor}\\
\begin{lstlisting}[frame=none]
public void addToFastGraphAccessor(FastGraphAccessor fga)\end{lstlisting} %end signature
}%end item
\item{ 
\index{edgesOf(V)}
{\bf  edgesOf}\\
\begin{lstlisting}[frame=none]
public java.util.Set edgesOf(Vertex vertex)\end{lstlisting} %end signature
}%end item
\item{ 
\index{getEdgeSet()}
{\bf  getEdgeSet}\\
\begin{lstlisting}[frame=none]
public java.util.Set getEdgeSet()\end{lstlisting} %end signature
\begin{itemize}
\item{{\bf  Returns} -- 
 
}%end item
\end{itemize}
}%end item
\item{ 
\index{getID()}
{\bf  getID}\\
\begin{lstlisting}[frame=none]
public java.lang.Integer getID()\end{lstlisting} %end signature
}%end item
\item{ 
\index{getName()}
{\bf  getName}\\
\begin{lstlisting}[frame=none]
public java.lang.String getName()\end{lstlisting} %end signature
}%end item
\item{ 
\index{getRegisteredLayouts()}
{\bf  getRegisteredLayouts}\\
\begin{lstlisting}[frame=none]
public java.util.List getRegisteredLayouts()\end{lstlisting} %end signature
}%end item
\item{ 
\index{getSource(E)}
{\bf  getSource}\\
\begin{lstlisting}[frame=none]
public Vertex getSource(DirectedEdge edge)\end{lstlisting} %end signature
\begin{itemize}
\item{
{\bf  Parameters}
  \begin{itemize}
   \item{
\texttt{edge} -- }
  \end{itemize}
}%end item
\item{{\bf  Returns} -- 
 
}%end item
\end{itemize}
}%end item
\item{ 
\index{getVertexSet()}
{\bf  getVertexSet}\\
\begin{lstlisting}[frame=none]
public java.util.Set getVertexSet()\end{lstlisting} %end signature
\begin{itemize}
\item{{\bf  Returns} -- 
 
}%end item
\end{itemize}
}%end item
\item{ 
\index{incomingEdgesOf(V)}
{\bf  incomingEdgesOf}\\
\begin{lstlisting}[frame=none]
public java.util.Set incomingEdgesOf(Vertex vertex)\end{lstlisting} %end signature
\begin{itemize}
\item{
{\bf  Description}

Get a set of all incoming edges of a vertex
}
\item{
{\bf  Parameters}
  \begin{itemize}
   \item{
\texttt{vertex} -- }
  \end{itemize}
}%end item
\item{{\bf  Returns} -- 
 
}%end item
\end{itemize}
}%end item
\item{ 
\index{indegreeOf(V)}
{\bf  indegreeOf}\\
\begin{lstlisting}[frame=none]
public java.lang.Integer indegreeOf(Vertex vertex)\end{lstlisting} %end signature
\begin{itemize}
\item{
{\bf  Description}

Get the indegree of a vertex
}
\item{
{\bf  Parameters}
  \begin{itemize}
   \item{
\texttt{vertex} -- }
  \end{itemize}
}%end item
\item{{\bf  Returns} -- 
 
}%end item
\end{itemize}
}%end item
\item{ 
\index{outcomingEdgesOf(V)}
{\bf  outcomingEdgesOf}\\
\begin{lstlisting}[frame=none]
public java.util.Set outcomingEdgesOf(Vertex vertex)\end{lstlisting} %end signature
\begin{itemize}
\item{
{\bf  Description}

Get a set of all outcoming edges of a vertex
}
\item{
{\bf  Parameters}
  \begin{itemize}
   \item{
\texttt{vertex} -- }
  \end{itemize}
}%end item
\item{{\bf  Returns} -- 
 
}%end item
\end{itemize}
}%end item
\item{ 
\index{outdegreeOf(V)}
{\bf  outdegreeOf}\\
\begin{lstlisting}[frame=none]
public java.lang.Integer outdegreeOf(Vertex vertex)\end{lstlisting} %end signature
\begin{itemize}
\item{
{\bf  Description}

Get the outdegree of a vertex
}
\item{
{\bf  Parameters}
  \begin{itemize}
   \item{
\texttt{vertex} -- }
  \end{itemize}
}%end item
\item{{\bf  Returns} -- 
 
}%end item
\end{itemize}
}%end item
\item{ 
\index{serialize()}
{\bf  serialize}\\
\begin{lstlisting}[frame=none]
public SerializedGraph serialize()\end{lstlisting} %end signature
}%end item
\end{itemize}
}
}
\subsection{\label{graphmodel.DefaultVertex}\index{DefaultVertex}Class DefaultVertex}{
\vskip .1in 
This is an DefaultVertex, which has basic functions and is provided by the mainapplication.\vskip .1in 
\subsubsection{Declaration}{
\begin{lstlisting}[frame=none]
public class DefaultVertex
 extends java.lang.Object implements Vertex\end{lstlisting}
\subsubsection{Constructors}{
\vskip -2em
\begin{itemize}
\item{ 
\index{DefaultVertex()}
{\bf  DefaultVertex}\\
\begin{lstlisting}[frame=none]
public DefaultVertex()\end{lstlisting} %end signature
}%end item
\end{itemize}
}
\subsubsection{Methods}{
\vskip -2em
\begin{itemize}
\item{ 
\index{addToFastGraphAccessor(FastGraphAccessor)}
{\bf  addToFastGraphAccessor}\\
\begin{lstlisting}[frame=none]
void addToFastGraphAccessor(FastGraphAccessor fga)\end{lstlisting} %end signature
\begin{itemize}
\item{
{\bf  Description copied from Vertex{\small \refdefined{graphmodel.Vertex}} }

Adds Values to FastGraphAccessor
}
\end{itemize}
}%end item
\item{ 
\index{getID()}
{\bf  getID}\\
\begin{lstlisting}[frame=none]
java.lang.Integer getID()\end{lstlisting} %end signature
\begin{itemize}
\item{
{\bf  Description copied from Vertex{\small \refdefined{graphmodel.Vertex}} }

Returns the ID of the vertex. Every vertex in one graph has a unique ID.
}
\item{{\bf  Returns} -- 
the ID of the vertex. 
}%end item
\end{itemize}
}%end item
\item{ 
\index{getLabel()}
{\bf  getLabel}\\
\begin{lstlisting}[frame=none]
java.lang.String getLabel()\end{lstlisting} %end signature
\begin{itemize}
\item{
{\bf  Description copied from Vertex{\small \refdefined{graphmodel.Vertex}} }

Returns the label of the vertex, that will be shown in the GUI. The label can be an empty string.
}
\item{{\bf  Returns} -- 
the label of the vertex 
}%end item
\end{itemize}
}%end item
\item{ 
\index{getName()}
{\bf  getName}\\
\begin{lstlisting}[frame=none]
java.lang.String getName()\end{lstlisting} %end signature
\begin{itemize}
\item{
{\bf  Description copied from Vertex{\small \refdefined{graphmodel.Vertex}} }

Returns the name of the vertex. A descriptive name of the vertex. Multiple vertices with equal name in one graph are allowed. Therefore don't use this as identifier, instead use \texttt{\small getID()}.
}
\item{{\bf  Returns} -- 
the name of the vertex. 
}%end item
\end{itemize}
}%end item
\item{ 
\index{getX()}
{\bf  getX}\\
\begin{lstlisting}[frame=none]
int getX()\end{lstlisting} %end signature
\begin{itemize}
\item{
{\bf  Description copied from Vertex{\small \refdefined{graphmodel.Vertex}} }

Returns the X-coordinate of the vertex.
}
\item{{\bf  Returns} -- 
the X-coordinate of this vertex. 
}%end item
\end{itemize}
}%end item
\item{ 
\index{getY()}
{\bf  getY}\\
\begin{lstlisting}[frame=none]
int getY()\end{lstlisting} %end signature
\begin{itemize}
\item{
{\bf  Description copied from Vertex{\small \refdefined{graphmodel.Vertex}} }

Returns the Y-coordinate of the vertex.
}
\item{{\bf  Returns} -- 
the Y-coordinate of the vertex. 
}%end item
\end{itemize}
}%end item
\item{ 
\index{serialize()}
{\bf  serialize}\\
\begin{lstlisting}[frame=none]
SerializedVertex serialize()\end{lstlisting} %end signature
}%end item
\end{itemize}
}
}
\subsection{\label{graphmodel.DirectedEdge}\index{DirectedEdge}Class DirectedEdge}{
\vskip .1in 
A \texttt{\small DirectedEdge}{\small 
\refdefined{graphmodel.DirectedEdge}} is an edge that has one source and one target vertex. So the direction of the edge is specified.\vskip .1in 
\subsubsection{Declaration}{
\begin{lstlisting}[frame=none]
public class DirectedEdge
 extends java.lang.Object implements Edge\end{lstlisting}
\subsubsection{Constructors}{
\vskip -2em
\begin{itemize}
\item{ 
\index{DirectedEdge()}
{\bf  DirectedEdge}\\
\begin{lstlisting}[frame=none]
public DirectedEdge()\end{lstlisting} %end signature
}%end item
\end{itemize}
}
\subsubsection{Methods}{
\vskip -2em
\begin{itemize}
\item{ 
\index{addToFastGraphAccessor(FastGraphAccessor)}
{\bf  addToFastGraphAccessor}\\
\begin{lstlisting}[frame=none]
void addToFastGraphAccessor(FastGraphAccessor fga)\end{lstlisting} %end signature
\begin{itemize}
\item{
{\bf  Description copied from Edge{\small \refdefined{graphmodel.Edge}} }

Adds Values to FastGraphAccessor.
}
\item{
{\bf  Parameters}
  \begin{itemize}
   \item{
\texttt{fga} -- }
  \end{itemize}
}%end item
\end{itemize}
}%end item
\item{ 
\index{getID()}
{\bf  getID}\\
\begin{lstlisting}[frame=none]
java.lang.Integer getID()\end{lstlisting} %end signature
\begin{itemize}
\item{
{\bf  Description copied from Edge{\small \refdefined{graphmodel.Edge}} }

Returns the ID of the edge.
}
\item{{\bf  Returns} -- 
id of the edge 
}%end item
\end{itemize}
}%end item
\item{ 
\index{getLabel()}
{\bf  getLabel}\\
\begin{lstlisting}[frame=none]
java.lang.String getLabel()\end{lstlisting} %end signature
\begin{itemize}
\item{
{\bf  Description copied from Edge{\small \refdefined{graphmodel.Edge}} }

Returns the label of the edge.
}
\item{{\bf  Returns} -- 
label of the edge 
}%end item
\end{itemize}
}%end item
\item{ 
\index{getName()}
{\bf  getName}\\
\begin{lstlisting}[frame=none]
java.lang.String getName()\end{lstlisting} %end signature
\begin{itemize}
\item{
{\bf  Description copied from Edge{\small \refdefined{graphmodel.Edge}} }

Returns the name of the edge.
}
\item{{\bf  Returns} -- 
name of the edge 
}%end item
\end{itemize}
}%end item
\item{ 
\index{getSource()}
{\bf  getSource}\\
\begin{lstlisting}[frame=none]
public Vertex getSource()\end{lstlisting} %end signature
\begin{itemize}
\item{
{\bf  Description}

Returns the source vertex of this directed edge.
}
\item{{\bf  Returns} -- 
the source vertex of this directed edge 
}%end item
\end{itemize}
}%end item
\item{ 
\index{getTarget()}
{\bf  getTarget}\\
\begin{lstlisting}[frame=none]
public Vertex getTarget()\end{lstlisting} %end signature
\begin{itemize}
\item{
{\bf  Description}

Returns the target vertex of this edge.
}
\item{{\bf  Returns} -- 
the target vertex of this directed edge 
}%end item
\end{itemize}
}%end item
\item{ 
\index{getVertices()}
{\bf  getVertices}\\
\begin{lstlisting}[frame=none]
java.util.List getVertices()\end{lstlisting} %end signature
\begin{itemize}
\item{
{\bf  Description copied from Edge{\small \refdefined{graphmodel.Edge}} }

Returns the vertices connected with this edge.
}
\item{{\bf  Returns} -- 
the vertices connected with the edge 
}%end item
\end{itemize}
}%end item
\item{ 
\index{serialize()}
{\bf  serialize}\\
\begin{lstlisting}[frame=none]
SerializedEdge serialize()\end{lstlisting} %end signature
\begin{itemize}
\item{{\bf  Returns} -- 
 
}%end item
\end{itemize}
}%end item
\end{itemize}
}
}
\subsection{\label{graphmodel.DirectedGraphLayoutOption}\index{DirectedGraphLayoutOption}Class DirectedGraphLayoutOption}{
\vskip .1in 
\subsubsection{Declaration}{
\begin{lstlisting}[frame=none]
public abstract class DirectedGraphLayoutOption
 extends plugin.LayoutOption\end{lstlisting}
\subsubsection{Constructors}{
\vskip -2em
\begin{itemize}
\item{ 
\index{DirectedGraphLayoutOption()}
{\bf  DirectedGraphLayoutOption}\\
\begin{lstlisting}[frame=none]
public DirectedGraphLayoutOption()\end{lstlisting} %end signature
}%end item
\end{itemize}
}
\subsubsection{Methods}{
\vskip -2em
\begin{itemize}
\item{ 
\index{setGraph(DirectedGraph)}
{\bf  setGraph}\\
\begin{lstlisting}[frame=none]
public void setGraph(DirectedGraph graph)\end{lstlisting} %end signature
}%end item
\end{itemize}
}
}
\subsection{\label{graphmodel.DirectedGraphLayoutRegister}\index{DirectedGraphLayoutRegister}Class DirectedGraphLayoutRegister}{
\vskip .1in 
\subsubsection{Declaration}{
\begin{lstlisting}[frame=none]
public class DirectedGraphLayoutRegister
 extends java.lang.Object implements plugin.LayoutRegister\end{lstlisting}
\subsubsection{Constructors}{
\vskip -2em
\begin{itemize}
\item{ 
\index{DirectedGraphLayoutRegister()}
{\bf  DirectedGraphLayoutRegister}\\
\begin{lstlisting}[frame=none]
public DirectedGraphLayoutRegister()\end{lstlisting} %end signature
}%end item
\end{itemize}
}
\subsubsection{Methods}{
\vskip -2em
\begin{itemize}
\item{ 
\index{addLayoutOption(DirectedGraphLayoutOption)}
{\bf  addLayoutOption}\\
\begin{lstlisting}[frame=none]
public void addLayoutOption(DirectedGraphLayoutOption option)\end{lstlisting} %end signature
}%end item
\item{ 
\index{getLayoutOptions()}
{\bf  getLayoutOptions}\\
\begin{lstlisting}[frame=none]
java.util.List getLayoutOptions()\end{lstlisting} %end signature
}%end item
\end{itemize}
}
}
\subsection{\label{graphmodel.FastGraphAccessor}\index{FastGraphAccessor}Class FastGraphAccessor}{
\vskip .1in 
\subsubsection{Declaration}{
\begin{lstlisting}[frame=none]
public class FastGraphAccessor
 extends java.lang.Object\end{lstlisting}
\subsubsection{Constructors}{
\vskip -2em
\begin{itemize}
\item{ 
\index{FastGraphAccessor()}
{\bf  FastGraphAccessor}\\
\begin{lstlisting}[frame=none]
public FastGraphAccessor()\end{lstlisting} %end signature
}%end item
\end{itemize}
}
\subsubsection{Methods}{
\vskip -2em
\begin{itemize}
\item{ 
\index{addEdgeForAttribute(String, Edge, int)}
{\bf  addEdgeForAttribute}\\
\begin{lstlisting}[frame=none]
public void addEdgeForAttribute(java.lang.String name,Edge edge,int value)\end{lstlisting} %end signature
\begin{itemize}
\item{
{\bf  Parameters}
  \begin{itemize}
   \item{
\texttt{name} -- }
   \item{
\texttt{edge} -- }
   \item{
\texttt{value} -- }
  \end{itemize}
}%end item
\end{itemize}
}%end item
\item{ 
\index{addEdgeForAttribute(String, Edge, String)}
{\bf  addEdgeForAttribute}\\
\begin{lstlisting}[frame=none]
public void addEdgeForAttribute(java.lang.String name,Edge edge,java.lang.String value)\end{lstlisting} %end signature
\begin{itemize}
\item{
{\bf  Parameters}
  \begin{itemize}
   \item{
\texttt{name} -- }
   \item{
\texttt{edge} -- }
   \item{
\texttt{value} -- }
  \end{itemize}
}%end item
\end{itemize}
}%end item
\item{ 
\index{addVertexForAttribute(Vertex, String, int)}
{\bf  addVertexForAttribute}\\
\begin{lstlisting}[frame=none]
public void addVertexForAttribute(Vertex vertex,java.lang.String value,int name)\end{lstlisting} %end signature
\begin{itemize}
\item{
{\bf  Parameters}
  \begin{itemize}
   \item{
\texttt{vertex} -- }
   \item{
\texttt{value} -- }
   \item{
\texttt{name} -- }
  \end{itemize}
}%end item
\end{itemize}
}%end item
\item{ 
\index{addVertexForAttribute(Vertex, String, String)}
{\bf  addVertexForAttribute}\\
\begin{lstlisting}[frame=none]
public void addVertexForAttribute(Vertex vertex,java.lang.String value,java.lang.String name)\end{lstlisting} %end signature
\begin{itemize}
\item{
{\bf  Parameters}
  \begin{itemize}
   \item{
\texttt{vertex} -- }
   \item{
\texttt{value} -- }
   \item{
\texttt{name} -- }
  \end{itemize}
}%end item
\end{itemize}
}%end item
\item{ 
\index{getEdgesByAttribute(String, int)}
{\bf  getEdgesByAttribute}\\
\begin{lstlisting}[frame=none]
public java.util.List getEdgesByAttribute(java.lang.String value,int name)\end{lstlisting} %end signature
\begin{itemize}
\item{
{\bf  Parameters}
  \begin{itemize}
   \item{
\texttt{value} -- }
   \item{
\texttt{name} -- }
  \end{itemize}
}%end item
\item{{\bf  Returns} -- 
 
}%end item
\end{itemize}
}%end item
\item{ 
\index{getEdgesByAttribute(String, String)}
{\bf  getEdgesByAttribute}\\
\begin{lstlisting}[frame=none]
public java.util.List getEdgesByAttribute(java.lang.String value,java.lang.String name)\end{lstlisting} %end signature
\begin{itemize}
\item{
{\bf  Parameters}
  \begin{itemize}
   \item{
\texttt{value} -- }
   \item{
\texttt{name} -- }
  \end{itemize}
}%end item
\item{{\bf  Returns} -- 
 
}%end item
\end{itemize}
}%end item
\item{ 
\index{getVerticesByAttribute(String, int)}
{\bf  getVerticesByAttribute}\\
\begin{lstlisting}[frame=none]
public java.util.List getVerticesByAttribute(java.lang.String name,int value)\end{lstlisting} %end signature
\begin{itemize}
\item{
{\bf  Parameters}
  \begin{itemize}
   \item{
\texttt{name} -- }
   \item{
\texttt{value} -- }
  \end{itemize}
}%end item
\item{{\bf  Returns} -- 
 
}%end item
\end{itemize}
}%end item
\item{ 
\index{getVerticesByAttribute(String, String)}
{\bf  getVerticesByAttribute}\\
\begin{lstlisting}[frame=none]
public java.util.List getVerticesByAttribute(java.lang.String name,java.lang.String value)\end{lstlisting} %end signature
\begin{itemize}
\item{
{\bf  Parameters}
  \begin{itemize}
   \item{
\texttt{name} -- }
   \item{
\texttt{value} -- }
  \end{itemize}
}%end item
\item{{\bf  Returns} -- 
 
}%end item
\end{itemize}
}%end item
\item{ 
\index{reset()}
{\bf  reset}\\
\begin{lstlisting}[frame=none]
public void reset()\end{lstlisting} %end signature
}%end item
\end{itemize}
}
}
\subsection{\label{graphmodel.GraphModel}\index{GraphModel}Class GraphModel}{
\vskip .1in 
A \texttt{\small GraphModel}{\small 
\refdefined{graphmodel.GraphModel}} contains one or more graphs. It is used to save nested or hierachical graphs in one class\vskip .1in 
\subsubsection{Declaration}{
\begin{lstlisting}[frame=none]
public abstract class GraphModel
 extends java.lang.Object\end{lstlisting}
\subsubsection{Constructors}{
\vskip -2em
\begin{itemize}
\item{ 
\index{GraphModel()}
{\bf  GraphModel}\\
\begin{lstlisting}[frame=none]
public GraphModel()\end{lstlisting} %end signature
}%end item
\end{itemize}
}
\subsubsection{Methods}{
\vskip -2em
\begin{itemize}
\item{ 
\index{getGraphs()}
{\bf  getGraphs}\\
\begin{lstlisting}[frame=none]
public abstract java.util.List getGraphs()\end{lstlisting} %end signature
\begin{itemize}
\item{
{\bf  Description}

Getter of graph
}
\item{{\bf  Returns} -- 
 
}%end item
\end{itemize}
}%end item
\end{itemize}
}
}
\subsection{\label{graphmodel.SerializedEdge}\index{SerializedEdge}Class SerializedEdge}{
\vskip .1in 
A \texttt{\small DirectedEdge}{\small 
\refdefined{graphmodel.DirectedEdge}} is an edge that has one source and one target vertex. So the direction of the edge is specified.\vskip .1in 
\subsubsection{Declaration}{
\begin{lstlisting}[frame=none]
public class SerializedEdge
 extends java.lang.Object implements Edge\end{lstlisting}
\subsubsection{Constructors}{
\vskip -2em
\begin{itemize}
\item{ 
\index{SerializedEdge()}
{\bf  SerializedEdge}\\
\begin{lstlisting}[frame=none]
public SerializedEdge()\end{lstlisting} %end signature
}%end item
\end{itemize}
}
\subsubsection{Methods}{
\vskip -2em
\begin{itemize}
\item{ 
\index{addToFastGraphAccessor(FastGraphAccessor)}
{\bf  addToFastGraphAccessor}\\
\begin{lstlisting}[frame=none]
void addToFastGraphAccessor(FastGraphAccessor fga)\end{lstlisting} %end signature
\begin{itemize}
\item{
{\bf  Description copied from Edge{\small \refdefined{graphmodel.Edge}} }

Adds Values to FastGraphAccessor.
}
\item{
{\bf  Parameters}
  \begin{itemize}
   \item{
\texttt{fga} -- }
  \end{itemize}
}%end item
\end{itemize}
}%end item
\item{ 
\index{getAttributes()}
{\bf  getAttributes}\\
\begin{lstlisting}[frame=none]
public java.util.Map getAttributes()\end{lstlisting} %end signature
}%end item
\item{ 
\index{getID()}
{\bf  getID}\\
\begin{lstlisting}[frame=none]
java.lang.Integer getID()\end{lstlisting} %end signature
\begin{itemize}
\item{
{\bf  Description copied from Edge{\small \refdefined{graphmodel.Edge}} }

Returns the ID of the edge.
}
\item{{\bf  Returns} -- 
id of the edge 
}%end item
\end{itemize}
}%end item
\item{ 
\index{getLabel()}
{\bf  getLabel}\\
\begin{lstlisting}[frame=none]
java.lang.String getLabel()\end{lstlisting} %end signature
\begin{itemize}
\item{
{\bf  Description copied from Edge{\small \refdefined{graphmodel.Edge}} }

Returns the label of the edge.
}
\item{{\bf  Returns} -- 
label of the edge 
}%end item
\end{itemize}
}%end item
\item{ 
\index{getName()}
{\bf  getName}\\
\begin{lstlisting}[frame=none]
java.lang.String getName()\end{lstlisting} %end signature
\begin{itemize}
\item{
{\bf  Description copied from Edge{\small \refdefined{graphmodel.Edge}} }

Returns the name of the edge.
}
\item{{\bf  Returns} -- 
name of the edge 
}%end item
\end{itemize}
}%end item
\item{ 
\index{getVertices()}
{\bf  getVertices}\\
\begin{lstlisting}[frame=none]
java.util.List getVertices()\end{lstlisting} %end signature
\begin{itemize}
\item{
{\bf  Description copied from Edge{\small \refdefined{graphmodel.Edge}} }

Returns the vertices connected with this edge.
}
\item{{\bf  Returns} -- 
the vertices connected with the edge 
}%end item
\end{itemize}
}%end item
\item{ 
\index{serialize()}
{\bf  serialize}\\
\begin{lstlisting}[frame=none]
SerializedEdge serialize()\end{lstlisting} %end signature
\begin{itemize}
\item{{\bf  Returns} -- 
 
}%end item
\end{itemize}
}%end item
\item{ 
\index{setAttribute(String, String)}
{\bf  setAttribute}\\
\begin{lstlisting}[frame=none]
public void setAttribute(java.lang.String name,java.lang.String value)\end{lstlisting} %end signature
}%end item
\end{itemize}
}
}
\subsection{\label{graphmodel.SerializedGraph}\index{SerializedGraph}Class SerializedGraph}{
\vskip .1in 
A \texttt{\small SerializedGraph}{\small 
\refdefined{graphmodel.SerializedGraph}} is a specific Graph that contains all informations as Strings\vskip .1in 
\subsubsection{Declaration}{
\begin{lstlisting}[frame=none]
public class SerializedGraph
 extends java.lang.Object implements Graph\end{lstlisting}
\subsubsection{Constructors}{
\vskip -2em
\begin{itemize}
\item{ 
\index{SerializedGraph()}
{\bf  SerializedGraph}\\
\begin{lstlisting}[frame=none]
public SerializedGraph()\end{lstlisting} %end signature
}%end item
\end{itemize}
}
\subsubsection{Methods}{
\vskip -2em
\begin{itemize}
\item{ 
\index{addEdge()}
{\bf  addEdge}\\
\begin{lstlisting}[frame=none]
public void addEdge()\end{lstlisting} %end signature
\begin{itemize}
\item{
{\bf  Description}

Adds a new Edge to the graph
}
\end{itemize}
}%end item
\item{ 
\index{addToFastGraphAccessor(FastGraphAccessor)}
{\bf  addToFastGraphAccessor}\\
\begin{lstlisting}[frame=none]
void addToFastGraphAccessor(FastGraphAccessor fga)\end{lstlisting} %end signature
\begin{itemize}
\item{
{\bf  Description copied from Graph{\small \refdefined{graphmodel.Graph}} }

adds this graph to a fastGraphAccessor
}
\item{
{\bf  Parameters}
  \begin{itemize}
   \item{
\texttt{vertex} -- }
  \end{itemize}
}%end item
\item{{\bf  Returns} -- 
 
}%end item
\end{itemize}
}%end item
\item{ 
\index{edgesOf(V)}
{\bf  edgesOf}\\
\begin{lstlisting}[frame=none]
java.util.Set edgesOf(Vertex vertex)\end{lstlisting} %end signature
\begin{itemize}
\item{
{\bf  Description copied from Graph{\small \refdefined{graphmodel.Graph}} }

get a list of all edges of a vertex
}
\item{
{\bf  Parameters}
  \begin{itemize}
   \item{
\texttt{vertex} -- }
  \end{itemize}
}%end item
\item{{\bf  Returns} -- 
 
}%end item
\end{itemize}
}%end item
\item{ 
\index{getAttributes()}
{\bf  getAttributes}\\
\begin{lstlisting}[frame=none]
public java.util.Map getAttributes()\end{lstlisting} %end signature
}%end item
\item{ 
\index{getEdgeSet()}
{\bf  getEdgeSet}\\
\begin{lstlisting}[frame=none]
java.util.Set getEdgeSet()\end{lstlisting} %end signature
\begin{itemize}
\item{
{\bf  Description copied from Graph{\small \refdefined{graphmodel.Graph}} }

Get a set of all edges in this graph
}
\item{{\bf  Returns} -- 
 
}%end item
\end{itemize}
}%end item
\item{ 
\index{getID()}
{\bf  getID}\\
\begin{lstlisting}[frame=none]
java.lang.Integer getID()\end{lstlisting} %end signature
\begin{itemize}
\item{
{\bf  Description copied from Graph{\small \refdefined{graphmodel.Graph}} }

Returns the ID of the graph.
}
\item{{\bf  Returns} -- 
id of the graph 
}%end item
\end{itemize}
}%end item
\item{ 
\index{getName()}
{\bf  getName}\\
\begin{lstlisting}[frame=none]
java.lang.String getName()\end{lstlisting} %end signature
\begin{itemize}
\item{
{\bf  Description copied from Graph{\small \refdefined{graphmodel.Graph}} }

Returns the name of the Graph.
}
\item{{\bf  Returns} -- 
name of the graph 
}%end item
\end{itemize}
}%end item
\item{ 
\index{getRegisteredLayouts()}
{\bf  getRegisteredLayouts}\\
\begin{lstlisting}[frame=none]
java.util.List getRegisteredLayouts()\end{lstlisting} %end signature
\begin{itemize}
\item{
{\bf  Description copied from Graph{\small \refdefined{graphmodel.Graph}} }

Returns a list of layouts which have been registered at the corresponding LayoutRegister for this graph type. The graph implementing this interface will be set as target of the LayoutOption.
}
\end{itemize}
}%end item
\item{ 
\index{getSource(E)}
{\bf  getSource}\\
\begin{lstlisting}[frame=none]
public SerializedVertex getSource(SerializedEdge edge)\end{lstlisting} %end signature
\begin{itemize}
\item{
{\bf  Parameters}
  \begin{itemize}
   \item{
\texttt{edge} -- }
  \end{itemize}
}%end item
\item{{\bf  Returns} -- 
 
}%end item
\end{itemize}
}%end item
\item{ 
\index{getVertexSet()}
{\bf  getVertexSet}\\
\begin{lstlisting}[frame=none]
java.util.Set getVertexSet()\end{lstlisting} %end signature
\begin{itemize}
\item{
{\bf  Description copied from Graph{\small \refdefined{graphmodel.Graph}} }

Get a set of all vertices in this graph
}
\item{{\bf  Returns} -- 
 
}%end item
\end{itemize}
}%end item
\item{ 
\index{serialize()}
{\bf  serialize}\\
\begin{lstlisting}[frame=none]
SerializedGraph serialize()\end{lstlisting} %end signature
}%end item
\item{ 
\index{setAttribute(String, String)}
{\bf  setAttribute}\\
\begin{lstlisting}[frame=none]
public void setAttribute(java.lang.String name,java.lang.String value)\end{lstlisting} %end signature
}%end item
\end{itemize}
}
}
\subsection{\label{graphmodel.SerializedVertex}\index{SerializedVertex}Class SerializedVertex}{
\vskip .1in 
This vertex interface specifies an vertex. An vertex contains an ID and a name\vskip .1in 
\subsubsection{Declaration}{
\begin{lstlisting}[frame=none]
public class SerializedVertex
 extends java.lang.Object implements Vertex\end{lstlisting}
\subsubsection{Constructors}{
\vskip -2em
\begin{itemize}
\item{ 
\index{SerializedVertex()}
{\bf  SerializedVertex}\\
\begin{lstlisting}[frame=none]
public SerializedVertex()\end{lstlisting} %end signature
}%end item
\end{itemize}
}
\subsubsection{Methods}{
\vskip -2em
\begin{itemize}
\item{ 
\index{addToFastGraphAccessor(FastGraphAccessor)}
{\bf  addToFastGraphAccessor}\\
\begin{lstlisting}[frame=none]
public void addToFastGraphAccessor(FastGraphAccessor fga)\end{lstlisting} %end signature
\begin{itemize}
\item{
{\bf  Description}

Adds Values to FastGraphAccessor
}
\item{{\bf  Returns} -- 
 
}%end item
\end{itemize}
}%end item
\item{ 
\index{getAttributes()}
{\bf  getAttributes}\\
\begin{lstlisting}[frame=none]
public java.util.Map getAttributes()\end{lstlisting} %end signature
}%end item
\item{ 
\index{getID()}
{\bf  getID}\\
\begin{lstlisting}[frame=none]
public java.lang.Integer getID()\end{lstlisting} %end signature
\begin{itemize}
\item{
{\bf  Description}

Get the ID of the vertex
}
\item{{\bf  Returns} -- 
 
}%end item
\end{itemize}
}%end item
\item{ 
\index{getLabel()}
{\bf  getLabel}\\
\begin{lstlisting}[frame=none]
java.lang.String getLabel()\end{lstlisting} %end signature
\begin{itemize}
\item{
{\bf  Description copied from Vertex{\small \refdefined{graphmodel.Vertex}} }

Returns the label of the vertex, that will be shown in the GUI. The label can be an empty string.
}
\item{{\bf  Returns} -- 
the label of the vertex 
}%end item
\end{itemize}
}%end item
\item{ 
\index{getName()}
{\bf  getName}\\
\begin{lstlisting}[frame=none]
public java.lang.String getName()\end{lstlisting} %end signature
\begin{itemize}
\item{
{\bf  Description}

Get the name of the vertex
}
\item{{\bf  Returns} -- 
 
}%end item
\end{itemize}
}%end item
\item{ 
\index{getX()}
{\bf  getX}\\
\begin{lstlisting}[frame=none]
int getX()\end{lstlisting} %end signature
\begin{itemize}
\item{
{\bf  Description copied from Vertex{\small \refdefined{graphmodel.Vertex}} }

Returns the X-coordinate of the vertex.
}
\item{{\bf  Returns} -- 
the X-coordinate of this vertex. 
}%end item
\end{itemize}
}%end item
\item{ 
\index{getY()}
{\bf  getY}\\
\begin{lstlisting}[frame=none]
int getY()\end{lstlisting} %end signature
\begin{itemize}
\item{
{\bf  Description copied from Vertex{\small \refdefined{graphmodel.Vertex}} }

Returns the Y-coordinate of the vertex.
}
\item{{\bf  Returns} -- 
the Y-coordinate of the vertex. 
}%end item
\end{itemize}
}%end item
\item{ 
\index{serialize()}
{\bf  serialize}\\
\begin{lstlisting}[frame=none]
SerializedVertex serialize()\end{lstlisting} %end signature
}%end item
\item{ 
\index{setAttribute(String, String)}
{\bf  setAttribute}\\
\begin{lstlisting}[frame=none]
public void setAttribute(java.lang.String name,java.lang.String value)\end{lstlisting} %end signature
}%end item
\end{itemize}
}
}
}
\printindex
