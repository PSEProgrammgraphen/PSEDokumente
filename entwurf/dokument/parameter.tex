\section{Package parameter}{
\label{parameter}\hskip -.05in
\hbox to \hsize{\textit{ Package Contents\hfil Page}}
\vskip .13in
\hbox{{\bf  Classes}}
\entityintro{BooleanParameter}{parameter.BooleanParameter}{BooleanParameter are parameters with an boolean value space.}
\entityintro{DoubleParameter}{parameter.DoubleParameter}{DoubleParameters are parameters with an double value space.}
\entityintro{IntegerParameter}{parameter.IntegerParameter}{IntegerParameters are parameters with an integer value space.}
\entityintro{MultipleChoiceParameter}{parameter.MultipleChoiceParameter}{MultipleChoiceParameter are parameters with an predefined String value space.}
\entityintro{Parameter}{parameter.Parameter}{An abstract parameter class.}
\entityintro{ParameterVisitor}{parameter.ParameterVisitor}{An abstract visitor class, as described in the Visitor-Pattern.}
\entityintro{Settings}{parameter.Settings}{A compound object to store parameters.}
\entityintro{StringParameter}{parameter.StringParameter}{StringParameter are parameters with an freely set String value space.}
\vskip .1in
\vskip .1in
\subsection{\label{parameter.BooleanParameter}\index{BooleanParameter}Class BooleanParameter}{
\vskip .1in 
BooleanParameter are parameters with an boolean value space.\vskip .1in 
\subsubsection{Declaration}{
\begin{lstlisting}[frame=none]
public class BooleanParameter
 extends parameter.Parameter\end{lstlisting}
\subsubsection{Constructors}{
\vskip -2em
\begin{itemize}
\item{ 
\index{BooleanParameter(String, boolean)}
{\bf  BooleanParameter}\\
\begin{lstlisting}[frame=none]
public BooleanParameter(java.lang.String name,boolean value)\end{lstlisting} %end signature
\begin{itemize}
\item{
{\bf  Description}

Constructs a new BooleanParameter, sets its name and its default value.
}
\item{
{\bf  Parameters}
  \begin{itemize}
   \item{
\texttt{name} -- The name of the parameter.}
   \item{
\texttt{value} -- The value of the parameter.}
  \end{itemize}
}%end item
\end{itemize}
}%end item
\end{itemize}
}
\subsubsection{Methods}{
\vskip -2em
\begin{itemize}
\item{ 
\index{accept(ParameterVisitor)}
{\bf  accept}\\
\begin{lstlisting}[frame=none]
public abstract void accept(ParameterVisitor visitor)\end{lstlisting} %end signature
\begin{itemize}
\item{
{\bf  Description copied from Parameter{\small \refdefined{parameter.Parameter}} }

Let the visitor visit this parameter.
}
\item{
{\bf  Parameters}
  \begin{itemize}
   \item{
\texttt{visitor} -- The visitor to visit}
  \end{itemize}
}%end item
\end{itemize}
}%end item
\item{ 
\index{compareTo(BooleanParameter)}
{\bf  compareTo}\\
\begin{lstlisting}[frame=none]
public int compareTo(BooleanParameter o)\end{lstlisting} %end signature
}%end item
\end{itemize}
}
}
\subsection{\label{parameter.DoubleParameter}\index{DoubleParameter}Class DoubleParameter}{
\vskip .1in 
DoubleParameters are parameters with an double value space.\vskip .1in 
\subsubsection{Declaration}{
\begin{lstlisting}[frame=none]
public class DoubleParameter
 extends parameter.Parameter\end{lstlisting}
\subsubsection{Constructors}{
\vskip -2em
\begin{itemize}
\item{ 
\index{DoubleParameter(String, Double, Double, Double)}
{\bf  DoubleParameter}\\
\begin{lstlisting}[frame=none]
public DoubleParameter(java.lang.String name,java.lang.Double value,java.lang.Double min,java.lang.Double max)\end{lstlisting} %end signature
\begin{itemize}
\item{
{\bf  Description}

Constructs a new DoubleParameter, sets its name, its default value and boundaries.
}
\item{
{\bf  Parameters}
  \begin{itemize}
   \item{
\texttt{name} -- The name of the parameter.}
   \item{
\texttt{value} -- The value of the parameter.}
   \item{
\texttt{min} -- The minimum boundary of the parameter.}
   \item{
\texttt{max} -- The maximum boundary of the parameter.}
  \end{itemize}
}%end item
\end{itemize}
}%end item
\end{itemize}
}
\subsubsection{Methods}{
\vskip -2em
\begin{itemize}
\item{ 
\index{accept(ParameterVisitor)}
{\bf  accept}\\
\begin{lstlisting}[frame=none]
public abstract void accept(ParameterVisitor visitor)\end{lstlisting} %end signature
\begin{itemize}
\item{
{\bf  Description copied from Parameter{\small \refdefined{parameter.Parameter}} }

Let the visitor visit this parameter.
}
\item{
{\bf  Parameters}
  \begin{itemize}
   \item{
\texttt{visitor} -- The visitor to visit}
  \end{itemize}
}%end item
\end{itemize}
}%end item
\item{ 
\index{compareTo(DoubleParameter)}
{\bf  compareTo}\\
\begin{lstlisting}[frame=none]
public int compareTo(DoubleParameter o)\end{lstlisting} %end signature
}%end item
\item{ 
\index{getMax()}
{\bf  getMax}\\
\begin{lstlisting}[frame=none]
public double getMax()\end{lstlisting} %end signature
\begin{itemize}
\item{
{\bf  Description}

Returns the maximum boundary.
}
\item{{\bf  Returns} -- 
The maximum boundary. 
}%end item
\end{itemize}
}%end item
\item{ 
\index{getMin()}
{\bf  getMin}\\
\begin{lstlisting}[frame=none]
public double getMin()\end{lstlisting} %end signature
\begin{itemize}
\item{
{\bf  Description}

Returns the minimum boundary.
}
\item{{\bf  Returns} -- 
The minimum boundary. 
}%end item
\end{itemize}
}%end item
\item{ 
\index{setMax(double)}
{\bf  setMax}\\
\begin{lstlisting}[frame=none]
public void setMax(double max)\end{lstlisting} %end signature
\begin{itemize}
\item{
{\bf  Description}

Sets the maximum boundary.
}
\item{
{\bf  Parameters}
  \begin{itemize}
   \item{
\texttt{min} -- The maximum boundary.}
  \end{itemize}
}%end item
\end{itemize}
}%end item
\item{ 
\index{setMin(double)}
{\bf  setMin}\\
\begin{lstlisting}[frame=none]
public void setMin(double min)\end{lstlisting} %end signature
\begin{itemize}
\item{
{\bf  Description}

Sets the minimum boundary.
}
\item{
{\bf  Parameters}
  \begin{itemize}
   \item{
\texttt{min} -- The minimum boundary.}
  \end{itemize}
}%end item
\end{itemize}
}%end item
\end{itemize}
}
}
\subsection{\label{parameter.IntegerParameter}\index{IntegerParameter}Class IntegerParameter}{
\vskip .1in 
IntegerParameters are parameters with an integer value space.\vskip .1in 
\subsubsection{Declaration}{
\begin{lstlisting}[frame=none]
public class IntegerParameter
 extends parameter.Parameter\end{lstlisting}
\subsubsection{Constructors}{
\vskip -2em
\begin{itemize}
\item{ 
\index{IntegerParameter(String, int, int, int)}
{\bf  IntegerParameter}\\
\begin{lstlisting}[frame=none]
public IntegerParameter(java.lang.String name,int value,int min,int max)\end{lstlisting} %end signature
\begin{itemize}
\item{
{\bf  Description}

Constructs a new IntegerParameter, sets its name, its default value and boundaries.
}
\item{
{\bf  Parameters}
  \begin{itemize}
   \item{
\texttt{name} -- The name of the parameter.}
   \item{
\texttt{value} -- The value of the parameter.}
   \item{
\texttt{min} -- The minimum boundary of the parameter.}
   \item{
\texttt{max} -- The maximum boundary of the parameter.}
  \end{itemize}
}%end item
\end{itemize}
}%end item
\end{itemize}
}
\subsubsection{Methods}{
\vskip -2em
\begin{itemize}
\item{ 
\index{accept(ParameterVisitor)}
{\bf  accept}\\
\begin{lstlisting}[frame=none]
public abstract void accept(ParameterVisitor visitor)\end{lstlisting} %end signature
\begin{itemize}
\item{
{\bf  Description copied from Parameter{\small \refdefined{parameter.Parameter}} }

Let the visitor visit this parameter.
}
\item{
{\bf  Parameters}
  \begin{itemize}
   \item{
\texttt{visitor} -- The visitor to visit}
  \end{itemize}
}%end item
\end{itemize}
}%end item
\item{ 
\index{compareTo(IntegerParameter)}
{\bf  compareTo}\\
\begin{lstlisting}[frame=none]
public int compareTo(IntegerParameter iw)\end{lstlisting} %end signature
}%end item
\item{ 
\index{getMax()}
{\bf  getMax}\\
\begin{lstlisting}[frame=none]
public int getMax()\end{lstlisting} %end signature
\begin{itemize}
\item{
{\bf  Description}

Returns the maximum boundary.
}
\item{{\bf  Returns} -- 
The maximum boundary. 
}%end item
\end{itemize}
}%end item
\item{ 
\index{getMin()}
{\bf  getMin}\\
\begin{lstlisting}[frame=none]
public int getMin()\end{lstlisting} %end signature
\begin{itemize}
\item{
{\bf  Description}

Returns the minimum boundary.
}
\item{{\bf  Returns} -- 
The minimum boundary. 
}%end item
\end{itemize}
}%end item
\item{ 
\index{setMax(int)}
{\bf  setMax}\\
\begin{lstlisting}[frame=none]
public void setMax(int max)\end{lstlisting} %end signature
\begin{itemize}
\item{
{\bf  Description}

Sets the maximum boundary.
}
\item{
{\bf  Parameters}
  \begin{itemize}
   \item{
\texttt{min} -- The maximum boundary.}
  \end{itemize}
}%end item
\end{itemize}
}%end item
\item{ 
\index{setMin(int)}
{\bf  setMin}\\
\begin{lstlisting}[frame=none]
public void setMin(int min)\end{lstlisting} %end signature
\begin{itemize}
\item{
{\bf  Description}

Sets the minimum boundary.
}
\item{
{\bf  Parameters}
  \begin{itemize}
   \item{
\texttt{min} -- The minimum boundary.}
  \end{itemize}
}%end item
\end{itemize}
}%end item
\end{itemize}
}
}
\subsection{\label{parameter.MultipleChoiceParameter}\index{MultipleChoiceParameter}Class MultipleChoiceParameter}{
\vskip .1in 
MultipleChoiceParameter are parameters with an predefined String value space.\vskip .1in 
\subsubsection{Declaration}{
\begin{lstlisting}[frame=none]
public class MultipleChoiceParameter
 extends parameter.Parameter\end{lstlisting}
\subsubsection{Constructors}{
\vskip -2em
\begin{itemize}
\item{ 
\index{MultipleChoiceParameter(String)}
{\bf  MultipleChoiceParameter}\\
\begin{lstlisting}[frame=none]
public MultipleChoiceParameter(java.lang.String name)\end{lstlisting} %end signature
\begin{itemize}
\item{
{\bf  Description}

Constructs a new MultipleChoiceParameter and sets its name. The possible choices and index are set as null.
}
\item{
{\bf  Parameters}
  \begin{itemize}
   \item{
\texttt{name} -- The name of the parameter.}
  \end{itemize}
}%end item
\end{itemize}
}%end item
\item{ 
\index{MultipleChoiceParameter(String, List, int)}
{\bf  MultipleChoiceParameter}\\
\begin{lstlisting}[frame=none]
public MultipleChoiceParameter(java.lang.String name,java.util.List choices,int init)\end{lstlisting} %end signature
\begin{itemize}
\item{
{\bf  Description}

Constructs a new MultipleChoiceParameter, sets its name, its possible choices and initialized index.
}
\item{
{\bf  Parameters}
  \begin{itemize}
   \item{
\texttt{name} -- The name of the parameter.}
   \item{
\texttt{choices} -- The choices of the parameter.}
   \item{
\texttt{init} -- The initialized index of the parameter.}
  \end{itemize}
}%end item
\end{itemize}
}%end item
\end{itemize}
}
\subsubsection{Methods}{
\vskip -2em
\begin{itemize}
\item{ 
\index{accept(ParameterVisitor)}
{\bf  accept}\\
\begin{lstlisting}[frame=none]
public abstract void accept(ParameterVisitor visitor)\end{lstlisting} %end signature
\begin{itemize}
\item{
{\bf  Description copied from Parameter{\small \refdefined{parameter.Parameter}} }

Let the visitor visit this parameter.
}
\item{
{\bf  Parameters}
  \begin{itemize}
   \item{
\texttt{visitor} -- The visitor to visit}
  \end{itemize}
}%end item
\end{itemize}
}%end item
\item{ 
\index{addChoice(String, int)}
{\bf  addChoice}\\
\begin{lstlisting}[frame=none]
public void addChoice(java.lang.String choice,int index)\end{lstlisting} %end signature
\begin{itemize}
\item{
{\bf  Description}

Adds a choice to the MultipleChoiceParameter.
}
\item{
{\bf  Parameters}
  \begin{itemize}
   \item{
\texttt{choice} -- The choice that will be added.}
   \item{
\texttt{index} -- The index in the list the new choice will be added.}
  \end{itemize}
}%end item
\end{itemize}
}%end item
\item{ 
\index{compareTo(MultipleChoiceParameter)}
{\bf  compareTo}\\
\begin{lstlisting}[frame=none]
public int compareTo(MultipleChoiceParameter o)\end{lstlisting} %end signature
}%end item
\item{ 
\index{getChoices()}
{\bf  getChoices}\\
\begin{lstlisting}[frame=none]
public java.util.List getChoices()\end{lstlisting} %end signature
\begin{itemize}
\item{
{\bf  Description}

Returns a list of all set possible choices in the MultipleChoiceParameter.
}
\item{{\bf  Returns} -- 
A list of all set possible choices in the MultipleChoiceParameter. 
}%end item
\end{itemize}
}%end item
\item{ 
\index{getSelectionIndex()}
{\bf  getSelectionIndex}\\
\begin{lstlisting}[frame=none]
public int getSelectionIndex()\end{lstlisting} %end signature
\begin{itemize}
\item{
{\bf  Description}

Returns the index of the currently selected choice.
}
\item{{\bf  Returns} -- 
The index of the currently selected choice. 
}%end item
\end{itemize}
}%end item
\item{ 
\index{remove(int)}
{\bf  remove}\\
\begin{lstlisting}[frame=none]
public void remove(int index)\end{lstlisting} %end signature
\begin{itemize}
\item{
{\bf  Description}

Removes a choice from the MultipleChoiceParameter.
}
\item{
{\bf  Parameters}
  \begin{itemize}
   \item{
\texttt{index} -- The index in the list of the choice to be removed.}
  \end{itemize}
}%end item
\end{itemize}
}%end item
\item{ 
\index{setValue(int)}
{\bf  setValue}\\
\begin{lstlisting}[frame=none]
public void setValue(int selected)\end{lstlisting} %end signature
\begin{itemize}
\item{
{\bf  Description}

Overloads the setValue of GAnsProperty. Sets the String at position selected in values as the value of the Parameter.
}
\item{
{\bf  Parameters}
  \begin{itemize}
   \item{
\texttt{selected} -- The position in values that has been selected and will be set as value.}
  \end{itemize}
}%end item
\end{itemize}
}%end item
\end{itemize}
}
}
\subsection{\label{parameter.Parameter}\index{Parameter}Class Parameter}{
\vskip .1in 
An abstract parameter class. A Parameter contains a value and a name. The value can be transformed into a string. Clients can set Listeners to track changes of the value. Classes inheriting from this class can visited by a ParameterVisitor.\vskip .1in 
\subsubsection{Declaration}{
\begin{lstlisting}[frame=none]
public abstract class Parameter
 extends objectproperty.GAnsProperty implements java.lang.Comparable\end{lstlisting}
\subsubsection{All known subclasses}{StringParameter\small{\refdefined{parameter.StringParameter}}, MultipleChoiceParameter\small{\refdefined{parameter.MultipleChoiceParameter}}, IntegerParameter\small{\refdefined{parameter.IntegerParameter}}, DoubleParameter\small{\refdefined{parameter.DoubleParameter}}, BooleanParameter\small{\refdefined{parameter.BooleanParameter}}}
\subsubsection{Constructors}{
\vskip -2em
\begin{itemize}
\item{ 
\index{Parameter(String, V)}
{\bf  Parameter}\\
\begin{lstlisting}[frame=none]
public Parameter(java.lang.String name,java.lang.Object value)\end{lstlisting} %end signature
\begin{itemize}
\item{
{\bf  Description}

Constructor, setting the name and value of the property.
}
\item{
{\bf  Parameters}
  \begin{itemize}
   \item{
\texttt{name} -- The string will be set as the name of the GAnsProperty.}
   \item{
\texttt{value} -- The value that will be set in the GAnsProperty.}
  \end{itemize}
}%end item
\end{itemize}
}%end item
\end{itemize}
}
\subsubsection{Methods}{
\vskip -2em
\begin{itemize}
\item{ 
\index{accept(ParameterVisitor)}
{\bf  accept}\\
\begin{lstlisting}[frame=none]
public abstract void accept(ParameterVisitor visitor)\end{lstlisting} %end signature
\begin{itemize}
\item{
{\bf  Description}

Let the visitor visit this parameter.
}
\item{
{\bf  Parameters}
  \begin{itemize}
   \item{
\texttt{visitor} -- The visitor to visit}
  \end{itemize}
}%end item
\end{itemize}
}%end item
\item{ 
\index{compareTo(T)}
{\bf  compareTo}\\
\begin{lstlisting}[frame=none]
int compareTo(java.lang.Object arg0)\end{lstlisting} %end signature
}%end item
\end{itemize}
}
}
\subsection{\label{parameter.ParameterVisitor}\index{ParameterVisitor}Class ParameterVisitor}{
\vskip .1in 
An abstract visitor class, as described in the Visitor-Pattern. Thought to give other elements access to a custom \texttt{\small Settings}{\small 
\refdefined{parameter.Settings}} and retrieve all informations of the specialized \texttt{\small Parameter}{\small 
\refdefined{parameter.Parameter}} interfaces.\vskip .1in 
\subsubsection{Declaration}{
\begin{lstlisting}[frame=none]
public abstract class ParameterVisitor
 extends java.lang.Object\end{lstlisting}
\subsubsection{Constructors}{
\vskip -2em
\begin{itemize}
\item{ 
\index{ParameterVisitor()}
{\bf  ParameterVisitor}\\
\begin{lstlisting}[frame=none]
public ParameterVisitor()\end{lstlisting} %end signature
}%end item
\end{itemize}
}
\subsubsection{Methods}{
\vskip -2em
\begin{itemize}
\item{ 
\index{visit(IntegerParameter)}
{\bf  visit}\\
\begin{lstlisting}[frame=none]
public abstract void visit(IntegerParameter parameter)\end{lstlisting} %end signature
\begin{itemize}
\item{
{\bf  Description}

Visits the specified parameter and performs some by the subclass chosen actions on it.
}
\item{
{\bf  Parameters}
  \begin{itemize}
   \item{
\texttt{parameter} -- The parameter to visit}
  \end{itemize}
}%end item
\end{itemize}
}%end item
\item{ 
\index{visit(MultipleChoiceParameter)}
{\bf  visit}\\
\begin{lstlisting}[frame=none]
public abstract void visit(MultipleChoiceParameter parameter)\end{lstlisting} %end signature
\begin{itemize}
\item{
{\bf  Description}

Visits the specified parameter and performs some by the subclass chosen actions on it.
}
\item{
{\bf  Parameters}
  \begin{itemize}
   \item{
\texttt{parameter} -- The parameter to visit}
  \end{itemize}
}%end item
\end{itemize}
}%end item
\item{ 
\index{visit(StringParameter)}
{\bf  visit}\\
\begin{lstlisting}[frame=none]
public abstract void visit(StringParameter parameter)\end{lstlisting} %end signature
\begin{itemize}
\item{
{\bf  Description}

Visits the specified parameter and performs some by the subclass chosen actions on it.
}
\item{
{\bf  Parameters}
  \begin{itemize}
   \item{
\texttt{parameter} -- The parameter to visit}
  \end{itemize}
}%end item
\end{itemize}
}%end item
\end{itemize}
}
}
\subsection{\label{parameter.Settings}\index{Settings}Class Settings}{
\vskip .1in 
A compound object to store parameters.\vskip .1in 
\subsubsection{Declaration}{
\begin{lstlisting}[frame=none]
public class Settings
 extends java.lang.Object\end{lstlisting}
\subsubsection{Constructors}{
\vskip -2em
\begin{itemize}
\item{ 
\index{Settings(Map)}
{\bf  Settings}\\
\begin{lstlisting}[frame=none]
public Settings(java.util.Map parameters)\end{lstlisting} %end signature
\begin{itemize}
\item{
{\bf  Description}

Constructs a new Settings-Object and sets its parameters.
}
\item{
{\bf  Parameters}
  \begin{itemize}
   \item{
\texttt{parameters} -- The parameters the Settings-Object will have.}
  \end{itemize}
}%end item
\end{itemize}
}%end item
\end{itemize}
}
\subsubsection{Methods}{
\vskip -2em
\begin{itemize}
\item{ 
\index{entrySet()}
{\bf  entrySet}\\
\begin{lstlisting}[frame=none]
public java.util.Set entrySet()\end{lstlisting} %end signature
\begin{itemize}
\item{
{\bf  Description}

Returns a Set of all the parameters.
}
\item{{\bf  Returns} -- 
A Set of all the parameters. 
}%end item
\end{itemize}
}%end item
\item{ 
\index{get(String)}
{\bf  get}\\
\begin{lstlisting}[frame=none]
public Parameter get(java.lang.String key)\end{lstlisting} %end signature
\begin{itemize}
\item{
{\bf  Description}

Returns the Parameter associated with the given key.
}
\item{
{\bf  Parameters}
  \begin{itemize}
   \item{
\texttt{key} -- The key which is associated with the Parameter.}
  \end{itemize}
}%end item
\item{{\bf  Returns} -- 
The Parameter associated with the given key. 
}%end item
\end{itemize}
}%end item
\item{ 
\index{size()}
{\bf  size}\\
\begin{lstlisting}[frame=none]
public int size()\end{lstlisting} %end signature
\begin{itemize}
\item{
{\bf  Description}

Returns the amount of parameters in the Settings.
}
\item{{\bf  Returns} -- 
The amount of parameters in the Settings. 
}%end item
\end{itemize}
}%end item
\item{ 
\index{values()}
{\bf  values}\\
\begin{lstlisting}[frame=none]
public java.util.Collection values()\end{lstlisting} %end signature
\begin{itemize}
\item{
{\bf  Description}

Returns all the Parameters in the Settings-Object.
}
\item{{\bf  Returns} -- 
All the Parameters in the Settings-Object. 
}%end item
\end{itemize}
}%end item
\end{itemize}
}
}
\subsection{\label{parameter.StringParameter}\index{StringParameter}Class StringParameter}{
\vskip .1in 
StringParameter are parameters with an freely set String value space.\vskip .1in 
\subsubsection{Declaration}{
\begin{lstlisting}[frame=none]
public class StringParameter
 extends parameter.Parameter\end{lstlisting}
\subsubsection{Constructors}{
\vskip -2em
\begin{itemize}
\item{ 
\index{StringParameter(String, String)}
{\bf  StringParameter}\\
\begin{lstlisting}[frame=none]
public StringParameter(java.lang.String name,java.lang.String value)\end{lstlisting} %end signature
\begin{itemize}
\item{
{\bf  Description}

Constructs a new StringParameter, sets its name and its default value.
}
\item{
{\bf  Parameters}
  \begin{itemize}
   \item{
\texttt{name} -- The name of the parameter.}
   \item{
\texttt{value} -- The value of the parameter.}
  \end{itemize}
}%end item
\end{itemize}
}%end item
\end{itemize}
}
\subsubsection{Methods}{
\vskip -2em
\begin{itemize}
\item{ 
\index{accept(ParameterVisitor)}
{\bf  accept}\\
\begin{lstlisting}[frame=none]
public abstract void accept(ParameterVisitor visitor)\end{lstlisting} %end signature
\begin{itemize}
\item{
{\bf  Description copied from Parameter{\small \refdefined{parameter.Parameter}} }

Let the visitor visit this parameter.
}
\item{
{\bf  Parameters}
  \begin{itemize}
   \item{
\texttt{visitor} -- The visitor to visit}
  \end{itemize}
}%end item
\end{itemize}
}%end item
\item{ 
\index{compareTo(StringParameter)}
{\bf  compareTo}\\
\begin{lstlisting}[frame=none]
public int compareTo(StringParameter o)\end{lstlisting} %end signature
}%end item
\end{itemize}
}
}
}
\printindex
