\section{Package sugiyama}{
\label{sugiyama}\hskip -.05in
\hbox to \hsize{\textit{ Package Contents\hfil Page}}
\vskip .13in
\hbox{{\bf  Classes}}
\entityintro{CrossMinimizer}{sugiyama.CrossMinimizer}{This class takes a sugiyama graph and rearranges its vertices on each layer to minimize the amount of edge crossings.}
\entityintro{DAGMaker}{sugiyama.DAGMaker}{This class takes a directed Graph G = (V, E) and removes a set of edges E\_\ so that the resulting Graph G' = (V, E\textbackslash E\_) is a DAG(Directed Acyclic Graph).}
\entityintro{EdgeDrawing}{sugiyama.EdgeDrawing}{This class takes a directed graph, as a SugiyamaClass.}
\entityintro{LayerAssigning}{sugiyama.LayerAssigning}{This class takes a directed graph and assigns every vertex in it a layer.}
\entityintro{LayerConstraint}{sugiyama.LayerConstraint}{A relative constraint, regarding layer assignment, between to sets of vertices.}
\entityintro{SugiyamaGraph}{sugiyama.SugiyamaGraph}{The SugiyamaGraph is a wrapper for a directed graph to enable easy and fast acessability of attributes and constructs needed during the computation of the hierarchical layout of a directed graph.}
\entityintro{SugiyamaLayoutAlgorithm}{sugiyama.SugiyamaLayoutAlgorithm}{This class controls the collaboration between every single step of Sugiyama Layout.}
\entityintro{VertexPositioning}{sugiyama.VertexPositioning}{This class takes a directed graph and position its vertices in order to look more clearly.}
\vskip .1in
\vskip .1in
\subsection{\label{sugiyama.CrossMinimizer}\index{CrossMinimizer}Class CrossMinimizer}{
\vskip .1in 
This class takes a sugiyama graph and rearranges its vertices on each layer to minimize the amount of edge crossings.\vskip .1in 
\subsubsection{Declaration}{
\begin{lstlisting}[frame=none]
public class CrossMinimizer
 extends java.lang.Object\end{lstlisting}
\subsubsection{Constructors}{
\vskip -2em
\begin{itemize}
\item{ 
\index{CrossMinimizer()}
{\bf  CrossMinimizer}\\
\begin{lstlisting}[frame=none]
public CrossMinimizer()\end{lstlisting} %end signature
}%end item
\end{itemize}
}
\subsubsection{Methods}{
\vskip -2em
\begin{itemize}
\item{ 
\index{minimizeCrossings(SugiyamaGraph)}
{\bf  minimizeCrossings}\\
\begin{lstlisting}[frame=none]
public void minimizeCrossings(SugiyamaGraph graph)\end{lstlisting} %end signature
\begin{itemize}
\item{
{\bf  Description}

Rearranges vertices in the graph argument in order to remove the amount of crosses of their edges.
}
\item{
{\bf  Parameters}
  \begin{itemize}
   \item{
\texttt{graph} -- input graph}
  \end{itemize}
}%end item
\end{itemize}
}%end item
\end{itemize}
}
}
\subsection{\label{sugiyama.DAGMaker}\index{DAGMaker}Class DAGMaker}{
\vskip .1in 
This class takes a directed Graph G = (V, E) and removes a set of edges E\_\ so that the resulting Graph G' = (V, E\textbackslash E\_) is a DAG(Directed Acyclic Graph).\vskip .1in 
\subsubsection{Declaration}{
\begin{lstlisting}[frame=none]
public class DAGMaker
 extends java.lang.Object\end{lstlisting}
\subsubsection{Constructors}{
\vskip -2em
\begin{itemize}
\item{ 
\index{DAGMaker()}
{\bf  DAGMaker}\\
\begin{lstlisting}[frame=none]
public DAGMaker()\end{lstlisting} %end signature
}%end item
\end{itemize}
}
\subsubsection{Methods}{
\vskip -2em
\begin{itemize}
\item{ 
\index{removeCycles(SugiyamaGraph)}
{\bf  removeCycles}\\
\begin{lstlisting}[frame=none]
public java.util.Set removeCycles(SugiyamaGraph graph)\end{lstlisting} %end signature
\begin{itemize}
\item{
{\bf  Description}

Searches for a acyclic subgraph in the graph argument and reversed the direction of the edges that are not part of this subgraph.
}
\item{
{\bf  Parameters}
  \begin{itemize}
   \item{
\texttt{graph} -- the input graph to remove cycles from}
  \end{itemize}
}%end item
\item{{\bf  Returns} -- 
a set of edges whose direction has been reversed in order to remove cycles from the graph 
}%end item
\end{itemize}
}%end item
\end{itemize}
}
}
\subsection{\label{sugiyama.EdgeDrawing}\index{EdgeDrawing}Class EdgeDrawing}{
\vskip .1in 
This class takes a directed graph, as a SugiyamaClass. It removes dummy vertices and reverses previously reversed edges. Afterwards it assigns every edge points it must run through.\vskip .1in 
\subsubsection{Declaration}{
\begin{lstlisting}[frame=none]
public class EdgeDrawing
 extends java.lang.Object\end{lstlisting}
\subsubsection{Constructors}{
\vskip -2em
\begin{itemize}
\item{ 
\index{EdgeDrawing()}
{\bf  EdgeDrawing}\\
\begin{lstlisting}[frame=none]
public EdgeDrawing()\end{lstlisting} %end signature
}%end item
\end{itemize}
}
\subsubsection{Methods}{
\vskip -2em
\begin{itemize}
\item{ 
\index{drawEdges(SugiyamaGraph)}
{\bf  drawEdges}\\
\begin{lstlisting}[frame=none]
public void drawEdges(SugiyamaGraph graph)\end{lstlisting} %end signature
\begin{itemize}
\item{
{\bf  Description}

Draws the edges from the graph argument and reverses the edges, which have been reversed earlier, so they have now the correct direction.
}
\item{
{\bf  Parameters}
  \begin{itemize}
   \item{
\texttt{graph} -- the input graph}
  \end{itemize}
}%end item
\end{itemize}
}%end item
\end{itemize}
}
}
\subsection{\label{sugiyama.LayerAssigning}\index{LayerAssigning}Class LayerAssigning}{
\vskip .1in 
This class takes a directed graph and assigns every vertex in it a layer.\vskip .1in 
\subsubsection{Declaration}{
\begin{lstlisting}[frame=none]
public class LayerAssigning
 extends java.lang.Object\end{lstlisting}
\subsubsection{Constructors}{
\vskip -2em
\begin{itemize}
\item{ 
\index{LayerAssigning()}
{\bf  LayerAssigning}\\
\begin{lstlisting}[frame=none]
public LayerAssigning()\end{lstlisting} %end signature
}%end item
\end{itemize}
}
\subsubsection{Methods}{
\vskip -2em
\begin{itemize}
\item{ 
\index{addConstraints()}
{\bf  addConstraints}\\
\begin{lstlisting}[frame=none]
public void addConstraints(<any> constraints)\end{lstlisting} %end signature
\begin{itemize}
\item{
{\bf  Description}

Assigns every vertex in the graph parameter e relative height.
}
\item{
{\bf  Parameters}
  \begin{itemize}
   \item{
\texttt{graph} -- input graph public void assignLayers(SugiyamaGraph graph) \{\ \}\ /** Defines a set of constraints which should be considered by the algorithm.}
   \item{
\texttt{constraints} -- relative layer constraints the algorithm should consider}
  \end{itemize}
}%end item
\end{itemize}
}%end item
\item{ 
\index{setMaxHeight(int)}
{\bf  setMaxHeight}\\
\begin{lstlisting}[frame=none]
public void setMaxHeight(int height)\end{lstlisting} %end signature
\begin{itemize}
\item{
{\bf  Description}

Reassigns the layer of vertices whose layer is greater than the height parameter.
}
\item{
{\bf  Parameters}
  \begin{itemize}
   \item{
\texttt{height} -- maximum height for vertices}
  \end{itemize}
}%end item
\end{itemize}
}%end item
\item{ 
\index{setMaxWidth(int)}
{\bf  setMaxWidth}\\
\begin{lstlisting}[frame=none]
public void setMaxWidth(int width)\end{lstlisting} %end signature
\begin{itemize}
\item{
{\bf  Description}

Reassigns the layer of vertices in case there are more than the width parameter in one layer.
}
\item{
{\bf  Parameters}
  \begin{itemize}
   \item{
\texttt{width} -- maximum amount of vertices in one layer}
  \end{itemize}
}%end item
\end{itemize}
}%end item
\end{itemize}
}
}
\subsection{\label{sugiyama.LayerConstraint}\index{LayerConstraint}Class LayerConstraint}{
\vskip .1in 
A relative constraint, regarding layer assignment, between to sets of vertices. Can describe if one set of vertices should be on top of the other. When the exact is set a layer distance can be set.\vskip .1in 
\subsubsection{Declaration}{
\begin{lstlisting}[frame=none]
public class LayerConstraint
 extends java.lang.Object\end{lstlisting}
\subsubsection{Constructors}{
\vskip -2em
\begin{itemize}
\item{ 
\index{LayerConstraint(Set, Set, boolean, int)}
{\bf  LayerConstraint}\\
\begin{lstlisting}[frame=none]
public LayerConstraint(java.util.Set top,java.util.Set bottom,boolean direct,int distance)\end{lstlisting} %end signature
}%end item
\end{itemize}
}
\subsubsection{Methods}{
\vskip -2em
\begin{itemize}
\item{ 
\index{bottomSet()}
{\bf  bottomSet}\\
\begin{lstlisting}[frame=none]
public java.util.Set bottomSet()\end{lstlisting} %end signature
\begin{itemize}
\item{
{\bf  Description}

Returns the set which should be below.
}
\item{{\bf  Returns} -- 
the bottom layer 
}%end item
\end{itemize}
}%end item
\item{ 
\index{getDistance()}
{\bf  getDistance}\\
\begin{lstlisting}[frame=none]
public int getDistance() throws java.lang.IllegalStateException\end{lstlisting} %end signature
\begin{itemize}
\item{
{\bf  Description}

Returns the distance the two sets should be apart, if this constraint is exact.
}
\item{{\bf  Returns} -- 
the number of layers between the sets 
}%end item
\item{{\bf  Throws}
  \begin{itemize}
   \item{\vskip -.6ex \texttt{java.lang.IllegalStateException} -- if the set is not exact}
  \end{itemize}
}%end item
\end{itemize}
}%end item
\item{ 
\index{isExact()}
{\bf  isExact}\\
\begin{lstlisting}[frame=none]
public boolean isExact()\end{lstlisting} %end signature
\begin{itemize}
\item{
{\bf  Description}

Returns true if the constraints describes an exact distance between the two sets, false otherwise.
}
\item{{\bf  Returns} -- 
true if exact 
}%end item
\end{itemize}
}%end item
\item{ 
\index{topSet()}
{\bf  topSet}\\
\begin{lstlisting}[frame=none]
public java.util.Set topSet()\end{lstlisting} %end signature
\begin{itemize}
\item{
{\bf  Description}

Returns the set which should be on top.
}
\item{{\bf  Returns} -- 
the top layer 
}%end item
\end{itemize}
}%end item
\end{itemize}
}
}
\subsection{\label{sugiyama.SugiyamaGraph}\index{SugiyamaGraph}Class SugiyamaGraph}{
\vskip .1in 
The SugiyamaGraph is a wrapper for a directed graph to enable easy and fast acessability of attributes and constructs needed during the computation of the hierarchical layout of a directed graph. All vertices are assigned to a layer.\vskip .1in 
\subsubsection{Declaration}{
\begin{lstlisting}[frame=none]
public class SugiyamaGraph
 extends graphmodel.DirectedGraph\end{lstlisting}
\subsubsection{Constructors}{
\vskip -2em
\begin{itemize}
\item{ 
\index{SugiyamaGraph(G)}
{\bf  SugiyamaGraph}\\
\begin{lstlisting}[frame=none]
public SugiyamaGraph(graphmodel.DirectedGraph graph)\end{lstlisting} %end signature
\begin{itemize}
\item{
{\bf  Description}

Constructs a new SugiyamaGraph and sets the Graph which is the underlying representation. To fulfill the invariant that all vertices are assigned to a layer, all vertices will be assigned to layer 0.
}
\item{
{\bf  Parameters}
  \begin{itemize}
   \item{
\texttt{graph} -- the graph used as underlying representation.}
  \end{itemize}
}%end item
\end{itemize}
}%end item
\end{itemize}
}
\subsubsection{Methods}{
\vskip -2em
\begin{itemize}
\item{ 
\index{assignToLayer(V, int)}
{\bf  assignToLayer}\\
\begin{lstlisting}[frame=none]
public void assignToLayer(graphmodel.Vertex vertex,int layerN)\end{lstlisting} %end signature
\begin{itemize}
\item{
{\bf  Description}

Assigns the specified vertex to the specified layer. A vertex must only be assigned to one layer at the time. When assigned to a new layer the vertex will be removed from layer it was assigned before.
}
\item{
{\bf  Parameters}
  \begin{itemize}
   \item{
\texttt{vertex} -- the vertex to assign to the layer}
   \item{
\texttt{layerN} -- the index of the layer}
  \end{itemize}
}%end item
\end{itemize}
}%end item
\item{ 
\index{getLayer(int)}
{\bf  getLayer}\\
\begin{lstlisting}[frame=none]
public java.util.List getLayer(int layerN)\end{lstlisting} %end signature
\begin{itemize}
\item{
{\bf  Description}

Returns a copy of the layer specified by the index.
}
\item{
{\bf  Parameters}
  \begin{itemize}
   \item{
\texttt{layerN} -- the index of the layer, which should be returned}
  \end{itemize}
}%end item
\item{{\bf  Returns} -- 
the layer 
}%end item
\end{itemize}
}%end item
\item{ 
\index{getLayers()}
{\bf  getLayers}\\
\begin{lstlisting}[frame=none]
public java.util.List getLayers()\end{lstlisting} %end signature
\begin{itemize}
\item{
{\bf  Description}

Returns a copy of all layers.
}
\item{{\bf  Returns} -- 
the layers. 
}%end item
\end{itemize}
}%end item
\item{ 
\index{getReplacedEdges()}
{\bf  getReplacedEdges}\\
\begin{lstlisting}[frame=none]
public java.util.Set getReplacedEdges()\end{lstlisting} %end signature
\begin{itemize}
\item{
{\bf  Description}

Returns the set of replaced edges.
}
\item{{\bf  Returns} -- 
the set of replaced edges 
}%end item
\end{itemize}
}%end item
\item{ 
\index{getReversedEdges()}
{\bf  getReversedEdges}\\
\begin{lstlisting}[frame=none]
public java.util.Set getReversedEdges()\end{lstlisting} %end signature
\begin{itemize}
\item{
{\bf  Description}

Returns the set of all with \texttt{\small reverseEdge(E edge)} reversed edges.
}
\item{{\bf  Returns} -- 
the set of all reversed edges. 
}%end item
\end{itemize}
}%end item
\item{ 
\index{replaceWithSupplementPath(E, int)}
{\bf  replaceWithSupplementPath}\\
\begin{lstlisting}[frame=none]
public void replaceWithSupplementPath(graphmodel.DirectedEdge edge,int length)\end{lstlisting} %end signature
\begin{itemize}
\item{
{\bf  Description}

Replaces the specified edge with a path of dummy vertices of the specified length. Replaced edges are removed from the set of edges but saved for later retrieval with \texttt{\small getReplacedEdges()} or restored with \texttt{\small restoreReplacedEdges}.
}
\item{
{\bf  Parameters}
  \begin{itemize}
   \item{
\texttt{edge} -- the edge to be replaced}
   \item{
\texttt{length} -- the length of the path which replaces the edge}
  \end{itemize}
}%end item
\end{itemize}
}%end item
\item{ 
\index{restoreReplacedEdges()}
{\bf  restoreReplacedEdges}\\
\begin{lstlisting}[frame=none]
public java.util.Set restoreReplacedEdges()\end{lstlisting} %end signature
\begin{itemize}
\item{
{\bf  Description}

Deletes all dummy vertices and edges connecting dummy vertices. Adds the replaced edges back to set of edges.
}
\item{{\bf  Returns} -- 
the set of edges, which has been restored 
}%end item
\end{itemize}
}%end item
\item{ 
\index{restoreReversedEdges()}
{\bf  restoreReversedEdges}\\
\begin{lstlisting}[frame=none]
public java.util.Set restoreReversedEdges()\end{lstlisting} %end signature
\begin{itemize}
\item{
{\bf  Description}

Deletes the supplement edges, which have been created when an edge was reversed. Adds all reversed edges back to the set of edges and returns them.
}
\item{{\bf  Returns} -- 
the set of edges which have been restored. 
}%end item
\end{itemize}
}%end item
\item{ 
\index{reverseEdge(E)}
{\bf  reverseEdge}\\
\begin{lstlisting}[frame=none]
public void reverseEdge(graphmodel.DirectedEdge edge)\end{lstlisting} %end signature
\begin{itemize}
\item{
{\bf  Description}

Reverses the specified edge (u, v) by replacing it with an supplement edge (v, u) The action will be saved and can later be retrieved with \texttt{\small getInvertedEdges()} or restored with \texttt{\small restoreReversedEdges()} of the old state. While reversed the specified edge won't be part of the edge set and therefore not returned by \texttt{\small getEdgeSet()}
}
\item{
{\bf  Parameters}
  \begin{itemize}
   \item{
\texttt{edge} -- the edge to reverse}
  \end{itemize}
}%end item
\end{itemize}
}%end item
\end{itemize}
}
}
\subsection{\label{sugiyama.SugiyamaLayoutAlgorithm}\index{SugiyamaLayoutAlgorithm}Class SugiyamaLayoutAlgorithm}{
\vskip .1in 
This class controls the collaboration between every single step of Sugiyama Layout.\vskip .1in 
\subsubsection{Declaration}{
\begin{lstlisting}[frame=none]
public class SugiyamaLayoutAlgorithm
 extends java.lang.Object implements plugin.LayoutAlgorithm\end{lstlisting}
\subsubsection{Constructors}{
\vskip -2em
\begin{itemize}
\item{ 
\index{SugiyamaLayoutAlgorithm()}
{\bf  SugiyamaLayoutAlgorithm}\\
\begin{lstlisting}[frame=none]
public SugiyamaLayoutAlgorithm()\end{lstlisting} %end signature
}%end item
\end{itemize}
}
\subsubsection{Methods}{
\vskip -2em
\begin{itemize}
\item{ 
\index{getSettings()}
{\bf  getSettings}\\
\begin{lstlisting}[frame=none]
parameter.Settings getSettings()\end{lstlisting} %end signature
\begin{itemize}
\item{
{\bf  Description copied from plugin.LayoutAlgorithm{\small \refdefined{plugin.LayoutAlgorithm}} }

Get the set of parameters for this instance of the algorithm.
}
\item{{\bf  Returns} -- 
the set of parameters 
}%end item
\end{itemize}
}%end item
\item{ 
\index{layout(G)}
{\bf  layout}\\
\begin{lstlisting}[frame=none]
void layout(graphmodel.Graph graph)\end{lstlisting} %end signature
\begin{itemize}
\item{
{\bf  Description copied from plugin.LayoutAlgorithm{\small \refdefined{plugin.LayoutAlgorithm}} }

Layout the specified Graph.
}
\item{
{\bf  Parameters}
  \begin{itemize}
   \item{
\texttt{graph} -- the graph to layout}
  \end{itemize}
}%end item
\end{itemize}
}%end item
\end{itemize}
}
}
\subsection{\label{sugiyama.VertexPositioning}\index{VertexPositioning}Class VertexPositioning}{
\vskip .1in 
This class takes a directed graph and position its vertices in order to look more clearly. (e.g. position vertices in a row or column)\vskip .1in 
\subsubsection{Declaration}{
\begin{lstlisting}[frame=none]
public class VertexPositioning
 extends java.lang.Object\end{lstlisting}
\subsubsection{Constructors}{
\vskip -2em
\begin{itemize}
\item{ 
\index{VertexPositioning()}
{\bf  VertexPositioning}\\
\begin{lstlisting}[frame=none]
public VertexPositioning()\end{lstlisting} %end signature
}%end item
\end{itemize}
}
\subsubsection{Methods}{
\vskip -2em
\begin{itemize}
\item{ 
\index{positionVertices(SugiyamaGraph)}
{\bf  positionVertices}\\
\begin{lstlisting}[frame=none]
public void positionVertices(SugiyamaGraph graph)\end{lstlisting} %end signature
\begin{itemize}
\item{
{\bf  Description}

Position all vertices relatively to each other.
}
\item{
{\bf  Parameters}
  \begin{itemize}
   \item{
\texttt{graph} -- input graph}
  \end{itemize}
}%end item
\end{itemize}
}%end item
\end{itemize}
}
}
}
\printindex
