\chapter{Änderungen am Entwurf}
\label{ch:aenderungen}

\newcounter{cnr}
\newcommand\change[2]{\textbf{\arabic{cnr}}\addtocounter{cnr}{1}. & \textbf{Aktion:} & #1 \\ & \textbf{Grund:} & #2 \\ [1ex] }
	
\section{General}
\setcounter{cnr}{1}

\begin{longtable}{llp{0.8\linewidth}}
	\change	{Methode \textit{getGraphBuilder()} verschoben von \textit{IVertexBuilder} zu \textit{IGraphBuilder}.} 
			{Dadurch können hierachische Graphen einfacher erstellt werden, da der \textit{IGraphBuilder} den \textit{IGraphModelBuilder} kennt.}
	\change	{Collabsable Graph Interface hinzugefügt} 
			{unterscheiden zwischen einem compoundgraph(z.B. FieldAccess) und einem normalen zusammengeklappten subgraphen}
	\change	{Abhängigkeiten zwischen \textit{JoanaGraphModelBuilder}, \textit{JoanaGraphBuilder}, \textit{JoanaVertexBuilder} und \textit{JoanaEdgeBuilder} hinzugefügt.} 
			{Jede Builder Klasse muss wissen von welchem Builder sie erstellt wurde, um das erstellte Produkt (z.B. \textit{JoanaVertex} oder \textit{JoanaEdge}) bei seiner Elternklasse abspeichern zu können. }
	\change	{Added AbstractPluginBase} 
			{Many functions in the concrete Plugins are nearly empty, but have to be overwritten. An AbstractPluginBase reduces identical code.}
	\change	{Removed build() functions from IGraphBuilder, IVertexBuilder, IEdgeBuilder.} 
			{build() is now only called on IGraphModelBuilder, which then calls recursively the specific build functions of the concrete classes. (Lucas)}
	\change	{Replaced nodeKind (String) field from JoanaVertex with enum. Adapted interface of JoanaVertex accordingly.} 
			{All reason why using enum is better when possibility are known at compile-time.  (Lucas)}
	\change	{Replaced edgeKind (String) field from JoanaEdge with enum. Adapted interface of JoanaEdge accordingly.} 
			{All reason why using enum is better when possibility are known at compile-time.  (Lucas)}
	\change	{Jede \textit{GraphView} besitzt ihre eigene \textit{GraphViewGraphFactory}.}
			{Die \textit{GraphViewGraphFactory} bietet Zugriff mittels eines Mappings von GUI-Elementen auf die Modelelementen und auf den dargestellten Graph.}
	\change	{\textit{SerializedGraph} verschoben. Der angezeigte Graph wird nun von der \textit{GraphViewGraphFactory} serialisiert und im Model.} 
			{Die \textit{GraphViewGraphFactory} besitzt mehr Information (z.B. Farbe, Gruppe usw.) über den Graphen als das Model. Diese Informationen sind vor allem für den \textit{SvgExporter} wichtig.}
	\change	{Added class DefaultDirectedEdge and changed DirectedEdge from class to interface. Changed occurences of DirectedEdge to DefaultDirectedEdge in most cases in whole project.} 
			{There was a need of an interface of DirectedEdge (Jonas M)}
	\change	{\textit{getColor()} in \textit{Edge} und \textit{Vertex} hinzugefügt} 
			{Die Farbe muss von der \textit{Edge}/\textit{Vertex} vorgegeben werden, damit die GUI verschiedene Farben vergeben kann, \textit{JoanaEdge} und \textit{JoanaVertex} geben jeweils eine Farbe zurück, die von ihrem Typ bestimmt wird.}
	\change	{Aktuelle \textit{LayoutOption} wird in \textit{GraphView} gespeichert} 
			{Um zu wissen welcher Layoutalgorithmus zuletzt angewendet wurde und um diesen wiederholt anwenden zu können. Dies ist unter anderem bei der Collapse bzw. Expand Funktion und dem ändern der Properties nötig.}
	\change	{Die Realisierung von Gruppen wurde in die GUI verschoben und jede \textit{VertexShape} speichert ihren VertexStyle}
			{Da Gruppen nicht direkt mit dem darunter liegenden Datenmodel zusammenhängen, werden sie komplett in der GUI realisiert. Dazu wird für eine bestimmte Untermenge der Knoten eine von JavaFX definierter Style, im CSS Format, gesetzt. Dieser wird in der \textit{VertexShape} gespeichert und mittels \textit{getVertexStyle()} und \textit{setVertexStyle()} zugegriffen.}
\end{longtable}

\section{Sugiyama}
\label{sec:change_sugiyama}
\setcounter{cnr}{1}

\begin{tabular}{llp{0.8\linewidth}}
	\change	{Changed method return type of reverseEdge(SugiyamaEdge edge) in ICycleRemoverGraph from Set<SugiyamaEdge> to void} 
			{Not really necessary to know which edges have been turned, it can be queried from the edge throug an instance of a SugiyamaGraph (Jonas M)}
	\change	{Added Interface ISugiyamaVertex and let SugiyamaVertex and DummyVertex implement it. Changed every occurence of SugiyamaVertex to ISugiyamaVertex in package sugiyama} 
			{It's necessary to treat SugiyamaVertex and DummyVertex the same way in a common list (Jonas M)}
	\change	{Moved class Point to from package sugiyama to package edu.kit.student.util} 
			{For a better overview (Lucas)}
	\change	{SupplementPath does not extends DirectedEdge anymore.} 
			{?}
\end{tabular}

