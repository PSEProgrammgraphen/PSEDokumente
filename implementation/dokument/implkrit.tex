\chapter{Implementierte Muss- und Wunschkriterien}
\label{ch:implkrit}

\section{Pflichtkriterien}

\subsection{Allgemein}
\begin{itemize}
	\item Anzeige von Graphen (/FA010/)
	\item Anzeige von geschachtelten Graphen (/FA020/)
		\begin{itemize}
			\item Tastenkürzel existieren nicht, da die meisten Steuerfunktionen sowieso über die Maus funktionieren müssen und deshalb einzelne Tastenkürzel nicht sinnvoll sind.
		\end{itemize}
	\item Graphen layouten (/FA030/)
	\item Knoten und Kanten filtern (/FA040/)
	\item Anwendung einer Arbeitsumgebung (/FA050/)
	\item Kollabieren  (/FA060/) und Ausklappen (/FA070/) von Subgraphen
		\begin{itemize}
			\item Der Graph wird automatisch neu gelayoutet, da die Positionierung des neuen Knoten durch den LayoutAlgorithmus geschehen muss.
		\end{itemize}
\end{itemize}

\subsection{Input/Output}
\begin{itemize}
	\item Graph aus Datei importieren (/FA100/)
		\begin{itemize}
			\item Generischen GraphML Dateien werden theoretisch vom Importer unterstützt können aber nicht geladen werden, da nur ein JOANA-Workspace existiert.
		\end{itemize}
	\item Export als Bilddatei (/FA110/)
\end{itemize}

\subsection{Steuerung}
\label{implkrit:steuerung}
\begin{itemize}
	\item Sichtfeld verschieben (/FA200/)
		\begin{itemize}
			\item Die im Pflichtenheft definierte Maustastenbelegung kann von Java leider nicht unter jedem Betriebssystem realisiert werden. Im Fall von Windows 8.1 gibt es eine Funktionalität, die auf den Klick des Mausrads reagiert. Aus diesem Grund wurde die Tastenbelegung leicht abgeändert
			\item Das Sichtfeld verschiebt man nun durch das Drücken von STRG und klicken und ziehen der rechten Maustaste
		\end{itemize}
	\item Zoom-Grad ändern (/FA210/)
	\item Knoten selektieren (/FA220/) und deselektieren (/FA230/)
		\begin{itemize}
			\item Um das Selektieren und Deselektieren von Knoten benutzerfreundlicher zu gestalten, wurde die Tastenbelegung an die der gängigen Dateisystemexplorer angepasst:
			\item Zum Hinzufügen oder Entfernen eines einzelnen Knotens aus der Selektion von mehreren Knoten muss beim Klicken STRG gehalten werden. Ansonsten wird die komplette Selektion aufgehoben und nur der angeklickte Knoten selektiert.
			\item Ist eine Menge von Knoten selektiert, STRG gedrückt und man selektiert durch ziehen der Maus mehrere Knoten, werden bereits selektierte Knoten aus der Menge entfernt und nicht selektierte Knoten hinzugefügt.
		\end{itemize}
	\item Knoten einer Gruppe hinzufügen (/FA240/)
	\item Gruppe löschen
		\begin{itemize}
			\item Gruppe kann nicht über Kontextmenü gelöscht werden, da dies unhandlich wird falls ein Knoten in mehreren Gruppen ist.
		\end{itemize}
	\item Wechsel zwischen Graphen (/FA260/)
	\item Informationsanzeige zu einzelnen Knoten und Kanten (/FA270/)
		\begin{itemize}
			\item Es existiert keine Informationsanzeige zu Kanten, weil diese wenige Informationen besitzen und eine Selektion nicht für weitere Operationen nötig ist
		\end{itemize}
	\item Statistiken zu Graphen (/FA280/)
	\item Kontextmenü (/FA290/)
\end{itemize}

%Eventuell anders reinbringen
\subsection{Nichtfunktionale Anforderungen}
\begin{itemize}
	\item Es werden mindestens Graphen mit bis zu 1000 Knoten unterstützt (/NFA010/)
	\item Es werden mindestens Graphdateien mit bis zu 100.000 Knoten insgesamt unterstützt (/NFA020/)
	\item Die maximale unterstützte Kantenanzahl pro Graph entspricht mindestens der 3-4 fachen Knotenzahl aus /NFA010/ und /NFA020/ (/NFA030/)
	\item Der Algorithmus, welcher das Graphlayout berechnet ist einfach auswechselbar (/NFA040/)
	\item Durch heuristische Methoden wird ein möglichst optimales Ergebnis bei der Einhaltung gegebener Constraints erreicht (/NFA050/)
		\begin{itemize}
			\item Da es keine benutzerdefinierte Constraints gibt, muss diese Einhaltung nicht beachtet werden
		\end{itemize}
	\item Ein einfacher Sprachwechsel der GUI soll möglich sein (/NFA100/)
		\begin{itemize}
			\item Der einfache Sprachwechsel hatte eine niedrige Priorität und wurde deshalb aus Zeitgründen nicht implementiert
		\end{itemize}
	\item Falls das Programm mit falschen Parametern über die Kommandozeile gestartet wird, soll das Programm nicht abstürzen (/NFA110/)
		\begin{itemize}
			\item Abfangen des Errors wurde aus Zeitgründen in die Testphase verschoben
		\end{itemize}
	\item Die Berechnung eines Graphen von 1000 Knoten und 4000 Kanten soll nicht länger als 2 Min dauern (/NFA210/)
		\begin{itemize}
			\item Die Berechnung dauert länger, da der CrossMinimizer noch nicht optimiert ist. Diese Optimierung wurde aus Zeitgründen in die Testphase verschoben.
		\end{itemize}
	\item Kanten sollen nicht direkt übereinander liegen (/NFA220/)
	\item Knoten dürfen sich nicht mit anderen Knoten überschneiden (/NFA230/)
	\item Beschriftung von Knoten darf nicht unkenntlich sein (/NFA240/)
	\item Die Anzahl der sich kreuzenden Kanten wird mittels der Barycenter Heuristik möglichst gering gehalten (/NFA250/)
\end{itemize}


\subsection{Plugins}
\begin{itemize}
	\item Schnittstellen für Plugins in den Bereichen Import, Export, Layoutalgorithmen, Filter für Knoten- und Kantentypen und weitere Operationen auf einzelne Knoten und Kanten. Die Schnittstelle für Darstellung wurde nicht implementiert, da diese im Entwurf nicht berücksichtigt wurde und eine Implementierung viele Änderungen erfordert hätte. (Pflichtenheft Kapitel 7)
	\item Es gibt ein Pluginmanagement-System, welches externe Plugins laden und verwalten kann. (Pflichtenheft Kapitel 7)
\end{itemize}

\subsection{Sonstiges}
\label{implkrit:sonstiges}
\begin{itemize}
	\item Gesetzte Einstellungen (z.B. Speicherort der zueltzt exportierten Datei oder Größe des Fensters) werden gespeichert
		\begin{itemize}
			\item Wurde aus Zeitgründen nicht implementiert
		\end{itemize}
	\item Akzeptieren von Kommandozeilenargumenten
		\begin{itemize}
			\item Das Eingabeformat der Kommandozeilenargumente hat sich zum Pflichtenheft geändert, um ein einfacheres parsen der Parameter zu  ermöglichen. Argumente werden nun im folgenden Format eingegeben: \glqq- -in=<datei>\grqq
		\end{itemize}
	\item Das Produkt wird unter einer freien Lizenz veröffentlicht	
\end{itemize}

\section{Wunschkriterien}
Es wurden, aufgrund von Zeitmangel, keine Wunschkriterien aus dem Pflichtenheft implementiert, 
