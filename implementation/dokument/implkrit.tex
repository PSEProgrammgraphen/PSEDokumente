\chapter{Implementierte Muss- und Wunschkriterien}
\label{ch:implkrit}

\section{Pflichtkriterien}

\subsection{Allgemein}
\begin{itemize}
	\item Anzeige von Graphen (/FA010/)
	\item Anzeige von geschachtelten Graphen (/FA020/)
		\begin{itemize}
			\item Öffnen eines geschachtelten Graphen über ein Kontextmenü eines Knoten wurde nicht implementiert (z.B. Callgraph)
			\item Tastenkürzel existiert nicht
		\end{itemize}
	\item Graphen layouten (/FA030/)
		\begin{itemize}
			\item Constraints für den Methodengraphen werden gesetzt, jedoch nicht im Layout-Algorithmus beachtet.
		\end{itemize}
	\item Knoten und Kanten filtern (/FA040/)
	\item Anwendung einer Arbeitsumgebung (/FA050/)
		\begin{itemize}
			\item CallgraphLayout wird nicht automatisch ausgewählt
		\end{itemize}
	\item Kollabieren von Subgraphen (/FA060/)
		\begin{itemize}
			\item Der Graph wird automatisch neugeladen und ist nicht optional
		\end{itemize}
	\item Ausklappen von kollabierten Subgraphen (/FA070/)
		\begin{itemize}
			\item Der Graph wird automatisch neugeladen und ist nicht optional
		\end{itemize}
\end{itemize}

\subsection{Input/Output}
\begin{itemize}
	\item Graph aus Datei importieren (/FA100/)
		\begin{itemize}
			\item Werden auch generische GraphML Dateien importiert?
		\end{itemize}
	\item Export als Bilddatei (/FA110/)
\end{itemize}

\subsection{Steuerung}
\label{implkrit:steuerung}
\begin{itemize}
	\item Sichtfeld verschieben (/FA200/)
		\begin{itemize}
			\item Die im Pflichtenheft definierte Maustastenbelegung kann von Java leider nicht unter jedem Betriebssystem realisiert werden. Im Fall von Windows 8.1 gibt es eine Funktionalität, die auf den Klick des Mausrads reagiert. Aus diesem Grund wurde die Tastenbelegung leicht abgeändert
			\item Das Sichtfeld verschiebt man nun durch das Drücken von STRG und klicken und ziehen der rechten Maustaste
		\end{itemize}
	\item Zoom-Grad ändern (/FA210/)
		\begin{itemize}
			\item Es existiert kein Fehlschlag (kein Maximum oder Minimum des Zooms
		\end{itemize}
	\item Knoten selektieren (/FA220/) und deselektieren (/FA230/)
		\begin{itemize}
			\item Um das Selektieren und Deselektieren von Knoten benutzerfreundlicher zu gestalten, wurde die Tastenbelegung an die der gängigen Dateisystemexplorer angepasst:
			\item Zum Hinzufügen oder Entfernen eines einzelnen Knotens aus der Selektion von mehreren Knoten muss beim Klicken STRG gehalten werden. Ansonsten wird die komplette Selektion aufgehoben und nur der angeklickte Knoten selektiert.
			\item Ist eine Menge von Knoten selektiert, STRG gedrückt und man selektiert durch ziehen der Maus mehrere Knoten, werden bereits selektierte Knoten aus der Menge entfernt und nicht selektierte Knoten hinzugefügt.
		\end{itemize}
	\item Knoten einer Gruppe hinzufügen (/FA240/)
	\item Gruppe löschen
		\begin{itemize}
			\item Gruppe kann nicht über Kontextmenü gelöscht werden
		\end{itemize}
	\item Wechsel zwischen Graphen (/FA260/)
	\item Informationsanzeige zu einzelnen Knoten und Kanten (/FA270/)
		\begin{itemize}
			\item Keine Informationsanzeige zu Kanten
		\end{itemize}
	\item Statistiken zu Graphen (/FA280/)
	\item Kontextmenü (/FA290/)
\end{itemize}

%Eventuell anders reinbringen
\subsection{Nichtfunktionale Anforderungen}
\begin{itemize}
	%TODO: richtige Formulierung? nicht dass Snelting wieder daran hängen bleibt
	\item Es werden mindestens Graphen mit bis zu 1000 Knoten unterstützt (/NFA010/)
	\item Es werden mindestens Graphdateien mit bis zu 100.000 Knoten insgesamt unterstützt (/NFA020/).. stimmt das?
	\item Die maximale unterstützte Kantenanzahl pro Graph entspricht mindestens der 3-4 fachen Knotenzahl aus /NFA010/ und /NFA020/. (/NFA030/).. stimmt das?
	\item Der Algorithmus, welcher das Graphlayout berechnet ist einfach auswechselbar (/NFA040/)
	%TODO: welche heuristische Methode?
	\item Durch heuristische Methoden wird ein möglichst optimales Ergebnis bei der Einhaltung gegebener Constraints erreicht (/NFA050/) FALSCH!!!
	\item Ein einfacher Sprachwechsel der GUI soll möglich sein (/NFA100/)
		\begin{itemize}
			\item Da Nicolas schlecht implementiert hat ist das nicht möglich
		\end{itemize}
	\item Falls das Programm mit falschen Parametern über die Kommandozeile gestartet wird, soll das Programm nicht abstürzen (/NFA110/)
		\begin{itemize}
			\item Abfangen des Errors wurde aus Zeitgründen in die Testphase verschoben
		\end{itemize}
\end{itemize}

\subsection{Plugins}
\begin{itemize}
	\item Schnittstellen für Plugins in den Bereichen Import, Export, Layoutalgorithmen, Filter für Knoten- und Kantentypen und weitere Operationen auf einzelne Knoten und Kanten (Pflichtenheft Kapitel 7)
	\item Es gibt ein Pluginmanagement-System, welches externe Plugins laden und verwalten kann. (Pflichtenheft Kapitel 7)
\end{itemize}

\subsection{Sonstiges}
\label{implkrit:sonstiges}
\begin{itemize}
	\item Gesetzte Einstellungen (z.B. Speicherort der zueltzt exportierten Datei oder Größe des Fensters) werden gespeichert
		\begin{itemize}
			\item Wurde nicht implementiert
		\end{itemize}
	\item Akzeptieren von Kommandozeilenargumenten
		\begin{itemize}
			\item Das Eingabeformat der Kommandozeilenargumente hat sich zum Pflichtenheft geändert, um ein einfacheres parsen der Parameter zu  ermöglichen. Argumente werden nun im folgenden Format eingegeben: \glqq- -in=<datei>\grqq
		\end{itemize}
	\item Das Produkt wird unter einer freien Lizenz veröffentlicht	
\end{itemize}

\section{Wunschkriterien}
Aus den Wunschkriterien die im Pflichtenheft definiert wurden, wurden aus Zeitmangel keine Implementiert.
