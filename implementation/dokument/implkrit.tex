\chapter{Implementierte Muss- und Wunschkriterien}
\label{ch:implkrit}

\section{Pflichtkriterien}

\subsection{Allgemein}
\begin{itemize}
	\item Hierarchisches Layout mit dem Sugiyama-Framework
	\item Ein Callgraph-Layout, welches übersichtlich die Abhängigkeiten der Methoden darstellt
	\item Ein Methodgraph-Layout, welches die Abhängigkeiten innerhalb einer Methode - mithilfe von vorgegebenen Constraints darstellt
	\item Kollabieren und Ausklappen von Subgraphen
	\item Informationsanzeige zu einzelnen Knoten und Kanten
	\item Statistiken über den Graphen und Subgraphen
	\item Filter für Knoten- und Kantentypen aus JOANA
	\item Tabs für geöffnete Graphen
	\item Akzeptieren von Kommandozeilenargumenten zur Angabe von Graphdatei und Layoutalgorithmus für ein schnelles Starten
	\item Das Produkt wird unter einer freien Lizenz veröffentlicht
	\item Die Anzeigesprache der GUI ist englisch. Ein Sprachwechsel soll aber leicht zu implementieren sein.
\end{itemize}

\subsection{Input/Output}
\begin{itemize}
	\item Import von generischen Graphen im GraphML-Format
	\item Export der visualisierten Graphen im SVG-Format
\end{itemize}

\subsection{Steuerung}
\begin{itemize}
	\item Navigation mittels Zoom und Verschieben
	\item Selektieren und Deselektieren von einzelnen oder mehreren Knoten
\end{itemize}

\subsection{Plugins}
\begin{itemize}
	\item Schnittstellen für Plugins in den Bereichen Import, Export, Layoutalgorithmen, Filter für Knoten- und Kantentypen und weitere Operationen auf einzelne Knoten und Kanten
	\item Es gibt ein Pluginmanagement-System, welches externe Plugins laden und verwalten kann.
\end{itemize}

\section{Wunschkriterien}

\subsection{Allgemein}
\begin{itemize}
	\item Automatisiertes Subgraph finden mittels Graph Pattern Matching
	\item Layout Constraints, die vom Nutzer angepasst werden
	\item Fortschrittsbalken bei der Berechnung des Layouts des Graphen
	\item Eine Übersicht des angezeigten Graphen
	\item Algorithmus zur Erreichbarkeit eines Knoten
	\item Die Darstellung von Kanten kann geändert werden
	\item Reload-Funktion
\end{itemize}

\subsection{Steuerung}
\begin{itemize}
	\item Tastaturkürzel (evtl. benutzerdefiniert)
\end{itemize}

\subsection{Input/Output}
\begin{itemize}
	\item weitere Exportfunktionen für das JPG- und das GraphML-Format (mit Koordinaten der Knoten und Kanten des aktuellen Layouts)
\end{itemize}