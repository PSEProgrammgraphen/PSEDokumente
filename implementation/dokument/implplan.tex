\chapter{Implementierungsplan}
\label{ch:implplan}
Um einen Implementierungsplan aufzustellen wurde das Projekt zuerst anhand der einzelnen Plugins/Pakete und dann noch feiner in einzelne möglichst voneinander unabhängige Aufgaben unterteilt. Der benötigte Arbeitsaufwand für die Aufgaben wurden von allen Projektmitgliedern abgeschätzt und deren Mittelwert als geplante Bearbeitungsdauer bis zur Fertigstellung der Aufgabe angenommen.\\
Anhand von bestehenden Abhängigkeiten der Aufgaben untereinander wurde folgender Implementierungsplan erstellt:
%TODO erster impl plan einfügen.
\section{Woche 1}
In der ersten Woche wurde der zuerst erstellte Implementierungsplan komplett überarbeitet. Die Aufgaben wurden in der Abarbeitungsreihenfolge umsortiert, verfeinert und für die ersten zwei Wochen einzelnen Personen zugewiesen. Zusätzlich wurde eine Tabelle erstellt, die den bereits existierenden Plan, durch eine Fortschrittsspalte, in der die Entwickler ihren Prozentualen Fortschritt der ihnen zugeteilten Aufgaben eintragen, und eine visuelle Repräsentation des Soll- und Ist-Zustands des gesamten Projekt erweitert.\\
\\
Zusätzlich zur Änderung des Plans wurde eine grobe Projektstruktur in Gradle aufgebaut und die bereits im Entwurf gemachten Interfaces in das Projekt überführt. Um die Gradle Projektkonfiguration zu testen wurde eine grobe GUI implementiert und die Entwicklung der allgemeinen genutzten Klassen aus dem Paket "graphmodel" begonnen.

%TODO bild erste soll-ist-kurve und ausschnitt des plans

\subsection{Verzögerungen}
Wie man in (ref aufs bild oben) sehen kann hing das Projekt nach der ersten Woche bereits ca. 25 Stunden hinter dem geplanten Zustand hinterher.
Dies liegt neben der Tatsache das die erste Hälfte der Woche, auf Grund von Erschöpfung aus der Entwurfsphase, weniger gearbeitet wurde, daran, dass das Team das für die Implementierung des Sugiyama-Layoutalgorithmus eingeteilt wurde relativ schnell auf konzeptionelle Probleme mit dem im Entwurf definierten Aufbau des SugiyamaGraphen und dessen Kanten/Knoten stieß.%TODO ref auf änderungen sugiyama.

\section{Woche 2}
\subsection{Verzögerungen}
\subsection{Änderungen}
\section{Woche 3}
\subsection{Verzögerungen}
\subsection{Änderungen}
\section{Woche 4}
\subsection{Verzögerungen}
\subsection{Änderungen}