\chapter{Zielbestimmung}

Graph von Ansicht soll Programmgraphen von externen Analyse Programmen visualisieren. Dafür wird eine bereits vorhandene Graphdatei importiert und ausgewertet. Der Benutzer verschiedene Constraints einstellen, welche im weiteren Dokument näher erläutert werden. 
%TODO: Verweis einfügen zu den konkreten Erläuterungen
Das Endergebnis soll eine übersichtliche Darstellung der Abhängigkeiten und des Steuerflusses eines Programmes anzeigen und dem Benutzer die Möglichkeit zu bieten den Programmfluss besser analysieren zu können.

\section{Pflichtkriterien}

\begin{itemize}
\item Schnittstellen
\begin{itemize}
\item Import von generischen Graphen im \gls{graphml}-Format
\item Export der visualisierten Graphen im \gls{svg}-Format
\item Erweiterbarkeit des Programmes durch Plugin Schnittstellen
\end{itemize}
\item User interface
\begin{itemize}
\item Navigation mittels Zoom und Translation
\item Selektieren von einzelnen oder mehreren Knoten
\item Einstellen von Constraints
\item Kollabieren und Ausklappen von Gruppen und Teilgraphen
\item Filter für Knoten- und Kantentypen
\item Ausblendung selektierter Knoten
\end{itemize}
\item Visualisierung
\begin{itemize}
\item Hierarchisches Layout mit dem \gls{sugiyama}
\item Kanten werden durch \gls{bezier} dargestellt
\item Informationsanzeige von Knoten und Kanten
\item Weitere Layoutalgorithmen sollen leicht integrierbar sein 
\end{itemize}
\end{itemize}

\section{Wunschkriterien}

\begin{itemize}
\item Schnittstellen
\begin{itemize}
\item Export der visualisierten Graphen im \gls{jpg}- und \gls{graphml}-Format
\end{itemize}
\item User interface
\begin{itemize}
\item Marker zu Bildschirmausschnitte (zu Funktionen springen)
\item Muster Definition mittels "GraphRegex"
\item Benutzerdefinierte Hotkeys
\end{itemize}
\item Visualisierung
\begin{itemize}
\item Fortschrittsbalken bei Berechnung der Visualisierung des Graphen mithilfe einer Zeitabschätzung
\item Minimap bzw. Übersicht des Graphens
\item Algorithmus zur Erreichbarkeit eines Knoten
\end{itemize}
\end{itemize}

\section{Abgrenzungskriterien}

\begin{itemize}
\item Das Produkt ist kein Graph Editor und unterstützt nicht das manuelle Zeichnen/Hinzufügen von neuen Kanten und Knoten.
\item Die GUI wird nicht von Grund auf neu entwickelt, sondern verwendet Bibliotheken um die Entwicklung zu erleichtern. 
\item Das Zeichnen einzelner Kanten und Knoten wird nicht selbst implementiert, sondern wird mithilfe von Bibliotheken umgesetzt.
\item Das Produkt ist kein Analysetool für Programme, sondern dient lediglich zur Visualisierung von bereits vorhandenen Graphdateien.
\end{itemize}