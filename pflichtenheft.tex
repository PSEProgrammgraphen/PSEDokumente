\documentclass[a4paper]{scrreprt}

\usepackage[german]{babel}
\usepackage[utf8]{inputenc}
\usepackage[T1]{fontenc}
\usepackage[titletoc]{appendix}
\usepackage{ae}
\usepackage{enumitem}
\usepackage[bookmarks,bookmarksnumbered]{hyperref}

\usepackage[toc,acronym]{glossaries}
\makeglossaries

\usepackage{xparse}
\DeclareDocumentCommand{\newdualentry}{ O{} O{} m m m m } {
  \newglossaryentry{gls-#3}{name={#5},text={#5\glsadd{#3}},
    description={#6},#1
  }
  \makeglossaries
  \newacronym[see={[Glossary:]{gls-#3}},#2]{#3}{#4}{#5\glsadd{gls-#3}}
}

\loadglsentries{glossary.tex}
\begin{document}

\title{Pflichtenheft\\
Graph von Ansicht}
\date{}
\author{Nicolas Boltz   \\ uweaw@student.kit.edu
  \and Jonas Fehrenbach \\ urdtk@student.kit.edu
  \and Sven Kummetz     \\ kummetz.sven@gmail.com
  \and Jonas Meier      \\ Meierjonas96@web.de
  \and Lucas Steinmann  \\ ucemp@student.kit.edu
}
\maketitle

% Ich denke einen Abstract brauchen wir nicht aber hier ist mal ein Template einfach die Kommentarzeichen wegmachen
%\begin{abstract}
%\end{abstract}

\tableofcontents

\chapter{Zielbestimmung}
Graph von Ansicht soll \glspl{pdg} von externen Analyse-Programmen visualisieren.
\glspl{pdg} können benutzt werden, um Daten- und Kontrollabhängigkeiten in Programmen deutlich zu machen.
Anwendungsbereiche (siehe \ref{ch:einsatz}) sind beispielsweise die Erkennung etwaiger Sicherheitslücken\cite{hammer09ijis} in Programmen
oder die Optimierung von Programmen beim Kompilieren\cite{Ferrante:1987:PDG:24039.24041}.\\
Um diese Graphen zu Visualisieren, muss eine vorhandene Graphdatei importiert werden.
Graph von Ansicht ist dann in der Lage die Elemente des Graphen übersichtlich anzuordnen (im weiteren Dokument ``layouten'' genannt) und in einer grafischen Oberfläche zu zeichnen.
%Der Benutzer kann verschiedene Constraints einstellen \ref{fa:constraints}, welche im weiteren Dokument näher erläutert werden.
%TODO: Verweis einfügen zu den konkreten Erläuterungen
Das Endergebnis soll eine übersichtliche Darstellung der Abhängigkeiten und des Steuerflusses eines Programmes zeigen, die dem Benutzer die Möglichkeit bietet, das Programm besser verstehen zu können.\\

\textit{Anmerkung zum Aufbau des Produktes:} Graph von Ansicht soll als Kernprogramm die Möglichkeit bieten,
mittels Plugins Unterstützung für weitere Arten von Graphen (z.B. DOM-Trees)
hinzuzufügen, ohne Anpassungen an dem Kernprogramm selbst machen zu müssen.
Die Schnittstellen für Plugins sind in Kapitel \ref{ch:plugschnitt} beschrieben.
Die notwendige Unterstützung für JOANA-PDGs, der Export in das \gls{svg}-Format, sowie der Import aus dem \gls{graphml}-Format werden auch über Plugins realisiert.
Da diese Plugins aber fest mit dem Kernprogramm ausgeliefert werden und eine Unterscheidung in Kernprogramm und Plugin für die Benutzung des
Programmes nicht relevant ist, wird von dieser Unterscheidung im Großteil des restlichen Dokumentes abgesehen und nur falls nötig gemacht.

\section{Pflichtkriterien}

\subsection{Allgemein}
  \begin{itemize}
    \item Hierarchisches Layout mit dem \gls{sugiyama} (siehe \ref{fa:layout})
    \item Ein \gls{callgraph}-Layout, welches übersichtlich die Abhängigkeiten der Methoden darstellt (siehe \ref{fa:layout})
    \item Ein \gls{methgraph}-Layout, welches die Abhängigkeiten innerhalb einer Methode - mithilfe von vorgegebenen Constraints (siehe \ref{fa:constraints}) - darstellt (siehe \ref{fa:layout})
    \item Kollabieren und Ausklappen von \glspl{subgraph} (siehe \ref{fa:kollabieren} und \ref{fa:ausklappen})
    \item Informationsanzeige zu einzelnen Knoten und Kanten (siehe \ref{fa:infoanzeige})
    \item Statistiken über den Graphen und \glspl{subgraph} (siehe \ref{fa:statistik})
    \item Filter für Knoten- und Kantentypen aus \gls{joana} (siehe \ref{fa:filter})
    \item Tabs für geöffnete Graphen (siehe \ref{fa:graphwechsel})
    \item Akzeptieren von Kommandozeilenargumenten zur Angabe von Graphdatei und Layoutalgorithmus für ein schnelles Starten (siehe \autoref{sec:uicmd})
    \item Das Produkt wird unter einer freien Lizenz veröffentlicht
    \item Die Anzeigesprache der \gls{gui} ist englisch. Ein Sprachwechsel soll aber leicht zu implementieren sein. (siehe \ref{nfa:sprachwechsel})
  \end{itemize}

\subsection{Input/Output}
  \begin{itemize}
    \item Import von generischen Graphen im \gls{graphml}-Format (beschrieben in \nameref{ch:daten}) (siehe \ref{fa:import})
    \item Export der visualisierten Graphen im \gls{svg}-Format (siehe \ref{fa:export_img})
  \end{itemize}

\subsection{Steuerung}
  \begin{itemize}
    \item Navigation mittels Zoom und Verschieben (siehe \ref{fa:zoom} und \ref{fa:verschieben})
    \item Selektieren und Deselektieren von einzelnen oder mehreren Knoten (siehe \ref{fa:selekt_knoten} und \ref{fa:deselekt_knoten})
  \end{itemize}

\subsection{Plugins}
  \begin{itemize}
    \item Schnittstellen für Plugins in den Bereichen Import, Export, Layoutalgorithmen, Filter für Knoten- und Kantentypen und weitere Operationen auf einzelne Knoten und Kanten (siehe \autoref{ch:plugschnitt})
    \item Es gibt ein Pluginmanagement-System, welches externe Plugins laden und verwalten kann. (siehe \autoref{ch:plugschnitt})
  \end{itemize}

\section{Wunschkriterien}

\subsection{Allgemein}
  \begin{itemize}
    \item Automatisierte \gls{subgraph} Findung mittels \gls{gpm} (siehe \ref{fa:gpm})
    \item Layout Constraints, die vom Nutzer angepasst werden (siehe \ref{fa:constraints})
    \item Fortschrittsbalken bei der Berechnung des Layouts des Graphen (siehe \ref{fa:fortschritt})
    \item Eine Übersicht des angezeigten Graphen (siehe \ref{fa:uebersicht}
    \item Algorithmus zur Erreichbarkeit eines Knoten (siehe \ref{fa:erreichbarkeit})
    \item Die Darstellung von Kanten kann geändert werden (siehe \ref{fa:darst-kanten})
    \item Reload-Funktion (siehe \ref{fa:reload})
  \end{itemize}

\subsection{Steuerung}
  \begin{itemize}
    \item Tastaturkürzel (evtl. benutzerdefiniert) (siehe \ref{fa:hotkey})
  \end{itemize}

\subsection{Input/Output}
  \begin{itemize}
    \item weitere Exportfunktionen für das \gls{jpg}- und das \gls{graphml}-Format (mit Koordinaten der Knoten und Kanten des aktuellen Layouts) (vgl. \ref{fa:export_img} )
  \end{itemize}
  
\section{Abgrenzungskriterien}

\subsection{Allgemein}
  \begin{itemize}
    \item Das Produkt ist kein Graph-Editor und unterstützt deshalb die Manipulation des Graphen hinsichtlich seiner Struktur (z.B. Kanten entfernen, Knoten hinzufügen/entfernen) nicht.
    \item Das \gls{gui} wird nicht von Grund auf neu entwickelt, es werden Bibliotheken verwendet, um die Entwicklung zu erleichtern.
    \item Das Zeichnen der Kanten und Knoten mit primitiven geometrischen Objekten wird mithilfe von Bibliotheken umgesetzt (siehe \autoref{sec:bibliotheken}).
    \item Das Produkt ist kein Analysetool für Programme, sondern dient lediglich zur Visualisierung von bereits vorhandenen Graphdateien.
  \end{itemize}
\subsection{Plugins}
  \begin{itemize}
    \item Die Plugins können nur auf die in \autoref{ch:plugschnitt} beschriebenen Schnittstellen zugreifen
    \item Neue Plugins können nicht zur Laufzeit hinzugefügt werden, sondern müssen beim Programmstart vorhanden sein, um genutzt werden zu können.
  \end{itemize}


\chapter{Produkteinsatz}\label{ch:einsatz}

\section{Anwendungsbereiche}
Das Produkt soll zur Visualisierung von Programmgraphen eingesetzt werden.
Der Nutzer soll dadurch die Abhängigkeiten und den Steuerfluss eines Programms besser verstehen.
Eine mögliche Anwendung ist das Finden von Sicherheitslücken, welche durch Abhängigkeiten von sicherheitsrelevanten Daten mit unvertraulichen Quellen/Senken in Programmen sichtbar werden.
Bei Präsentationen kann das Produkt benutzt werden um dem Publikum einen (Ausschnitt eines) Graphen vorzustellen.

\section{Zielgruppen}
Institute und Forschungsgruppen, die sich mit der Analyse von Programmen beschäftigen.
Das Produkt kann auch zu Lehrzwecken an Schulen oder Hochschulen angewendet werden, um Schülern und Studenten Programmabhänghigkeitsgraphen näher zubringen.

\section{Betriebsbedingungen}
Das Produkt wird als Standalone Produkt zur Visualisierung von Programmgraphen ausgeliefert.
Somit benötigt es keine weiteren Installationen (abgesehen von der \gls{jre}).
Es benötigt auch keine aktive Internetverbindung oder ein Netzwerk.
Es wird lediglich eine Graphdatei als Input benötigt.
Das Programm ist für Linux und Windows Betriebssysteme ausgelegt und funktioniert nur auf diesen garantiert (siehe \autoref{ch:umgebung}).


\chapter{Produktumgebung}
\label{ch:umgebung}

\section{Software}
Das Produkt muss in folgenden Desktop-Systemen ausführbar und wie im restlichen Dokument beschrieben benutzbar sein:
\begin{itemize}
  \setlength\itemsep{0em}
  \item Linux Fedora 22/23 % nochmal die Versionen in der ATIS nachsehen und abgleichen.
  \item Linux Ubuntu 15.10/16.04 LTS
  \item Windows 7 und höher
\end{itemize}
Zur Programmierung wird die Programmiersprache Java benutzt. Daher ist eine Installation der \gls{jre} 8+ zur Ausführung notwendig.

\section{Hardware}
Das Produkt ist als \gls{jfx}-Anwendung zur Ausführung auf Desktop-Systemen konzipiert.
Durch die Verwendung von Java ist das Produkt unabhängig von Details der unterliegenden Hardware, sofern diese in der Lage ist die benötigte \gls{jre} (siehe oben) auszuführen.
Um die in Kapitel~\ref{ch:leistungen} beschriebenen nichtfunktionalen Anforderungen einhalten zu können, sind folgende Mindestanforderungen an das System, auf dem das Produkt ausgeführt soll, notwendig:

\begin{itemize}
  \setlength\itemsep{0em}
  \item Arbeitsspeicher: 4 GB
  \item Prozessor: Intel Core i5-4210U %Jonas F. Prozessor; vlt. durch gleichwertigen Desktopprozessor ersetzen.
  \item Festplattenspeicher: 100 MB
  \item Display: 1280x960
\end{itemize}


\chapter{Funktionale Anforderungen}
\label{ch:funktionen}

\newcounter{fanr}[chapter]
\setcounter{fanr}{10}
\newcommand{\fano}[1]{\subsection{#1}\addtocounter{fanr}{10}}
\newcommand{\subfano}[1]{\subsubsection{#1}\addtocounter{fanr}{1}}
\renewcommand\thesubsection{/FA\ifnum\value{fanr}<10 000\else\ifnum\value{fanr}<100 00\else\ifnum\value{fanr}<1000 0\fi\fi\fi\arabic{fanr}/}
\renewcommand\thesubsubsection{/FA\ifnum\value{fanr}<10 000\else\ifnum\value{fanr}<100 00\else\ifnum\value{fanr}<1000 0\fi\fi\fi\arabic{fanr}/}

\section{Muss-Funktionen} %TODO: Evtl. Umbennen

\fano{Graph aus Graph-Datei laden}\label{fa:laden}
\textbf{Ziel:} Der in der Graph-Datei beschriebene Graph soll visuell dargestellt werden. \\
\textbf{Kategorie:} \gls{io} \\
\textbf{Vorbedingung:} Der Nutzer hat eine korrekte Graph-Datei in seinem Dateissystem, auf welche zugegriffen werden kann. \\ % TODO: 'korrekt' vlt. in einem anderen Abschnitt definieren (GraphML)
\textbf{Nachbedingung (Erfolg):} Der Graph wird dargestellt. \\
\textbf{Nachbedingung (Fehlschlag):}
Eine Fehlermeldung wird ausgegeben, dass die Datei nicht geöffnet werden konnte, bzw. dass die Datei keinen korrekten Graph beschreibt. \\
\textbf{Auslösende Ereignisse:}
\begin{enumerate}[nolistsep, label=(\alph*)]
  \item Der Nutzer ruft das Programm auf und übergibt den Pfad zur Graph-Datei als Argument.
  \item Der Nutzer wählt die Graph-Datei über das Menü aus, nachdem das Programm geöffnet wurde.
\end{enumerate}
\textbf{Beschreibung:}
\begin{enumerate}[nolistsep]
  \item (a) Nutzer öffnet das Programm mit dem Pfad zur Graph-Datei als Argument.
  \item (a) Das Programm öffnet sich.
  \item Ein Fenster mit Präferenzmöglichkeiten bzgl. der Darstellung und Interpretation des Graphen öffnet sich.
  \item Der Nutzer wählt die gewünschten Einstellungen und bestätigt diese.
  \item Die Darstellung des Graphen wird berechnet und schließlich angezeigt.
\end{enumerate}
\textbf{Alternativen:}
\begin{enumerate}[nolistsep]
  \item (b) Der Nutzer wählt die Funktion Datei-->Öffnen.. aus dem Menübalken aus. %TODO: Pfeil schöner
  \item (b) Der Nutzer wählt die Graph-Datei über ein Dateiverzeichnis-Menü aus.%TODO: Dateiverzeichnis-Menü auch in GUI-Entwurf oder klar?
\end{enumerate}


\fano{Sichtfeld verschieben}\label{fa:ziehen} %Ref GUI Sichtfeld
\textbf{Ziel:} Das Sichtfeld soll verschoben werden. \\
\textbf{Kategorie:} Navigation \\
\textbf{Vorbedingung:} Ein Graph wurde geladen und das Sichtfeld deckt nicht alle Elemente des Graphens ab. \\
\textbf{Nachbedingung (Erfolg):}  Das Sichtfeld deckt nun einen anderen Teil des Graphen ab.\\
\textbf{Nachbedingung (Fehlschlag):} Das Sichtfeld deckt einen Randabschnitt ab und es wurde versucht das Sichtfeld über den Rand hinaus zu bewegen. \\
Das Sichtfeld wird nicht über den Rand bewegt. Keine Fehlermeldung. \\
\textbf{Auslösende Ereignisse:}
\begin{enumerate}[nolistsep, label=(\alph*)]
  \item Der Nutzer klickt und zieht mit der mittleren Maustaste in einem leeren Bereich des Sichtfeldes.
  \item Der Nutzer betätigt eine der zum Verschieben des Sichtfeldes designierten Tasten (siehe FAXXXX). %TODO: Referenz auf Tasten Auswahl funktion
  \item Der Nutzer bewegt die Scroll-Balken am Rand des Sichtfeldes. % TODO: Referenzt auf GUI Entwurf, wo scrollbalken sichtbar sind
\end{enumerate}
\textbf{Beschreibung:}
\begin{enumerate}[nolistsep]
  \item (a) Der Nutzer klickt und zieht mit der mittleren Maustaste in einem leeren Bereich des Sichtfeldes.
  \item (a) Das Sichtfeld verschiebt sich entgegen der Richtung, in welche die Maus bewegt wird.
\end{enumerate}
\textbf{Alternativen:}
\begin{enumerate}[nolistsep]
  \item (b) Eine Navigationstaste wird betätigt.
  \item (b) Das Sichtfeld bewegt sich um eine feste Länge (in Abhängigkeit vom Zoom-Grad) in die entsprechende Richtung.
\end{enumerate}
\begin{enumerate}[nolistsep]
  \item (c) Einer der beiden Scroll-Balken (horizontal/vertikal) wird bewegt.
  \item (c) Das Sichtfeld bewegt sich relativ zur Größe des gesamten Graphen um das gleiche Maß wie der Scroll-Balken zur Länge seiner Fahrbahn in die entsprechende Richtung.
\end{enumerate}

\fano{Zoom-Grad ändern}\label{fa:zoomen}
\textbf{Ziel:} Den Zoom-Grad des Sichtfeldes soll vergrößert bzw. verkleinert werden. \\
\textbf{Kategorie:} Navigation \\
\textbf{Vorbedingung:} Ein Graph wurde geladen. \\
\textbf{Nachbedingung (Erfolg):} Der Zoom-Grad hat sich geändert. \\
\textbf{Nachbedingung (Fehlschlag):} Der Zoom-Grad bleibt gleich, falls ein Maximum/Minimum erreicht wurde. Keine Fehlermeldung. \\
\textbf{Auslösende Ereignisse:}
\begin{enumerate}[nolistsep, label=(\alph*)]
  \item Der Nutzer dreht am Mausrad.
  \item Der Nutzer betätigt die zum Zoomen designierten Tasten. (siehe FAXXXX). %TODO: Referenz auf Tasten Auswahl funktion
\end{enumerate}
\textbf{Beschreibung:}
\begin{enumerate}[nolistsep]
  \item (a) Der Zoom-Grad ändert sich mit dem Drehen. Die Richtung kann sich abhängig vom Betriebssystem ändern.
  \item Falls das Maximum/Minimum erreicht wurde, wird dieses nicht überschritten.
\end{enumerate}
\textbf{Alternativen:}
\begin{enumerate}[nolistsep]
  \item (b) Der Zoom-Grad ändert sich um einen in Abhängigkeit zum derzeitigen Zoom-Grad festen Wert.
\end{enumerate}

\fano{Knoten selektieren}\label{fa:selekt_knoten}
\textbf{Ziel:} Ein oder mehrere Knoten sollen der Auswahl hinzugefügt werden. \\
\textbf{Kategorie:} Auswahl \\
\textbf{Vorbedingung:} Ein Graph wurde geladen. \\
\textbf{Nachbedingung (Erfolg):} Ein oder mehrere Knoten wurden der Auswahl hinzugefügt. \\
\textbf{Nachbedingung (Fehlschlag):} - \\
\textbf{Auslösende Ereignisse:}
\begin{enumerate}[nolistsep, label=(\alph*)]
  \item Der Nutzer klickt mit der linken Maustaste auf einen Knoten.
  \item Der Nutzer klickt mit der linken Maustaste in einen leeren Bereich des Sichtfeldes und zieht die Maus nach rechts unten.
  \item Der Nutzer klickt mit der linken Maustaste in einen leeren Bereich des Sichtfeldes und zieht die Maus nach links oben.
\end{enumerate}
\textbf{Beschreibung:}
\begin{enumerate}[nolistsep]
  \item Falls die Shift-Taste nicht gedrückt ist werden alle Knoten von der Auswahl entfernt.
  \item (a) Der Knoten wird der Auswahl hinzugefügt.
\end{enumerate}
\textbf{Alternativen:}
\begin{enumerate}[nolistsep]
  \setcounter{enumi}{1}
  \item (b) Es wird eine Auswahlbox vom Startpunkt bis zur jetztigen Position des Mauszeigers gezeichnet.
  \item (b) Der Nutzer lässt die linke Maustaste los, oder drückt eine andere Taste.
  \item (b) Alle Knoten, die in der Auswahlbox komplett enthalten, sind werden selektiert.
\end{enumerate}
\begin{enumerate}[nolistsep]
  \setcounter{enumi}{1}
  \item (c) Es wird eine Auswahlbox vom Startpunkt bis zur jetztigen Position des Mauszeigers gezeichnet.
  \item (c) Der Nutzer lässt die linke Maustaste los, oder drückt eine andere Taste.
  \item (c) Alle Knoten, die in der Auswahlbox enthalten sind oder von ihr geschnitten werden, werden selektiert.
\end{enumerate}

\fano{Export als Bilddatei}\label{fa:export_img}
\textbf{Ziel:} Der Graph soll in seiner aktuellen Darstellung als Bilddatei exportiert werden. \\
\textbf{Kategorie:} \gls{io} \\
\textbf{Vorbedingung:} Ein Graph wurde geladen.  \\
\textbf{Nachbedingung (Erfolg):} Es wurde eine Bilddatei an einer gewünschten, schreibbaren Stelle im Dateissystem erstellt, welche ein Abbild des derzeit angezeigten Graphens enthält. \\
\textbf{Nachbedingung (Fehlschlag):} Die Bilddatei konnte nicht erstellt werden. Es wird eine Fehlermeldung ausgegeben. \\
\textbf{Auslösendes Ereignis:}
Der Nutzer wählt die Exportfunktion im Menübalken aus.
\textbf{Beschreibung:}
\begin{enumerate}[nolistsep]
  \item Der Nutzer wählt die Exportfunktion im Menübalken unter Datei-->Export--><Dateiformat> aus. Wobei <Dateiformat> durch das gewünschte Bilddatei-Format zu ersetzen ist. Unterstützt werden mindestens das \gls{svg}- und \gls{jpg}-Format. %TODO: Referenz auf Dateiformate und Erweiterbarkeit durch Plugins
  \item Der Nutzer gibt über ein Dateisystem-Menü den gewünschten Pfad zum Abspeichern der Bilddatei an.
  \item Eventuell wird der Nutzer nach weiteren, zur Abspeicherung relevanten, Details abgefagt. (z.B. Stärke der Komprimierung bei JPEG)
  \item Es wird versucht die Bilddatei am ausgewählten Ort abzuspeichern.
\end{enumerate}
\textbf{Alternativen:} -

%\fano{<++>}\label{fa:<++>}
%\textbf{Ziel:} <++> \\
%\textbf{Kategorie:} <++> \\
%\textbf{Vorbedingung:} <++> \\
%\textbf{Nachbedingung (Erfolg):} <++> \\
%\textbf{Nachbedingung (Fehlschlag):} <++> \\
%\textbf{Auslösende Ereignisse:}
%\begin{enumerate}[nolistsep, label=(\alph*)]
%  \item <++>
%\end{enumerate}
%\textbf{Beschreibung:}
%\begin{enumerate}[nolistsep]
%  \item <++>
%\end{enumerate}
%\textbf{Alternativen:}
%\begin{enumerate}[nolistsep]
%  \item <++>
%\end{enumerate}


\chapter{Produktdaten}\label{ch:daten}

\begin{itemize}
  \item Gesetzte Einstellungen (z.B. Speicherort der zuletzt exportierten Datei oder Größe des Fensters) werden mithilfe der \gls{prefapi} von Java gespeichert.
  \item Als Inputformat für die zu zeichnenden Graphen wird das \gls{graphml}-Format unterstützt.\\
    Das GraphML-Format ist so aufgebaut, dass nicht alle Konstrukte welche mit GraphML beschrieben werden können (wie z.B. \gls{hyperkante}),
    von einer Applikation unterstützt werden müssen.
    Nicht unterstützte Konstrukte können ignoriert werden, ohne strukturelle Informationen des restlichen Graphens zu verlieren.\\
    Wir listen hier die GraphML-Konstrukte, welche von dem Produkt unterstützt werden müssen und für welche eventuell eine Unterstützung später hinzugefügt wird, auf:
    \begin{itemize}
      \item Muss-Konstrukte: graph, node, edge, desc, key, data, default, hierarchische/geschachtelte Graphen  % Ist Port ein Musskriterium? Beim ersten Treffen wurde es erwähnt
      \item Kann-Konstrukte: hyperedge, port, endpoint, Multi-Graphen
    \end{itemize}
    Für den Fall das ein Kann-Konstrukt nicht unterstützt wird, wird es, wie in der GraphML Spezifikation vorgeschrieben, behandelt.
  \item Das Produkt wird den resultierenden Graphen als \gls{svg}-Format exportieren können.
\end{itemize}


\chapter{Produktleistungen}\label{ch:leistungen}

\chapter{Benutzungsschnittstelle}

\chapter{Globale Testfälle}

\chapter{Qualitätsbestimmungen}

\chapter{Systemmodelle}

\clearpage
\printglossary[type=\acronymtype]
\printglossary[title=Glossar,toctitle=Glossar]

\listoffigures

\end{document}
