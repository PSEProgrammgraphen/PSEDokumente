\documentclass[a4paper]{scrreprt}

\usepackage[german]{babel}
\usepackage[utf8]{inputenc}
\usepackage[T1]{fontenc}
\usepackage[titletoc]{appendix}
\usepackage{ae}
\usepackage[bookmarks,bookmarksnumbered]{hyperref}

\usepackage[toc,acronym]{glossaries}
\makeglossaries

\usepackage{xparse}
\DeclareDocumentCommand{\newdualentry}{ O{} O{} m m m m } {
  \newglossaryentry{gls-#3}{name={#5},text={#5\glsadd{#3}},
    description={#6},#1
  }
  \makeglossaries
  \newacronym[see={[Glossary:]{gls-#3}},#2]{#3}{#4}{#5\glsadd{gls-#3}}
}

\loadglsentries{glossary.tex}
\begin{document}

\title{Pflichtenheft\\
Graph von Ansicht}
\date{}
\author{Nicolas Boltz   \\ uweaw@student.kit.edu
  \and Jonas Fehrenbach \\ urdtk@student.kit.edu
  \and Sven Kummetz     \\ kummetz.sven@gmail.com
  \and Jonas Maier      \\ Meierjonas96@web.de
  \and Lucas Steinmann  \\ ucemp@student.kit.edu
}
\maketitle

% Ich denke einen Abstract brauchen wir nicht aber hier ist mal ein Template einfach die Kommentarzeichen wegmachen
%\begin{abstract}
%\end{abstract}

\tableofcontents

\chapter{Zielbestimmung}

\section{Musskriterien}

\section{Wunschkriterien}

\section{Abgrenzungskriterien}

\chapter{Produkteinsatz}
\section{Anwendungsbereiche}

\section{Zielgruppen}

\section{Betriebsbedingungen}

\chapter{Produktumgebung}
\label{ch:umgebung}

\section{Software}
Das Produkt muss in folgenden Desktop-Systemen ausführbar und wie im restlichen Dokument beschrieben benutzbar sein:
\begin{itemize}
  \setlength\itemsep{0em}
  \item Linux Fedora 22/23 % nochmal die Versionen in der ATIS nachsehen und abgleichen.
  \item Linux Ubuntu 15.10/16.04 LTS
  \item Windows 7 und höher
\end{itemize}
Zur Programmierung wird die Programmiersprache Java benutzt. Daher ist eine Installation der \gls{jre} 8+ zur Ausführung notwendig.

\section{Hardware}
Das Produkt ist als \gls{jfx}-Anwendung zur Ausführung auf Desktop-Systemen konzipiert.
Durch die Verwendung von Java ist das Produkt unabhängig von Details der unterliegenden Hardware, sofern diese in der Lage ist die benötigte \gls{jre} (siehe oben) auszuführen.
Um die in Kapitel~\ref{ch:leistungen} beschriebenen nichtfunktionalen Anforderungen einhalten zu können, sind folgende Mindestanforderungen an das System, auf dem das Produkt ausgeführt soll, notwendig:

\begin{itemize}
  \setlength\itemsep{0em}
  \item Arbeitsspeicher: 4 GB
  \item Prozessor: Intel Core i5-4210U %Jonas F. Prozessor; vlt. durch gleichwertigen Desktopprozessor ersetzen.
  \item Festplattenspeicher: 100 MB
  \item Display: 1280x960
\end{itemize}


\chapter{Produktfunktionen}

\chapter{Produktdaten}

\chapter{Produktleistungen}

\chapter{Benutzungsschnittstelle}

\chapter{Globale Testfälle}

\chapter{Qualitätsbestimmungen}


\clearpage
\printglossary[type=\acronymtype]
\printglossary

\listoffigures

\end{document}
