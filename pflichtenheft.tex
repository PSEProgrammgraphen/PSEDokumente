\documentclass[a4paper]{scrreprt}

\usepackage[german]{babel}
\usepackage[utf8]{inputenc}
\usepackage[T1]{fontenc}
\usepackage[titletoc]{appendix}
\usepackage{ae}
\usepackage[bookmarks,bookmarksnumbered]{hyperref}

\begin{document}

\title{Pflichtenheft\\
Visualisierung von Programmgraphen}
\date{}
\author{Nicolas Boltz   \\ uweaw@student.kit.edu
  \and Jonas Fehrenbach \\ urdtk@student.kit.edu
  \and Sven Kummetz     \\ kummetz.sven@gmail.com
  \and Jonas Maier      \\ Meierjonas96@web.de
  \and Lucas Steinmann  \\ ucemp@student.kit.edu
}
\maketitle

% Ich denke einen Abstract brauchen wir nicht aber hier ist mal ein Template einfach die Kommentarzeichen wegmachen
%\begin{abstract}
%\end{abstract}

\tableofcontents

\chapter{Zielbestimmung}
\section{Musskriterien}

\section{Wunschkriterien}

\section{Abgrenzungskriterien}

\chapter{Produkteinsatz}

\section{Anwendungsbereiche}

\section{Zielgruppen}

\section{Betriebsbedingungen}

\chapter{Produktumgebung}

\section{Software}

\section{Hardware}

\chapter{Produktfunktionen}

\chapter{Produktdaten}

\chapter{Produktleistungen}

\chapter{Benutzungsschnittstelle}

\chapter{Globale Testfälle}

\chapter{Qualitätsbestimmungen}

\begin{appendices}
\chapter{Glossar}
\end{appendices}

\listoffigures

\end{document}
