\documentclass[a4paper]{scrreprt}

\usepackage[german]{babel}
\usepackage[utf8]{inputenc}
\usepackage[T1]{fontenc}
\usepackage[titletoc]{appendix}
\usepackage{ae}
\usepackage[bookmarks,bookmarksnumbered]{hyperref}

\usepackage[toc,acronym]{glossaries}
\makeglossaries

\usepackage{xparse}
\DeclareDocumentCommand{\newdualentry}{ O{} O{} m m m m } {
  \newglossaryentry{gls-#3}{name={#5},text={#5\glsadd{#3}},
    description={#6},#1
  }
  \makeglossaries
  \newacronym[see={[Glossary:]{gls-#3}},#2]{#3}{#4}{#5\glsadd{gls-#3}}
}

\loadglsentries{glossary.tex}
\begin{document}

\title{Pflichtenheft\\
Graph von Ansicht}
\date{}
\author{Nicolas Boltz   \\ uweaw@student.kit.edu
  \and Jonas Fehrenbach \\ urdtk@student.kit.edu
  \and Sven Kummetz     \\ kummetz.sven@gmail.com
  \and Jonas Maier      \\ Meierjonas96@web.de
  \and Lucas Steinmann  \\ ucemp@student.kit.edu
}
\maketitle

% Ich denke einen Abstract brauchen wir nicht aber hier ist mal ein Template einfach die Kommentarzeichen wegmachen
%\begin{abstract}
%\end{abstract}

\tableofcontents

<<<<<<< HEAD
\chapter{Zielbestimmung}

\section{Musskriterien}

\section{Wunschkriterien}

\section{Abgrenzungskriterien}

\chapter{Produkteinsatz}
\section{Anwendungsbereiche}

\section{Zielgruppen}

\section{Betriebsbedingungen}
=======
\chapter{Zielbestimmung}

Graph von Ansicht soll Programmgraphen von externen Analyse-Programmen visualisieren. Dafür wird eine bereits vorhandene Graphdatei importiert und ausgewertet. Der Benutzer kann verschiedene Constraints einstellen, welche im weiteren Dokument näher erläutert werden. 
%TODO: Verweis einfügen zu den konkreten Erläuterungen
Das Endergebnis soll eine übersichtliche Darstellung der Abhängigkeiten und des Steuerflusses eines Programmes zeigen, das dem Benutzer die Möglichkeit bietet, das Programm besser analysieren zu können.

\section{Pflichtkriterien}

\begin{itemize}
\item Schnittstellen
\begin{itemize}
\item Import von generischen Graphen im \gls{graphml}-Format
\item Export der visualisierten Graphen im \gls{svg}-Format
\end{itemize}
\item User interface
\begin{itemize}
\item Navigation mittels Zoom und Translation
\item Selektieren von einzelnen oder mehreren Knoten
\item Einstellen von Constraints
\item Kollabieren und Ausklappen von Gruppen und Teilgraphen
\item Filter für Knoten- und Kantentypen
\item Ausblendung selektierter Knoten
\end{itemize}
\item Visualisierung
\begin{itemize}
\item Hierarchisches Layout mit dem \gls{sugiyama}
\item Kanten werden durch \gls{bezier} dargestellt
\item Informationsanzeige von Knoten und Kanten
\item Weitere Layoutalgorithmen sollen leicht integrierbar und austauschbar sein
\end{itemize}
\item Layout
\begin{itemize}
	\item Sprungpunkte (aka "Knubbel", wie bei yComp) etc...
	%evtl. genauer, anders, besser, mehr ... ;)
\end{itemize}
\end{itemize}

\section{Wunschkriterien}

\begin{itemize}
\item Schnittstellen
\begin{itemize}
\item Export der visualisierten Graphen im \gls{jpg}- und \gls{graphml}-Format
\item Erweiterbarkeit des Programmes durch Plugin Schnittstellen
\end{itemize}
\item User interface
\begin{itemize}
\item Muster Definition mittels "GraphRegex"
\item Benutzerdefinierte Hotkeys
\end{itemize}
\item Visualisierung
\begin{itemize}
\item Fortschrittsbalken bei Berechnung der Visualisierung des Graphen mithilfe einer Zeitabschätzung
\item Minimap bzw. Übersicht des Graphens
\item Algorithmus zur Erreichbarkeit eines Knoten
\item Die visuelle Darstellung von Kanten kann geändert werden (\gls{bezier}, orthogonale Kanten oder direkte Kanten)
\end{itemize}
\end{itemize}

\section{Abgrenzungskriterien}

\begin{itemize}
\item Das Produkt ist kein Graph Editor und unterstützt nicht das manuelle Zeichnen/Hinzufügen von neuen Kanten und Knoten.
\item Die GUI wird nicht von Grund auf neu entwickelt, es werden Bibliotheken verwendet, um die Entwicklung zu erleichtern. 
\item Das Darstellen von Kanten und Knoten selbst wird mithilfe von Bibliotheken umgesetzt.
\item Das Produkt ist kein Analysetool für Programme, sondern dient lediglich zur Visualisierung von bereits vorhandenen Graphdateien.
\end{itemize}

\chapter{Produkteinsatz}

\section{Anwendungsbereiche}
Graph von Ansicht soll zur Visualisierung von Programmgraphen eingesetzt werden. Der Nutzer soll dadurch die Abhängigkeiten und den Steuerfluss eines Programms besser verstehen.
\section{Zielgruppen}
Institute und Forschungsgruppen, die sich mit der Analyse von Programmen beschäftigen. Das Programm richtet sich auch an Menschen welche vorhandene Graphdateien visualisieren und sich mit dem Thema Programmgraphen vertraut machen wollen.

\section{Betriebsbedingungen}
Das Produkt besteht aus einem Graphviewer (Graph von Ansicht) und wird mit einem JOANA-Plugin ausgeliefert. Somit benötigt es keine weiteren Installationen (abgesehen von der \gls{jre}), damit es funktioniert. Es benötigt auch keine aktive Internetverbindung oder ein Netzwerk. Es wird lediglich eine Graphdatei als Input benötigt. Das Programm ist für Linux und Windows Betriebssysteme ausgelegt und funktioniert nur auf diesen garantiert (siehe \nameref{ch:umgebung}).
>>>>>>> 3c659f53f6048faa135ff8d666fa8b8e287c2559

\chapter{Produktumgebung}
\label{ch:umgebung}

\section{Software}\label{sec:software}
Das Produkt muss in folgenden Desktop-Systemen ausführbar und, wie im restlichen Dokument beschrieben, benutzbar sein:
\begin{itemize}
  \setlength\itemsep{0em}
  \item Linux Fedora 22/23 %TODO nochmal die Versionen in der ATIS nachsehen und abgleichen.
  \item Linux Ubuntu 15.10/16.04 LTS
  \item Windows 7 und höher
\end{itemize}
Zur Programmierung wird die Programmiersprache Java benutzt. Daher ist eine Installation der \gls{jre} 8+ zur Ausführung notwendig.

\section{Hardware}
Das Produkt ist als \gls{jfx}-Anwendung zur Ausführung auf Desktop-Systemen konzipiert.
Durch die Verwendung von Java ist das Produkt unabhängig von Details der unterliegenden Hardware, sofern diese in der Lage ist, die benötigte \gls{jre} (siehe \ref{sec:software}) auszuführen.
Um die in Kapitel~\ref{ch:nfa} beschriebenen nichtfunktionalen Anforderungen einhalten zu können, sind folgende Mindestanforderungen an das System, auf dem das Produkt ausgeführt werden soll, notwendig:

\begin{itemize}
  \setlength\itemsep{0em}
  \item Arbeitsspeicher: 4 GB
  \item Prozessor: Intel Core i5-4210U %TODO Jonas F. Prozessor; vlt. durch gleichwertigen Desktopprozessor ersetzen.
  \item Festplattenspeicher: 200 MB 
  \item Display: 1280x960
\end{itemize}


\chapter{Produktfunktionen}

\chapter{Produktdaten}

\chapter{Produktleistungen}

\chapter{Benutzungsschnittstelle}

\chapter{Globale Testfälle}

\chapter{Qualitätsbestimmungen}


\clearpage
\printglossary[type=\acronymtype]
\printglossary

\listoffigures

\end{document}
