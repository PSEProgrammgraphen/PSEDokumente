\chapter{Durchgeführte Tests}
\label{ch:durchgefuehrtetests}

\section{Globale Testfälle aus dem Pflichtenheft}

\textbf{/T010/: }Import und Darstellung von einem JOANA Graphen\\
\textbf{Anmerkungen: }-\\

\textbf{/T020/: }Öffnen eines JOANA-Methodengraphen\\
\textbf{Anmerkungen: }-\\

\textbf{/T030/: }Selektieren mehrerer Knoten und Kanten\\
\textbf{Anmerkungen: }Da es sich während der Implementierung als wenig nützlich erwiesen hat, wurde das selektieren von Kanten nicht implementiert. Deswegen ist es in diesem Test auch nicht möglich, Kanten zu selektieren.\\
	Es werden außerdem bei mehreren ausgewählten Knoten weder eine Statistik noch eine Information zu diesen angezeigt. Lediglich von einem einzigen Knoten werden Informationen angezeigt.\\

\textbf{/T040/: }Navigation\\
\textbf{Anmerkungen: }Das Verschieben des Sichtfeldes geschieht nach der Implementierung nicht per Mittelmaus-klick halten und ziehen, sondern über Strg + Rechts-klick gedrückt halten und ziehen mit der Maus.\\

\textbf{/T050/: }Constraint zu Knoten eines geladenen Graphen hinzufügen\\
\textbf{Anmerkungen: }Durch Verschieben des Kriteriums der manuell hinzufügbaren Constraints zu Gruppen in die Wunschkriterien (siehe Pflichtenheft: 4.2 Wunschfunktionen, /FA300/ Constraints hinzufügen), wurde dies in der Implementierungsphase nicht hinzugefügt. Es ist daher hier lediglich möglich, Gruppen zu erstellen (bis einschließlich Punkt 4 des Tests).\\

\textbf{/T060/: }Filtern von Kanten\\
\textbf{Anmerkungen: }Die Menüführung zum Erreichen der Filter von Kanten ist "Other->Edit Filter", danach klick auf Edges und dann hinzufügen eines Haken zum filtern dieser Kante, anstatt "Editieren->Filter anpassen" mit anschließendem klick auf Joana und entfernen von Haken.\\

\textbf{/T070/: }Export von einem geladenen JOANA-Graphen als SVG\\
\textbf{Anmerkungen: }Die Menüführung zum exportieren des geladenen JOANA-Graphen gechieht über "File->Export", anstatt "Datei->Export->SVG". Da nur ein Export-Format zur Auswahl steht, wurde auf den letzten Schritt der geplanten Menüführung verzichtet.\\


\section{JUnit Tests}
%alle gemachten JUnit tests. Falls welche sich mit denne aus dem Implementierungsheft überschneiden, bzw. dieselben sind, dann Verweis auf den jeweiligen Eintrag im Implementierungsheft


\section{Hallway Usability Testing}
%hier die Einträge zu den Tests mit "unwissenden" leuten. ggf. Datum des testens und commithash des verwendeten Programmes beifügen. Format der Darstellung noch offen.
Hier werden unwissende Probanden vor das Programm gesetzt, ihnen die Funktionsweise und Möglichkeiten der Funktionen des Programmes erläutert, sie dann teste lassen und Auffälligkeiten notiert. Auffälligkeiten können sowohl in der Benutzbarkeit des Programmes, also wie die Probanden zurechtkommen, vorkommen, aber auch in der Funktionsweise des Programmen, indem man selbst oder die Probanden mögliche unerwünschte Nebeneffekte oder Fehler im Programm sehen.\\

\textbf{Proband 1: } Datum 16.08.2016, commit hash: 6b06dd0\\
\textbf{Auffälligkeiten: } Es traten keine unerwarteten Fehler im Programm auf.\\
Es wurde vor allem mit collapse herumgespielt, den Menüpunkten zm Filtern, neu importieren, exportieren wurde kaum Aufmerksamkeit geschenkt.\\

\textbf{Proband 2:} Datum 16.08.2016, commit hash: 6b06dd0\\
\textbf{Auffälligkeiten: } Auch hier traten keine nicht schon bekannten Fehler im Programm auf.\\
Verwirrend war zum einen, dass man durch Doppelklick auf einen Methodengraphen in der Strukturansicht diesen öffnen kann, jedoch durch einen Doppelklick auf einen Knoten im Callgraphen diesen dazugehörigen Mathodengraphen nicht öffnen konnte, nur mit "Rechtsklick->open".\\
Zum anderen war das Graph verschieben mithilfe der Tastenkombination Strg + Recktsklick ungewöhnlich.\\
Ebenso unschön wurde das Filtern von Kanten über das checken der checkboxes gefunden, sowie die Unwissenheit über momentan vorhandene Knoten und Kanten im Graphen. Man kann auch Knoten und Kanten filtern, die gar nicht im Graph vorhanden sind.\\




\section{Monkey Testing}
%Falls dahingehend etwas getan wurde

\section{Andere durchgeführte Tests}
%hier alles wa sgetan wurde, um z.b. javadocs zu fixen, mögliche bugs mit werkzeug zu fixen, codequalität zu verbessern, etc. reinschreiben mit Beschreibung und Zweck 