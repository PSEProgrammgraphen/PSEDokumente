\chapter{Durchgeführte Tests}
\label{ch:durchgefuehrtetests}

\section{Globale Testfälle aus dem Pflichtenheft}

\textbf{/T010/: }Import und Darstellung von einem JOANA Graphen\\
\textbf{Anmerkungen: }-\\

\textbf{/T020/: }Öffnen eines JOANA-Methodengraphen\\
\textbf{Anmerkungen: }-\\

\textbf{/T030/: }Selektieren mehrerer Knoten und Kanten\\
\textbf{Anmerkungen: }Da es sich während der Implementierung als wenig nützlich erwiesen hat, wurde das selektieren von Kanten nicht implementiert. Deswegen ist es in diesem Test auch nicht möglich, Kanten zu selektieren.\\
	Es werden außerdem bei mehreren ausgewählten Knoten weder eine Statistik noch eine Information zu diesen angezeigt. Lediglich von einem einzigen Knoten werden Informationen angezeigt.\\

\textbf{/T040/: }Navigation\\
\textbf{Anmerkungen: }Das Verschieben des Sichtfeldes geschieht nach der Implementierung nicht per Mittelmaus-klick halten und ziehen, sondern über Strg + Rechts-klick gedrückt halten und ziehen mit der Maus.\\

\textbf{/T050/: }Constraint zu Knoten eines geladenen Graphen hinzufügen\\
\textbf{Anmerkungen: }Durch Verschieben des Kriteriums der manuell hinzufügbaren Constraints zu Gruppen in die Wunschkriterien (siehe Pflichtenheft: 4.2 Wunschfunktionen, /FA300/ Constraints hinzufügen), wurde dies in der Implementierungsphase nicht hinzugefügt. Es ist daher hier lediglich möglich, Gruppen zu erstellen (bis einschließlich Punkt 4 des Tests).\\

\textbf{/T060/: }Filtern von Kanten\\
\textbf{Anmerkungen: }Die Menüführung zum Erreichen der Filter von Kanten ist ''Other->Edit Filter'', danach klick auf Edges und dann hinzufügen eines Haken zum filtern dieser Kante, anstatt "Editieren->Filter anpassen" mit anschließendem klick auf Joana und entfernen von Haken.\\

\textbf{/T070/: }Export von einem geladenen JOANA-Graphen als SVG\\
\textbf{Anmerkungen: }Die Menüführung zum exportieren des geladenen JOANA-Graphen gechieht über "File->Export", anstatt "Datei->Export->SVG". Da nur ein Export-Format zur Auswahl steht, wurde auf den letzten Schritt der geplanten Menüführung verzichtet.\\


\section{JUnit Tests}
%alle gemachten JUnit tests. Falls welche sich mit denne aus dem Implementierungsheft überschneiden, bzw. dieselben sind, dann Verweis auf den jeweiligen Eintrag im Implementierungsheft


\section{Hallway Usability Testing}
%hier die Einträge zu den Tests mit "unwissenden" leuten. ggf. Datum des testens und commithash des verwendeten Programmes beifügen. Format der Darstellung noch offen.
Hier wird das Programm durch fachfremde Probanden, die das Programm nicht kennen, getestet.\\
Es wird ihnen erst die Funktionsweise und des Programmes erläutert, anschließend dürfen sie frei testen.\\
Alle hierbei auftretenden Auffälligkeiten, seien es mögliche Fehler oder Verbesserungsmöglichkeiten in der Bedienbarkeit, unerwünschte Nebeneffekte oder gar schwerwiegende Fehler, die das Programm abstürzen lassen, werden notiert.\\

\textbf{Proband 1: } Datum 16.08.2016, commit hash: 6b06dd0\\
\textbf{Auffälligkeiten: } Es traten keine unerwarteten Fehler im Programm auf.\\
Es wurde vor allem mit collapse gearbeitet, den Menüpunkten zm Filtern, neu importieren, exportieren wurde kaum Aufmerksamkeit geschenkt.\\
Der Proband war mit der Menüführung etwas überfordert, da es dort viele für ihn unbekannte Funktionen gab, welche er somit nur selten testete.

\textbf{Proband 2:} Datum 16.08.2016, commit hash: 6b06dd0\\
\textbf{Auffälligkeiten: } Auch hier traten keine unerwarteten Fehler im Programm auf.\\
Verwirrend war zum einen, dass man durch Doppelklick auf einen Methodengraphen in der Strukturansicht diesen öffnen kann, jedoch durch einen Doppelklick auf einen Knoten im Callgraphen diesen dazugehörigen Methodengraphen nicht öffnen konnte, nur mit "Rechtsklick->open".\\
Zum anderen war das Graph verschieben mithilfe der Tastenkombination Strg + Recktsklick ungewöhnlich.\\
Ebenso unschön wurde das Filtern von Kanten über das checken der checkboxes gefunden, sowie die Unwissenheit über momentan vorhandene Knoten und Kanten im Graphen. Man kann auch Knoten und Kanten filtern, die gar nicht im Graph vorhanden sind.\\




\section{Manuelle Tests}

\subsection{Randfalltests}
Es wurden Randfälle des Programmverlaufes getestet, die entweder zu Fehlern führen könnten oder schon zu welchen geführt haben.\\
Ausgeführt wurden Randfalltests zum Beispiel im Import, dem Layoutalgorithmus.\\
Dem Importer wurden .graphml Dateien übergeben, welche nicht einen JOANA-Graphen darstellen\\
Im Layoutalgorithmus wurden Graphen mit isolierten Knoten, Knoten auf dem selben Layer, Kanten die zumindest ein Layer überspringen und somit intern einen Pfad darstellen und Knoten mit Selbstzykel getestet. Diese Fälle müssen einzeln betrachtet werden, da davon ausgegangen wurde, dass das die Knotenstruktur auf dan Layer hierarchisch ist, also die Kanten nur von oben nach unten gehend.\\


\section{Andere durchgeführte Tests}
%hier alles wa sgetan wurde, um z.b. javadocs zu fixen, mögliche bugs mit werkzeug zu fixen, codequalität zu verbessern, etc. reinschreiben mit Beschreibung und Zweck 
\subsection{Überdeckungstests}
Es wurden Überdeckungstests sowohl über JUnit-Tests der einzelnen Projekte, als auch über alle Projekte während eines Programmlaufes mithilfe von EclEmma durchgeführt.
%hier dann bilder der Überdeckungstests hinein

\subsection{Performance Tests}
Getestet wurden hier Methodengraphen aus erstellten Graphen des JOANA-Graph-Analyzer.\\
Gemessen wird die Zeit für das Berechnen des Layout des Methodengraphen, also von der Übergabe des Methodengraphen in interner Repräsentation nach dem Import an den Layoutalgorithmus bis dieser erst einen SuguyamaGraph erstellt und anschließend die fünf Sugiyama-Steps ausgeführt hat.\\

\textbf{Datei: }
\textbf{Methodengraph: }
\textbf{Information zum Graph: }
\textbf{Programmeinstellungen: }
\textbf{Zeit: }



%\textbf{Datei: }
%\textbf{Methodengraph: }
%\textbf{Information zum Graph: }
%\textbf{Programmeinstellungen: }
%\textbf{Zeit: }