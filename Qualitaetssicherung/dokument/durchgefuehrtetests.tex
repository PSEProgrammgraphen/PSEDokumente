\chapter{Durchgeführte Tests}
\label{ch:durchgefuehrtetests}

\newcounter{tnr}
\newcommand\test[2]{\textbf{\arabic{tnr}}\addtocounter{tnr}{1}. & \textbf{Test:} & #1 \\ & \textbf{Aufgabe:} & #2 \\ [1ex] }

\section{Globale Testfälle aus dem Pflichtenheft}

\textbf{/T010/: }Import und Darstellung von einem JOANA Graphen\\
\textbf{Anmerkungen: }-\\

\textbf{/T020/: }Öffnen eines JOANA-Methodengraphen\\
\textbf{Anmerkungen: }-\\

\textbf{/T030/: }Selektieren mehrerer Knoten und Kanten\\
\textbf{Anmerkungen: }Da es sich während der Implementierung als wenig nützlich erwiesen hat, wurde das selektieren von Kanten nicht implementiert. Deswegen ist es in diesem Test auch nicht möglich, Kanten zu selektieren.\\
	Es werden außerdem bei mehreren ausgewählten Knoten weder eine Statistik noch eine Information zu diesen angezeigt. Lediglich von einem einzigen Knoten werden Informationen angezeigt.\\

\textbf{/T040/: }Navigation\\
\textbf{Anmerkungen: }Das Verschieben des Sichtfeldes geschieht nach der Implementierung nicht per Mittelmaus-klick halten und ziehen, sondern über Strg + Rechts-klick gedrückt halten und ziehen mit der Maus.\\

\textbf{/T050/: }Constraint zu Knoten eines geladenen Graphen hinzufügen\\
\textbf{Anmerkungen: }Durch Verschieben des Kriteriums der manuell hinzufügbaren Constraints zu Gruppen in die Wunschkriterien (siehe Pflichtenheft: 4.2 Wunschfunktionen, /FA300/ Constraints hinzufügen), wurde dies in der Implementierungsphase nicht hinzugefügt. Es ist daher hier lediglich möglich, Gruppen zu erstellen (bis einschließlich Punkt 4 des Tests).\\

\textbf{/T060/: }Filtern von Kanten\\
\textbf{Anmerkungen: }Die Menüführung zum Erreichen der Filter von Kanten ist \glqq{}Other $\to$ Edit Filter\grqq, danach klick auf Edges und dann hinzufügen eines Haken zum filtern dieser Kante, anstatt \glqq{}Editieren $\to$ Filter anpassen\grqq mit anschließendem klick auf Joana und entfernen von Haken.\\

\textbf{/T070/: }Export von einem geladenen JOANA-Graphen als SVG\\
\textbf{Anmerkungen: }Die Menüführung zum exportieren des geladenen JOANA-Graphen gechieht über \glqq{}File $\to$ Export\grqq, anstatt \glqq{}Datei $\to$ Export $\to$ SVG\grqq{}. Da nur ein Export-Format zur Auswahl steht, wurde auf den letzten Schritt der geplanten Menüführung verzichtet.\\


\section{JUnit Tests}
%alle gemachten JUnit tests. Falls welche sich mit denne aus dem Implementierungsheft überschneiden, bzw. dieselben sind, dann Verweis auf den jeweiligen Eintrag im Implementierungsheft
\subsection{Sugiyama Plugin}
\subsubsection{CycleRemover}
\setcounter{tnr}{1}
\begin{longtable}{llp{0.8\linewidth}}
	\test{testSimpleCycle()}{Erstellt einen Testgraphen aus drei Knoten die jeweils mit einem anderen Knoten über eine gerichtete Kante verbunden sind, sodass ein Kreis entsteht. Diesen übergibt es an einen \textit{SugiyamaGraph}. Dann ruft es auf dem Cycle remover \textit{removeCycles()} mit dem \textit{SugiyamaGraph} als Parameter und testet danach ob der Graph azyklisch ist.}
	\test{testDoubleCycle()}{Erstellt einen Testgraphen aus vier Knoten und fünf gerichteten Kanten. Die Kanten sind so miteinander verknüpft, dass zwei Zykel entstehen. Ein kleiner Zykel drei Knoten beinhaltender und ein großer Zykel, der alle vier Knoten und damit auch den kleineren Zykel enthaltenden. Diesen übergibt es an einen \textit{SugiyamaGraph}. Dann ruft es auf dem Cycle remover \textit{removeCycles()} mit dem \textit{SugiyamaGraph} als Parameter und testet danach ob der Graph azyklisch ist.}
	\test{RandomGraphsTest()}{Erstellt zwanzig zufällige, zyklische \textit{SugiyamaGraphen}. Diese bestehen beim n-ten Graphen aus aus 2n Knoten und haben eine Kantendichte von $0.95^{n}$. Dann ruft es auf dem Cycle remover \textit{removeCycles()} mit dem \textit{SugiyamaGraph} als Parameter und testet danach ob der Graph azyklisch ist.}
	\test{SelfLoopTest()}{Erstellt einen Testgraphen aus einem Knoten mit einer gerichteten Kante als Selbstschleife. Diesen übergibt es an einen \textit{SugiyamaGraph}. Dann ruft es auf dem Cycle remover \textit{removeCycles()} mit dem \textit{SugiyamaGraph} als Parameter und testet danach ob diese Kante umgedreht wurde.}
\end{longtable}

\subsubsection{LayerAssigner}
\setcounter{tnr}{1}
\begin{longtable}{llp{0.8\linewidth}}
	\test{assignLayers()}{Erstellt einen azyklischen Testgraphen aus fünf Knoten, welche die korrekte Ebene für die Knoten angeben. Diese werden mit fünf Kanten verbunden. Diesen übergibt es an einen \textit{SugiyamaGraph}. Dann ruft es auf dem LayerAssigner \textit{assignLayers()} mit dem \textit{SugiyamaGraph} als Parameter und testet mithilfe der Labels, ob die Knoten den richtigen Schichten zugewiesen wurden.}
	\test{LayerAssignerTest2()}{Erstellt einen azyklischen Testgraphen aus sieben Knoten, welche die korrekte Ebene für die Knoten angeben. Diese werden mit zehn Kanten verbunden. Diesen übergibt es an einen \textit{SugiyamaGraph}. Dann ruft es auf dem  LayerAssigner \textit{assignLayers()} mit dem \textit{SugiyamaGraph} als Parameter und testet mithilfe der Labels, ob die Knoten den richtigen Schichten zugewiesen wurden.}
\end{longtable}

\subsubsection{CrossMinimizer}
\setcounter{tnr}{1}
\begin{longtable}{llp{0.8\linewidth}}
	\test{singleRandomTest()}{Erstellt einen zufälligen, zyklischen, topologisch gelayerten \textit{SugiyamaGraphen}. Diese bestehen aus 20 Knoten, die jeweils 2-8 Kanten haben. Dann ruft es auf dem CrossMinimizer \textit{minimizeCrossings()} mit dem \textit{SugiyamaGraph} als Parameter und testet danach ob die Anzahl der Kreuzungen verringert wurde oder zumindest gleich bleibt.}
	\test{randomTests()}{Erstellt 20 zufälligen, zyklischen, topologisch gelayerten \textit{SugiyamaGraphen}. Diese bestehen beim n-ten durchlauf aus ${n} + 10$ Knoten, die jeweils 3-4 Kanten haben. Dann ruft es auf dem CrossMinimizer \textit{minimizeCrossings()} mit dem \textit{SugiyamaGraph} als Parameter und testet danach ob die Anzahl der Kreuzungen verringert wurde oder zumindest gleich bleibt.}
	\test{performanceTest()}{Erstellt 20 zufälligen, zyklischen, topologisch gelayerten \textit{SugiyamaGraphen}. Diese bestehen aus 75 Knoten, die jeweils 2-8 Kanten haben. Dann ruft es auf dem CrossMinimizer \textit{minimizeCrossings()} mit dem \textit{SugiyamaGraph} als Parameter und testet danach ob die Anzahl der Kreuzungen verringert wurde oder zumindest gleich bleibt. Dabei geht es besonders um die Zeit die der Test benötigt.}
	\test{hugeTest()}{Erstellt einen zufälligen, zyklischen, topologisch gelayerten \textit{SugiyamaGraphen}. Dieser besteht aus 250 Knoten, die jeweils 2-6 Kanten haben. Dann ruft es auf dem CrossMinimizer \textit{minimizeCrossings()} mit dem \textit{SugiyamaGraph} als Parameter und testet danach ob die Anzahl der Kreuzungen verringert wurde oder zumindest gleich bleibt. Dabei geht es besonders um die Zeit die der Test benötigt.}
\end{longtable}

\subsubsection{SugiyamaLayoutAlgorithm}
\setcounter{tnr}{1}
\begin{longtable}{llp{0.8\linewidth}}
	\test{testSmallGraph()}{Erstellt einen kleinen Graphen mit 4 Knoten und 5 Kanten, auf diesem wird dann der Sugiyama Algorithmus ausgeführt und überprüft ob dieser ohne Exceptions durchläuft.}
	\test{testIsolatedGraph()}{Erstellt einen kleinen Graphen mit 4 Knoten und 2 Kanten, sodass zwei voneinander isolierte Teilgraphen entstehen. Auf diesem wird dann der Sugiyama Algorithmus ausgeführt und überprüft ob dieser ohne Exceptions durchläuft.}
	\test{testRandomGraph()}{Erstellt 3 zufällige Graphen mit 100 Knoten und über 400 Kanten, der auch Selbstschleifen enthalten kann. Auf diesem wird dann der Sugiyama Algorithmus ausgeführt und überprüft ob dieser ohne Exceptions durchläuft. Dieser Test wird auch zum Performance überprüfen verwendet.}
\end{longtable}

\subsection{SVG-Export}
\setcounter{tnr}{1}
\begin{longtable}{llp{0.8\linewidth}}
	\test{fileEndingTest()}{Testet ob der \textit{SvgExporter} die untersützten Datei Formate korrekt zurückgibt.}
	\test{exportSucessfullTest()}{Testet ob beim Exportieren eine Fehlermeldung geworfen wird.}
\end{longtable}

\section{Hallway Usability Testing}
%hier die Einträge zu den Tests mit "unwissenden" leuten. ggf. Datum des testens und commithash des verwendeten Programmes beifügen. Format der Darstellung noch offen.
Hier wird das Programm durch fachfremde Probanden, die das Programm nicht kennen, getestet.\\
Es wird ihnen erst die Funktionsweise und des Programmes erläutert, anschließend dürfen sie frei testen.\\
Alle hierbei auftretenden Auffälligkeiten, seien es mögliche Fehler oder Verbesserungsmöglichkeiten in der Bedienbarkeit, unerwünschte Nebeneffekte oder gar schwerwiegende Fehler, die das Programm abstürzen lassen, werden notiert.\\

\textbf{Proband 1: } Datum 16.08.2016, commit hash: 6b06dd0\\
\textbf{Auffälligkeiten: } Es traten keine unerwarteten Fehler im Programm auf.\\
Es wurde vor allem mit collapse gearbeitet, den Menüpunkten zm Filtern, neu importieren, exportieren wurde kaum Aufmerksamkeit geschenkt.\\
Der Proband war mit der Menüführung etwas überfordert, da es dort viele für ihn unbekannte Funktionen gab, welche er somit nur selten testete.

\textbf{Proband 2:} Datum 16.08.2016, commit hash: 6b06dd0\\
\textbf{Auffälligkeiten: } Auch hier traten keine unerwarteten Fehler im Programm auf.\\
Verwirrend war zum einen, dass man durch Doppelklick auf einen Methodengraphen in der Strukturansicht diesen öffnen kann, jedoch durch einen Doppelklick auf einen Knoten im Callgraphen diesen dazugehörigen Methodengraphen nicht öffnen konnte, nur mit "Rechtsklick->open".\\
Zum anderen war das Graph verschieben mithilfe der Tastenkombination Strg + Recktsklick ungewöhnlich.\\
Ebenso unschön wurde das Filtern von Kanten über das checken der checkboxes gefunden, sowie die Unwissenheit über momentan vorhandene Knoten und Kanten im Graphen. Man kann auch Knoten und Kanten filtern, die gar nicht im Graph vorhanden sind.\\


\section{Manuelle Tests}
\subsection{Genereller GUI Test}
Um die GUI zu testen wurde, neben den Globalen Systemtests, ein Test formuliert, der händisch durchgeführt wird. Dabei werden die verschiedenen Funktionen grob in ihrer Funktionalität getestet und die Überdeckung des Codes überprüft. \\ \\
Der Test besteht aus folgenden Schritten: \\
\textbf{1:} Import der Datei Fibonacci.pdg.graphml beim Start per Kommandozeile. 

\textbf{->} Graph von Ansicht wird geöffnet und der Callgraph der Fibonacci-Datei wird angezeigt.\\
\textbf{2:} Import der Datei Fields.pdg.graphml über den Menüeintrag Import.

\textbf{->} Der Callgraph von Fibonacci schließt sich und der Callgraph von Fields wird angezeigt.\\
\textbf{3:} Selektieren des Graphen java.io.printStram.print() in der StrukturView.

\textbf{->} InformationView zeigt Informationen über den Graphen.\\
\textbf{4:} Öffnen des Graphen java.io.printStram.print() per Kontextmenü.

\textbf{->} Der Methodgraph zu java.io.printStram.print() wird in einem neuen Tab geöffnet.\\
\textbf{5:} Maximieren des Fensters, zoomen und Graph bewegen das er in die Ansicht passt.

\textbf{->} Graph passt in die Ansicht, die Steuerung funktioniert wie definiert.\\
\textbf{6:} Selektieren mehrerer Knoten und kollabieren über das Kontextmenü.

\textbf{->} Selektierte Knoten sind zu einem Knoten kollabiert, der Graph wurde neu gelayouted.\\
\textbf{7:} Selektieren mehrerer Knoten, hinzufügen zu einer Gruppe, Farbe der Gruppe ändern und schließen des Dialogs über Ok.

\textbf{->} Die selektierten Knoten haben eine Umrandung in der ausgewählten Farbe.\\
\textbf{8:} Erneut Knoten zu einer neuen Gruppe hinzufügen, ein Knoten der zuvor ausgewählt war mit auswählen.

\textbf{->} Die selektierten Knoten haben eine Umrandung in der ausgewählten Farbe. Der zuvor bereits in einer Gruppe enthaltene Knoten ist in der Farbe der ersten Gruppe umrandet.\\
\textbf{9:} Farben und Position beider Gruppen im Dialog ändern und Dialog per X schließen.

\textbf{->} Ansicht bleibt gleich. Erneutes öffnen des Dialogs zeigt die gleichen Werte wie beim zuvorigen öffnen.\\
\textbf{10:} Entfernen der ersten der beiden Gruppen über den Dialog.

\textbf{->} Der Knoten der in zwei Gruppen war, wird nun in Farbe der zweiten Gruppe angezeigt.\\
\textbf{11:} Öffnen des Graphen fields.t2() per Doppelklick in der StrukturView.

\textbf{->} Der Methodengraph wurde in einem neuen Tab geöffnet.\\
\textbf{12:} Öffnen des Preferences-Dialogs, ändern der Einstellungen und schließen über Ok.

\textbf{->} Graph wird neu gelayouted angezeigt.\\
\textbf{13:} Erneutes öffnen des Preferences-Dialogs, ändern der Einstellungen und schließen über X.

\textbf{->} Erneutes öffnen des Dialogs zeigt die alten Einstellungen an.\\
\textbf{14:} Anwenden eines neuen Methodgraph-Layouts, erhöhen der Einstellungen auf gechecked, 0,01 und 1500.

\textbf{->} Der Graph wird neu gelayouted angezeigt. Öffnen des Preferences-Dialogs zeigt die gemachten Einstellungen.\\
\textbf{15:} Exportieren von fields.t2() als .svg.

\textbf{->} Datei wird am angegebenen Ort exportiert und enthält den passenden Graphen.\\
\textbf{16:} Schließen des Tabs von fields.t2().

\textbf{->} Tab wurde geschlossen. Ansicht ist wieder bei "java.io.printStram.print()".\\
\textbf{17:} Wechseln zum Tab callgraph.

\textbf{->} callgraph wird angezeigt.\\
\textbf{18:} Doppelklick auf Graph java.io.printStram.print() in der StrukturView.

\textbf{->} Automatischer Wechsel zum Tab von java.io.printStram.print().\\
\textbf{19:} Öffnen des Filter Dialogs und Filtern von allen Vertex-Typen der ersten zwei Spalten. Dialog über Ok schließen.

\textbf{->} Graph wird neu gelayouted und verschiedene Vertices sind nicht mehr vorhanden.\\
\textbf{20:} Entfernen der Vertex-Filter und Dialog über Ok schließen.

\textbf{->} Graph wird wieder neu gelayouted mit allen Vertices.\\
\textbf{21:} Öffnen des Filter Dialogs und Filtern von allen Edge-Typen der ersten zwei Spalten.

\textbf{->} Der Graph wird neu gelayouted und verschiedene Edges sind nicht mehr vorhanden.\\
\textbf{22:} Entfernen der Edge-Filter und Dialog über Ok schließen.

\textbf{->} Graph wird neu gelayoutetd und alle Edges werden wieder angezeigt.\\
\textbf{23:} Export von java.io.printStram.print() als .svg.

\textbf{->} Datei wird am angegebenen Ort exportiert und enthält den passenden Graphen.\\
\textbf{24:} Schließen von Graph von Ansicht über den Menüeintrag Exit.

\textbf{->} Graph von Ansicht schließt sich.\\
\\
\textbf{Überdeckung app: }86,4\%\\
\textbf{Überdeckung graphml: }80,9\%\\
\textbf{Überdeckung joana: }82,4\%\\
\textbf{Überdeckung shared: }57,9\%\\
\textbf{Überdeckung sugiyama: }84,2\%\\
\textbf{Überdeckung svg: }89,9\%\\
Die Werte können sich noch verbessern, da noch einiges an Code ungenutzt ist, aber in Zukunft noch genutzt werden könnte.


\subsection{Randfalltests}
Es wurden Randfälle des Programmverlaufes getestet, die entweder zu Fehlern führen könnten oder schon zu welchen geführt haben.\\
Ausgeführt wurden Randfalltests zum Beispiel im Import, dem Layoutalgorithmus.\\
Dem Importer wurden .graphml Dateien übergeben, welche nicht einen JOANA-Graphen darstellen\\
Im Layoutalgorithmus wurden Graphen mit isolierten Knoten, Knoten auf dem selben Layer, Kanten die zumindest ein Layer überspringen und somit intern einen Pfad darstellen und Knoten mit Selbstzykel getestet. Diese Fälle müssen einzeln betrachtet werden, da davon ausgegangen wurde, dass das die Knotenstruktur auf dan Layer hierarchisch ist, also die Kanten nur von oben nach unten gehend.\\


\section{Andere durchgeführte Tests}
%hier alles wa sgetan wurde, um z.b. javadocs zu fixen, mögliche bugs mit werkzeug zu fixen, codequalität zu verbessern, etc. reinschreiben mit Beschreibung und Zweck 
\subsection{Überdeckungstests}
Es wurden Überdeckungstests sowohl über JUnit-Tests der einzelnen Projekte, als auch über alle Projekte während eines Programmlaufes mithilfe von EclEmma durchgeführt.\\
Das Programm enthält folgende Projekte, die nach Möglichkeit einzeln getestet wurden und für die es auch, falls möglich, extra JUnit Tests gibt:\\

\begin{longtable}{l p{0.8\linewidth}}
	\textbf{app} & In diesem Projekt befindet sich die View und alles zu GUI gehörende. Aufgrund der enorm hohen Zahl an Userinteraktionen (Klick, Klick + Ziehen, Rechtsklick, Graph verschieben, Knoten selektieren...) gibt es keine expliziten JUnit Tests und es kann auch nicht vollständig abgedeckt werden. \\

	\textbf{Graph von Ansicht} & Dieses Projekt verbindet die einzelne Plugins. Es existieren Testfälle die einzelne Projekte und Klassen testet, als auch Testfälle, welche komplette Prozesse und Programmabläufe automatisiert testen. \\
	
	\textbf{graphml} & Hier befindet sich der GraphML-Importer. Es existieren keine expliziten Testfälle, da zum Import die GraphBuilder Klassen benötigt werden, auf welche des graphml-Projekt keinen Zugriff besitzt. Die Testfälle hierzu liegen in \textbf{Graph von Ansicht}.\\
	
	\textbf{joana} & Dieses Projekt enthält JOANA-spezifische Funktionalitäten und Strukturen (z.B. Call- und Methodengraph, FieldAccess....)\\
	
	\textbf{shared} & Hier sind das allgemeine Graphmodel und die Graphbuilder-Interfaces vorhanden. Explizite Testfälle liegen keine vor, da keine große Algorithmen durchgeführt werden, sondern meist nur Getter- und Setter-Methoden oder Interfaces.\\
	
	\textbf{sugiyama} & In diesem Projekt befindet sich alles zum Layoutalgorithmus Sugiyama-Layout. Dazu gehören unter anderen die fünf Sugiyama-Schritte, SugiyamaGraph und der SugiyamaLayoutAlgorithm. Es existieren viele Testfälle da die einzelnen Schritte explizit getestet werden müssen. \\
	
	\textbf{svg} & Der SVG-Exporter wird explizit auf Exceptions getestet. \\

\end{longtable}


\subsubsection{Testfälle aus dem Pflichtenheft}
Alle Testfallszenarien aus dem Pflichtenheft werden direkt nacheinander ausgeführt.\\
Die Überdeckung über alle Projekte ist sehr stark abhängig von der importierten Datei und dem benutzten Graphen dieser Datei, da diese Graphen zum Teil unterschiedliche Eigenschaften haben(selfloop, kanten auf dem selben layer...).\\

\textbf{Datei: }Fibonacci.pdg.graphml\\
\textbf{Graph: }Fibonacci.mult()\\
\textbf{Überdeckung app: }72,1\%\\
\textbf{Überdeckung graphml: }86,4\%\\
\textbf{Überdeckung joana: }80,8\%\\
\textbf{Überdeckung shared: }54,2\%\\
Es werden Parameter, die vor dem Layouten des Methodengraphen verändert werden können, durch diese Tests nicht abgedeckt.\\
\textbf{Überdeckung sugiyama: }75,2\%\\
Auch hier können keine änderbaren Parameter verarbeitet werden, weil keine gesetzt wurden.\\
\textbf{Überdeckung svg: }90\%\\
\subsubsection{JUnit Tests}

\textbf{Überdeckung app: }\\
\textbf{Überdeckung graphml: }\\
\textbf{Überdeckung joana: }\\
\textbf{Überdeckung shared: }\\
\textbf{Überdeckung sugiyama: }\\
\textbf{Überdeckung svg: }90\%\\

\subsubsection{Test über Programmdurchlauf}
Es werden alle Method- und Callgraphen mit einer Knotenzahl von unter 1000 aus allen Dateien außer BigCG nacheinander importiert und gelayouted.\\

\textbf{Überdeckung app: }0\%\\
Keine GUI Interaktionen.\\
\textbf{Überdeckung graphml: }80.9\%\\
\textbf{Überdeckung joana: }80.1\%\\
\textbf{Überdeckung shared: }47.2\%\\
Parameter werden nicht angepasst, FastGraphAccessor ist komplett unbenutzt.\\
\textbf{Überdeckung sugiyama: }78.6\%\\
SugiyamaGraph bietet manche unbenutzte Funktionalität, die nie benutzt wird.\\
CrossMinimizer hat nur eine Überdeckung von 57.4\%, da dieser veränderte Parameter berücksichtigt, werden diese Abschnitte aufgrund von nicht veränderten Parametern nicht ausgeführt.\\
\textbf{Überdeckung svg: }0.4\%\\
Kein Export wird ausgeführt.\\

\subsection{Performance Tests}
Getestet wurden hier Methodengraphen aus erstellten Graphen des JOANA-Graph-Analyzer.\\
Gemessen wird die Zeit für das Berechnen des Layout des Methodengraphen, also von der Übergabe des Methodengraphen in interner Repräsentation nach dem Import an den Layoutalgorithmus bis dieser erst einen SuguyamaGraph erstellt und anschließend die fünf Sugiyama-Steps ausgeführt hat.\\
Die benötigte Zeit zum Layouten ist nicht nur abhängig von der Anzahl der Knoten und Kanten eines Graphen, sondern auch von dessen Aufbau (Anzahl an logischen Layern, einzelne dichte Stellen im Graphen, ...).\\
Ebenso hängt die benötigte Zeit stark von der Anzahl der Läufe des Kreuzungsminimierers bei der Kreuzungsminimierung zwischen zwei Layern ab. Die Anzahl an Läufen kann über den Menüpunkt \glqq{}Layout $\to$ Properties\grqq angepasst werden.\\
Im folgenden werden Graphen mit einer Anzahl von 10 Läufen gelayouted.\\

\textbf{Datei: }BigControl.pdg.graphml\\
\textbf{Methodengraph: }BigControl.main()\\
\textbf{Information zum Graph: }779 Knoten, 4917 Kanten, 1 FieldAccess\\
\textbf{Programmeinstellungen: }CrossMinimizer runs: 10, use threshold: false\\
\textbf{Zeit: }6.6s, 162.4s, 30.5s, 8.9s, 6.6s, 7.6s, 369.1s, 6.9s \\

\textbf{Datei: }PhiMadness.pdg.graphml\\
\textbf{Methodengraph: }PhiMadness.phiMadness()\\
\textbf{Information zum Graph: }393 Knoten, 2937 Kanten, 25 FieldAccesses\\
\textbf{Programmeinstellungen: }CrossMinimizer runs: 10, use threshold: false\\
\textbf{Zeit: }44.5s, 79.6s, 46.6s, 58.3s, 51.7s, 38.3s, 43.4s, 45.1s\\

%Hier Satz, warum Zeiten so extrem abweichen können, auch mit constraints und fieldaccesses argumentieren!



%\textbf{Datei: }
%\textbf{Methodengraph: }
%\textbf{Information zum Graph: }
%\textbf{Programmeinstellungen: }
%\textbf{Zeit: }