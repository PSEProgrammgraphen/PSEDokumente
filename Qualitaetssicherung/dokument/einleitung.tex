\chapter{Einleitung}
\label{ch:einleitung}

Nach der Implementierung von Graph von Ansicht ist nun die Qualitätssicherung an der Reihe. Ziel ist es, Fehler im Programm durch verschiedene Vorgehensweisen zu finden und diese anschließend zu beheben. Ziel ist es auch, durch sowohl zufällige als auch wohl überlegte Testfälle und der Überprüfung dieser auf Korrektheit, die richtige Funktionsweise des Programmes ein Stück weit mehr garantieren zu können. Je mehr Testfälle funktionieren, desto mehr mögliche Fehlerquellen können wahrscheinlicher ausgeschlossen werden.\\%unschön
Da das Programm nach der Implementierung nicht komplett fertig war und es noch einige Stellen gab, an denen aus Zeitgründen nicht weiter gearbeitet werden konnte und diese somit auch teilweise nicht vollständig funktionsfähig sind, wurden diese zuallererst fertig implementiert. Noch zu erledigen waren in diesem Projekt: das korrekte Darstellen von Feldzugriffen (eine bestimmte Teilmenge von Knoten und Kanten des Graphen), dessen Kanten nicht richtig gezeichnet wurden, zeitliche Zusicherungen, welche versprochen wurden (1000Knoten+4000Kanten unter 2 Minuten, 500Knoten+2000Kanten unter einer Minute), ein Programmabsturz bei falschen Kommandozeilenargumenten soll verhindert werden.
Graph von Ansicht ist ein Graphviewer, seine Hauptaufgabe besteht darin, gegebene Graphen zu layouten. Deswegen liegt das Hauptaugenmerk des Testens auf dem Layoutalgorithmus. Die korrekte Funktionsweise des Algorithmus kann aufgrund der hohen Anzahl an möglichen Eingaben nicht garantiert werden, auch bietet das Programm durch eine eingebaute Filterfunktion für Knoten und Kanten eine Vielzahl an Möglichkeiten den Graphen nochmals zu modifizieren bevor der Algorithmus erneut ausgeführt werden kann. Deswegen wurden viele Tests erstellt, die vor allem Randfälle (zum Beispiel ein Graph ohne Knoten oder Kanten, mit einem Selbstzyklus (Kante in sich selbst), etc.) abdecken.
Ebenso wird die Benutzbarkeit des Programmes getestet über das Testen des Programmes durch Personen, die sowohl von dem Thema als auch der Funktionsweise des Programmes unwissend sind. Auffälligkeiten hinsichtlich der Handhabung und Navigation in dem Programm werden angepasst, sodass sich auch fachfremde Personen in der Menüführung zurecht finden.



%Nach der Implementierung ist nun die Qualitätssicherungsphase an der Reihe, mit eine der wichtigsten Phasen eines Softwareprojektes. Denn hier wird garantiert, dass ein Programm auch sicher das tut, was es soll. \\
%Zuallererst werden hier Programmfunktionalitäten, die aus Gründen mangelnder Zeit nicht mehr bzw. nicht komplett oder korrekt implementiert wurden, hinzugefügt und Fehler behoben.
%Anschließend gilt es, die in der Implementierungsphase implementierten Funktionalitäten des Programmes auf Herz und Nieren zu testen und sowohl schon bekannte, als auch mögliche Fehler zu beseitigen.
%Da Graph von Ansicht eine graphische Benutzerschnittstelle bietet, ist es nicht möglich, alle nur denkbaren Szenarien durch zu testen, da innerhalb dieser Benutzerschnittstelle viele Aktionen beliebig oft wiederholt werden können und diese sich gegenseitig beeinflussen.\\
%Vielmehr wird nun darauf gesetzt mit Testfällen, die einerseits einen ordnungsgemäßen Ablauf aber auch andererseits einen möglichen Randfall oder gar einen fehlerhaften Ablauf darstellen können, das Programm zu simulieren und das Ergebnis mit dem erwarteten Wert abzugleichen.\\
%Begleitet werden diese Tests von unzähligen manuellen Durchläufen und Testen des Programmes, sowohl durch die Entwickler, als auch durch unwissende Personen. Letzteres dient vor allem zur Verbesserung der Benutzbarkeit des Programmes und zur besseren Abdeckung von Randfällen des Programmes.\\