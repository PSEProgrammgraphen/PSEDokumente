\chapter{Bugs und Reparatur}
\label{ch:bugsundreparatur}

\newcounter{cnr}
\newcommand{\bug}[4]{\textbf{\##1} & \textbf{Fehlersymptom:} & #2 \\ & \textbf{Fehlergrund:} & #3 \\ & \textbf{Fehlerbehebung:} & #4 \\ [1ex] }


\begin{longtable}{llp{0.8\linewidth}}
\bug{1}
       {Graphische Hervorhebung von FieldAccessen durch eine farbige Box im Hintergrund wird nicht angezeigt.}
       {Die Struktur im Hintergrund ist leicht fehlerhaft und in der Erstellung der Box gab es einen Bug.}
       {Die Struktur im Hintergrund ist erneuert worden und der Bug beim Erstellen des Hintergrunds ist behoben.}
\bug{2}
       {Einige Kanten werden übereinanderliegend gezeichnet.}
       {Dummy Vertices werden übereinander platziert, da der Codeabschnitt zur Überprüfung von Kantenüberlagerungen fehlerhaft ist. Außerdem werden die Kanten die mehrere Ebenen überdecken nicht immer weit genug verschoben. Auch Mehrfachzuweisungen von Knoten in zu begradigenden Abschnitten treten auf.}
       {Ein fehlerhafter Vergleich bei der Überprüfung von Kantenschnitten im VertexPositioner worde korrigiert. Er kann nun auch mit 1 breiten Kanten arbeiten. Die Verschiebung der Kanten wird nun so oft ausgeführt, bis sich Kanten nicht mehr überdecken. Der schwerwiegender Fehler, welcher zu den Mehrfachzuweisungen führte wurde auch behoben. Hier hat ein rekursiver Teilalgorithmus die Ergebnisse verdoppelt.}

\bug{3}
	{Kanten, die von außen in einen FieldAccess hinein oder von einem FieldAccess nach außen gehen, werden nur bis an die Box des FieldAccesses gezeichnet, und nicht bis zu dem dazugehörigen Knoten.}
	{Nach dem Layouten des kompletten Graphen werden die Kanten nicht von der Box nach innen weiter gezeichnet.}
	{Alle Kanten innerhalb eines FieldAccesses werden nun nach dem Layouten des kompletten Graphen nochmals neu gezeichnet. Sowohl die alten, als auch die neuen. Die Positionen der Knoten und die außerhalb liegende Kanten bleiben dabei gleich.}

\bug{7}
       {Kanten, welche keine Ebene überspannen werden nie begradigt}
       {Diese Kanten werden beim Suchen nach Segmenten (Kantenketten, die im VertexPositioner später begradigt werden) nicht beachtet}
       {Diese Kanten werden nun beim Starten des VertexPositioner ebenfalls hinzugefügt. Im Zuge dieser Korrektur wurde es außerdem ermöglicht die Endpunkte von Kanten, welche eine oder mehr Ebenen überspringen, zu den aus ihnen enstehenden Segmenten hinzuzufügen.}
       
\bug{8}
	   {Kollabierte Knoten in einem FieldAccessGraph verschwinden bei Ihrer Erstellung.}
	   {Bei dem Layouten von MethodGraphen werden alle FieldAccesse betrachtet. Auch FieldAccesse die teilweise kollabiert werden.}
	   {Es werden nun nur noch FAs welche vollständig ausgeklappt sind im MethodGraphLayout betrachtet.}
	   
\bug{9}
       {Beim Öffnen eines Graphens über die Kommandozeile wird nicht das Default Layout des ausgewählten Workspace übernommen.}
       {Es wird immer der LayoutSelectionDialog aufgerufen, obwohl ein Workspace als Kommandozeilenargument angegeben wurde.}
       {Der Aufruf des Dialogs wurde gelöscht und das Default Layout aus dem Workspace wird angewendet.}

\bug{10}
       {Keine Informationen oder Warnung über falsch eingegebene Kommandozeilenparameter.}
       {Keine explizite Überprüfung der Eingaben. Es wurde angenommen, dass der User alle Eingaben richtig macht.}
       {Hinzufügen von expliziten Überprüfungen und Anzeigen von Fehlermeldungen mit hilfreicher Information.}
       
\bug{11}
       {Falls das Joana-Workspace für eine generische GraphML-Datei ausgewählt wird, kommt keine Warnung und es passiert nichts.}
       {Keine Überprüfung ob ein GraphBuilder für diesen Graphentyp existiert.}
       {Der Importer überprüft ob der GraphBuilder gesetzt wird. Falls nicht wird eine Fehlermeldung angezeigt.}
       
\bug{12}
       {Der Graph springt beim verschieben mit der Maus in der Graphview zu Beginn ein Stück nach unten.}
       {Für die Berechnung der Mausposition werden die Koordinaten des Fensters genutzt.}
       {Die Mausposition wird nun direkt aus dem Mausevent, welches durch das Ziehen aufgerufen wird, berechnet.}
       
\bug{13}
       {Es gibt keine Option zur Einstellung ob eine Abbruchgrenze beim Kreuzungsminimieren in Abhängigkeit von den vorhandenen Kreuzungen gewählt werden soll.}
       {Diese Option wird in den CrossMinimizer Einstellungen nicht bereitgestellt.}
       {Die Option wurde als Checkbox hinzugefügt und wird während der Kreuzungsminimierung beachtet.}
       
\bug{17}
	   {Das Filtern von Knoten bestimmten Types (z.B. EXPR) im MethodGraphLayout führt zu einer NullPointerException}
	   {Für das Layouten der FieldAccesse und des ganzen MethodGraphen wurde der selbe Layoutalgorithmus, mit den selben Constraints gewählt.}
	   {Separate Instanzen des SugiyamaAlgorithm für FieldAccesse und MethodGraphen.}
	   
\bug{22}
       {Die alte Strukturansicht kann nach dem Import einer neuen Datei nicht zurückgebracht werden.}
       {Alte Graphen werden nach einem neuen Import nicht geschlossen, obwohl die Anwendung nur eine einzelne Datei unterstützt.}
       {Bereits offene Graphen werden nach erfolgreichem Import nun geschlossen.}
       
\bug{23}
       {Ähnlich wie in Bug 22 wird versucht nach einem Import über den noch offenen alten Callgraphen einen Methodengraphen zu öffnen.}
       {Siehe Bug 22}
       {Siehe Bug 22}
       
\bug{24}
       {Programm stürzt ab, falls der eingegebene Dateipfad über den Kommandozeilenparameter -~-in keinen Punkt enthält.}
       {Falls Dateipfad keinen Punkt erhält gibt Java Methode lastIndexOf('.') als Ergebnis -1 zurück. Dieses Ergebniss wird bei substring() als ungültiger Index verwendet was zum Abstürz führt.}
       {Vor Aufruf der Methode substring() wird überprüft ob der Dateipfad einen Punkt enthält. Falls nicht wird eine Fehlermeldung ausgegeben.}
       
\bug{25}
       {Es wird ein Fehler geworfen, wenn man die ENTR Knoten filtert.}
       {Der VertexPositioner benötigt mindestens ein Vertex pro Graph.}
       {Der VertexPositioner überprüft nun am Anfang, ob der Graph leer ist und läuft in diesem Fall nicht durch.}      
        
\bug{27}
       {Text in exportierter SVG-Datei ist größer als die Textboxen der Knoten.}
       {Default Textgröße in SVG ist größer als Default Textgröße innerhalb der GUI. Die Größe der Textboxen richten sich jedoch nach der Textgröße der GUI.}
       {Hinzufügen einer expliziten Textgröße zur SVG-Datei, welche der Textgröße der GUI entspricht.}
		
\bug{28}
	{Kanten innerhalb eines FieldAccess laufen ggf. aus der Box hinaus.}
	{Kantenläufe werden bei der Berechnung der Größe der Box nicht berücksichtigt.}
	{Kantenläufe werden nun bei der Berechnung der Größe der Box berücksichtigt.}
		
\bug{30}
	{Kanten laufen durch Knoten durch und manchmal von unten rein und verbinden sich dann oben auf dem Knoten.}
	{Der Kantenzeichner geht davon aus, dass Kanten immer von oben nach unten verlaufen, also der source-Knoten sich immer oberhalb des target-Knoten befindet. Durch relative- und absolute-layer-constraints kann es aber vorkommen, dass miteinander verbundene Knoten auf dem selben layer liegen, weswegen die Kanten teilweise unvorhersehbar durch Knoten gezeichnet werden.}
	{Kanten zwischen Knoten auf dem selben layer werden nun gesondert gezeichnet, und zwar unterhalb des layers.}
		
\bug{31}
        {Beim Zoomen wird nicht vom Mauszeiger weg oder zum Mauszeiger hin gezoomt. Der Fokus und Sichtbereich springt herum.}
        {Die GraphView wird durch das Zoomen vergrößert, dadurch gibt es von den äußeren Oberflächenelementen, die um die GraphView gelegt sind, Interferenzen, welche das Springen und unfokussierte Zoomen ausgelöst haben.}
        {Die Ansicht des Graphen besteht nun aus 4 ineinander verschachtelten Panes, wobei das innerste die GraphView und das äußerste ein ScrollPane ist. Vergrößert wird beim Zoomen nun das Pane welches direkt um der GraphView liegt. Das Pane darum passt sich per Listener immer an die Größe der beiden inneren Panes an. Dadurch entstehen keinen Interferenzen mehr und anderes ungewolltes Verhalten tritt auch nicht mehr auf.}
        
\bug{40}
        {Die Richtung mancher Kanten zwischen zwei Knoten ändert sich manchmal nach neuem Importieren oder neuem Layouten des Graphen.}
        {Im Sugiyama werden notwendigerweise zuallererst Kanten gedreht, die Zykel im Graph verursachen. Der Kantenzeichner im letzten Schritt des Sugiyama hat nun nur die gedrehten Kanten, die sich über ein Layer erstrecken in ihrer Richtung angepasst. Kanten, die über mehrere Layer gingen wurden ignoriert, da diese in einem anderen Set lagen.}
        {Der Kantenzeichner passt nun durch Berücksichtigung des Sets der sich über mehrere Layer erstreckenden Kanten, von diesen ehemals gedrehte Kanten in ihrer Richtung an.}

\bug{42}
	{ArrayIndexOutOfBoundsException im VertexPositioner nach dem Filtern von NORM Knoten und dann EXPR Knoten}
	{Nach einem commit verschwand dieser Fehler und äußert sich durch die selben Merkmale wie Bug 44, siehe daher 44.}
	{siehe Bug 44}

\bug{44}
	{Nach dem Filtern von NORM Knoten aus einem Graphen, tritt eine NullPointerException im EdgeDrawer auf.}
	{Der Graph enthält nach dem Filtern einen isolierten Knoten.}
	{Der EdgeDrawer schaut nun nach isolierten Knoten, bevor er auf Knoten zugreift.}

\bug{45}
	{FieldAccess Boxen werden in der exportierten SVG-Datei nicht angezeigt.}
	{Die FieldAccess Boxen werden nicht zum serialisierten Graphen hinzugefügt.}
	{Die FieldAccess Boxen werden nun als zusätzliche Knoten im serialisierten Graphen gespeichert.}
	
\bug{46}
	{Exportierte SVG-Datei wird in manchen SVG-Viewer nicht komplett angezeigt.}
	{Einige SVG-Viewer können die Standard Größe 100\% einer SVG-Datei nicht korrekt interpretieren.}
	{Die komplette Größe eines Graphens wird berechnet und als feste Werte in die SVG-Datei geschrieben.}

\bug{51}
  {Vertices werden hinter anderen Vertices positioniert}
  {Der VertexPositioner positioniert Vertices, die auf einer Ebene liegen aber durch eine Kante verbunden sind fehlerhaft}
  {Im VertexPositioner werden Vertices, welche auf einer Ebene liegen und durch eine Kante verbunden sind nun nicht mehr versucht auf eine Linie zu bringen.}

\bug{52}
  {Im Callgraph finden sich Kantenkreuzungen}
  {Die Anzahl der CrossMinimizer durchläufe beim callgraph war zu gering. Dies liegt vor allem an einer prozentualen Verbesserungsgrenze für Callgraphen, diese macht es wahrscheinlich, dass nur ein einzelner Durchlauf des CrossMinimizers stattfindet.}
  {Der Callgraph wird nun per Default mindestens 10 mal durchgeführt. Einer prozentualen Grenze für die Verbesserung wird nun nicht mehr verwendet.}

\bug{53}
	{Das Programm stürzt ab falls beim Import eines Joana Graphen ein falscher Knotentyp eingelesen wird.}
	{Alle mögliche Joana Knotentypen werden in einem Enum gespeichert. Falls der Typ nicht im Enum existiert wird eine IllegalArgumentException geworfen.}
	{Die Exception wird nun im Importer abgefangen und als Fehlermeldungen dem Nutzer angezeigt.}

\bug{57}
	{Im Callgraph werden keine Schleifen (Kante, welche einen Knoten mit sich selbst verbindet) angezeigt.}
	{Beim Import-Vorgang werden, beim Erstellen des Callgraphen, Schleifen ignoriert. Dies liegt daran, dass Source- und Target-Knoten einer Schleife im selben MethodGraph liegen und somit diese Kante zum MethodGraph hinzugefügt wird und nicht zum CallGraph. }
	{Falls eine Kante vom \glqq{}EdgeKind CL\grqq{} ist (welche Call-Kanten definieren), wird diese zum CallGraph hinzugefügt. }

		
%		\bug{}
%		{}
%		{}
%		{}
\end{longtable}