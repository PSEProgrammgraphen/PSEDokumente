\chapter{Offene Bugs}
\label{ch:offenebugs}

\newcommand{\openbug}[3]{\textbf{\##1} & \textbf{Fehlersymptom:} & #2 \\ & \textbf{Bemerkung} & #3 \\ [1ex] }


\begin{longtable}{llp{0.7\linewidth}}

\openbug{21}
		{Der Scrollbalken wird nicht richtig angepasst, wenn man mit Rechtsklick die Zeichenfläche zieht.}
		{Das Berechnen der Position des Scrollbalken ist mit Berücksichtung des Zoomgrades schwieriger als zu Beginn gedacht. Es gibt keine Unterstützung des Frameworks.}
		
\openbug{33}
		{Wenn Knoten aus einem FieldAccess kollabiert wird, wird eine Exception geworfen. Das Programm stürzt nicht ab, aber der Graph wird nicht neu gelayoutet.}
		{Siehe issue 35.}
		
\openbug{35}
		{Wenn Knoten, welche in einem FieldAccess vorhanden sind, gefiltert werden, stürzt der LayoutAlgorithmus ab.}
		{Durch das späte Hinzufügen von Filtern und Berücksichtigung von FieldAccess als hervorgehobene SubGraphen entstanden viele im Nachhinein Korrelationen zwischen verschiedenen Programmteile, die nicht leicht gelöst werden konnten.}
		
\openbug{36}
 		{Das Constraint, dass alle Parameter Knoten für Methoden Calls direkt unter dem Call-Knoten platziert werden, wird nicht eingehalten.}
 		{Aus zeitlichen Gründen wurde nicht implementiert, dass der Sugiyama-Algorithmus Constraints umsetzt, die einen festen Layerabstand zwischen zwei Knoten definiert. Für Call-Knoten wurde dies benötigt.}

\openbug{43}
       		{Der Sugiyama-Algorithmus ohne Spezialisierung auf MethodGraph/CallGraph ist nicht auswählbar.}
       		{Es werden keine Verbesserungen bei der Benutzung des Sugiyama-Algorithmus ohne Spezialisierung erwartet. Daher wurde dieses Feature zurückgestellt.}
       
\openbug{54}
 		{Größe der FieldaccessBox ist nach dem Filtern falsch.}
 		{Auftreten einer NullPointerException nach einem Commit. Kann erst nach Bug 35 gefixt werden.}
 		
\openbug{59}
 		{Gruppenfarbe von zwei nebeneinanderliegenden Knoten überlappen sich.}
 		{Wurde zurückgestellt da es schwerwiegendere Probleme gab und dies kein gravierender Fehler ist.}
\end{longtable}