\chapter{Bugs und Reparatur}
\label{ch:bugsundreparatur}

\newcounter{cnr}
\newcommand{\bug}[4]{\textbf{\##1} & \textbf{Fehlersymptom:} & #2 \\ & \textbf{Fehlergrund:} & #3 \\ & \textbf{Fehlerbehebung:} & #4 \\ [1ex] }


\begin{longtable}{llp{0.8\linewidth}}

\bug{9}
       {Beim Öffnen eines Graphens über die Kommandozeile wurde nicht das Default Layout des ausgewählten Workspace übernommen.}
       {Es wurde immer der LayoutSelectionDialog aufgerufen, obwohl ein workspace angegeben wurde}
       {Aufruf gelöscht und das Default Layout aus dem Workspace angewendet}

\bug{10}
       {Keine Informationen oder Warnung über falsch eingegebene Kommandozeilenparameter.}
       {Keine explizite Überprüfung der Eingaben. Es wurde angenommen, dass der User alle Eingaben richtig macht.}
       {Hinzufügen von expliziten Überprüfungen und Anzeigen von Fehlermeldungen mit hilfreicher Information.}

\bug{24}
       {Programm stürzt ab, falls der eingegebene Dateipfad über den Kommandozeilenparameter --in keinen Punkt enthält. }
       {Falls Dateipfad keinen Punkt erhält gibt Java Methode lastIndexOf('.') als Ergebnis -1 zurück. Dieses Ergebniss wird bei substring() als ungültiger Index verwendet was zum Abstürz führt.}
       {Vor Aufruf der Methode substring() wird überprüft ob der Dateipfad einen Punkt enthält. Falls nicht wird eine Fehlermeldung ausgegeben.}
       
\bug{27}
       {Text in exportierter SVG-Datei ist größer als die Textboxen der Knoten}
       {Default Textgröße in SVG ist größer als Default Textgröße innerhalb der GUI. Die Größe der Textboxen richten sich jedoch nach der Textgröße der GUI.}
       {Hinzufügen einer expliziten Textgröße zur SVG-Datei, welche der Textgröße der GUI entspricht.}
		
\end{longtable}