\chapter{Einleitung}
\label{ch:einleitung}

Nach der Implementierung ist nun die Qualitätssicherungsphase an der Reihe, mit eine der wichtigsten Phasen eines Softwareprojektes. Denn hier wird garantiert, dass ein Programm auch sicher das tut, was es soll. \\
Zuallererst werden hier Programmfunktionalitäten, die aus Gründen mangelnder Zeit nicht mehr bzw. nicht komplett oder korrekt implementiert wurden, hinzugefügt und Fehler behoben.
Anschließend gilt es, die in der Implementierungsphase implementierten Funktionalitäten des Programmes auf Herz und Nieren zu testen und sowohl schon bekannte, als auch mögliche Fehler zu beseitigen.
Da Graph von Ansicht eine graphische Benutzerschnittstelle bietet, ist es nicht möglich, alle nur denkbaren Szenarien durch zu testen, da innerhalb dieser Benutzerschnittstelle viele Aktionen beliebig oft wiederholt werden können und diese sich gegenseitig beeinflussen.\\
Vielmehr wird nun darauf gesetzt mit Testfällen, die einerseits einen ordnungsgemäßen Ablauf aber auch andererseits einen möglichen Randfall oder gar einen fehlerhaften Ablauf darstellen können, das Programm zu simulieren und das Ergebnis mit dem erwarteten Wert abzugleichen.\\
Begleitet werden diese Tests von unzähligen manuellen Durchläufen und Testen des Programmes, sowohl durch die Entwickler, als auch durch unwissende Personen. Letzteres dient vor allem zur Verbesserung der Benutzbarkeit des Programmes und zur besseren Abdeckung von Randfällen des Programmes.\\